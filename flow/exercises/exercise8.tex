% --------------------------------------------------------------
% This is all preamble stuff that you don't have to worry about.
% Head down to where it says "Start here"
% --------------------------------------------------------------
 
\documentclass[12pt]{article}
 
\usepackage[margin=1in]{geometry} 
\usepackage{amsmath,amsthm,amssymb}
\usepackage{graphicx}
 
\newcommand{\N}{\mathbb{N}}
\newcommand{\Z}{\mathbb{Z}}
\newcommand{\bh}{\boldsymbol{h}}
\newcommand{\bx}{\boldsymbol{x}}
\newcommand{\bu}{\boldsymbol{u}}
\newcommand{\bv}{\boldsymbol{v}}
\newcommand{\bw}{\boldsymbol{w}}
\newcommand{\bomega}{\boldsymbol{\omega}}
 
\newenvironment{theorem}[2][Theorem]{\begin{trivlist}
\item[\hskip \labelsep {\bfseries #1}\hskip \labelsep {\bfseries #2.}]}{\end{trivlist}}
\newenvironment{lemma}[2][Lemma]{\begin{trivlist}
\item[\hskip \labelsep {\bfseries #1}\hskip \labelsep {\bfseries #2.}]}{\end{trivlist}}
\newenvironment{exercise}[2][Exercise]{\begin{trivlist}
\item[\hskip \labelsep {\bfseries #1}\hskip \labelsep {\bfseries #2.}]}{\end{trivlist}}
\newenvironment{problem}[2][Problem]{\begin{trivlist}
\item[\hskip \labelsep {\bfseries #1}\hskip \labelsep {\bfseries #2.}]}{\end{trivlist}}
\newenvironment{question}[2][Question]{\begin{trivlist}
\item[\hskip \labelsep {\bfseries #1}\hskip \labelsep {\bfseries #2.}]}{\end{trivlist}}
\newenvironment{corollary}[2][Corollary]{\begin{trivlist}
\item[\hskip \labelsep {\bfseries #1}\hskip \labelsep {\bfseries #2.}]}{\end{trivlist}}

\setlength\parindent{0pt}
 
\begin{document}
 
\title{Finite Element Methods for Flow Simulations\\ - Exercise Sheet 8 -}
\author{Daniel Arndt}
\date{}
\maketitle

\begin{exercise}{14}
Let $p, q, r \in [1, \infty]$ with $\frac{1}{p}+\frac{1}{q}+\frac{1}{r}=1$ and $\Omega\subset\mathbb{R}^d$.
\begin{enumerate}
\item Show for $u,v,w\colon\Omega\to\mathbb{R}$
\begin{align*}
\left|\int_\Omega u v w \, dx\right|\leq \|u\|_{L^p(\Omega)}\|v\|_{L^q(\Omega)}\|w\|_{L^r(\Omega)}.
\end{align*}

\item Consider the convection term of the Navier-Stokes problem in the weak form
\begin{align*}
t(\bu, \bv, \bw) = \int_\Omega (\bu \cdot \nabla)\bv \cdot \bw \,dx.
\end{align*}
and show
\begin{align*}
|t(\bu, \bv, \bw)| \leq \|\bu\|_{L^p(\Omega)}\|\nabla \bv\|_{L^q(\Omega)} \|\bw\|_{L^r(\Omega)}.
\end{align*}
\end{enumerate}
\end{exercise}

\begin{exercise}{15}
Exercise 30:
Using the notation of the Sec. 7 of the lecture script, the projection step of pressure-correction methods
for time-dependent incompressible flow problems reads:
Given ũ +1 , find u n+1 \in H̃ := {v \in [L 2 (\Omega)] d : div v = 0, v \cdot n = 0 on ∂\Omega} and φ n+1 \in W 1,2 (\Omega) ∩ L 2 ∗ (\Omega)
such that
u n+1 + \nablaφ n+1 = ũ n+1 ;
div u n+1 = 0.
(i) Show that this projection step corresponds to a Poisson problem for φ n+1 with appropriate source
term and boundary conditions.
(ii) Why is this problem uniquely solvable ?
\end{exercise}

\begin{exercise}{16}
Exercise 26:
Consider the variational formulation of the Oseen problems: Find (u, p) \in X × Q := [H 0 1 (\Omega)] d × L 2 ∗ (\Omega)
such that
a Os (u, v) + b(p, v) = (f , v) ∀v \in X
b(q, u) = 0
∀q \in Q
(1)
(2)
with
1
a Os (u, v) := ν(\nablau, \nablav) + ((b \cdot \nabla)u, v) + ( (div b)u, v),
b(q, v) := −(q, div v).
2
Let P : X → X be the mapping which relates the vector field b to the solution u = P (b) of the Oseen
problem. Show that this mapping P is uniformly bounded, i.e. for u = T (b) there holds independently
of b resp. \nablab the estimate
kuk X \leq C(ν, \Omega, kf k X ∗ ) < \infty.
Argue why the fixed point u ∗ = P (u ∗ ) of the mapping P : X → X is the solution of the stationary
incompressible Navier-Stokes problem, i.e. of (1)-(2) with b = u.
\end{exercise}
\end{document}