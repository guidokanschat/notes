% --------------------------------------------------------------
% This is all preamble stuff that you don't have to worry about.
% Head down to where it says "Start here"
% --------------------------------------------------------------
 
\documentclass[12pt]{article}
 
\usepackage[margin=1in]{geometry} 
\usepackage{amsmath,amsthm,amssymb}
\usepackage{graphicx}
 
\newcommand{\N}{\mathbb{N}}
\newcommand{\Z}{\mathbb{Z}}
\newcommand{\bh}{\boldsymbol{h}}
\newcommand{\bx}{\boldsymbol{x}}
\newcommand{\bu}{\boldsymbol{u}}
\newcommand{\bv}{\boldsymbol{v}}
\newcommand{\bw}{\boldsymbol{w}}
\newcommand{\bomega}{\boldsymbol{\omega}}
 
\newenvironment{theorem}[2][Theorem]{\begin{trivlist}
\item[\hskip \labelsep {\bfseries #1}\hskip \labelsep {\bfseries #2.}]}{\end{trivlist}}
\newenvironment{lemma}[2][Lemma]{\begin{trivlist}
\item[\hskip \labelsep {\bfseries #1}\hskip \labelsep {\bfseries #2.}]}{\end{trivlist}}
\newenvironment{exercise}[2][Exercise]{\begin{trivlist}
\item[\hskip \labelsep {\bfseries #1}\hskip \labelsep {\bfseries #2.}]}{\end{trivlist}}
\newenvironment{problem}[2][Problem]{\begin{trivlist}
\item[\hskip \labelsep {\bfseries #1}\hskip \labelsep {\bfseries #2.}]}{\end{trivlist}}
\newenvironment{question}[2][Question]{\begin{trivlist}
\item[\hskip \labelsep {\bfseries #1}\hskip \labelsep {\bfseries #2.}]}{\end{trivlist}}
\newenvironment{corollary}[2][Corollary]{\begin{trivlist}
\item[\hskip \labelsep {\bfseries #1}\hskip \labelsep {\bfseries #2.}]}{\end{trivlist}}

\setlength\parindent{0pt}
 
\begin{document}
 
\title{Finite Element Methods for Flow Simulations\\ - Exercise Sheet 8 -}
\author{Daniel Arndt}
\date{}
\maketitle

\begin{exercise}{14}
Let $p, q, r \in [1, \infty]$ with $\frac{1}{p}+\frac{1}{q}+\frac{1}{r}=1$ and $\Omega\subset\mathbb{R}^d$.
\begin{enumerate}
\item Show for $u,v,w\colon\Omega\to\mathbb{R}$
\begin{align*}
\left|\int_\Omega u v w \, dx\right|\leq \|u\|_{L^p(\Omega)}\|v\|_{L^q(\Omega)}\|w\|_{L^r(\Omega)}.
\end{align*}

\item Consider the convection term of the Navier-Stokes problem in the weak form
\begin{align*}
t(\bu, \bv, \bw) = \int_\Omega (\bu \cdot \nabla)\bv \cdot \bw \,dx.
\end{align*}
and show
\begin{align*}
|t(\bu, \bv, \bw)| \leq \|\bu\|_{L^p(\Omega)}\|\nabla \bv\|_{L^q(\Omega)} \|\bw\|_{L^r(\Omega)}.
\end{align*}
\end{enumerate}
\end{exercise}

\begin{exercise}{15}
Using the notation of Section 7, the projection step of the pressure-correction method
for time-dependent incompressible flow problems reads:\\
Given $\widetilde{\bu}^{n+1}$, 
find $\bu^{n+1} \in \widetilde{H}$ and $\varphi^{n+1} \in W^{1,2} (\Omega) \cap L^2_∗ (\Omega)$
such that
\begin{align*}
\bu^{n+1} + \nabla\varphi^{n+1} &= \widetilde{\bu}^{n+1}\\
\nabla \cdot \bu^{n+1} &= 0
\end{align*}
where $\widetilde{H} := \{\bv \in [L^2(\Omega)]^d : \nabla \cdot \bv = 0, \bv \cdot \boldsymbol{n}|_{\partial\Omega} = 0\}$.
\begin{enumerate}
\item Show that this projection step corresponds to a Poisson problem for $\varphi^{n+1}$ with appropriate source
term and boundary conditions.
\item Why is this problem uniquely solvable?
\end{enumerate}
\end{exercise}

\begin{exercise}{16}
Consider the variational formulation of the Oseen problems: \\
Find $(\bu, p) \in X \times Q := [H_0^1 (\Omega)]^d \times L^2_∗ (\Omega)$
such that
\begin{align*}
&&a_{Os} (\bu, \bv) + b(p, \bv) &= (\boldsymbol{f} , \bv) &&\forall \bv \in X&&\\
&&b(q, \bu) &= 0 &&\forall q \in Q&&
\end{align*}
with
\begin{align*}
a_{Os}(\bu, \bv) &:= \nu(\nabla \bu, \nabla \bv) + ((\boldsymbol{b} \cdot \nabla)\bu, \bv) + ( (\nabla \cdot \boldsymbol{b})\bu, \bv)\\
b(q, \bv) &:= −(q, \nabla \cdot \bv).
\end{align*}
Let $P : X \to X$ be the mapping which relates the vector field $\boldsymbol{b}$ to the solution $\bu = P (\boldsymbol{b})$
of the Oseen problem. Show that this mapping $P$ is uniformly bounded, i.e. for $\bu = T (\boldsymbol{b})$ there holds independently
of $\boldsymbol{b}$ resp. $\nabla \boldsymbol{b}$ the estimate
\begin{align*}
\|\bu\|_X \leq C(\nu, \Omega, \|f\|_{X^∗}) < \infty.
\end{align*}
Argue why the fixed point $\bu^∗ = P (\bu^∗ )$ of the mapping $P : X \to X$ is the solution of the stationary
incompressible Navier-Stokes problem.
\end{exercise}
\end{document}