% --------------------------------------------------------------
% This is all preamble stuff that you don't have to worry about.
% Head down to where it says "Start here"
% --------------------------------------------------------------
 
\documentclass[12pt]{article}
 
\usepackage[margin=1in]{geometry} 
\usepackage{amsmath,amsthm,amssymb}
\usepackage{graphicx}
 
\newcommand{\N}{\mathbb{N}}
\newcommand{\Z}{\mathbb{Z}}
\newcommand{\bh}{\boldsymbol{h}}
\newcommand{\bx}{\boldsymbol{x}}
\newcommand{\bu}{\boldsymbol{u}}
\newcommand{\bv}{\boldsymbol{v}}
\newcommand{\bomega}{\boldsymbol{\omega}}
 
\newenvironment{theorem}[2][Theorem]{\begin{trivlist}
\item[\hskip \labelsep {\bfseries #1}\hskip \labelsep {\bfseries #2.}]}{\end{trivlist}}
\newenvironment{lemma}[2][Lemma]{\begin{trivlist}
\item[\hskip \labelsep {\bfseries #1}\hskip \labelsep {\bfseries #2.}]}{\end{trivlist}}
\newenvironment{exercise}[2][Exercise]{\begin{trivlist}
\item[\hskip \labelsep {\bfseries #1}\hskip \labelsep {\bfseries #2.}]}{\end{trivlist}}
\newenvironment{problem}[2][Problem]{\begin{trivlist}
\item[\hskip \labelsep {\bfseries #1}\hskip \labelsep {\bfseries #2.}]}{\end{trivlist}}
\newenvironment{question}[2][Question]{\begin{trivlist}
\item[\hskip \labelsep {\bfseries #1}\hskip \labelsep {\bfseries #2.}]}{\end{trivlist}}
\newenvironment{corollary}[2][Corollary]{\begin{trivlist}
\item[\hskip \labelsep {\bfseries #1}\hskip \labelsep {\bfseries #2.}]}{\end{trivlist}}

\setlength\parindent{0pt}
 
\begin{document}
 
\title{Finite Element Methods for Flow Simulations\\ - Exercise Sheet 5 -}
\author{Daniel Arndt}
\date{}
 
\maketitle

\begin{exercise}{9}
Consider the following penalty formulation of the Stokes problem:
\begin{align*}
\hfill&&-\Delta \bu_\epsilon + \nabla p_\epsilon &= \boldsymbol{f} &&\text{ in } \Omega, &&\hfill\\
\hfill&&\nabla \cdot \bu_\epsilon + \epsilon p_\epsilon &= 0       &&\text{ in } \Omega, &&\hfill\\
\hfill&&\bu_\epsilon &= \boldsymbol{0} &&\text{ on } \partial\Omega && \hfill
 \end{align*}
 \begin{enumerate}
\item Write the corresponding variational formulation. How and why can the pressure $p_\epsilon$ be eliminated ?
\item Formulate the FEM problem for the penalty form. Prove an a priori estimate of the form
\begin{align*}
\|\nabla \bu^h_\epsilon\|_{L^2 (\Omega)} + \|p^h_\epsilon\|_{L^2(\Omega)} 
\leq C(\epsilon)(\|\nabla \bu\|_{L^2(\Omega)} +\|p\|_{L^2(\Omega)}).
\end{align*}
\end{enumerate}


\end{exercise}

\begin{exercise}{10}
Prove the following variant of the Gronwall Lemma:\\
Let 
\begin{itemize}
 \item $\beta \in \mathbb{R}$,
 \item $\psi \in C^1([0, T ]; \mathbb{R})$ and 
 \item $f \in C([0, T ]; \mathbb{R})$ 
\end{itemize}
such that 
$\psi' \leq \beta \psi + f$. Then there holds for all $t \in [0, T ]$ that
\begin{align*}
\psi(t) \leq e^{\beta t}\psi(0) + \int_0^t e^{\beta(t-\tau )}f (\tau )\, \mathrm{d}\tau.
\end{align*}
\end{exercise}
\end{document}