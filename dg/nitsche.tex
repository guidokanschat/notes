\section{Nitsche's method}

\begin{intro}
Let us consider the inhomogeneous Dirichlet boundary value problem
\begin{gather}
  \label{eq:nitsche:1}
    \arraycolsep1pt
  \begin{array}{rclrl}
    -\Delta u &=& f
    &\text{ in }&\Omega\\
    u &=& u^D &\text{ on }&\partial\Omega.
  \end{array}
\end{gather}

We already know that for $u^d\equiv 0$ we can discretize this problem
by choosing a finite element space $V_h \subset H^1_0(\Omega)$. For
general $u^D$, there are two options:
\begin{enumerate}
\item Compute an arbitrary ``lifting'' $u^D\in H^1(\Omega)$ such that
  it is equal to $u^D$ on the boundary and compute
  $w=u-u^D \in H^1_0(\Omega)$ as the solution to the weak formulation
  \begin{gather}
    \label{eq:nitsche:1a}
    \int_\Omega \nabla w\cdot\nabla v\dx
    = \int_\Omega f v\dx
    + \int_\Omega \nabla u^D\cdot\nabla v\dx.
  \end{gather}
\item Compute an interpolation or projection $u^D_h$ of the boundary
  data $u^D$. Then, eliminate each row of the matrix corresponding to
  a degree of freedom on the boundary. In particular, let $i$ be the
  index of such a degree of freedom and let $k$ be an index
  corresponding to an interior degree of freedom not constraint by a
  boundary condition, but such that $a_{ik}\neq 0 \neq a_{ki}$. Then,
  replace the rows
  \begin{gather}
    \arraycolsep1pt
    \begin{array}{ccccccccccl}
      \cdots &+& a_{ii} u_i &+& \cdots &+& a_{ik} u_k &+& \cdots &=& f_i \\
      \cdots &+& a_{ki} u_i &+& \cdots &+& a_{kk} u_k &+& \cdots &=& f_k
    \end{array}
  \end{gather}
  by the rows
  \begin{gather}
    \arraycolsep1pt
    \begin{array}{ccccccccccl}
      && u_i &&  && &&  &=& u^D_i \\
      \cdots &+& 0 &+& \cdots &+& a_{kk} u_k &+& \cdots &=& f_k - a_{ki} u^D_i
    \end{array}
  \end{gather}
\end{enumerate}

The first option introduces the complication of finding a function in
$H^1(\Omega)$, which cannot be achieved automatically. The second can
be implemented in an automatic way, but it complicates code, in
particular for nonlinear problems.

A completely different approach modifying the bilinear form was first
suggested in the 60s and then perfected by Joachim Nitsche in 1971. In
this section, we motivate this method and derive its error
estimates. Its key feature is the transition from
$V_h \subset H^1_0(\Omega)$ to $V_h \subset H^1(\Omega)$.
\end{intro}

\begin{intro}
  If we simply derive our weak formulation in $H^1(\Omega)$, we end up
  with an additional boundary term
  \begin{gather}
    \int_\Omega \nabla u \cdot \nabla v \dx
    =
    - \int_\Omega \Delta u v \dx
    + \int_{\partial\Omega} \partial_n u v \ds.
  \end{gather}
  Thus, we obtain the natural boundary condition $\partial_n u=0$,
  which is not consistent with the original BVP. The first step for
  deriving Nitsche's method is subtracting this boundary term on both
  sides. The result is the equation
  \begin{gather}
    \label{eq:nitsche:2}
    \int_\Omega \nabla u \cdot \nabla v \dx
    -\int_{\partial\Omega} \partial_n u v \ds
    = \int_\Omega fv\dx
    \qquad\forall v\in H^1(\Omega).
  \end{gather}
  We observe that the left hand side vanishes for any constant
  function $u$. Thus, we do not have unique solvability and we will
  have to fix this problem. Furthermore, the boundary data $u^D$ does
  not appear in this formulation. We enforce $u=u^D$ in our
  formulation by a so called ``penalty term'' with penalty parameter
  $\alpha$, modifying~\eqref{eq:nitsche:2} to
  \begin{multline}
    \label{eq:nitsche:3}
    \int_\Omega \nabla u \cdot \nabla v \dx
    -\int_{\partial\Omega} \partial_n u v \ds
    + \int_{\partial\Omega} \alpha u v \ds
    \\
    = \int_\Omega fv\dx
    + \int_{\partial\Omega} \alpha u^D v \ds
    \qquad\forall v\in H^1(\Omega).
  \end{multline}
  Integrating by parts, we see that $u$ is a solution to this weak
  formulation. Following Nitsche, we make one additional modification
  which restores the symmetry of our form:
  \begin{multline}
    \label{eq:nitsche:4}
    \int_\Omega \nabla u \cdot \nabla v \dx
    -\int_{\partial\Omega} \partial_n u v \ds
    -\int_{\partial\Omega} \partial_n v u \ds
    + \int_{\partial\Omega} \alpha u v \ds
    \\
    = \int_\Omega fv\dx
    -\int_{\partial\Omega} \partial_n v u^D \ds
    + \int_{\partial\Omega} \alpha u^D v \ds
    \qquad\forall v\in H^1(\Omega).
  \end{multline}
  We abbreviate this equation to
  \begin{gather}
    \label{eq:nitsche:5}
    a_h(u,v) = f_h(v) \qquad\forall v\in H^1(\Omega).
  \end{gather}
\end{intro}

\begin{note}
  Unfortunately, the problem~\eqref{eq:nitsche:5} is not well-posed
  for any finite parameter $\alpha$. Thus, it cannot be used to
  determine $u\in H^1(\Omega)$. Nevertheless, we can establish
  well-posedness on discrete spaces $V_h$ in order to compute a
  discrete solution $u_h$ and use the fact that $u$ is already
  determined by the continuous problem. Our immediate goals are thus:
  \begin{enumerate}
  \item Establish the assumptions of the Lax-Milgram theorem on $V_h$,
    which in this case involves a suitable new norm for measuring the
    error.
  \item Establish a relation between the discrete and continuous
    solution replacing Galerkin orthogonality.
  \item Deriving error estimates in suitable norms.
  \end{enumerate}
\end{note}

\begin{notation}
  From now on, we will use the inner product notation
  \begin{gather*}
    \form(u,v) \equiv \int_\Omega uv\dx
    \qquad
    \form(\nabla u,\nabla v) \equiv \int_\Omega \nabla u\cdot\nabla v\dx,
  \end{gather*}
  on $\Omega$ as well as
  \begin{gather*}
    \forme(u,v) \equiv \int_{\partial\Omega} uv\ds,
  \end{gather*}
  on its boundary.
\end{notation}

\begin{definition}
  A discrete problem~\eqref{eq:nitsche:5} is called \define{consistent},
  if for the solution $u$ of the BVP~\eqref{eq:nitsche:1} there holds
  \begin{gather}
    a_h(u,v_h) = f_h(v_h) \qquad \forall v_h\in V_h.
  \end{gather}
\end{definition}

\begin{corollary}
  Let $u \in H^1(\Omega)$ be the weak solution to~\eqref{eq:nitsche:1}
  in the sense of~\eqref{eq:nitsche:1a}. Then, the discrete
  problem~\eqref{eq:nitsche:5} is consistent.
\end{corollary}

\begin{proof}
  Since $u \in H^1(\Omega)$ and $v_h\in V_h$, all boundary terms in
  $a_h(u,v-h)$ and $f_h(v_h)$ are well-defined and consistency follows
  from $u=u^D$ in the sense of $L^2(\partial\Omega)$.
\end{proof}

\begin{definition}
  The problem dependent norm used for the analysis of Nitsche's method
  is defined by
  \begin{gather}
    \tnorm{v}^2 = \form(\nabla v,\nabla v) + \forme(\alpha v,v).
  \end{gather}
\end{definition}

This choice is justified by
\begin{lemma}
  Let $V_h \subset V = H^1(\Omega)$ be a piecewise polynomial finite
  element space on a shape-regular mesh $\T_h$. Then, if $\alpha$
  sufficiently large, there exist constants $M$ and $\gamma$ such that
  \begin{align*}
    a_h(u_h, v_h) &\le M \tnorm{u_h}\tnorm{v_h}\\
    a_h(u_h,u_h) &\ge \gamma \tnorm{u_h}^2.
  \end{align*}
\end{lemma}

\begin{proof}
  Key for the proof is the inverse trace estimate
  \begin{gather*}
    |v|_{H^1(\partial T)} \le c h^{-1/2} |v|_{H^1(T)},
  \end{gather*}
  which holds with a constant $c$ depending on shape regularity and
  polynomial degree. Thus, for a cell $T$ at the boundary, there holds
  \begin{align*}
    |\forme(\partial_n u_h, v_h)_{\partial T \cap \partial\Omega}|
    & \le c h^{-1/2} |u_h|_{H^1(T)} \norm{v_h}_{L^2(\partial T \cap \partial\Omega)}
    \\
    \le \tfrac{1}{2} |u_h|_{H^1(T)}^2
    + \tfrac{c^2}{2h_T}\norm{v_h}_{L^2(\partial T \cap \partial\Omega)},
  \end{align*}
  We apply this to the lower bound to obtain
  \begin{gather*}
    a_h(u_h, u_h) \ge \left(1-\tfrac{1}{2}\right)
    |u_h|_{H^1(\Omega)}^2
    + \left(\alpha - \tfrac{c^2}{2h_T}\right)
    \norm{u_h}_{L^2(\partial\Omega)}^2.
  \end{gather*}
  Choosing $\alpha(x) = c^2/h(x)$, where $h(x)$ is the size of the
  cell such that $x\in\partial T$, we obtain
  \begin{gather*}
    a_h(u_h, u_h) \ge \frac12 \tnorm{u_h}^2.
  \end{gather*}
\end{proof}

%%% Local Variables: 
%%% mode: latex
%%% TeX-master: "main"
%%% End: 
