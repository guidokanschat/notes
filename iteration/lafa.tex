
\begin{intro}
  Linear algebra deals with abstract vector spaces, but most results
  on linear mappings are restricted to finite dimensional spaces,
  since they exploit the fact that we can choose a basis.

  The choice of a basis becomes a more involved endeavor if we allow
  for spaces that do not have a finite basis. We can actually go by
  two very different routes. The route of Hamel bases, which are bases
  of inifintely many vectors, but in order to represent a vector in
  such a basis, we only allow for finite linear combinations.
  
  The other route defines a Schauder basis as a set of vectors, such
  that every vector in the space is the linear combination of
  infinitely many basis vectors. In order to define such a linear
  combination, we have to define the meaning of such an infinite sum,
  namely the convergence of the sum. In the course of such a
  definition, we will learn about a natural extension of
  Euclidean\footnote{And we will not have to distinguish between
    Euclidean real spaces and unitary complex spaces anymore.}
  spaces, namely pre-Hilbert and Hilbert spaces.
\end{intro}

\begin{definition}
  Let $V$ be a vector space over $\mathbb K$ with $\mathbb K = \mathbb
  C$ or $\mathbb K = \R$. An \define{inner product} on $V$ is a mapping
  $\scal(.,.): V\times V \to \mathbb K$ with the properties
  \begin{xalignat}2
    \scal(\alpha x+y,z) &= \alpha \scal(x,z) + \scal(y,z)
    && \forall x,y,z \in V; \alpha \in \mathbb K\\
    \scal(x,y) &= \overline{\scal(y,x)} && \forall x,y \in V \\
    \scal(x,x) & \ge 0 \quad\forall x\in V &\text{and}
    & \scal(x,x)=0 \Leftrightarrow x=0,
  \end{xalignat}
  usually referred to as (bi-)linearity, symmetry, and
  definiteness. We note that linearity in the second argument follows
  immediately by symmetry.
\end{definition}

\begin{definition}
  A vector space $V$ equipped with an inner product $\scal(.,.)$ and a
  norm defined by
  \begin{gather*}
    \|v\| = \sqrt{\scal(v,v)}
  \end{gather*}
  is called a \define{pre-Hilbert space}. A \define{Hilbert space} is
  a pre-Hilbert space which is also \putindex{complete}, that is,
  every \putindex{Cauchy sequence} with elements in the space has a
  limit in the space.
\end{definition}

\begin{example}
  For any positive integer, the space $\R^n$ equipped with the
  Euclidean inner product
  \begin{gather*}
    \scal(x,y) = \sum_{i=1}^n x_i y_i
  \end{gather*}
  is a Hilbert space. The same holds for $\mathbb C^n$ and
  \begin{gather*}
    \scal(x,y) = \sum_{i=1}^n x_i \overline{y_i}.
  \end{gather*}
\end{example}

\begin{example}
  The space $\ell^2$ of sequences $\{x_i\}_{i=1,\dots}$ of complex
  numbers is a Hilbert space if equipped with the inner product
  \begin{gather*}
    \scal(x,y) = \sum_{i=1}^\infty x_i \overline{y_i}
    = \lim_{n\to\infty}\sum_{i=1}^n x_i \overline{y_i}.
  \end{gather*}
\end{example}

\begin{example}
  On the space of continuous functions on the interval
  $[-\pi/2,\pi/2]$ define the inner product
  \begin{gather*}
    \scal(f,g) = \int_{-\pi/2}^{\pi/2} f(x)g(x)\,dx.
  \end{gather*}
  Let
  \begin{gather*}
    V = \bigl\{f\in C[-\pi/2,\pi/2] \big| \scal(f,f) < \infty \bigr\}.
  \end{gather*}
  Then $V$ is a vector space with an inner product and thus a
  pre-Hilbert space, but it is not a Hilbert space, since for any $n$
  the sum
  \begin{gather*}
    f_n(x) = \frac4\pi \sum_{k=1}^n \frac{\sin\bigl((2k-1) x\bigr)}{2k-1}
  \end{gather*}
  is continuous, but
  \begin{gather*}
    \lim_{n\to\infty} f_n =
    \begin{cases}
      -1 & x<0 \\
      0 & x=0 \\
      1 & x>0
    \end{cases}
  \end{gather*}
  is not.
\end{example}

%%% Local Variables: 
%%% mode: latex
%%% TeX-master: "main"
%%% End: 
