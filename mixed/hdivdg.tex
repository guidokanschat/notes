\begin{intro}
  In the previous chapter, we studied discretizations with
  $\div V_h = Q_h$ with two advantages. First, due to
  \blockref{Corollary}{galerkin-mixed-u-kerb} the velocity error is
  independent of the pressure. Second, the divergence converges faster
  than the gradient. A natural question arising is whether we can do
  something similar for the Stokes problem. There, the equation
  \begin{gather*}
    \form(\div v_h, q_h) = 0 \qquad\forall q_h\in Q_h,
  \end{gather*}
  would immediately imply $\div v_h=0$, that is, the discrete solution
  is exactly divergence free.
  
  The answer to this question is a current research topic. So far,
  beginning with the element by Scott and Vogelius, several methods
  have been proposed for special mesh geometry or macro meshes. The
  difficulty is balancing the condition $\div V_h = Q_h$ with the
  $H^1$-conformity of the velocity space. All the spaces in the
  previous chapter were only $\Hdiv$-conforming with discontinuous
  tangential components.
  
  A fairly simple solution to this question though can be obtained by
  using discontinuous Galerkin methods. These were introduced to
  obtain formulations \emph{consistent} with $H^1$ while not
  \emph{conforming}. Thus, we can apply them directly to
  Raviart-Thomas oder Brezzi-Douglas-Marini elements to obtain a
  consistent method with divergence free solutions.

  We begin this chapter by a quick review of the interior penalty
  method before diving into divergence conforming methods.
\end{intro}


% \begin{intro}
%   In this section we extend the weakening of continuity, which we
%   explored for boundary values in Section~\ref{sec:nitsches-method}
%   using Nitsche's method to interior interfaces between mesh
%   cells. While the methods obtained may look much more complicated,
%   the mathematical analysis is completely analogue to that
%   section. Thus, we can be fairly brief.
% \end{intro}

\begin{intro}
  We review the basic definitions necessary to describe discontinuous
  Galerkin (DG) methods. In particular, we need the sets of faces
  $\F_h$ of a mesh, discontinuous piecewise polynomial spaces and
  broken integrals.
\end{intro}

\begin{Definition}{dg-faces}
  Let $\T_h$ be a mesh of $\Omega \subset \R^d$ consisting of mesh
  cells $T_i$. For every boundary facet $F\subset \partial T_i$, we
  assume\footnote{This assumption can indeed be relaxed} that either
  $F \subset \partial \Omega$ or $F$ is a boundary facet of another
  cell $T_j$. In the second case, we indicate this relation by
  labeling this facet $F_{ij}$. The set of all facets $F_{ij}$ is the
  set of interior faces $\F_h^i$. The set of facets on the boundary is
  $\F_h^\partial$.
\end{Definition}

\begin{Definition}{dg-spaces}
  The discontinuous finite element space on $\T_h$ is constructed by
  concatenation of all shape function spaces $P_T$ for $T\in \T_h$
  without additional continuity requirements:
  \begin{gather}
    V_h = \bigl\{v\in L^2(\Omega) \big|
    v_{|T} \in P_T \;\forall T\in \T_h\bigr\}.
  \end{gather}
\end{Definition}

\begin{Definition}{broken-integrals}
  For any set of cells $\mesh_h$ or faces $\faces_h$, we define the bilinear
  forms
  \begin{align}
    \form(u,v)_{\mesh_h} &= \sum_{\cell\in\mesh_h} \form(u,v)_\cell, \\
    \forme(u,v)_{\faces_h} &= \sum_{\face\in\faces_h} \forme(u,v)_\face. \\
  \end{align}
\end{Definition}

\begin{intro}
  We start out with the equation
  \begin{gather*}
    -\Delta u = f.
  \end{gather*}
  Integrating by parts on each mesh cell yields
  \begin{gather*}
    \form(-\Delta u,v )_\cell
    = \form(\nabla u, \nabla v)_\cell - \forme(\d_n u, v)_{\d\cell} = \form(f,v)_T.
  \end{gather*}
  We realize that the choice of discontinuous finite element spaces
  introduces a consistency term on the interfaces between cells and on
  the boundary.

  On interior faces, there is the issue that $u$ and
  $\d_n u$ actually have two values on the interface, one from the
  left cell and one from the right. Therefore, we have to consolidate
  these two values into one. To this end, we introduce the concept of
  a numerical flux, which constructs a single value out of these
  two. Thus, we introduce on the interface $\face$ between two cells
  $\cell^+$ and $\cell^-$
  \begin{gather*}
    \mathcal F(\nabla u) = \frac{\nabla u^+ + \nabla u^-}{2} = :
    \mvl{\nabla u}.
  \end{gather*}

  Using $\forme(\d_n u,v) = \forme(\nabla u,v\n)$ we change our point
  of view and instead of integrating over the boundary $\d\cell$, we
  integrate over a face $\face$ between two cells $\cell^+$ and
  $\cell^-$. Adding up integrals from both sides, we obtain the term
  \begin{gather*}
    -\forme(\mvl{\nabla u},v^+\n^+ +v^-\n^-)_{\face}
    = -2\forme(\mvl{\nabla u},\mvl{v\n})_{\face}.
  \end{gather*}
  On boundary faces, we simply get
  \begin{gather*}
    \forme(\d_\n u,v)_{\face}.
  \end{gather*}

  Adding over all cells and faces, we obtain the equation
  \begin{gather*}
    \form(\nabla u,\nabla v)_{\T_h}
    -2\forme(\mvl{\nabla u},\mvl{v\n})_{\F_h^i}
    -\forme(\d_\n u,v)_{\F_h^\d} = \form(f,v)_{\domain}.
  \end{gather*}

  Following the idea of Nitsche, we symmetrize this term
  to obtain
  \begin{multline*}
    \form(\nabla u,\nabla v)_{\T_h}
    -2\forme(\mvl{\nabla u},\mvl{v\n})_{\F_h^i}
    -2\forme(\mvl{u\n},\mvl{\nabla v})_{\F_h^i}
    \\
    -\forme(\d_\n u,v)_{\F_h^\d}
    -\forme(u,\d_\n v)_{\F_h^\d}
    = \form(f,v)_{\domain}
    - \forme(u^o,\d_n v)_{\F_h^\d}.
  \end{multline*}
  Here the second term on the right was introduced for consistency.
  Finally, it turns out that this method is not stable and needs
  stabilization by a jump term. This will be done in
  \blockref{Definition}{ip}. Before, we introduce the notation for
  averaging and jump operators.
\end{intro}

\begin{Notation}{dg-operators}
  Let $\face$ be a face between the cells $\cell^+$ and $\cell^-$. Let
  $\n^+$ and $\n^-=-\n^+$ be the outer normal vectors of the cells at a
  point $x\in \face$. For a function $u\in V_h$, the traces $u^+$ and
  $u_-$ of $u$ on $\face$ taken from the cell $\cell^+$
  and $\cell^-$ are defined as:
  \begin{align*}
    u^+(x) &= \lim_{\epsilon\searrow 0} u(x-\epsilon\n^+), \\
    u^-(x) &= \lim_{\epsilon\searrow 0} u(x-\epsilon\n^-).
  \end{align*}
  We define the \define{averaging operator} $\mvl{.}$ and the
  \define{jump operator} $\jmp{.}$ as
  \begin{gather}
    \label{eq:ip:1}
    \mvl{u} = \frac{u^++u^-}{2},
    \qquad
    \jmp{u} = u^+-u^-.
  \end{gather}
  Not that the sign of the jump of $u$ depends on the choice of the
  cells $\cell^+$ and $\cell^-$. It will only be used in quadratic
  terms.
\end{Notation}

\begin{remark}
  The jump can be denoted as the mean value of the product of a
  function and the normal vector,
  \begin{gather}
    \jmp{u} = 2\mvl{u\n}\cdot\n^+ = -2\mvl{u\n}\cdot\n^-.
  \end{gather}
\end{remark}

\begin{Definition}{ip}
  The \define{interior penalty method}\footnote{Also known as
    symmetric interior penalty (SIPG) or IP-DG.} uses the bilinear
  form
  \begin{multline}
    \label{eq:ip:2}
    a_h(u,v) = \form(\nabla u,\nabla v)_{\mesh_h}
    + \forme(\ipp_h\jmp{u},\jmp{v})_{\faces_h^i}
    + \forme(\ipp_h u,v)_{\faces_h^\d}
    \\
    -2\forme(\mvl{\nabla u},\mvl{v\n})_{\faces_h^i}
    -2\forme(\mvl{u\n},\mvl{\nabla v})_{\faces_h^i}
    \\
    - \forme(\d_n u,v)_{\faces_h^\d}
    - \forme(u,\d_n v)_{\faces_h^\d},
  \end{multline}
  and the linear form
  \begin{gather}
    \label{eq:ip:3}
    f_h(v) = \form(f,v)_{\domain} - \forme(u^D,\d_n v)_{\faces_h^\d}
    + \forme(\ipp_h u,v)_{\faces_h^\d},
  \end{gather}
  where $f$ is the right hand side of the equation and $u^D$ the
  Dirichlet boundary value.
\end{Definition}

\begin{Definition}{ip-norm}
  On the space $V_h$ we define the norm $\norm{.}_{1,h}$ by
  \begin{gather}
    \label{eq:ip:4}
    \norm{v}_{1,h}^2 = \sum_{\cell\in\mesh_h} \norm{\nabla v}_\cell^2
    + \sum_{\face\in\faces_h^i} \norm{\sqrt{\ipp_h}\jmp{v}}_\face^2
    + \sum_{\face\in\faces_h^\d} \norm{\sqrt{\ipp_h}v}_\face^2.
  \end{gather}
\end{Definition}

\begin{Problem}{ip-norm}
  Prove that the norm defined in (\ref{eq:ip:4}) is indeed a norm on $V_h$.
\begin{solution}
\begin{align*}
  0=\norm{v}_{1,h}^2 &= \sum_{\cell\in\mesh_h} \norm{\nabla v}_\cell^2
    + \sum_{\face\in\faces_h^i} \norm{\sigma_h\jmp{v}}_\face^2
    + \sum_{\face\in\faces_h^\d} \norm{\sigma_hv}_\face^2
\end{align*}
implies first of all $v|_T \equiv const.$ for all $T \in \mathbb{T}_h$.
Furthermore, $\norm{\sigma_h\jmp{v}}_\face^2=0$ implies $v \equiv const.$
and the last condition gives $v \equiv 0$.

 We use the trace inequality
 \begin{align*}
  \norm{v}_{0,\partial T}^2
  \lesssim \left( h_T^{-1} \norm{v}_{0,T}^2+\norm{v}_{0,T}\norm{\nabla v}_{0,T}\right)
  \quad \forall v \in H^1(T).
 \end{align*}
 Then, we can estimate
 \begin{align*}
  &\sum_{\face\in\faces_h^i} \norm{\sigma_h\jmp{v}}_\face^2 + \sum_{\face\in\faces_h^\d} \norm{\sigma_hv}_\face^2\\
  &\lesssim \min_{F\in F_h^i\cup F_h^\partial} \sigma_{h,F} \sum_{T \in \mathbb{T}_h}\norm{v}_{0,\partial T}^2 \\
  &\lesssim \min_{F\in F_h^i\cup F_h^\partial} \sigma_{h,F} \sum_{T \in \mathbb{T}_h}\left( h_T^{-1}
  \norm{v}_{0,T}^2+\norm{v}_{0,T}\norm{\nabla v}_{0,T}\right)\\
  &\lesssim \min_{F\in F_h^i\cup F_h^\partial} \sigma_{h,F} \sum_{T \in \mathbb{T}_h}\left( h_T^{-1}
  \norm{v}_{0,T}^2+h_T\norm{\nabla v}_{0,T}\right)
 \end{align*}
 and thus
 \begin{align*}
    \norm{v}_{1,h}^2 &= \sum_{\cell\in\mesh_h} \norm{\nabla v}_\cell^2
    + \sum_{\face\in\faces_h^i} \norm{\sigma_h\jmp{v}}_\face^2
    + \sum_{\face\in\faces_h^\d} \norm{\sigma_hv}_\face^2\\
    &\lesssim \min_{F\in F_h^i\cup F_h^\partial} \sigma_{h,F}
      \sum_{T \in \mathbb{T}_h}\left( h_T^{-1} \norm{v}_{0,T}^2+\norm{\nabla v}_{0,T}\right)\\
  &\lesssim \norm{v}_1^2
  \end{align*}
\end{solution}
\end{Problem}

\begin{Lemma}{ip-stability}
  Let $\T_h$ be shape-regular and chosen on each face $\face$ as
  $\sigma_h = \sigma_0/h_\face$, where $h_T$ is the minimal diameter
  of a cell adjacent to $\face$. Then, there is a $\sigma_0>0$ such
  that there exists a constant $\ellipa>0$, such that independent of
  $h$ there holds
  \begin{gather}
    \label{eq:ip:5}
    a_h(u_h,u_h) \ge \ellipa \norm{u_h}_{1,h}^2 \quad \forall u_h\in V_h.
  \end{gather}
\end{Lemma}

\begin{Problem}{ip-stability}
  Prove \blockref{Lemma}{ip-stability}.
\begin{solution}
We first note the estimate
\begin{align*}
 (\boldsymbol{n}\cdot\nabla v_h)_e^2
 &\leq C (h_K^{-1}\norm{\nabla v_h}_{0,K}^2+\norm{\nabla v_h}_{0,K}\norm{\nabla^2 v_h}_{0,K}) \\
 &\leq C \left(\frac{1}{h_K}+\frac{p_K^2}{h_K}\right) \norm{\nabla v_h}_{0,K}^2\\
 &\leq C \frac{p_K^2}{h_K} \norm{\nabla v_h}_{0,K}^2
 = C\frac{\sigma_h}{\delta} \norm{\nabla v_h}_{0,K}^2
\end{align*}

Testing the bilinear form symmetrically, we obtain
\begin{align*}
 a_h(u_h,u_h) &= \norm{\nabla u_h}_0^2
    + \ipp_h\norm{\jmp{u_h}}_{\faces_h^i}
    + \ipp_h\norm{u_h}_{\faces_h^\d}
    \\
    &-4\forme(\mvl{\nabla u_h},\mvl{u_h\n})_{\faces_h^i}
    -2 \forme(\d_n u_h,u_h)_{\faces_h^\d}.
\end{align*}
and the last two terms can be estimated by
\begin{align*}
 \forme(\mvl{\nabla u_h},\mvl{u_h\n})_{\faces_h^i} &=\forme(n^+\cdot\nabla u_h^+-n^-\cdot\nabla u_h^-, u_h^+-u_h^-)\\
 &\leq \frac{\epsilon}{2}\norm{n^+\{\{\nabla u_h\}\}}_0^2+\frac{1}{2\epsilon}\norm{[[u_h]]}_0^2\\
 &\leq C\frac{\sigma_h\epsilon}{2\delta}\norm{\nabla u_h}_{0,K}^2+\frac{1}{2\epsilon}\norm{[[u_h]]}_0^2
\end{align*}
and therefore
\begin{align*}
 a_h(u_h,u_h) -\gamma \norm{u_h}_{1,h}^2&\geq \norm{\nabla u_h}_0^2 \left(1-\gamma-C\frac{\sigma_h\epsilon}{2\delta}\right)\\&\quad
    + \ipp_h\norm{\jmp{u_h}}_{\faces_h^i}\left(1-\gamma-\frac{1}{2\epsilon\sigma_h}\right).
\end{align*}

Hence, we have to choose $\epsilon, \delta>0$ such that
\begin{align*}
 1-\gamma-C\frac{\sigma_h\epsilon}{2\delta}>0 \\
 1-\gamma-\frac{1}{2\epsilon\sigma_h}>0 .
\end{align*}
This is possible for all $\gamma\in(0,1)$ and in fact we get the lower limits
\begin{align*}
\epsilon&>\frac{1}{2(1-\gamma)\sigma_h}\\
 \delta&>C \frac{\sigma_h \epsilon}{2(1-\gamma)}.
\end{align*}


\end{solution}

\end{Problem}

\begin{Lemma}{ip-consistence}
  Let $f\in L^2(\domain)$ and let the boundary conditions admit that
  for the solution to
  \begin{xalignat*}2
    -\Delta u &= f &\text{in }&\domain, \\
    u &= u^D &\text{on }&\d\domain,
  \end{xalignat*}
  there holds $u\in H^{1+\epsilon}(\domain)$ for a positive
  $\epsilon$. Then, the interior penalty method is consistent, that
  is,
  \begin{gather}
    a_h(u,v_h) = f_h(v_h)\quad\forall v_h\in V_h.
  \end{gather}
\end{Lemma}

\begin{proof}
  From $f\in L^2(\domain)$ we deduce that
  $\nabla u\in \Hdiv(\domain)$. Thus, with the extra regularity, the
  traces of $\d_n u$ on faces are well-defined and coincide from both
  sides. The remainder is integration by parts.
\end{proof}

\begin{Theorem}{ip-convergence}
  For $k\ge 1$ let $\P_k\subset P_\cell$ and $u\in H^{s+1}(\domain)$ with
  $1/2 \le s \le k$. Then, the interior penalty method admits the
  error estimate
  \begin{gather}
    \norm{u-u_h}_{1,h} \le c h^s \snorm{u}_{s+1}.
  \end{gather}
  If furthermore the boundary condition admits \putindex{elliptic
    regularity},
there holds
  \begin{gather}
    \norm{u-u_h}_{0} \le c h^{s+1} \snorm{u}_{s+1}.
  \end{gather}
\end{Theorem}

%%% Local Variables:
%%% mode: latex
%%% TeX-master: "main"
%%% End:






\begin{remark}
  The extension of the interior penalty method to vector-valued
  problems is obvious. Furthermore, since the method generates an
  elliptic bilinear form on the discontinuous space $V_h$, this
  ellipticity is inherited by any subspace of
  $V_h\cap\Hdiv(\domain)$. Thus, we can write down the weak
  formulation of a divergence conforming DG method for the Stokes
  equations. In the following definition, we assume slip or no-slip
  boundary conditions, that is, $v\cdot\n=0$ on the whole boundary.
\end{remark}

\begin{Definition}{hdiv-ip}
  A divergence conforming DG method for the Stokes equations consists
  of a discrete velocity space $V_h\subset \Hdiv_0(\domain)$ and a
  pressure space $Q_h\subset L^2_0(\domain)$ such that
  \begin{gather}
    \label{eq:hdivdg:1}
    \div V_h = Q_h.
  \end{gather}
  Using the interior penalty bilinear form $a_h(.,.)$, we search for
  solutions $(u_h,p_h)\in V_h\times Q_h$ such that for all $(v,q)\in
  V_h\times Q_h$ there holds
  \begin{gather}
    \label{eq:hdivdg:2}
    a_h(u_h,v) +\form(\div v,p_h)+\form(\div u_h,q) = f(v).
  \end{gather}
\end{Definition}

\begin{Lemma}{dg-fortin}
  The \putindex{canonical interpolation} operators of the
  Brezzi-Douglas-Marini and Raviart-Thomas elements admit the bound
  \begin{gather}
    \label{eq:hdivdg:4}
    \norm{v-I_h v}_{1,h} \le c \norm{v}_1
  \end{gather}
\end{Lemma}

\begin{Lemma}{hdivdg-infsup}
  The method in \blockref{Definition}{hdiv-ip}
  admits the inf-sup condition
  \begin{gather}
    \label{eq:hdivdg:3}
    \inf_{q_h\in Q_h} \sup_{v_h\in V_h}
    \frac{\form(\div v_h,q_h)}{\norm{v_h}_{1,h}\norm{q_h}_0} \ge \beta,
  \end{gather}
  with a constant $\beta >0$ independent of $h$.
\end{Lemma}

% \begin{proof}
%   First, we make use of the fact that $q_h\in Q_h \subset Q$ to deduce
%   that 
% \end{proof}

%%% Local Variables: 
%%% mode: latex
%%% TeX-master: "main"
%%% End: 
