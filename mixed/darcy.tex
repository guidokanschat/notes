
\section{Modelling diffusion problems}

\begin{intro}
  Diffusion problems arise when a balance law, for instance for mass
  in ground water flow or for energy in temperature conduction is
  coupled with a constitutive equation relating the direction of
  movement to the gradient of the quantity of interest.
\end{intro}

\begin{intro}
  Let $\rho$ be the density of a conserved quantity. Then, for any given
  volume we have the ``mass''
  \begin{gather*}
    m = \int_V \rho\dx.
  \end{gather*}
  Changes of this mass can be due to two processes:
  \begin{enumerate}
  \item Generation of additional mass by a source $g$,
  \item Flow of mass over the boundary of $V$ at a velocity $v$.
  \end{enumerate}
  In formulas, we have
  \begin{gather*}
    \tfrac{d}{dt} m = \int_V g\dx - \oint_{\d V} J\cdot \n \ds,
  \end{gather*}
  also known as \define{Reynolds transport theorem}. Here, $J$ is the
  \define{flux}. The exact form of the flux will be modelled later.
  The formula above is somewhat unwieldy, since it combines volume and
  surface integrals. Therefore, we apply the \putindex{Gauss theorem}
  to obtain
  \begin{gather}
    \label{eq:darcy:2}
    \frac{d}{dt} \int_V \rho\dx = \int_V g\dx - \int_V \div J \dx.
  \end{gather}
  Concentrating and assuming sufficient regularity, we arrive at the
  equation
  \begin{gather}
    \label{eq:darcy:3}
    \d_t \rho + \div J = g.
  \end{gather}
  As before in these notes, we ignore the time dependence and only
  look at stationary limits. In this case, this reduces to
  \begin{gather}
    \label{eq:darcy:4}
    \div J = g.
  \end{gather}
\end{intro}

\begin{example}
  Next we consider constitutive relations between $\rho$ and $J$ such
  that we can complement equation~\eqref{eq:darcy:4} by a second
  equation and obtain a solvable system. To this end, we consider
  thermal diffusion and ground water flow.

  \begin{description}
  \item[Heat conduction:] Here, the conserved quantity is not the
    density $\rho$, but the temperature $T$. \define{Fourier's law}
    states that the flux is proportional to the gradient of the
    temperature, pointing in opposite direction:
    \begin{gather*}
      J = -k\nabla T.
    \end{gather*}
    The constant of proportionality $k$ is the heat conductivity.
  \item[Porous media flow:] The conserved quantity is the amount of
    fluid, represented by the hydraulic head or pressure
    $p$. \define{Darcy's law} says that the flux is the product of the
    hydraulic \define{permeability} of the media and the gradient of the
    pressure:
    \begin{gather*}
      J = -K\nabla p.
    \end{gather*}
    Here, the permeability $K$ is either a positive scalar function or
    a symmetric, positive definite matrix. Note that in the latter
    case, $J$ and $\nabla p$ do not point in the same direction.
    \item[General diffusion processes:] \define{Fick's law} states,
      that the flux of a diffusion process is determined by the
      gradient of the diffusing quantity $p$ by the relation
    \begin{gather*}
      J = -D\nabla p.
    \end{gather*}
    $D$ is the symmetric, positive definite \define{diffusion tensor}.
  \end{description}
\end{example}

\begin{intro}
  From the two equations for $J$, we derive the following system of
  PDE, where we replace the letter $J$ by the more familiar $u$:
  \begin{gather}
    \label{eq:darcy:5}
    \arraycolsep2pt
    \begin{matrix}
      K^{-1} u &+& \nabla p &=& 0 \\
      \div u &&&=& f.
    \end{matrix}
  \end{gather}
  This system is closed by boundary conditions. Let $\Gamma_D$ be the
  Dirichlet boundary and $\Gamma_N$ be the Neumann boundary such that
  $\Gamma_D \cap \Gamma_N = \emptyset$ and
  $\Gamma_D\cup\Gamma_N = \d\domain$. Then, we let
  \begin{gather}
    \label{eq:darcy:6}
    \begin{aligned}
      p(x) &= p^D(x) & x & \in \Gamma_D, \\
      u(x)\cdot\n &= u^N(x)\cdot n & x & \in \Gamma_N.
    \end{aligned}
  \end{gather}

  Following the concept of finding spaces such that we have an inf-sup
  condition, we are looking for a pair with minimal regularity, such
  that we have a stable and bounded inf-sup condition. We begin the
  usual way by multiplying with a test function and integrating:
  \begin{gather}
    \label{eq:darcy:7}
    \arraycolsep2pt
    \begin{matrix}
      \displaystyle\int_\domain K^{-1} u\cdot v\dx
      &+&
      \displaystyle\int_\domain \nabla p \cdot v\dx
      &=& 0 \\
      \displaystyle\int_\domain \div u q\dx
      &&&=&
      \displaystyle\int_\domain fq\dx.
    \end{matrix}
  \end{gather}
  It turns out, we have two immediate options: first, we can integrate
  the first equation by parts, having all derivatives on $u$. On the
  other hand, we can integrate by parts in the second equation,
  leaving all derivatives on $p$ on $p$ and $q$. In the second case,
  we obtain the equation
  \begin{gather*}
    -\int_\domain u\cdot\nabla q\dx + \int_{\d\domain} u\cdot\n q \ds
    = \int_\domain fq\dx.
  \end{gather*}
  Applying the boundary condition, we first follow the recipe of
  elliptic partial differential equations and implement $p=p^D$ as an
  \putindex{essential boundary condition}, that is, the test function
  space has zero trace on $\Gamma_D$. Then, we can swap in $u^N$ for
  $u$ on $\Gamma_N$, such that the boundary term ends up on the right
  hand side.
\end{intro}

\begin{Definition}{primal-mixed}
  The \define{primal mixed formulation} of the mixed diffusion
  problem~\eqref{eq:darcy:5} reads: find $(u,p)\in V\times Q$ such
  that for all $v\in V$ and $q\in Q$ holds
  \begin{gather}
    \label{eq:darcy:8}
    \arraycolsep2pt
    \begin{array}{rcccl}
      \form(K^{-1} u, v) &+& \form( \nabla p, v)
      &=& 0 \\
      -\form(u,\nabla q)
      &&&=& \form(f,q) - \forme(u^N\cdot\n, q)_{\Gamma_N}.
    \end{array}
  \end{gather}
  The spaces are
  \begin{gather}
    \label{eq:darcy:9}
    \begin{split}
      V_h &= L^2(\domain;\R^d), \\
      Q_h &= H^1_{\Gamma_D}(\domain) = \bigl\{
      q\in H^1(\domain) \big| \;q_{|\Gamma_D} = 0
      \bigr\}.
    \end{split}
  \end{gather}
\end{Definition}

\begin{remark}
  Since the first equation is tested with the test function $v$ itself
  in all terms, we can eliminate this equation and there holds
  $u= K\nabla P$ in $L^2(\domain;\R^d)$. Entering this into the second
  equation, we obtain the well-known \putindex{primal formulation}
  \begin{gather*}
    \form(K \nabla p,\nabla q) = \form(f,q)
    - \forme(u^N\cdot\n, q)_{\Gamma_N}.
  \end{gather*}
  Just keep in mind that the ``\putindex{natural boundary condition}''
  in this case is
  \begin{gather*}
    K\nabla p\cdot n = 0.
  \end{gather*}
  Hence, the primal mixed formulation does not provide any advantages
  compared to the primal formulation, and we are not going to pursue
  it further.
\end{remark}

\begin{intro}
  Now we return to the first alternative, namely integrating by parts
  in the first equation of~\eqref{eq:darcy:7}:
  \begin{gather*}
    \int_\domain K^{-1} u \cdot v \dx - \int_\domain p \div v\dx 
    + \int_{\d\domain} v\cdot \n p\ds = 0.
  \end{gather*}
  Ensuing is a formulation multiplying and integrating the divergences
  of $u$ and $v$, respectively, with functions in $Q$. In order to fit
  this into our standard framework, we have to introduce a new Sobolev
  space. In addition, since $u\cdot\n$ does not appear as a boundary
  integral, we must make this an \putindex{essential boundary
    condition}. Thus, we require that the test functions have zero
  normal trace on $\Gamma_N$ (and justify this below). Note that now
  the Dirichlet condition $p=0$ has become a ``\putindex{natural
    boundary condition}''!
\end{intro}

\begin{Definition}{hdiv}
  Let $\domain \subset \R^d$ be a domain.  We define the
  Sobolev space
  \begin{gather}
    \Hdiv(\domain) = \bigl\{
    v\in L^2(\domain;\R^d) \big\vert
    \div v\in L^2(\domain)\bigr\},
  \end{gather}
  and its inner product
  \begin{gather}
    \scal(u,v)_{\Hdiv} = \form(u,v)_0 + \form(\div u,\div v)_0.
  \end{gather}
  Furthermore, let $C^\infty_{00}(\domain)$ be the space of smooth
  functions with compact support in $\domain$. Then, we define its
  closure in $\Hdiv(\domain)$:
  \begin{gather}
    \Hdiv_0(\domain) = \overline{C^\infty_{00}(\domain)}.
  \end{gather}
  For subset $\Gamma\subset\d\domain$, the space
  $\Hdiv_\Gamma(\domain)$ is defined accordingly (compare to
  $H^1_\Gamma(\domain)$)
\end{Definition}

Using the space $\Hdiv$ and for the moment the assumption, that
$\Hdiv_0$ and $\Hdiv_\Gamma$ serve to set boundary conditions, we can
write down our second weak fromulation of the mixed diffusion problem:

\begin{Definition}{dual-mixed}
  The \define{dual mixed formulation} of the mixed diffusion
  problem~\eqref{eq:darcy:5} reads: find $(u,p) \in V\times Q$ such
  that for all $v\in V$ and $q\in Q$ holds
  \begin{gather}
    \label{eq:darcy:10}
    \arraycolsep2pt
    \begin{array}{rcccl}
      \form(K^{-1} u, v) &-& \form(p, \div v)
      &=& \forme(p^D,v\cdot \n)_{\Gamma_D} \\
      \form(\div u, q)
      &&&=& \form(f,q).
    \end{array}
  \end{gather}
  The spaces are
  \begin{gather}
    \label{eq:darcy:11}
    V_h = \Hdiv_{\Gamma_N}(\domain),
    \qquad
    Q_h = L^2(\domain).
  \end{gather}  
\end{Definition}

\begin{Problem}{mixed-inhomogeneous-bc}
  In both the primal and the dual mixed formulation, we ignored
  inhomogeneous essential boundary conditions. Show that the usual
  lifting method applies. Determine the modified equations and the
  spaces needed for the liftings.
\end{Problem}

\subsection{Properties of $\Hdiv(\domain)$}

\begin{Theorem}{Hdiv-separable}
  Let $\domain$ be a bounded Lipschitz domain. Then, the space
  $C^\infty(\overline\domain)$ is dense in $\Hdiv(\domain)$.
\end{Theorem}

\begin{proof}
  
\end{proof}

\begin{remark}
  The condition of boundedness entered the assumptions since we use
  the space $C^\infty(\overline\domain)$. It could be dropped, if we
  used a more appropriate space (cf.~\cite[Theorem
  2.4]{GiraultRaviart86}).
\end{remark}

\begin{Theorem}{Hdiv-trace}
  The \putindex{trace operator}
  $\gamma_n\colon C^\infty(\overline\domain;\R^d) \to
  C^\infty(\overline{\d\domain})$
  which maps $v\mapsto v\cdot\n_{|\d\domain}$ can be extended to a
  continuous, linear mapping
  \begin{gather}
    \gamma_n\colon \Hdiv(\domain) \to H^{-1/2}(\d\domain),
  \end{gather}
  where $H^{-1/2}(\d\domain)$ is the dual of $H^{1/2}(\d\domain)$.
\end{Theorem}

\begin{proof}
  
\end{proof}

\begin{remark}
  The trace theorem tells us that our interpretation of the spaces
  $\Hdiv_0(\domain)$ and $\Hdiv_\Gamma(\domain)$ as spaces with zero
  boundary condition of the normal component is justified. This notion
  will be fortified by the following two theorems.
\end{remark}

\begin{Problem}{trace-dnu}
  Show the following resut. Let $p\in H^1(\domain)$ and
  $\Delta p \in L^2(\domain)$. Then, $\d_n p\in H^{-1/2}(\d\domain)$
  and
  \begin{gather*}
    \form(\nabla p,\nabla q) = -\form(\Delta p,q) + \forme(\d_n
    p,q)_{\d\domain} \quad\forall q\in H^1(\domain).
  \end{gather*}
\end{Problem}

\begin{Theorem}{Hdiv-trace-surjective}
  The trace theorem is optimal in the sense that
  $\gamma_n\colon \Hdiv(\domain) \to H^{-1/2}(\d\domain)$ is
  surjective.
\end{Theorem}

\subsection{Well-posedness of the dual mixed formulation}

%%% Local Variables:
%%% mode: latex
%%% TeX-master: "main"
%%% End:
