\begin{intro}
  We can embed finite element methods for the Darcy problem, also for
  the Maxwell problem, into a common framework based on the de Rham
  complex. If we wanted to do this in its full mathematical beauty, we
  would have to spend some time introducing the concept and notation
  of differential forms. As an alternative, we can use the concrete
  vector spaces $\Hdiv(\domain)$ and $\Hcurl(\domain)$. The drawback
  is, that we have to prove several particular cases, where the
  abstract theory only knows one common case. Nevertheless, it is
  worthwhile to begin this way, such that the reader has an easier
  task reading the full theory
  in~\cite{ArnoldFalkWinther06acta,ArnoldFalkWinther10}. As a byproduct,
  we will prove in generality some of the properties of polynomial
  spaces in Chapter~\ref{cha:darcy}.
\end{intro}

\subsection{Excursion to differential forms}

\begin{intro}
  This section of the notes gives a very brief overview over
  alternating multilinear forms, integrals, and differential forms and
  their relation to vector fields. It is inspired by \cite{Hiptmair02}
  and serves as a motivation for the usage of such objects.
\end{intro}

\begin{Definition}{alternating-k-form}
  An \define{alternating} $k$-form on a vector space $V$ is a map
  \begin{gather}
    \omega\colon\underbrace{V\times \dots\times V}_{k \text{ factors}} \to \R,
  \end{gather}
  which is linear in each argument and such that
  \begin{gather}
    \omega(v_1,\dots,v_k) = 0
  \end{gather}
  whenever two or more of the vectors $v_i$ coincide.

  The vector space of alternating $k$-forms on $V$ is called
  $\Alt^k V$. The space $\Alt^1 V$ is the dual space of $V$.
\end{Definition}

\begin{Lemma}{alternating-k-forms}
  For an alternating $k$-form $\omega$, there holds
  \begin{enumerate}
  \item Whenever the set $v_1,\dots,v_k$ is linearly dependent, then
    \begin{gather}
      \omega(v_1,\dots,v_k) = 0.
    \end{gather}
  \item There holds
    \begin{gather}
      \omega(\ldots,v_i,\ldots,v_j,\ldots) = - \omega(\ldots,v_j,\ldots,v_i,\ldots).
    \end{gather}
  \end{enumerate}
\end{Lemma}

\begin{Notation}{dual-basis}
  Let $x_1,\dots,x_n$ be a basis of $V$. Then we denote by $\dx_1,\dots,\dx_n$ the basis of $\Alt^1V$,
  such that
  \begin{gather}
    \dx_i(x_j) = \delta_{ij}.
  \end{gather}
\end{Notation}


\begin{example}
  Let there be two alternating 1-forms $\omega, \eta \in \Alt^1V$.
  Then, a 2-form $\mu \in \Alt^2V$ can be defined by
  \begin{gather}
    \mu(v_1, v_2) = \omega(v_1)\eta(v_2) - \eta(v_1)\omega(v_2).
  \end{gather}
  This suggests the concept of an ``alternating product'' denoted by $\omega \wedge \eta$. In
  particular, we can assign to basis vectors
  \begin{align}
    \dx_i \wedge \dx_j (x_k,x_\ell)
    &= \dx_i(x_k)\dx_j(x_\ell) - \dx_j(x_k)\dx_i(x_\ell) \\
    = - \dx_j \wedge \dx_i(x_k,x_\ell) &= \dx_i(x_\ell)\dx_j(x_k) - \dx_j(x_\ell)\dx_i(x_k) \\
    &= \delta_{ik}\delta_{j\ell} - \delta_{i\ell}\delta_{jk}.
  \end{align}
  It can be shown that the products $\dx_i\wedge \dx_j$ with $i<j$ form a basis of $\Alt^2V$.
\end{example}

\begin{Definition}{wedge-product}
  The \define{wedge product} of two alternating forms
  $\omega\in \Alt^k\R^n$ and $\eta\in\Alt^\ell\R^n$ is a form in $\Alt^{k+\ell}\R^n$ such that for
  $v_1,\dots,v_{k+\ell}\in\R^n$ holds
  \begin{gather}
    (\omega\wedge\eta)(v_1,\dots,v_{k+\ell})
    = \sum_{\sigma} \operatorname{sgn}(\sigma) \omega(v_{\sigma_1},\dots,v_{\sigma_k})
    \eta(v_{\sigma_{k+1}},\dots,v_{\sigma_{k+\ell}}),
  \end{gather}
  where the sum is taken over all permutations $\sigma$ of the set
  $\{1,\dots,k+\ell\}$ such that
  \begin{gather}
    \begin{split}
      \sigma_1 &< \dots < \sigma_k,\\
      \sigma_{k+1} & < \dots < \sigma_{k+\ell}.
    \end{split}
  \end{gather}
\end{Definition}

\begin{remark}
  Alternatively, we can sum the wedge product over all permutations of
  $k+\ell$ elements without restriction of the order. Then, we sum up
  similar terms mutliple times, such that
  \begin{gather}
    (\omega\wedge\eta)(v_1,\dots,v_{k+\ell})
    = \frac1{k!l!}\sum_{\sigma} \operatorname{sgn}(\sigma) \omega(v_{\sigma_1},\dots,v_{\sigma_k})
    \eta(v_{\sigma_{k+1}},\dots,v_{\sigma_{k+\ell}})
  \end{gather}
\end{remark}

\begin{Lemma}{basis-alt-k}
  The dimension of $\Alt^k V$ for an $n$-dimensional vector space $V$
  is $\binom nk$. The set
  \begin{gather}
    \left\{ \dx_{\sigma_1}\wedge\dots\wedge\dx_{\sigma_k} \;\middle|\;
      0<\sigma_1 < \dots < \sigma_k\le n \right\}
  \end{gather}
  is a basis for $\Alt^k V$.
\end{Lemma}
\begin{todo}
  There holds
  \begin{gather}
    \dx_{\sigma_1}\wedge\dots\wedge\dx_{\sigma_k} (x_{\rho_1},\dots,x_{\rho_k})
    = \prod_{i=1}^k \delta_{\sigma_i\rho_i}
  \end{gather}
  among all combinations $0<\rho_1 < \dots < \rho_k\le n$.
\end{todo}

\begin{remark}
  The alternating $n$-forms on $\R^n$ can be used to measure volume. In particular,
  we have
  \begin{gather}
    \dx_1\wedge\dots\wedge \dx_n(v_1,\dots,v_n)
  \end{gather}
  is the (oriented) volume of the parallelepiped spanned by the
  vectors $v_1$ to $v_n$. The $n$-form $\dx_1\wedge\dots\wedge \dx_n$ is
  the determinant.

  Similarly, alternating $k$-forms measure the volume or area of oriented
  parallelepipeds of dimension $k$ in $\R^n$. But, while
  $\dim\Alt^n\R^n = 1$ and therefore the choice of a basis determines
  the volume up to a constant only, the dimension of $\Alt^k\R^n$ for
  $0<k<n$ is nontrivial, the area has the character of a vector itself.
\end{remark}

\begin{intro}
  Let $M \subset \R^n$ be a smooth manifold of dimension $k$. We
  approximate $M$ by a set of parallelepipeds $\cell_i$ of dimension
  $k$. Then, we can define an integral of a $k$-form
  $\omega\in \Alt^k\R^n$ over $M$ in the Riemannian fashion by
  \begin{gather}
    \int_M \omega \approx \sum_{\cell_i} \omega(v_{i,1},\dots, v_{i,k}),
  \end{gather}
  where the vectors $v_{i,1}$ to $v_{i,k}$ span the cell
  $\cell_i$. The integral is then defined by a suitable limit process
  for finer and finer subdivisions of $M$.

  It is clear, that in this process, the form $\omega$ may be
  different on each cell, which leads to the more general concept of
  an integral $k$-form $\omega$ on $\R^n$, which assigns every
  submanifold of dimension $k$ a value through the integral
  ``defined'' above.

  Looking at Maxwell's equations, the electric field $E$ and the
  magnetic field $B$ in the integral form of
  equations~\eqref{eq:mixed:maxwell:faraday}
  and~\eqref{eq:mixed:maxwell:ampere} are exactly such forms, which
  assign a value to each loop in $\R^n$, or more generally, to each
  smooth curve. Thus, they qualify as integral 1-forms in this
  context.

  As all $k$-forms can be constructed as linear combinations of the
  basis of alternating $k$-forms, we can define forms depending on
  position in $\R^n$ by
  \begin{gather}
    \omega(\vx) = \sum_{\sigma} \alpha_{\sigma}(\vx)
    \dx_{\sigma_1}\wedge\dots\wedge \dx_{\sigma_k}.
  \end{gather}
\end{intro}

\begin{Definition}{differential-form}
  A \define{differential $k$-form} on $\R^n$ is a mapping
  \begin{gather}
    \begin{split}
      \omega\colon \R^n &\to  \Alt^k\R^n\\
      \omega(\vx) &= \sum_{\sigma} a_{\sigma}(\vx)
      \dx_{\sigma_1}\wedge\dots\wedge \dx_{\sigma_k},
    \end{split}
  \end{gather}
  where $\sigma$ runs over all increasing sequences of $k$ elements
  out of $n$ and the coefficient functions $a_\sigma$ are
  differentiable. More specifically, we define the spaces
  \index{lambdak@$\Lambda^k$}
  $\Lambda^k(\R^n)$ of such forms, where
  $a_\sigma \in C^\infty(\R^n)$.
  The space $\Lambda^0(\R^n)$ is introduced for closure and
  consists of differentiable functions in $C^\infty(\R^n)$.
\end{Definition}

\begin{remark}
  Note that $\Alt^k\R^n$ is a finite dimensional vector space and
  thus, we can view elements from $\Lambda^k(\domain)$ as
  vector-valued functions with an additional structure. If we define a
  norm on $\Alt^k\R^n$, we can define differential forms without going
  through coefficients.

  Thus, definition above extends naturally to all types of function
  spaces we know. Then, we prefix $\Lambda^k$ by the function
  space. Thus, $C^m\Lambda^k(\domain)$ is the space of $m$-times
  differentiable $k$-forms on the domain $\domain$.  We can also
  replace differentiablility of $\alpha_\sigma$ by integrability to
  obtain spaces like $L^p\Lambda^k(\domain)$ and even Sobolev spaces
  $H^m\Lambda^k(\domain)$.

  The short hand notation $\Lambda^k$ for $C^\infty\Lambda^k$
  introduced in the definition above is used when we do not want to
  worry about order of differentiability in order to keep arguments
  more readable.
\end{remark}

\begin{remark}
  \label{remark:proxy-fields}
  Integral and differential forms are fairly abstract objects, which
  gain their meaning by being integrated over their respective
  domains. Let us therefore relate them to functions and vector
  fields, also referred to as \define{proxy fields}.

  First 0-forms. We have already defined them as differentiable
  functions. Furthermore, 0-forms define integrals over sets of
  dimension zero, thus point evaluation.
  Thus, a form $\omega_0$ is related to a scalar function $p$ by
  \begin{gather}
    \omega_0(\vx) = p(\vx),\qquad \int_\vx \omega_0 = p(\vx).
  \end{gather}

  The space of 3-forms is also one-dimensional and can thus be
  represented by a scalar function. They represent integrals over
  three-dimensional sets $\domain$. It is known that all elements of
  $\Alt^n\R^n$ are multiples of the determinant. Thus, we conclude the
  relation
  \begin{gather}
    \omega_3(\vx)(\vv_1,\vv_2,\vv_3) = p(\vx) \det(\vv_1,\vv_2,\vv_3)
    ,\qquad \int_\domain \omega_3 = \int_\domain p(\vx) \dvx.
  \end{gather}

  The spaces of 1-forms and 2-forms on $\R^n$ are of dimension
  $n$. Thus they can be represented by vector fields $\vu$. 1-forms are
  related to line integrals and their proxy fields follow the scheme
  \begin{gather}
    \omega_1(\vx)(\vv_1) = \vu(\vx) \cdot \vv_1
    ,\qquad \int_\Gamma \omega_1 = \int_\Gamma \vu(\vx) \cdot \dl.
  \end{gather}

  Finally, 2-forms represent area integrals over vector fields such as
  \begin{gather}
    \omega_2(\vx)(\vv_1,\vv_2) = \vu(\vx) \cdot (\vv_1 \times ,\vv_2)
    ,\qquad \int_A \omega_2 = \int_A \vu(\vx) \cdot \n \ds.
  \end{gather}
\end{remark}

\begin{Definition}{proxy-fields}
  A differential form $\omega_k \in \C^m\Lambda^k(\domain)$ for
  $\domain \subseteq \R^3$ is generated by a \define{proxy field}
  $p\in C^m(\domain)$ or $\vu\in\C^m(\domain;\R^3)$ by assigning
  \begin{align}
    \omega_0 &= p\\
    \omega_1 &= u_1 \dx_1 + u_2 \dx_2 + u_3 \dx_3\\
    \omega_2 &= u_1 \dx_2\wedge \dx_3 + u_2 \dx_3 \wedge \dx_1 + u_3 \dx_1 \wedge \dx_2\\
    \omega_3 &= p \, \dx_1 \wedge \dx_2 \wedge \dx_3.
  \end{align}
\end{Definition}
\begin{todo}
  Should $\dx_i$ be the dual basis of $e_i$ or of any basis $x_i$?
\end{todo}


\begin{Problem}{proxy-fields}
  Show that the definition of the proxy fields in coordinates in
  \slideref{Definition}{proxy-fields} is consisten with the discussion
  in Remark~\ref{remark:proxy-fields}.
\end{Problem}

\begin{Definition}{exterior-derivative}
  The \define{exterior derivative} $d$ of an integral $k$-form is
  defined such that the relation
  \begin{gather}
    \label{eq:mixed:k-forms-stokes}
    \int_M d\omega = \int_{\d M} \omega
  \end{gather}
  holds for any smooth manifold $M$ of dimension $k$. The exterior
  derivative on differential forms is defined as
  $d:C^1\Lambda^k \to C^0\Lambda^{k+1}$ by
  \begin{gather}
    \label{eq:mixed:exterior-derivative}
    d\omega(v_1,\dots,v_{k+1})
    = \sum_{j=1}^{k+1} (-1)^{j+1} \tfrac{\d}{\d v_j} \omega(v_1,\dots,\widehat{v_j},\dots,v_{k+1}).
  \end{gather}
  In coordinates, we have
  \begin{gather}
    d(\alpha_\sigma \dx_{\sigma_1} \wedge\dots\wedge \dx_{\sigma_k})
    = \sum_{j=1}^n \frac{\partial \alpha_\sigma}{\d x_j}
    \dx_j \wedge \dx_{\sigma_1} \wedge\dots\wedge \dx_{\sigma_k}.
  \end{gather}
\end{Definition}

\begin{remark}
  The definitions of the exterior derivative according to
  formulas~\eqref{eq:mixed:exterior-derivative}
  and~\eqref{eq:mixed:k-forms-stokes} are consistent by the
  \putindex{Stokes theorem} for differential forms.
\end{remark}

\begin{Lemma}{proxy-derivatives}
  Let $p$ be a proxy field for $\omega_0\in C^1\Lambda^0(\R^3)$ and
  $\vu$ be a proxy field for $\omega_1\in C^1\Lambda^1(\R^3)$ and
  $\omega_2\in C^1\Lambda^2(\R^3)$, respectively. Then, the
  corresponding proxy fields of the exterior derivatives are
  \begin{gather}
    \begin{matrix}
      d\omega_0 & \leftrightarrow & \nabla p\\
      d\omega_1 & \leftrightarrow & \curl \vu\\
      d\omega_2 & \leftrightarrow & \div \vu.
    \end{matrix}
  \end{gather}
\end{Lemma}

\begin{Problem}{proxy-derivatives}
  Show \slideref{Lemma}{proxy-derivatives} by using thee coordinate
  representations of the exterior derivative and the proxy fields.
\end{Problem}

\begin{Lemma}{leibnitz-rule}
  The exterior derivative admits the \define{Leibnitz rule}: for
  $\omega\in C^1\Lambda^j$ and $\eta\in C^1\Lambda^k$, there holds
  \begin{gather}
    d(\omega\wedge\eta) = d\omega\wedge\eta + (-1)^j\omega\wedge d\eta.
  \end{gather}
\end{Lemma}

\begin{Definition}{pullback-algebra}
  Let $L\colon V\to W$ be a linear mapping and $\omega \in \Alt^k
  W$. Then, a $k$-form on $V$ is defined by \define{pullback}, namely
  \begin{gather}
    L^* \omega(v_1,\dots,v_k) = \omega(Lv_1,\dots,Lv_k)
    \qquad\forall v_1,\dots,v_k\in V.
  \end{gather}
\end{Definition}

\begin{remark}
  The definition of the pullback is consistent with the dual
  operator $L^*\colon \Alt^1 W \to \Alt^1 V$. Furthermore, there holds
  \begin{gather}
    L^*(\omega\wedge\eta) = L^*\omega \wedge L^*\eta.
  \end{gather}
\end{remark}

\begin{Definition}{pullback}
  Let $\widehat\domain,\domain$ be domains in $\R^n$ such that there is a
  smooth, surjective, invertible, orientation preserving transformation
  $\Phi\colon \widehat\domain \to \domain$. Then, the \define{pullback} of a
  differential form $\omega \in\Lambda^k(\domain)$ is the
  differential form $\Phi^*\omega \in \Lambda^k(\widehat\domain)$ defined by
  \begin{gather}
    (\Phi^*\omega)(\refvx) = \bigl(\jacobian\refmap(\refvx)\bigr)^*\omega(\vx),
  \end{gather}
  or more explicitly
  \begin{gather}
    (\Phi^*\omega)(\refvx)(v_1,\dots,v_k)
    = \omega(\vx)\bigl(\jacobian\refmap(\refvx)v_1,\dots,\jacobian\refmap(\refvx)v_k\bigr).
  \end{gather}
  Here, $\vx = \Phi(\refvx)$.
\end{Definition}

\begin{Lemma}{pullback-structure}
  The pullback for differential forms preserves the structure of such
  forms, in particular, for all forms applicable
  \begin{align}
    \Phi^*(\omega\wedge \eta) &= \Phi^* \omega \wedge \Phi^*\eta\\
    \Phi^*(d\omega) &= d(\Phi^*\omega)\\
    \int_{\widehat\domain} \Phi^*\omega &= \int_{\domain} \omega.
  \end{align}
\end{Lemma}

\begin{Lemma}{pullback-proxy}
  The pullback under the mapping $\Phi$ applied to the proxy fields
  yields the following transformations
  \begin{xalignat}2
    \Lambda^0&:& \refp(\refvx) &= p(\vx)\\
    \Lambda^1&:& \refvu(\refvx) &= \jacobian\refmap(\refvx)^T \vu(\vx)\\
    \Lambda^2&:& \refvu(\refvx) &= \det \jacobian\refmap(\refvx) \; \jacobian\refmap(\refvx)^{-1} \vu(\vx)\\
    \Lambda^3&:& \refp(\refvx) &= \det \jacobian\refmap(\refvx) \,p(\vx).
  \end{xalignat}
\end{Lemma}

\begin{Definition}{forms-trace}
  Let $i\colon \d\domain \to \overline\domain$ be the embedding
  operator mapping the point $\vx\in\d\domain$ to itself as
  $\vx\in\overline\domain$.  We define the trace operator
  \begin{align}
    \gamma\colon \Lambda^k(\overline\domain) &\to \Lambda^k(\d\domain)\\
    \omega & \mapsto i^*\omega.
  \end{align}
\end{Definition}

\begin{remark}
  The boundary of $\domain$ is of codimension one. Hence, for a
  smooth, open part $U\subset \d\domain$, we can find a set
  $\widehat U\in \R^{n-1}$ and a transformation
  $\Phi\colon\widehat U\to U$. Then, we have for
  $\omega\in\Lambda^k(\domain)$ and $\vx\in U$ with $\vx = \Phi\refvx$
  \begin{gather}
    \Phi^*\gamma\omega(\refvx)(v_1,\dots,v_k)
    =
    \gamma\omega(\vx)(\jacobian\refmap v_1,\dots,\jacobian\refmap v_k).
  \end{gather}
  There holds $\Phi^*\gamma\omega\in \Lambda^k(\widehat U)$ and the
  vectors $v_1,\dots,v_k$ are in $\R^{n-1}$. Their images
  $\jacobian\refmap v_i$ are tangential to $\d\domain$. Thus, the trace only
  depends on the tangential component of $\omega$. In particular,
  nothing is known about $\gamma\omega$ applied to vectors which are
  not tangential.
\end{remark}


%%% Local Variables:
%%% mode: latex
%%% TeX-master: "main"
%%% End:


\subsection{Well-posedness of the Maxwell problem}

\begin{Definition}{hlambda}
  The space $H\Lambda^k(\domain)$ is the closure of $\Lambda^k(\domain)$
  under the norm corresponding to the inner product
  \begin{gather}
    \scal(u,v)_{H\Lambda^k}
    = \scal(u,v)_{L^2\Lambda^k}
    + \scal(d u, d v)_{L^2\Lambda^{k+1}}.
  \end{gather}
  The space $H\Lambda^k_0(\domain)$ is the closure of
  $C^{\infty}_{00}\Lambda^k(\domain)$ with respect to the same norm.

  Both spaces are Hilbert spaces and we have
  \begin{gather}
    \gamma\omega(v_1,\dots,v_k) = 0 \qquad
    \forall \omega\in H\Lambda^k_0(\domain),
  \end{gather}
  and tangential vector fields $v_1,\dots,v_k$.
\end{Definition}

\begin{Notation}{hlambda}
  We can summarize the results of the previous section by the notation
  of the \define{Hilbert cochain complex}\index{cochain complex} of
  differential forms and the corresponding cochain complex for proxy
  fields
  \begin{gather}\minCDarrowwidth20pt
    \label{eq:derham:9}
    \begin{CD}
      \R
      @>{d}>> H\Lambda^0(\domain)
      @>{d}>> H\Lambda^1(\domain)
      @>{d}>> H\Lambda^2(\domain)
      @>{d}>> H\Lambda^3(\domain)
      @>>> 0
      \\
      @.
      @V{\cong}VV
      @V{\cong}VV
      @V{\cong}VV
      @V{\cong}VV
      \\
      \R
      @>{\subset}>> H^1(\domain)
      @>{\nabla}>> \Hcurl(\domain)
      @>{\curl}>> \Hdiv(\domain)
      @>{\div}>> L^2(\domain)
      @>>> 0,
    \end{CD}
  \end{gather}
  such that $d=d_k\colon H\Lambda^k(\domain) \to H\Lambda^{k+1}(\domain)$ and
  \begin{gather}
    d^2 = d\circ d = d_{k+1} \circ d_k = 0.
  \end{gather}
\end{Notation}

\begin{remark}
  The spaces $H\Lambda^k(\domain)$ are Hilbert spaces with values in
  the spaces of alternating $k$-forms on $\R^d$. From linear algebra,
  we know that all alternating $k$-forms are zero if $k$ exceeds the
  dimension of the vector space.  Therefore, the sequence above is
  only valid in three dimensions, and it must be shorter by one member
  in two dimensions. Changing our view back to differential operators,
  we realize that there are two relevant sequences in two
  dimensions. In the following diagram, the sequence on top can be
  used to formulate Maxwell problems in $\Hcurl$ in two dimensions,
  while the sequence on the bottom relates to the mixed form of the
  Laplacian.

  We introduce the sequences in two dimensions and afterwards will
  focus our arguments on the more general case of three dimensions
  again. Specialization to two dimensions are straight forward.
\end{remark}

\begin{Notation}{hlambda-2d}
  In two dimensions, we consider the de Rham sequences
  \begin{gather}\minCDarrowwidth20pt
    \label{eq:derham:8}
    {\small
    \begin{CD}
      \R
      @>{\subset}>> H^1(\domain)
      @>{\nabla}>> \Hcurl(\domain)
      @>{\curl}>> L^2(\domain)
      @>>> 0
      \\
      @.
      @A{\cong}AA
      @A{\cong}AA
      @A{\cong}AA
      \\
      \R
      @>{d}>> H\Lambda^0(\domain)
      @>{d}>> H\Lambda^1(\domain)
      @>{d}>> H\Lambda^2(\domain)
      @>>> 0
      \\
      @.
      @V{\cong}VV
      @V{\cong}VV
      @V{\cong}VV
      \\
      \R
      @>{\subset}>> H^1(\domain)
      @>{\curl}>> \Hdiv(\domain)
      @>{\div}>> L^2(\domain)
      @>>> 0,
    \end{CD}
    }
  \end{gather}
\end{Notation}

The value of this notation lies in the following theorem by de Rham,
which describes the relation between the elements of the sequence. It
is cited here without proof.

\begin{Definition}{cohomology-space}
  The \define{cohomology group} in $\Lambda^k(\domain)$ is defined in
  differential geometry as
  \begin{gather}
    \mathcal H_k(\domain) = \nicefrac{\ker{d_k}}{\range{d_{k-1}}}.
  \end{gather}
  Since we have a Hilbert space structure, we can much more
  conveniently choose a representative in $\ker{d_k} \subset H\Lambda^k(\domain)$
  \begin{gather}
    \mathcal H_k(\domain) = \bigl\{ \omega\in \ker{d_k}
    \;\big|\;
    \form(\omega,d\eta) = 0 \quad\forall \eta\in H\Lambda^{k-1}(\domain)
    \bigr\}.
  \end{gather}
  For spaces $H\Lambda^k_0(\domain)$ defined below, $\mathcal H_k$ is
  defined in an analogous way.
\end{Definition}

\begin{Theorem*}{de-rham-1}{de Rham}
  The dimension of the cohomology groups is equal to the Betti
  numbers, which in turn depend only on the topology of the
  domain. These depend on the number of ``holes of dimension
  $k$''. For a Lipschitz domain, they are finite. For a simply
  connected domain, they are all zero.
\end{Theorem*}

\begin{Theorem*}{de-rham}{de Rham for Hilbert complexes}
  Assume the domain $\domain$ is Lipschitz.  If $\domain$ is simply
  connected, the sequences in equations~\eqref{eq:derham:9}
  and~\eqref{eq:derham:8} are exact, that is, there holds
  \begin{gather}
    \label{eq:derham:7}
    \ker {d_{k+1}} = \range{d_k}.
  \end{gather}
  If it is not simply connected, the codimension of $\range{d_k}$ in
  $\ker{d_{k+1}}$ is finite and only determined by the topology of
  $\domain$. In particular, $\range{d_k}$ is closed in
  $H\Lambda^{k+1}(\domain)$.
\end{Theorem*}

\begin{Corollary}{de-rham-1}
  Let
  \begin{gather}
    V_0 = \bigl\{ \omega\in H\Lambda^k(\domain)
    \; \big| \;
    \form(\omega, d\eta)_{L^2\Lambda^k(\domain)} = 0
    \quad\forall\eta\in H\Lambda^{k-1}(\domain)\bigr\}.
  \end{gather}
  The problem: find $\omega\in V_0$ such that
  \begin{gather}
    \form(d \omega, d\mu) = \form(f,\mu)
    \qquad\forall \mu\in V_0,
  \end{gather}
  has a unique solution on a simply connected domain.
\end{Corollary}

\begin{Example}{de-rham-1}
  Consider the case $H\Lambda^0(\domain) = H^1(\domain)$. Then, $V_0$
  is the space of functions orthogonal to constants, and we seek a
  solution $p\in  V_0$ such that
  \begin{gather}
    \form(\nabla p, \nabla q) = (f,q) \qquad\forall q\in V_0.
  \end{gather}
\end{Example}

\begin{Corollary}{de-rham-2}
  Let
  \begin{gather}
    V_0 = \left\{ \omega\in H\Lambda^k(\domain)
        \; \middle| \;
        \arraycolsep1pt
        \begin{array}{rcll}
          \form(\omega, d\eta)
          &=&0
          &\;\forall\eta\in H\Lambda^{k-1}(\domain)\\
          \form(\omega,\tau)
          &=&0
          &\;\forall \tau\in \mathcal H_k(\domain)
        \end{array}
  \right\}.
  \end{gather}
  The problem: find $\omega\in V_0$ such that
  \begin{gather}
    \form(d \omega, d\mu) = \form(f,\mu)
    \qquad\forall \mu\in V_0,
  \end{gather}
  has a unique solution.
\end{Corollary}


So far, we have not considered boundary conditions. The next lemma,
which is again stated without proof, indicates that the properties of
the de Rham complex are inherited, if the appropriate boundary
conditions are applied to each space, namely, function values in
$H^1$, tangential traces in $\Hcurl$, and normal traces in
$\Hdiv$. The last restriction from $L^2$ to $L^2_0$ is not a boundary
condition, but it is the compatibility condition implied by the Gauss
theorem on $\Hdiv$.

\begin{Lemma}{hlambda-0}
  The Hilbert cochain complex with boundary values
  \begin{gather}\minCDarrowwidth20pt
    {\small
    \begin{CD}
      0
      @>{d}>> H\Lambda^0_0(\domain)
      @>{d}>> H\Lambda^1_0(\domain)
      @>{d}>> H\Lambda^2_0(\domain)
      @>{d}>> H\Lambda^3_0(\domain)
      @>>> 0
      \\
      @.
      @V{\cong}VV
      @V{\cong}VV
      @V{\cong}VV
      @V{\cong}VV
      \\
      0
      @>>> H^1_0(\domain)
      @>{\nabla}>> \Hcurl_0(\domain)
      @>{\curl}>> \Hdiv_0(\domain)
      @>{\div}>> L^2_0(\domain)
      @>>> 0,
    \end{CD}
    }
  \end{gather}
  has the same properties as stated for the Hilbert complex without
  boundary conditions.
\end{Lemma}

% \begin{Example}{not-simply-connected}

% \end{Example}

\begin{remark}
  The complex does not start with $\R$ on the left, but with zero,
  since the constant functions are not members of $H^1_0(\domain)$.

  On the other hand, we could have replaced the right end of the
  complex by
  \begin{gather}
    L^2(\domain) \xrightarrow{\frac1{\abs{\domain}}\int} \R,
  \end{gather}
  where the arrow is the mean value operator.
\end{remark}

\begin{Lemma}{de-rham-3}
  The problem: find
  \begin{gather}
    (\omega,\eta,\theta)
    \in H\Lambda^1_0(\domain)
    \times H\Lambda^{0}_0(\domain)
    \times \mathcal H_1(\domain)
  \end{gather}
   such that
  \begin{gather}
    \arraycolsep1pt
    \begin{array}{cccccclll}
      \form(d \omega, d\mu) &+& \form(\mu,d\eta) &+& \form(\theta,\mu)
      &=& \form(f,\mu)
      & \qquad & \forall \mu\in H\Lambda^1_0(\domain)
      \\
      \form(\omega,d\zeta) && && &=&0
      & \qquad & \forall \zeta\in H\Lambda^{0}_0(\domain)
      \\
      \form(\omega,\tau) && && &=& 0
      & \qquad & \forall \tau\in \mathcal H_1(\domain).
    \end{array}
  \end{gather}
  is well-posed.
\end{Lemma}

\begin{Theorem}{div-curl-well-posed}
  Let $\domain$ be simply connected. Then, the Maxwell problem in
  \slideref{Definition}{Maxwell-mixed-0} is well posed.
\end{Theorem}

\begin{proof}
  We have to show the inf-sup condition and the ellipticity of the
  curl-curl bilinear form. Let us introduce
  \begin{gather}
    a(u,v) = \form(\curl u, \curl v),
    \qquad
    b(v,q) = \form(v,\nabla q).
  \end{gather}
  From the fact that the de Rham complex starts with zero, we obtain
  that the kernel of the gradient is zero. Thus, for any $q\in
  H^1_0(\domain)\setminus\{0\}$, we have $v = \nabla q \neq 0$ and
  $\norm{v}_{\Hcurl} = \norm{v}_{L^2} \le \norm{q}_{H^1}$. Thus, the
  inf-sup condition holds.

  We show now that $a(.,.)$ is elliptic on $\ker B$. From the
  definition of $b(.,.)$, we deduce that
  $\ker B \perp \nabla H^1_0(\domain) = \ker A$. Thus, $A$ is an
  isomorphism between $\ker B$ and its dual, and consequently
  elliptic.
\end{proof}

\begin{Problem}{darcy-derham}
  Prove well-posedness for the Darcy problem using the de Rham complex
  for proving \slideref{Lemma}{darcy-reduced-wellposed} and
  \slideref{Lemma}{darcy-infsup}.
\begin{solution}
From the last step
 \begin{gather}\minCDarrowwidth20pt
    \begin{CD}
      \Hdiv_0(\domain)
      @>{\div}>> L^2_0(\domain)
      @>>> 0,
    \end{CD}
  \end{gather}
we deduce $L_0^2(\Omega)=\range{\nabla\cdot}$. In particular, this means
that for all $q\in L_0^2(\Omega)$ there exists $v\in H_0^{\text{div}}$ such that
$\nabla \cdot v=q$ and $\norm{v}_{H^{\text{div}}}\leq C \norm{q}_0$.
Hence, it holds the estimate
\begin{align}
\inf_{q\in Q}\sup_{v\in V}\frac{(\nabla \cdot v, q)}{\norm{q}_Q\norm{v}_V}
  \geq \inf_{q\in Q} \frac{\norm{q}_0}{\norm{v}_{H^{\text{div}}}}\geq C.
\end{align}
where $q = \nabla \cdot v$. Ellipticity of the bilinear form is obvious
\begin{align}
  \norm{u}_{H^{\text{div}}}=\norm{u}_0 \quad \forall u\in \ker B.
\end{align}
\end{solution}
\end{Problem}

\section{Polynomial complexes for simplicial meshes}

\begin{intro}
  We have already seen that adding $\vx\P_k$ to the space $\P_k^\sdim$, we
  obtain a surjective divergence operator from the Raviart-Thomas
  element to the pressure space $\P_k$. In this section, we see that
  there is a general principle behind this concept and it can be
  extended to the curl and gradient operators.
\end{intro}

\begin{Notation}{pk-complex}
  The polynomial complex $\P_r\Lambda^k$ consists of $k$-forms with
  coefficients in the polynomial space $\P_r$.
  
  The homogeneous polynomial spaces $\breve\P_k$ and their proxy
  fields form the cochain complex
  \begin{gather}\minCDarrowwidth15pt
    \begin{CD}
      0
      @>{d}>> \breve\P_r\Lambda^0
      @>{d}>> \breve\P_{r-1}\Lambda^1
      @>{d}>> \breve\P_{r-2}\Lambda^2
      @>{d}>> \breve\P_{r-3}\Lambda^3
      @>{d}>> 0
      \\
      @.
      @V{\cong}VV
      @V{\cong}VV
      @V{\cong}VV
      @V{\cong}VV
      \\
      0
      @>{\subset}>> \breve\P_r
      @>{\nabla}>> \breve\P_{r-1}^3
      @>{\curl}>> \breve\P_{r-2}^3
      @>{\div}>> \breve\P_{r-3}
      @>>> 0,
    \end{CD}
  \end{gather}
  and $d_{k+1}\circ d_k = 0$.
\end{Notation}

\begin{remark}
  Since the polynomial space $\P_r$ is the direct sum
  \begin{gather}
    \P_r = \bigoplus_{s=0}^r \breve\P_s,
  \end{gather}
  the homogeneous polynomial complex above can be extended to a
  general polynomial complex in a straightforward way.
\end{remark}

\subsection{The Koszul complex}

\begin{Definition}{koszul-differential}
  We define the \define{Koszul differential} as a map
  \begin{gather}
    \kappa\colon \Lambda^k(\R^\sdim) \to \Lambda^{k-1}(\R^\sdim),
  \end{gather}
  such that
  \begin{gather}
    \kappa\omega(\vx) (v_1,\dots,v_{k-1})
    = \omega(\vx) (-\vx,v_1,\dots,v_{k-1}).
  \end{gather}
  As with the exterior differentail, we write $\kappa_k$ if we want to
  specify the degree of $\Lambda^k$.  Application to proxy fields yields
  \begin{gather}
    \label{eq:derham:13}
    \begin{aligned}
      \kappa\omega_1 &\leftrightarrow \vx\cdot\vu,\\
      \kappa\omega_2 &\leftrightarrow -\vx\times\vu,\\
      \kappa\omega_3 &\leftrightarrow \vx p.
    \end{aligned}
  \end{gather}
\end{Definition}

\begin{Lemma}{koszul-differential}
  There holds
  \begin{gather}
    \label{eq:derham:14}
    \kappa\circ\kappa = \kappa_k\circ\kappa_{k+1} = 0.
  \end{gather}
  Furthermore, we have the Leibnitz rule for $\omega\in \Lambda^k$ and
  $\eta\in\Lambda^\ell$
  \begin{gather}
    \kappa(\omega\wedge\eta) = \kappa\omega\wedge\eta
    + (-1)^k \omega\wedge\kappa\eta.
  \end{gather}
  If
  $\omega(\vx) = a_\sigma(\vx)
  \dx_{\sigma_1}\wedge\dots\wedge\dx_{\sigma_k}$, then we have
  \begin{gather}
    \kappa\omega(\vx) = \sum_{i=1}^k (-1)^{i+1} a_\sigma(\vx)x_{\sigma_i}
    \dx_{\sigma_1}\wedge\dots\wedge\widehat{\dx_{\sigma_i}}
    \wedge\dots\wedge\dx_{\sigma_k}.
  \end{gather}
  Finally, the Koszul differential commutes with the pullback.
\end{Lemma}

\begin{Definition}{Koszul-complex}
  The homogeneous \define{Koszul complex} is a polynomial complex of
  the form
  \begin{gather}\minCDarrowwidth15pt
    \label{eq:derham:12}
    \begin{CD}
      0
      @<<< \breve \P_r\Lambda^0
      @<{\kappa_1}<< \breve \P_{r-1}\Lambda^1
      @<{\kappa_2}<< \breve \P_{r-2}\Lambda^2
      @<{\kappa_3}<< \breve \P_{r-3}\Lambda^3
      @<<< 0
    \end{CD}.
  \end{gather}
\end{Definition}

Note that the ``Koszul differential'' increases the polynomial order
and lowers the order of the form, thus acts in the opposite way of the
usual exterior derivative $d$.

\begin{Lemma}{kd-plus-dk}
  For $\omega\in \breve \P_r\Lambda^k$ there holds
  \begin{gather}
    \label{eq:derham:15}
    \bigl(d\kappa+\kappa d\bigr)\omega = (r+k) \omega.
  \end{gather}
\end{Lemma}

\begin{proof}
  We prove this here
  for each $k$ directly. The proof for forms is in the video and can be found in ~\cite{ArnoldFalkWinther06acta}. For $k=0$, we have $\kappa\omega = 0$, thus
  we have to show
  \begin{gather}
    \kappa d\omega = r\omega.
  \end{gather}
  Due to linearity of $\kappa$ and $d$, it suffices to prove the
  result for $\omega = p=x_1^ax_2^bx_3^c$. We note that $dp/d_{x_1} =
  a/x_1 p$ and $d(x_1 p)/d_{x_1} = (a+1) p$ and analogue for the other
  coordinates.
  \begin{gather}
    \kappa_1 d_0\omega = x\cdot \nabla p = x\cdot
    \begin{pmatrix}
      a/x_1\\b/x_2\\c/x_3
    \end{pmatrix}p
    = (a+b+c)p.
  \end{gather}
  The second easy case is $k=3$ such that $d\omega = 0$. Let again
  $\omega = p$ to obtain
  \begin{gather}
    d_2\kappa_3 \omega = \div(xp) = \div
    \begin{pmatrix}
      x_1 p \\x_2 p \\x_3 p
    \end{pmatrix}
    = (a+1+b+1+c+1) p = (r+3) \omega.
  \end{gather}
  For the two vector valued cases, we note that it suffices to prove
  the result for $\omega = (p,0,0)^\transpose$ and to note that the results for
  nonzero second and third component follow suite. Thus, for $k=1$
  \begin{multline}
    \nabla (x\cdot \omega) - x\times \curl \omega
    = \nabla(x_1 p) - x\times
    \begin{pmatrix}
      0\\c/x_3 \\ -b/x_2
    \end{pmatrix}p
    \\
    =
    \begin{pmatrix}
      a+1 \\ bx_1/x_2\\ cx_1/x_3
    \end{pmatrix}p
    +
    \begin{pmatrix}
      b+c \\ -bx_1/x_2\\cx_1/x_3
    \end{pmatrix}p
    =
    \begin{pmatrix}
      a+b+c+1 \\0\\0
    \end{pmatrix}p
    = (r+1)\omega.
  \end{multline}
  Finally, for $k=2$
  \begin{multline}
    \curl(-x\times \omega) + x \div \omega
    = \curl
    \begin{pmatrix}
      0 \\ -x_3 \\ x_2
    \end{pmatrix}p
    +
    \begin{pmatrix}
      x_1 a/x_1\\x_2 a/x_1\\x_3 a/x_1\\
    \end{pmatrix}p
    \\=
    \begin{pmatrix}
      b+1+c+1\\-ax_2/x_1 \\ -a x_3/x_1
    \end{pmatrix}p
    +
    \begin{pmatrix}
      a\\ax_2/x_1\\ax_3/x_1
    \end{pmatrix}p
    =
    \begin{pmatrix}
      a+b+c+2\\0\\0
    \end{pmatrix}p
    = (r+2)\omega.
  \end{multline}
\end{proof}

\begin{Lemma}{d-kappa-injective}
  The restriction of operator $d$ to $\range \kappa$ is injective and
  vice versa, or equivalently for any polynomial form
  $\omega\in \breve \P_r\Lambda^k$ there holds
  \begin{gather}
    \label{eq:derham:16}
    \begin{aligned}
      d\kappa\omega &= 0 &\Longrightarrow&& \kappa\omega &= 0,\\
      \kappa d\omega &= 0 &\Longrightarrow&& d\omega &= 0.
    \end{aligned}
  \end{gather}
\end{Lemma}

\begin{proof}
  If $r=k=0$, then $\kappa\omega = d\omega = 0$, such that the lemma
  holds trivially. For $r+k\neq 0$, we apply $\kappa$ to
  equation~\eqref{eq:derham:15} to obtain
  \begin{gather}
    \kappa\omega = \frac1{r+k}
    \bigl(\kappa d\kappa\omega + \kappa^2d\omega\bigr)
    = \frac1{r+k}\kappa d\kappa\omega.
  \end{gather}
  Thus, we have proven $d\kappa\omega=0$ implies $\kappa\omega=0$. The
  second implication is proven by applying $d$ to~\eqref{eq:derham:15}.
\end{proof}

\begin{Theorem}{polynomial-exact}
  The polynomial de Rham complex and the Koszul complex are exact for
  $r\ge 1$. Furthermore, for $k,r\ge 0$ and $k+r>0$, there holds
  \begin{gather}
    \label{eq:derham:17}
    \breve \P_r\Lambda^k = \kappa \breve\P_{r-1}\Lambda^{k+1}
    \oplus d\breve\P_{r+1}\Lambda^{k-1}.
  \end{gather}
\end{Theorem}

\begin{proof}
  We already know $\range{\kappa_{k-1}} \subset \ker{\kappa_k}$. Thus,
  it remains to show the opposite inclusion. Let therefore $\omega\in
  \breve \P_r\Lambda^k$ such that $\kappa\omega=0$. Then,
  \begin{gather}
    \omega = \frac1{r+k} (d\kappa\omega+\kappa d\omega)
    = \frac1{r+k} \kappa d\omega =: \kappa\eta
  \end{gather}
  with $\eta \in \breve\P_{r-1}\Lambda^{k+1}$. Thus,
  $\omega\in \range{\kappa_{k-1}}$. Again, the proof for the de Rham
  complex is obtained by replacing $\kappa$ by $d$.

  In order to see that $\breve \P_r\Lambda^k$ is the sum of the two
  spaces, we let for arbitrary $\omega\in \breve \P_r\Lambda^k$
  \begin{gather}
    \eta = \frac1{r+k} d\omega \in \breve\P_{r-1}\Lambda^{k+1},
    \qquad
    \mu = \frac1{r+k} \kappa\omega \in \breve\P_{k+1}\Lambda^{k-1}.
  \end{gather}
  By equation~\eqref{eq:derham:15}, we have
  $\omega = \kappa\eta + d\mu$. It remains to show that the
  intersection of the spaces is zero. Therefore, let $\omega$ be
  chosen from the intersection. Then, $\omega = \kappa\eta = d \mu$
  and
  \begin{gather}
    (r+k)\omega = d\kappa \omega + \kappa d \omega
    = d \kappa^2 \eta + \kappa d^2 \mu = 0.
  \end{gather}
\end{proof}

\begin{Corollary}{pk-complexes}
  \slideref{Theorem}{polynomial-exact} holds as well for the
  polynomial complexes
  \begin{gather}\minCDarrowwidth15pt
    \begin{CD}
      0
      @>>> \P_r\Lambda^0
      @>{d}>> \P_{r-1}\Lambda^1
      @>{d}>> \P_{r-2}\Lambda^2
      @>{d}>> \P_{r-3}\Lambda^3
      @>>> 0
    \end{CD},
  \end{gather}
  and
  \begin{gather}\minCDarrowwidth15pt
    \begin{CD}
      \R
      @<<< \P_r\Lambda^0
      @<{\kappa}<< \P_{r-1}\Lambda^1
      @<{\kappa}<< \P_{r-2}\Lambda^2
      @<{\kappa}<< \P_{r-3}\Lambda^3
      @<<< 0
    \end{CD}.
  \end{gather}
\end{Corollary}

\begin{proof}
  This is due to the fact that the polynomial spaces $\P_r$ are the
  direct sums of homogeneous polynomial space $\breve\P_s$.
\end{proof}

\begin{Definition}{pk-plus}
  \index{prpl@$\P_r+\Lambda^k$}
  The polynomial space of $k$-forms $\P_r^+\Lambda^k$ is defined as
  \begin{gather}
    \P_r^+\Lambda^k = \P_r\Lambda^k \oplus \kappa \breve\P_r\Lambda^{k+1}.
  \end{gather}
  \index{prml@$\P_r-\Lambda^k$}
  It is also referred to as $\P_{r+1}^-\Lambda^k$. Furthermore,
  \begin{gather}
    \P_r^+\Lambda^0 = \P_{r+1}\Lambda^0,
    \qquad
    \P_r^+\Lambda^\sdim = \P_r\Lambda^\sdim.
  \end{gather}
\end{Definition}

\begin{remark}
  We have used the construction principle
  \begin{gather}
    \P_r\Lambda^k = \P_{r-1}\Lambda^k \oplus \breve \P_r\Lambda^k.
  \end{gather}
  Using its decomposition, we obtain
  \begin{gather}
    \P_r\Lambda^k =
    \bigoplus_{s=1}^{r-1}\kappa \breve\P_{s-1}\Lambda^{k+1}
    \bigoplus_{s=1}^{r-1} d\breve\P_{s+1}\Lambda^{k-1}
    \oplus \kappa \breve\P_{r-1}\Lambda^{k+1}
    \oplus d\breve\P_{r+1}\Lambda^{k-1}.
  \end{gather}
  If we leave out the last factor, we get the new space
  $\P_{r-1}^+\Lambda^k$.
\end{remark}

\begin{Lemma}{pk-plus-d}
  If $\omega\in \P_r^+\Lambda^k$ and $d\omega=0$, then
  $\omega\in\P_r\Lambda^k$.
\end{Lemma}

\begin{proof}
  Let $\omega = \omega_1 + \kappa\eta$ with $\omega_1\in\P_r\Lambda^k$
  and $\eta\in \breve\P_r\Lambda^{k+1}$. Then,
  $d\omega_1\in\P_{r-1}\Lambda^{k+1}$ and
  $d\kappa\eta\in \breve\P_r\Lambda^{k+1}$. Therefore,
  $d\omega_1 = d\kappa\eta = 0$. By
  \slideref{Lemma}{d-kappa-injective}, $\kappa\eta=0$, such that
  $\omega=\omega_1$.
\end{proof}

\begin{Problem}{rt-bdm-forms}
  \begin{enumerate}
  \item   Identify the spaces $\P_r \Lambda^{\sdim-1}$ and
    $\P_r^+ \Lambda^{\sdim-1}$ with finite element spaces for
    $\Hdiv$.
  \item Where have we stated a special case of
    \slideref{Lemma}{pk-plus-d} before?
  \item Suggest polynomial spaces for $\Hcurl$ corresponding to
    $\P_r \Lambda^{1}$ and $\P_r^+ \Lambda^{1}$
  \end{enumerate}
\end{Problem}

% AFW 2006 Lemma 3.8
\begin{Lemma}{pr-pr-plus}
  For $r\ge 1$ and $0\le k < n$ there holds
  \begin{gather}
    d\P_{r}^+\Lambda^k \subset d\P_{r+1}\Lambda^k
    \subset \P_r\Lambda^{k+1}
    \subset \P_r^+\Lambda^{k+1}.
  \end{gather}
  The following four mappings $d$ have the same kernel:
  \begin{gather}
    \begin{aligned}
      d\colon \P_{r+1}\Lambda^k &\to \P_{r}\Lambda^{k+1}
      &
      d\colon \P_r^+\Lambda^k &\to \P_{r}\Lambda^{k+1}
      \\
      d\colon \P_{r+1}\Lambda^k &\to \P_{r}^+\Lambda^{k+1}
      &
      d\colon \P_r^+\Lambda^k &\to \P_{r}^+\Lambda^{k+1}
    \end{aligned}
  \end{gather}
  The following four mappings $d$ have the same range:
  \begin{gather}
    \begin{aligned}
      d\colon \P_{r+1}\Lambda^k &\to \P_{r}\Lambda^{k+1}
      &
      d\colon \P_{r}^+\Lambda^k &\to \P_{r}\Lambda^{k+1}
      \\
      d\colon \P_{r+1}\Lambda^k &\to \P_{r}^+\Lambda^{k+1}
      &
      d\colon \P_{r}^+\Lambda^k &\to \P_{r}^+\Lambda^{k+1}
    \end{aligned}
  \end{gather}
\end{Lemma}


\begin{Corollary}{pr-sequences}
  For polynomial forms in $\R^3$, 
  all four sequences in this diagram are exact:
  {\small
  \begin{tikzcd}
    \R \arrow[r,hook]
    & \P_{r+3}\Lambda^0 \arrow[r,"\diffd"]
    & \P_{r+2}\Lambda^1 \arrow[rd,"\diffd"]\\
    \R \arrow[r,hook]
    & \P_{r+2}\Lambda^0 \arrow[r,"\diffd"]
    & \P_{r+1}^+\Lambda^1 \arrow[r,"\diffd"]
    & \P_{r+1}\Lambda^2 \arrow[rdd,"\diffd"]\\
    \R \arrow[r,hook]
    & \P_{r+2}\Lambda^0 \arrow[r,"\diffd"]
    & \P_{r+1}\Lambda^1 \arrow[rd,"\diffd"]\\
    \R \arrow[r,hook]
    & \P_{r+1}\Lambda^0 \arrow[r,"\diffd"]
    & \P_{r}^+\Lambda^1 \arrow[r,"\diffd"]
    & \P_{r}^+\Lambda^2 \arrow[r,"\diffd"]
    & \P_{r}\Lambda^3 \arrow[r,"\diffd"]
    & 0
  \end{tikzcd}
  }
\end{Corollary}

\begin{proof}
  The first statement follows from the inclusions of $\P_r$ and
  $\P_r^+$. The horizontal equality of the second statement follows
  from \slideref{Lemma}{pk-plus-d}. The vertical identities from the
  decomposition~\eqref{eq:derham:17}. For the last set of identities,
  we observe that by construction
  \begin{gather}
    \P_r\Lambda^k = \P_{r-1}^+\Lambda^k \oplus d \P_{r+1}\Lambda^{k-1}.
  \end{gather}
  Thus, $d\P_r\Lambda^k = d\P_{r-1}^+\Lambda^k\subset \P_{r-1}\Lambda^{k+1}$.
\end{proof}

\begin{Theorem}{dimension-pr-lambda}
  Let $r\ge 0$ and $1\le k \le \sdim$. Then,
  \begin{gather}
    \begin{split}
      \dim \kappa \breve P_r\Lambda^k(\R^\sdim)
      & = \dim d \breve P_{r+1}\Lambda^{k-1}(\R^\sdim)
      \\
      &= \binom{\sdim+r}{\sdim-k}\binom{r+k-1}{k-1}.
    \end{split}
  \end{gather}
\end{Theorem}

\begin{proof}
  First, we prove the equality of the two dimensions by applying $\kappa$
  to equation~\eqref{eq:derham:17}, yielding
  \begin{gather}
    \kappa \breve P_r\Lambda^k(\R^d)
    = \kappa \ediff \breve P_{r+1}\Lambda^{k-1}(\R^\sdim).
  \end{gather}
  By \slideref{Lemma}{d-kappa-injective}, the two spaces are
  isomorphic and the equality holds.

  The dimension formula is proven first for $r=0$ and $k\ge 1$. The
  Koszul operator is injective on $\P_0\Lambda^K(\R^\sdim)$ since the
  first factor in equation~\eqref{eq:derham:17} vanishes. It is
  also injective on $\breve P_r\Lambda^\sdim(\R^\sdim)$ for $r\ge 0$.

  For all other combinations of $r$ and $k$ it is proven by induction
  over $k$. For $k=\sdim$,
  \begin{gather}
    \dim \breve\P_r\Lambda^\sdim(\R^\sdim) = \dim\breve\P_r(\R^\sdim)
    = \binom{\sdim+r-1}{\sdim-1}.
  \end{gather}
  For $k<\sdim$, we assume the formula proven for $k+1$. We have
% Move this to an earlier point or an appendix
  \begin{gather}
    \breve P_r\Lambda^k(\R^d) = \binom{\sdim+r-1}{\sdim-1}\binom{\sdim}{k}.
  \end{gather}
  Now, the dimension formula
  \begin{gather}
    \dim \range \phi = \dim V - \dim \ker \phi,
  \end{gather}
  yields
  \begin{gather}
    \dim\kappa\breve \P_r\Lambda^k(\R^\sdim) =
    \dim\breve \P_r\Lambda^k(\R^\sdim)
    - \dim\kappa\breve \P_{r-1}\Lambda^{k+1}(\R^\sdim),
  \end{gather}
  where we have used the exactness of the Koszul complex.
  Using the induction hypothesis yields by the binomial identity
% Olkhovskiy
  \begin{align}
    \dim\kappa\breve \P_r\Lambda^k(\R^\sdim)
    &= \binom{\sdim+r-1}{\sdim-1}\binom{\sdim}{k} - \binom{\sdim+r-1}{\sdim-k-1}\binom{r+k-1}{k}
    \\
    &= \frac{(\sdim+r-1)!{\color{green}\sdim!}}{{\color{green}(\sdim-1)!}{\color{purple}r!}k!{\color{blue}(\sdim-k)!}}
      - \frac{(\sdim+r-1)!{\color{red}(r+k-1)!}}{{\color{blue}(\sdim-k-1)!}
      {\color{red}(r+k)!}k!{\color{purple}(r-1)!}}
    \\
    &=
      \frac{(\sdim+r-1)!{\color{green}\sdim}{\color{red}(r+k)}-{\color{blue}(\sdim-k)}(\sdim+r-1)!{\color{purple}r}}
      {{\color{purple}r!}k!{\color{blue}(\sdim-k)!}{\color{red}(r+k)}}
    \\
    &= \frac{(\sdim+r-1)!\bigl(\sdim(r+k)-(\sdim-k)r\bigr)}{r!k!(\sdim-k)!(r+k)}
    \\
    &= \frac{(\sdim+r)!}{r!(k-1)!(\sdim-k)!(r+k)}
    \\
    &= \binom{\sdim+r}{\sdim-k}\binom{r+k-1}{k-1}.
  \end{align}
\end{proof}

\subsection{Degrees of freedom and bases for simplicial meshes}

\begin{intro}
  After having studied the properties of the de Rham complex and the
  Koszul complex of polynomial spaces, we continue like with standard
  finite elements and define a basis of shape functions and sets of
  degrees of freedom dual to this basis. Note that the following
  definition subsumes the definitions of conforming finite elements
  for $H^1$, $\Hcurl$ and $\Hdiv$ in a single statement.
\end{intro}

\begin{Definition}{mesh-pk}
  Given a space of polynomial forms
  $\P_r\Lambda^k=\P_r\Lambda^k(\R^\sdim)$, we define the space of finite
  element polynomial forms on a mesh $\mesh_h$ covering the domain
  $\domain\subset\R^\sdim$ as
  \begin{gather}
    \P_r\Lambda^k(\mesh) = \bigl\{
    \omega \in H\Lambda^k \big|
    \;\forall \cell\in\T\colon \omega_{|\cell} \in \P_r\Lambda^k
    \bigr\}.
  \end{gather}
  Furthermore, we define $\Delta_m(\T_h)$ as the set of all
  $m$-dimensional subsimplices of the whole mesh, where shared
  subsimplices of several cells are identified.
\end{Definition}

\begin{intro}
  The degrees of freedom have to be designed such that they guarantee
  the necessary continuity between cells. To this end, we have to
  study the traces of polynomial forms on the boundaries (called
  subsimplices below) of the simplex $\cell$. Then, we can start
  decomposing degrees of freedom and node values such that they can be
  allocated to these subsimplices.
\end{intro}

\begin{Theorem}{hlambda-continuity}
  Let $\omega\in L^2\Lambda^k(\domain)$ be polynomial on each mesh
  cell $\cell$ of a mesh $\mesh_h$ covering $\domain$. Then, the
  following statements are equivalent:
  \begin{enumerate}
  \item $\omega\in H\Lambda^k(\domain)$
  \item $\gamma_f\omega$ is single-valued for all $f\in\Delta_{\sdim-1}(\T_h)$
  \item $\gamma_f\omega$ is single-valued for all $f\in\Delta_{m}(\T_h)$ for
    $k\le m \le n-1$.
  \end{enumerate}
\end{Theorem}

\begin{proof}
  By the Stokes theorem, every smooth $k$-form on a mesh cell $\cell$
  has a well-defined trace $\gamma_f\omega$ on
  $f\in\Delta_{n-1}(\cell)$. Thus, $\omega\in H\Lambda^k(\domain)$ if
  and only if these traces coincide from both cells sharing this face.

  Once this is established, it is also clear that the two traces can
  only be the same if their traces to the boundary of $f$ coincide as
  well.
\end{proof}

\subsubsection{Geometric decomposition of $\P_r(\cell)$}

For the geometric decomposition of simplices and the consequences on barycentric coordinates, see \cref{sec:barycentric}

\begin{remark}
  When we introduced barycentric coordinates in order to define
  standard shape functions on simplices, we generated a basis for
  $\P_r(\R^\sdim)$ by selecting polynomials of the $\lambda_i$. Closer
  inspection reveals that these polynomials were
  homogeneous. Therefore, we defined an isomorphism
  \begin{gather}
    \breve \P_k(\R^{\sdim+1}) \cong \P_k(\R^\sdim),
  \end{gather}
  which reads: for every $p\in \P_k(\R^\sdim)$ there is
  $q\in\breve\P_k(\R^{\sdim+1})$ such that
  \begin{gather}
    p(x_1,\dots,x_\sdim) = q(\lambda_0,\dots,\lambda_\sdim).
  \end{gather}
% Argue why isomorphism: count?
\end{remark}

\begin{Definition}{pr-f}
  For each $k$-dimensional subsimplex $f_\sigma$ of $\cell$ with
  $\sigma = \sigma_0,\dots,\sigma_k$, the space
  $\P_r(f_\sigma) \cong \breve\P_r(\R^{k+1})$ is defined as
  \begin{gather}
    \P_r(f_\sigma) = \bigl\{
    q(\lambda_{\sigma_0},\dots,\lambda_{\sigma_k})
    \;\big\vert\;
    q\in\breve \P_r(\R^{k+1})\bigr\}.
  \end{gather}
  
  By $\overset{\circ}{\P}_r(f_\sigma)$ we denote the space of
  polynomials in $\P_r(f_\sigma)$ which vanish on the boundary $\d f_\sigma$.

  The \putindex{bubble function} associated with $f_\sigma$ is
  \begin{gather}
    b_{f\sigma} = \lambda_{\sigma_0}\cdots\lambda_{\sigma_k}
    \in \overset{\circ}{\P}_{k+1}(f_\sigma).
  \end{gather}
\end{Definition}

\begin{Lemma}{p0-bubble}
  Let $f$ be a $k$-simplex. Then,
  \begin{gather}
    \overset{\circ}{\P}_r(f) =
    \begin{cases}
      b_f \P_{r-k-1}(f) &\text{if } r\ge k+1\\
      \{0\} &\text{if } r \le k.
    \end{cases}
  \end{gather}
\end{Lemma}


\begin{Definition}{pr-f-extension}
  The \define{extension operator} $E_{f_\sigma\to\cell}$ is defined as
  \begin{gather}
    \begin{split}
      E_{f_\sigma\to\cell}\colon \P_r(f_\sigma) &\to \P_r(\R^\sdim),\\
      p(\lambda_0,\dots,\lambda_\sdim) &= q(\lambda_{\sigma_0},\dots,\lambda_{\sigma_k}),
    \end{split}
  \end{gather}
  where $q$ is chosen as in the definition of $\P_r(f_\sigma)$.
\end{Definition}

\begin{Lemma}{subsimplex-polynomials}
  Every function in $\P_r(f)$ vanishes on every subsimplex
  $g\in\Delta(\cell)$ which is disjoint from $f$.

  The bubble function $b_f$ vanishes on every subsimplex in
  $\Delta(\cell)$ not containing $f$.
\end{Lemma}

\begin{Problem}{subsimplex-polynomials}
  Show: $\P_r(f_\sigma)$ is isomorphic to $\P_r(\R^k)$. Prove
  \slideref{Lemma}{subsimplex-polynomials}.
\begin{solution}
  Since $\P_r(f_\sigma)$ is isomorphic to $\breve\P_r(\R^{k+1})$ and
  $\P_r(f_\sigma)$ is a vector space
  all we have to do is to show that the dimensions are equal.
  In fact, $\dim \P_r(\R^k) = \binom{r+k}{k}$
  (show by distributiung separators) and
  $\dim \breve\P_r(\R^k) = \binom{r+k-1}{k-1}$.
  Thus, $\dim \breve\P_r(\R^{k+1}) = \binom{r+k}{k} = \dim \P_r(\R^k)$
  and the two spaces are isomorphic.

  Let $g$ be a subsimplex that is disjoint from $f$.
  \begin{align}
      f_\sigma &= \bigl\{x\in\cell \big\vert
      \lambda_j=0 \text{ for } j\not\in\sigma\bigr\}.
      \\
      &= \biggl\{ x = \sum_{i\in\sigma}\lambda_i\bigg\vert
    \lambda_i \ge 0,\quad\sum_{i\in\sigma}\lambda_i = 1\biggr\}.
  \end{align}
  Then $\sigma_f$ and $\sigma_g$ are disjoint and in particular
  $\lambda_i = 0$ for all $i\in \sigma_f$ in the decomposition
  $g \ni x = \sum_i \lambda_i(x)$. Thus, $p\in \P_r(f)$
  is identified by $q(\lambda_{\sigma_0},\dots,\lambda_{\sigma_k})$
  which vanishes for all $x \in g$.

  Let $g$ be a subsimplex not containing $f$. Then, $\sigma_f\setminus \sigma_g$ is non-empty.
  In the decomposition $g\ni x = \sum_i\lambda_i$ such that $\lambda_i \ge 0,\quad\sum_i\lambda_i = 1$ at most
  the $\lambda_i$ ($i\in\sigma_f$) for $i\in\sigma_f\cap\sigma_g$ are non-zero. Due to to our assumption
  $\sigma_f\cap\sigma_g$ is a proper subset of $\sigma_f$ and hence there exist for each $x \in f\cap g$
  a $i \in \sigma_f$ such that $\lambda_i=0$. This implies $b_{f_\sigma}\equiv 0$ on $g$.
\end{solution}
\end{Problem}

\begin{Example}{h1-moment-dofs}
  \begin{center}
    \includegraphics[width=.3\textwidth]{fig/p1-p}
    \includegraphics[width=.3\textwidth]{fig/p2-p}
    \includegraphics[width=.3\textwidth]{fig/p3-p}
  \end{center}
  Unisolvent interpolation conditions for $\P_r(\cell)$
  \begin{xalignat*}3
    u(f) &= 0 &&& \dim f &= 0\\
    \form(u,q)_f &= 0 & q&\in \P_{r-2}(f) & \dim f &= 1 \\
    \form(u,q)_f &= 0 & q&\in \P_{r-3}(f) & \dim f &= 2 \\
    &\vdots && \vdots && \vdots
  \end{xalignat*}
\end{Example}

We are now generalizing and formalizing this example in order to
derive a geometric decomposition of $\P_r(\cell)$ and its dual.

\begin{Definition}{v-of-f}
  For every $f\in\Delta(\cell)$%
  %with $\sigma=\{\sigma_0,\dots,\sigma_k\}$
  , we define
  $V(f) \subset \P_r(\cell)$  for $\dim f>0$ as
  \begin{gather}
    V(f) = \bigl\{ p = E_{f\to\cell} b_f q
    \;\big\vert\;
    q \in \P_{r-\dim f -1}(f) \bigr\},
  \end{gather}
  and for $\dim f=0$
  \begin{gather}
    V(f) = \bigl\{ \lambda_i^r
    \;\big\vert\;
    f = \{x_i\}\bigr\}.
  \end{gather}
  For $r\le \dim f$ holds
  \begin{gather}
    V(f) = \{0\}.
  \end{gather}
\end{Definition}

\begin{Definition}{w-of-f}
  For every $f\in\Delta(\cell)$, we define
  $W(f) \subset \P_r(\cell)^*$ for $\dim f>0$ as
  \begin{gather}
    W(f) = \bigl\{\phi(p) = \form(p,q)_f\big\vert
    \;q\in\P_{r-\dim f-1}(f) \bigr\},
  \end{gather}
  and for $\dim f=0$
  \begin{gather}
    W(f) = \bigl\{\phi(p)= p(x_i) \big|
    \;f=\{x_i\}
    \bigr\}.
  \end{gather}
  For $r \le \dim f$ there holds
  \begin{gather}
    W(f) = \{0\}.
  \end{gather}
\end{Definition}

\begin{Lemma}{pr-geometric}
  There holds
  \begin{gather}
    \P_r(\cell) = \bigoplus_{f\in\Delta(\cell)}V(f),
    \qquad
    \P_r(\cell)^* = \bigoplus_{f\in\Delta(\cell)}W(f).
  \end{gather}
\end{Lemma}

\begin{proof}
  We begin to show that
  \begin{gather}
    \P_r(\cell) = \bigoplus_{f\in\Delta(\cell)}V(f).
  \end{gather}
  First, we note that for any $f\in\Delta(\cell)$ every function in $V(f)$
  is also in $\P_r(\cell)$.  For $\dim f=0$, that is, $f=\{x_i\}$ for some
  vertex $x_i$, the only homogeneous polynomial of order $r$ is $\lambda_i^r$.

  For $\dim f > 0$ we have by the first statement of
  \slideref{Lemma}{subsimplex-polynomials}, that the spaces $V(f)$
  where $f$ is a vertex are disjoint. By the second statement of the
  same lemma, the spaces $V(f)$ for all $f$ with equal dimension are
  disjoint. Therefore, the sum
  \begin{gather}
    V_\sdim(\cell) = \sum_{k=0}^{\sdim-1} \sum_{f\in \Delta_k(\cell)} V(f),
  \end{gather}
  is direct. But, $V(\cell) \cap V_\sdim(\cell) = \{0\}$, since all elements in
  $V(\cell)$ contain a bubble function factor. Therefore,
  \begin{gather}
    \bigoplus_{f\in\Delta(\cell)}V(f) \subset \P_r(\cell).
  \end{gather}
  We conclude by showing that dimensions on both sides are equal.
  On the right, we use
  \begin{gather}
    \dim \P_r(\R^\sdim) = \binom{r+\sdim}{r} = \frac{(r+\sdim)!}{\sdim!r!}
  \end{gather}
  On the left, we have
  \begin{multline}
    \dim \bigoplus_{f\in\Delta(\cell)}V(f) =
    \sum_{k=0}^{\sdim} \binom{\sdim+1}{k+1} \binom{r+k}{k}
    \\
    = \frac{(d+1)!}{r!}
    \sum_{k=0}^{\sdim} \frac{1}{(k+1)!(d-k)!} \frac{(r+k)!}{k!}.
  \end{multline}

% Finish this!

  It remains to show the decomposition for $\P_r(\cell)^*$. To this
  end, we first notice that for any $f\in\Delta(\cell)$ there holds
  $\dim W(f) = \dim V(f)$ by their definition. Furthermore, for
  $p\in V(f)$ there holds
  \begin{gather}
    \Bigl(\phi(p) = 0 \quad\forall \phi\in W(f)\Bigr)
    \quad\Rightarrow\quad
    p=0.
  \end{gather}
  Thus, for $p\in \P_r(\cell)$ there holds
  \begin{gather}
    \Bigl(\phi(p) = 0 \quad\forall \phi\in \sum W(f)\Bigr)
    \quad\Rightarrow\quad
    p=0.
  \end{gather}
  Consequently,
  \begin{gather}
    \P_r(\cell)^* = \sum W(f).
  \end{gather}
  since we have already proven that
  \begin{gather}
    \dim \P_r(\cell)^* = \sum \dim W(f),
  \end{gather}
  the sum on the right must be direct.
\end{proof}

\subsubsection{Results for $\P_r\Lambda^k$ and proxy fields}

\begin{Lemma}{pr-lambda-dimension-decomposition}
  There holds
  \begin{align}
    \dim \P_r\Lambda^k(\cell)
    &= \sum_{f\in\Delta(\cell)}
      \dim \P_{r+k-\dim f-1}^+\Lambda^{\dim f-k}(f) \\
    \dim \P_r^+\Lambda^k(\cell)
    &= \sum_{f\in\Delta(\cell)}
      \dim \P_{r+k-\dim f}\Lambda^{\dim f-k}(f)
  \end{align}
\end{Lemma}

% \begin{Notation}{volume-form}
%   The \define{volume form} of a $k$-dimensional subsimplex
%   $f\in\Delta(\cell)$ is defined by the relation
%   \begin{gather}
%     \abs{f} = \int_f \vol_f.
%   \end{gather}
%   Using the fact that $n! \abs{T} = \vol_T(t_1,\dots,t_n)$ we see
%   \begin{gather}
%     \dlambda_{\sigma_1}\wedge\dots\wedge\dlambda_{\sigma_k}
%     = \pm \frac1{k! \abs{f}} \vol_f.
%   \end{gather}
% \end{Notation}

\begin{Lemma}{afw06-4-7}
  Let $\omega \in \overset{\circ}{\P}_r \Lambda^k(\cell)$ and assume
  \begin{gather}
    \int_\cell\omega\wedge\eta = 0,
    \qquad\forall \eta\in\P_{r-n+k-1}^+\Lambda^{n-k}(\cell).
  \end{gather}
  Then, $\omega = 0$.
\end{Lemma}

\begin{proof}
  The proof consists of Lemmas 4.5 to 4.7 in
  \cite{ArnoldFalkWinther06acta}. It skilfully exploits representations of
  polynomial forms with zero traces by barycentric coordinates, but is
  rather technical.
\end{proof}

\begin{Theorem}{pr-lambda-unisolvent}
  Let $r\ge 1$ and $0\le k \le n$. For $f\in \Delta(\cell)$, define
  subspaces of $W(f) \subset \P_r\Lambda^k(\cell)^*$ by
  \begin{gather}
    W(f) = \left\{
      \nodal_{f,_\eta}(\omega) = \int_f \gamma_f \omega\wedge\eta,
      \;\middle\vert\;
       \eta \in \P_{r+k-\dim f-1}^+\Lambda^{\dim f-k}(f)
      \right\}.
  \end{gather}
  Then, if $\omega \in \P_r\Lambda^k(\cell)$ satisfies
  \begin{gather}
    \nodal_{f,\eta}(\omega) = 0
    \qquad \forall f\in \Delta(\cell)
    \quad\forall \nodal_{f,\eta} \in W(f),
  \end{gather}
  then $\omega = 0$. Thus, the space $\P_r\Lambda^k(\cell)$ together
  with the node functionals $\nodal_{f,\eta}$ forms a unisolvent
  finite element.  Furthermore, $\gamma_f\omega$ for any subsimplex
  $f\in\Delta(\cell)$ is uniquely determined by the degrees of freedom
  in $\bigoplus_g W(g)$ where $g\in\Delta(f)$.
\end{Theorem}

\begin{proof}
  First we note, that by
  \slideref{Lemma}{pr-lambda-dimension-decomposition} the dimensions
  of the spaces $W(f)$ add up to the dimension of
  $\P_r\Lambda^k(\cell)$. Thus, if we can show the first statement, we
  have shown that the sum of the spaces $W(f)$ is a direct sum, and
  thus unisolvence is obtained by the usual argument.

  Let now $f\in\Delta_k(\cell)$. Then, $\gamma\omega$ vanishes on
  $\d f$ as a $k$-form on a simplex of dimension $k-1$. Thus, we have
  $\gamma_f\omega\in \overset{\circ}{\P}_r\Lambda^k(f)$. Since
  furthermore by the assumption
  \begin{gather}
    \int_f \gamma_f\omega\wedge \eta = 0,
    \qquad \forall \eta \in \P_{r-1}^+\Lambda^0(f),
  \end{gather}
  \slideref{Lemma}{afw06-4-7} yields $\gamma_f\omega = 0$, which holds
  for all $f\in\Delta_k(\cell)$.

  Let now $f\in \Delta_{k+1}(\cell)$. Then, by the result of the
  previous argument,
  $\gamma_f\omega\in \overset{\circ}{\P}_r\Lambda^k(f)$, and by the
  argument itself $\gamma_f \omega = 0$. In particular,
  $\gamma_f\omega$ is uniquely determined at this point of the
  construction process, which proves the last statement of the theorem.

  We can now do induction by the dimension of $f$ until we reach
  $\Delta_n(\cell) = \{\cell\}$ to obtain the result.
\end{proof}

\begin{remark}
  Note that the space $\P_s^+\Lambda^k$ vanishes if $s<0$ or
  $k<0$. Therefore, the spaces $W(f)$ are nontrivial only if
  \begin{gather}
    k \le \dim f \le r+k-1.
  \end{gather}
  Comparing this to \slideref{Example}{h1-moment-dofs} for
  $H^1 = H\Lambda^0$, we see that one-dimensional subsimplices carry
  degrees of freedom for $r\ge 2$ and two-dimensional for $r\ge 3$.
\end{remark}

\begin{Theorem}{decomp-pr}
  Let $k,r\ge 1$. Then, $\P_r(\cell)$ admits a geometric decomposition
  \begin{gather}
    \P_r\Lambda^k(\cell) = \bigoplus_{f\in\Delta(\cell)} V(f),
  \end{gather}
  where
  \begin{gather}
    V(f) \cong W(f) \cong
    \begin{cases}
      0 & \dim f < k\\
      \P_{r+k-\dim f-1}^+ \Lambda^{\dim f-k}(f) &\text{else}\\
      0 & \dim f \le r+k.
    \end{cases}
  \end{gather}
\end{Theorem}

\begin{Example}{bdm-complex-decomp}
  \begin{gather}
    \begin{array}{c|cc|cc}
      \dim f
      & \P_r\Lambda^1 & N^{2e}_r
      & \P_r\Lambda^2 & BDM_r \\\hline
      3 & \P_{r-3}^+\Lambda^2 & RT_{r-2} & \P_{r-1}^+\Lambda^1 & N^{1e}_{r-1} \\
      2 & \P_{r-2}^+\Lambda^1 & RT_{r-1} & \P_{r}^+\Lambda^0 & \P_r \\
      1 & \P_{r-1}^+\Lambda^0 & \P_r & --& --
%      0 & \R & \R & -- & --
    \end{array}
  \end{gather}
  The spaces $\P_r\Lambda^k$ and their proxy fields and the spaces
  $W(f)$ of degrees of freedom.
  
  \begin{itemize}
  \item [$N^{2e}$] ($\Hcurl$) Nedelec 2nd family edge element
  \item [$BDM$] ($\Hdiv$) Brezzi-Douglas-Marini (also Nedelec 2nd face in 3D)
  \item [$N^{1e}$] ($\Hcurl$) Nedelec 1st family edge element
  \end{itemize}
\end{Example}

\subsubsection{Results for $\P_r^+\Lambda^k$ and proxy fields}


\begin{Theorem}{decomp-pr-plus}
  Let $k,r\ge 1$. Then, $\P_r^+(\cell)$ admits a geometric decomposition
  \begin{gather}
    \P_r^+\Lambda^k(\cell) = \bigoplus_{f\in\Delta(\cell)} V(f),
    \qquad
    \P_r^+\Lambda^k(\cell)^* = \bigoplus_{f\in\Delta(\cell)} W(f),
  \end{gather}
  where
  \begin{align}
    V(f) \cong
    \begin{cases}
      0 & \dim f < k\\
      \P_{r+k-\dim f} \Lambda^{\dim f-k}(f) &\text{else}\\
      0 & \dim f > r+k.
    \end{cases}
    \\
    W(f) \cong
    \begin{cases}
      0 & \dim f < k\\
      \P_{r+k-\dim f} \Lambda^{\dim f-k}(f) &\text{else}\\
      0 & \dim f > r+k.
    \end{cases}
  \end{align}
\end{Theorem}

\begin{Example}{rt-complex-decomp}
  \begin{gather}
    \begin{array}{c|cccccc}
      \dim f
      & \P_r^+\Lambda^0 & \P_{r+1}
      & \P_r^+\Lambda^1 & N^{1e}_r
      & \P_r^+\Lambda^2 & RT_r \\\hline
      3 & \P_{r-3}\Lambda^3 & \P_{r-3} & \P_{r-2}\Lambda^2 & BDM_{r-2} & \P_{r-1}\Lambda^1 & N^{2e}_{r-1} \\
      2 & \P_{r-2}\Lambda^2 & \P_{r-2} & \P_{r-1}\Lambda^1 & BDM_{r-1} & \P_{r}\Lambda^0 & \P_r\\
      1 & \P_{r-1}\Lambda^1 & \P_{r-1} & \P_{r}\Lambda^0   & \P_r  & -- & -- \\
      0 & \R & \R & -- & -- & --& --
    \end{array}
  \end{gather}
  Geometric decomposition of $\P_r^+\Lambda^k$ and their spaces of degrees of freedom.
  \begin{itemize}
  \item [$N^{1e}$] ($\Hcurl$) Nedelec 1st family edge element
  \item [$RT$] ($\Hdiv$) Raviart-Thomas (also Nedelec 1st face in 3D)
  \item [$N^{2e}$] ($\Hcurl$) Nedelec 2nd family edge element
  \end{itemize}
\end{Example}

\begin{Definition}{k-form-interpolation}
  By choosing a basis for each of the spaces $W(f)$ with
  $f\in\Delta(\cell)$, we obtain a finite number of node functionals
  $\nodal_{f,i}$ which in turn induces bases $\{\omega_{f,i}\}$ for $V(f)$ by
  duality.
  The \define{canonical interpolation} operators
  \begin{gather}
    \Pi_k\colon C\Lambda^k(\cell) \to \P_r\Lambda^k(\cell),
    \qquad
    \Pi_k^+\colon C\Lambda^k(\cell) \to \P_r^+\Lambda^k(\cell),
  \end{gather}
  are defined such that for all $\omega\in C\Lambda^k(\cell)$
  \begin{gather}
    \Pi_k \omega, \Pi_k^+\omega = \sum_{f\in\Delta(\cell)}
  \sum_{i=1}^{\dim W(f)} \nodal_{f,i}(\omega) \omega_{f,i},
\end{gather}
where the inner sum is determined by the spaces in \slideref{Theorem}{pr-lambda-unisolvent} and \slideref{Theorem}{decomp-pr-plus}, respectively.
\end{Definition}

\begin{Theorem}{canonical-commute}
  The diagram
  \begin{center}
    \begin{tikzcd}
      \Lambda^k(\cell)
      \arrow[r,"\diffd"]
      \arrow[d,"\Pi"]
      & \Lambda^{k+1}(\cell)
      \arrow[d,"\Pi"]
      \\
      \P\Lambda^k(\cell) \arrow[r,"\diffd"]
      &\P\Lambda^{k+1}(\cell)
    \end{tikzcd}
  \end{center}
  commutes for all combinations of the spaces
  $\P\Lambda^k \in \{\P_{r+1}\Lambda^k,\P_r^+\Lambda^k\}$ and
  $\P\Lambda^{k+1}\in \{\P_r\Lambda^{k+1}, \P_r^+\Lambda^{k+1}\}$.
  Namely, the canonical interpolation operators commute with the
  exterior derivative,
  \begin{gather}
    \Pi(d\omega) = d(\Pi\omega).
  \end{gather}
\end{Theorem}


\section{The complex of tensor product polynomials}

\begin{intro}
  The other multilinear map we know is the tensor product, which we
  used to define finite elements on squares and cubes. This section is
  now concerned with the interplay of alternating and differential
  forms and tensor products. In particular, we are looking into the
  construction of $k$-forms on $\R^\sdim$ as tensor products of
  one-dimensional forms.

  We will avoid the functional analysis of tensor products of Hilbert
  spaces and refer the readers
  to~\cite{ReedSimon80,Hackbusch14,Hackbusch19}.

  We focus instead on the finite dimensional construction of
  polynomial $k$-forms on $\R^\sdim$ by tensor products of
  one-dimensional forms.
\end{intro}

\begin{Definition}{pr-complex-1d}
  The one-dimensional de Rham complex on the interval $I = [0,1]$ and
  its polynomial subcomplex are
  \begin{gather}
    \begin{CD}
    \R
    @>{\subset}>>
    H^1(I) = H\Lambda^0(I)
    @>{\tfrac{\diffd}{\dx}}>>
    L^2(I) = H\Lambda^1(I)
    @>>> 0
    \\
    @.
    @A{\subset}AA
    @A{\subset}AA
    \\
    \R
    @>{\subset}>>
    \P_{r+1} = \P_{r+1}\Lambda^0
    @>{\tfrac{\diffd}{\dx}}>>
    \P_r = \P_r\Lambda^1
    @>>> 0
    \end{CD}
  \end{gather}
  We also use the simplified notation $\P\Lambda^k$, indicating that
  the polynomial degrees are chosen such that $r+k$ is constant.
\end{Definition}

\begin{remark}
  The polynomials sequence is exact due to \slideref{Theorem}{polynomial-exact}.
\end{remark}

\begin{Definition}{pr-1d-basis}
  The node functionals for $\P_{r+1}\Lambda^0$ are
  \begin{gather}
    \begin{matrix}
      \nodal_{0,0}^0(p) = p(1) - p(0),\\
      \nodal_{0,1}^0(p) = p(1) + p(0),
    \end{matrix}\qquad
    \nodal_{1.q}^0(p) = \int_{I} p' q \dx,\quad\forall q\in\nicefrac{\P_{r-1}}{\R}.
  \end{gather}
  The degrees of freedom for $\P_{r}\Lambda^1$ are
  \begin{gather}
    \nodal_{1,q}^1(p) = \int_{I} p q \dx,\quad\forall q\in\P_{r}.
  \end{gather}
  Bases for the shape function spaces are defined by duality.
\end{Definition}

\begin{remark}
  The degrees of freedom for $\P_{r+1}\Lambda^0$ are chosen such that
  the finite element function on a subdivision of $I$ is continuous,
  thus in $H^1$. This is true, even if the function values at the end
  points only appear in linear combinations.

  For $\P_{r}\Lambda^1$, we do not require continuity
  and thus only need interior degrees of freedom.
\end{remark}

\begin{Lemma}{commute-nodal}
  Let finite element $k$-forms $\P\Lambda^k$ on the cell $\cell\subset\R^\sdim$ be
  defined by node values $\nodal_i^k$ for $i=1,\dots,m_k$. Let the
  basis for the shape function spaces $\{\phi^k_i\}$ and
  $\{\phi^{k+1}_i\}$, respectively defined by duality.  Assume for $r=\dim\range d$
  \begin{gather}
    \label{eq:commute-basis}
      \begin{aligned}
          d\phi^k_i &= \phi^{k+1}_i & \qquad i&=1,\dots,r,\\
          d\phi^k_i &= 0 & i&=r+1,\dots,\dim\P\Lambda^{k}(\cell).
      \end{aligned}
  \end{gather}
Moreover, assume for any $\omega\in \Lambda^k(\cell)$
  \begin{gather}
    \label{eq:commute-nodal}
      \begin{aligned}
      \nodal^{k+1}_i(d \omega) &= \nodal^k_i(\omega)& \qquad i&=1,\dots,r,\\
      \nodal^{k+1}_i(d \omega) &= 0 & i&=r+1,\dots,\dim\P\Lambda^{k+1}.
      \end{aligned}
  \end{gather}
  Then, the \putindex{canonical interpolation} operators $\Pi$ commute
  with the exterior derivative, namely, there holds:
  \begin{gather}
    d_k \Pi_k \omega = \Pi_{k+1} d_k \omega.
  \end{gather}
\end{Lemma}

\begin{proof}
  By linearity, we have
  \begin{gather}
      d\Pi_k \omega = d\left(\sum_{i=1}^{\dim\P\Lambda^k}\nodal^k_i(\omega) \phi^k_i\right)
      = \sum_{i=1}^{r}\nodal^k_i(\omega) d \phi^k_i
      = \sum_{i=1}^{r}\nodal^k_i(\omega) \phi^{k+1}_i.
  \end{gather}
  On the other hand,
  \begin{gather}
      \Pi_{k+1} d \omega = \sum_{i=1}^{\dim\P\Lambda^{k+1}}\nodal^{k+1}_i(d \omega) \phi^{k+1}_i.
  \end{gather}
  Employing~\eqref{eq:commute-nodal} concludes the proof.
\end{proof}


\begin{Lemma}{pr-1d-commute}
  Let
  \begin{gather}
    \begin{split}
      \Pi^0_{r+1}\colon \Lambda^0(I) &\to \P_{r+1}\Lambda^0(I)\\
      \Pi^1_{r}\colon \Lambda^1(I) &\to \P_{r}\Lambda^1(I).
    \end{split}
  \end{gather}
 Then, the diagram
  \begin{center}
    \begin{tikzcd}
      \Lambda^0(I)
      \arrow[r,"\diffd"]
      \arrow[d,"\Pi"]
      & \Lambda^{1}(I)
      \arrow[d,"\Pi"]
      \\
      \P\Lambda^0(I) \arrow[r,"\diffd"]
      &\P\Lambda^{1}(I)
    \end{tikzcd}
  \end{center}
  commutes.
\end{Lemma}

\begin{proof}
  We prove that the assumptions of \slideref{Lemma}{commute-nodal} are
  fulfilled. To this end, we have to choose a particular basis for the
  node functional spaces. Let this be the \putindex{Legendre
    polynomials} of degrees zero to $r$ for $\nodal_{1,q}^1$. For
  $\nodal_{1,q}^0$, we choose
  \begin{gather}
    q_i = \plegendre_i,\qquad i=1,\dots, r-1
  \end{gather}

  We note that for Legendre polynomials on $[0,1]$ there holds
  \begin{gather}
    2(2m+1) \int_0^x \plegendre_m(t)\dt = \plegendre_{m+1}(x) - \plegendre_{m-1}(x).
  \end{gather}
  The basis functions for $\P_r\Lambda^0$ are
  \begin{gather}
    \begin{split}
      \phi_{0,0}^0 &= x-\tfrac12,\\
      \phi_{1,q_i} &= \tfrac{1}{2(2i+1)}(\plegendre_{i+1} - \plegendre_{i-1}),
      \qquad i=1,\dots,n-1,\\
      \phi_{0,1}^0 &= \tfrac12.
    \end{split}
  \end{gather}
  Clearly, the basis functions for $\P_r\Lambda^1$ are the Legendre polynomials themselves.
  Then, we have for $i=1,\dots,r-1$
  \begin{gather}
    d \phi_{1,q_i}^0(x) = \frac{\diffd}{\diffd x} \phi_{1,q_i}(x)
    = \plegendre_i(x)
    = \phi_{1,i}(x).
  \end{gather}
  Furthermore, $d\phi_{0,1}^0(x)=0$ and
  $d\phi_{0,0}^0(x)=1 = \phi_{1,0}^1(x)$. It thus remains to verify
  that the node functionals commute, which is obvious for
  $\nodal_{1,q_i}^0$ for $i=1,\dots,r-1$. For the remaining one, this
  is due to the fundamantal theorem of calculus.
\end{proof}

\begin{Definition}{k-form-tensor-product}
  Let $\omega\in \Lambda^k$ and $\eta\in\Lambda^\ell$, then their
  tensor product $\omega\otimes\eta\in\Lambda^{k+\ell}$ is defined through
  their basis representations
  \begin{gather}
    \omega = \sum_{\sigma\in\Sigma(k,n)} a_\sigma \dx_\sigma,
    \qquad
    \eta = \sum_{\tau\in\Sigma(\ell,n)} b_\tau \dx_\tau,
  \end{gather}
  as
  \begin{gather}
    \omega\otimes\eta = \sum_{\sigma\in\Sigma(k,n)}\sum_{\tau\in\Sigma(\ell,n)}
    (a_\sigma\otimes b_\tau) \dx_\sigma \wedge \dx_\tau.
  \end{gather}
  The space of $n$-fold tensor product polynomial $k$-forms is
  \begin{gather}
    \left(\P\Lambda^{\otimes n}\right)^k
    = \bigoplus_{\sigma_\in\Sigma(k,n)}
    \P\Lambda^{\chi_1} \otimes \dots \otimes \P\Lambda^{\chi_n},
  \end{gather}
  where $\chi= \chi_\sigma$ is the \putindex{characteristic vector} of $\sigma$.
\end{Definition}

\begin{Lemma}{tensor-product-exterior-derivative}
  The \putindex{exterior derviative} of a tensor product $k$-form
  $\omega\otimes\eta$ with $\omega\in\Lambda^i$,
  $\eta\in\Lambda^j$, and $i+j=k$ is the $(k+1)$-form obtained by the
  \putindex{Leibnitz rule}
  \begin{gather}
    d_k(\omega\otimes\eta) = d_i\omega \otimes \eta +  (-1)^i \omega \otimes d_j\eta.
  \end{gather}
\end{Lemma}

\begin{example}
  Let $\omega_1^0 = p_1 = p_1(x_1)$ and $\omega_2 = p_2(x_2)$.
  The two-dimensional tensor product $0$-form $q=p_1\otimes p_2$ on $[0,1]^2$ is
  \begin{gather}
    q(\vx) = \omega_1^0\otimes \omega_2^0(\vx) = p_1(x_1)p_2(x_2).
  \end{gather}
  Its exterior derivative is
  \begin{gather}
    d q(\vx) = d\omega_1^0 \tensor \omega_2^0 - \omega_1^0 \tensor d\omega_2^0
    = p_1'(x_1) p_2(x_2) \dx_1 - p_1(x_1) p_2'(x_2) \dx_2.
  \end{gather}
  Note that $d^2 = 0$ since $\dx_i\wedge \dx_i=0$.
  
  Let $\omega_1^1 = p_1(x_1)\dx_1$ and $\omega_2^1 = p_2(x_2)\dx_2$. Then, the
  possible tensor product 1-forms on $[0,1]^2$ are
  \begin{align}
    \omega_1^1 \otimes \omega_2^0 (\vx) &= p_1(x_1)p_2(x_2) \dx_1 \\
    \omega_1^0 \otimes \omega_2^1 (\vx) &= p_1(x_1)p_2(x_2) \dx_2,
  \end{align}
  Since forms of index 1 are one polynomial degree lower, we see that
  the tensor product function coefficient in front of $\dx_i$ is one
  degree lower in $x_i$ than in $x_j$ for $j\neq i$.

  Finally, we obtain the 2-form
  \begin{gather}
    \omega_1^1 \otimes \omega_2^1 (\vx) = p_1(x_1)p_2(x_2) \dx_1\wedge\dx_2.    
  \end{gather}
\end{example}

\begin{Lemma}{tensor-product-complex-exact}
  The tensor product complex
  \begin{gather}
    \R
    \overset{\subset}{\longrightarrow}
    \left(\P_r\Lambda^{\otimes n}\right)^0
    \overset{d}{\longrightarrow}    
    \left(\P_r\Lambda^{\otimes n}\right)^1
    \overset{d}{\longrightarrow}
    \dots
    \overset{d}{\longrightarrow}    
    \left(\P_r\Lambda^{\otimes n}\right)^n
    \longrightarrow 0
  \end{gather}
  is exact.
\end{Lemma}

\begin{proof}
  By the Leibnitz rule, $\omega\times\eta \in \ker d$ if
  $\omega\in\ker d$ and $\eta\in\ker d$, individually. Thus,
\end{proof}


\begin{Lemma}{qr-complex}
  The coefficient functions for $(\P_r^{\otimes 3})^k$ and thus the
  polynomial proxy fields are from the spaces
  \begin{gather}
    \R
    \overset{\subset}{\longrightarrow} \Q_{r+1}
    \overset{\nabla}{\longrightarrow}
    \begin{pmatrix}
      \Q_{r,r+1,r+1}\\\Q_{r+1,r,r+1}\\\Q_{r+1,r+1,r}
    \end{pmatrix}
    \overset{\curl}{\longrightarrow}
    \begin{pmatrix}
      \Q_{r+1,r,r}\\\Q_{r,r+1,r}\\\Q_{r,r,r+1}
    \end{pmatrix}
    \overset{\div}{\longrightarrow}
    \Q_r
    \longrightarrow 0,    
  \end{gather}
\end{Lemma}


\begin{Definition}{tensor-product-node-values}
  The 
\end{Definition}


% \begin{Lemma}{qr-complex}
%   Tensor product polynomials form the exact sequence
%   \begin{gather}
%     \R
%     \overset{\subset}{\longrightarrow} \Q_{r+1}
%     \overset{\nabla}{\longrightarrow}
%     \begin{pmatrix}
%       \Q_{r,r+1,r+1}\\\Q_{r+1,r,r+1}\\\Q_{r+1,r+1,r}
%     \end{pmatrix}
%     \overset{\curl}{\longrightarrow}
%     \begin{pmatrix}
%       \Q_{r+1,r,r}\\\Q_{r,r+1,r}\\\Q_{r,r,r+1}
%     \end{pmatrix}
%     \overset{\div}{\longrightarrow}
%     \Q_r
%     \longrightarrow 0,    
%   \end{gather}
% \end{Lemma}

% \begin{proof}
%   First, we show that the differential operators map into the right
%   spaces. Let $q\in\Q_{r+1}$ such that
%   $q(x) = q_1(x_1) q_2(x_2)q_3(x_3)$ with $q_i\in\Q_{r+1}$. Then
%   \begin{gather}
%     \nabla q =
%     \begin{pmatrix}
%       q_1'q_2q_3\\q_1q_2'q_3\\q_1q_2q_3'
%     \end{pmatrix}
%     \in
%     \begin{pmatrix}
%       \Q_{r,r+1,r+1}\\\Q_{r+1,r,r+1}\\\Q_{r+1,r+1,r}
%     \end{pmatrix}.
%   \end{gather}
%   Similarly, we can compute this directly for $\curl$ and
%   $\div$. Since polynomials are differentiable, we have $d^2=0$. It
%   remains to show that the sequence is exact, which we will prove at
%   the example of $\Hcurl$. Let $u\in \ker{\curl}$,
%   \begin{gather}
%     u=
%     \begin{pmatrix}
%       \phi_1\phi_2\phi_3\\
%       \psi_1\psi_2\psi_3\\
%       \pi_1\pi_2\pi_3
%     \end{pmatrix},
%     \qquad
%     0 = \curl u =
%     \begin{pmatrix}
%       \pi_1\pi_2'\pi_3 - \psi_1\psi_2\psi_3'\\
%       \phi_1\phi_2\phi_3' - \pi_1'\pi_2\pi_3\\
%       \psi_1'\psi_2\psi_3 - \phi_1\phi_2'\phi_3
%     \end{pmatrix},
%   \end{gather}
%   where each polynomial with index $i$ only depends on
%   $x_i$. Furthermore, $\phi_1,\psi_2,\pi_3\in\P_r$ and all other in
%   $\P_{r+1}$.  From the continuous de Rham complex, we know that there
%   is a function $p$ such that $u=\nabla p$. It remains to show that
%   $p\in\Q_{r+1}$. There holds
%   \begin{gather}
%     \d_1 p = \phi_1\phi_2\phi_3.
%   \end{gather}
%   Thus, we make the ansatz
%   \begin{gather}
%     p = \Phi_1\phi_2\phi_3,
%   \end{gather}
%   where $\Phi_1\in\P_{r+1}$ is the antiderivative of $\phi_1$. Thus,
%   $p\in\Q_{r+1}$ It remains to show that this ansatz is consistent
%   with the other two derivatives, thus,
%   \begin{gather}
%     p = \Phi_1\phi_2\phi_3 = \psi_1\Psi_2\psi_3 = \pi_1\pi_2\Pi_3.
%   \end{gather}
%   Integrating the first component of $\curl u$ with respect to $x_2$
%   and $x_3$, we obtain
%   \begin{gather}
%     \pi_1\pi_2\Pi_3 = \psi_1\Psi_2\psi_3.
%   \end{gather}
%   Doing the same with the second component, we see indeed that the
%   function $p$ is consistently defined and thus $u=\nabla p$.
% \end{proof}
% \begin{Definition}{tensor-dofs}
%   Let
%   \begin{gather}
%     \nodal_{0,0} p = p(0),\qquad \nodal_{0,1} p = p(1), \qquad
%     \nodal_{1,i} = \int_I p q_i\dx,
%   \end{gather}
%   be the degrees of freedom of a one-dimensional element where the
%   $q_i$ are a basis for $\P_{r-1}$ and $\P_r$ in case of
%   $\P_{r+1}\Lambda^0$ and $\P_{r}\Lambda^1$, respectively. Then, the
%   tensor product of these degrees of freedom applied to the function
%   $p(x_1,x_2) = p_1(x_1)p_2(x_2)$ is defined as
%   \begin{gather}
%     \begin{split}
%       \nodal_{0,i}\otimes\nodal_{0,j}(p) &= p_1(\chi_i) p_2(\chi_j),\\
%       \nodal_{0,i}\otimes\nodal_{1,j}(p) &= \int_I p_1(\chi_i)  p_2(x)q_j(x)\dx,\\
%       \nodal_{1,i}\otimes\nodal_{0,j}(p) &= \int_I p_1(x)q_j(x) p_2(\chi_j) \dx,\\
%       \nodal_{1,i}\otimes\nodal_{1,j}(p) &= \iint_I p_1(x_1)
%       p_2(x_2) \dx_1\dx_2.
%     \end{split}
%   \end{gather}
% \end{Definition}

% \begin{Lemma}{2d-tensor-rt-nedelec}
%   For the two elements
%   \begin{gather}
%     RT_r =
%     \begin{pmatrix}
%       \P_{r+1}\Lambda^0 \otimes \P_{r}\Lambda^1\\
%       \P_{r}\Lambda^1 \otimes \P_{r+1}\Lambda^0
%     \end{pmatrix},
%     \qquad
%     N^{1e}_r =
%     \begin{pmatrix}
%       \P_{r}\Lambda^1 \otimes \P_{r+1}\Lambda^0\\
%       \P_{r+1}\Lambda^0 \otimes \P_{r}\Lambda^1      
%     \end{pmatrix},
%   \end{gather}
%   the tensorized degrees of freedom uniquely determine the normal and
%   tangential components, respectively, on the face of a square.

%   Both elements with their tensor degrees of freedom are unisolvent.
% \end{Lemma}

% \begin{proof}
%   The elements are unisolvent by the following argument: Let
%   $\{\phi_i\}$ and $\{\psi_j\}$ be the basis dual to the degrees of
%   freedom of $\P_{r+1}\Lambda^0$ or $\P_r \Lambda^1$,
%   respectively. Then, (renumbering the degrees of freedom)
%   \begin{gather}
%     \nodal_k\otimes\nodal_l(\phi_i\otimes\psi_j) = \delta_{ik}\delta_{jl}.
%   \end{gather}
%   Thus, the mapping between tensorized degrees of freedom and
%   tensorized basis functions is one-to-one.

%   For the Raviart-Thomas element $RT_r$, the normal component is
%   $\P_r\Lambda^1$. Take for instance the face $x_1 = 0$. Then,
%   the degrees of freedom associated to this face are
%   \begin{gather}
%     \nodal_{0,0} \otimes \nodal{1,q} p
%     = \int_I p_1(1) p_2(x_2) q(x_2) \dx_2
%     \quad\forall q\in \P_r.
%   \end{gather}
%   Since the trace of $p$ on this face is $p_2 \in \P_r$, this
%   polynomial is uniquely determined by the degree of freedom.

%   The argument for the Nedelec edge element $N^{1e}_r$ follows by
%   exchanging tangential and normal component.
% \end{proof}

% \begin{remark}
%   The construction extends to the three-dimensional products.
% \end{remark}



%%% Local Variables:
%%% mode: latex
%%% TeX-master: "main"
%%% End:
