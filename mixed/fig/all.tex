\documentclass{article}
\usepackage{tikz,tikzscale}
\usetikzlibrary{calc}
\usetikzlibrary{math}
\usetikzlibrary{arrows.meta}
\usetikzlibrary{shapes.geometric}
\tikzset{shape veloxy/.style={color=black,draw,fill=red,thick}}
\tikzset{shape pressure/.style={color=black,draw,fill=cyan,thick}}


\newcommand{\showtikz}[2]{\begin{minipage}{#1\textwidth}
    \begin{center}
      \includegraphics[width=\textwidth]{./#2.tikz}
      \\
      #2
    \end{center}
  \end{minipage}
}

\begin{document}

\includegraphics[width=.24\textwidth]{./triangle.tikz}
\includegraphics[width=.24\textwidth]{./rectangle.tikz}
\includegraphics[width=.24\textwidth]{./p-mini-v.tikz}
\includegraphics[width=.24\textwidth]{./q-mini-v.tikz}

\showtikz{.24}{p0-p}
\showtikz{.24}{p1-p}
\showtikz{.24}{p2-p}
\showtikz{.24}{p3-p}

\showtikz{.24}{p2-v}
\showtikz{.24}{p3-v}

\showtikz{.24}{q0-p}
\showtikz{.24}{q1-p}
\showtikz{.24}{q2-p}

\showtikz{.24}{q2-v}
\showtikz{.24}{q-p1-p}
\showtikz{.24}{q3-v}
\showtikz{.24}{q-p2-p}

\showtikz{.24}{bdm1-tri}
\showtikz{.24}{bdm2-tri}
\showtikz{.24}{dgp1-p-tri}
\showtikz{.24}{dgp2-p-tri}

\showtikz{.24}{rt0-tri}
\showtikz{.24}{rt1-tri}
\showtikz{.24}{rt2-tri}

\showtikz{.24}{rt0-quad}
\showtikz{.24}{rt1-quad}
\showtikz{.24}{rt2-quad}

\includegraphics[width=.4\textwidth]{./patch1.tikz}
\includegraphics[width=.4\textwidth]{./patch2.tikz}

\includegraphics[width=.8\textwidth]{./p1-p1-1d.tikz}

\includegraphics[width=.2\textwidth]{./q-p1-p.tikz}
\includegraphics[width=.3\textwidth]{./q-p1-p.tikz}
\includegraphics[width=.4\textwidth]{./q-p1-p.tikz}

\includegraphics[width=.2\textwidth]{./q-p2-p.tikz}
\includegraphics[width=.3\textwidth]{./q-p2-p.tikz}
\includegraphics[width=.4\textwidth]{./q-p2-p.tikz}

\fbox{\includegraphics[width=.2\textwidth]{./rt5-tri.tikz}}
\fbox{\includegraphics[width=.3\textwidth]{./rt5-tri.tikz}}
\fbox{\includegraphics[width=.4\textwidth]{./rt5-tri.tikz}}

\end{document}

%%% Local Variables:
%%% mode: latex
%%% TeX-master: t
%%% End:
