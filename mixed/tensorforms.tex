\begin{intro}
  From one dimensional polynomial space $\P_r(\R)$, we construct the
  tensor product space $\Q_r$ by the construction
  \begin{gather}
    \begin{split}
      \Q_r(\R^d) & = \bigotimes_{i=1}^d \P_r(\R^1),
      = \P_r(\R^1) \otimes\dots\otimes\P_r(\R^1),\\
      q &= p_1 \otimes \dots \otimes p_d,\\
      q(x_1,\dots,x_d) &= \prod_{i=1}^d p_1(x_1)\cdots p_d(x_d).
    \end{split}
  \end{gather}
  We can even allow for different orders in different dimenions,
  obtaining $\Q_{r_1,\dots,r_d}$. Examining the product, we see that
  $\R^d$ was decomposed into the direct sum of one-dimensional spaces
  such that
  \begin{gather}
    \begin{split}
      \R^d &= \bigtimes_{i=1}^d \R,\\
      x &= \sum_{i=1}^d \pi_i x,\\
      q(x) &= \prod_{i=1}^d p_i(\pi_i x),
    \end{split}
  \end{gather}
  where $\pi_i$ is the projection to the $i$th component of the vector.

  We can generalize this construction in an obvious way to the direct
  sum of spaces of higher dimension, say $\R^d\oplus \R^d$ and
  functionals $f\colon \R^d\to \R$ and $g\colon \R^e\to \R$. Then,
  \begin{gather}
    f\otimes g
    \begin{pmatrix}
      x\\y
    \end{pmatrix}
    = f(x) g(y).
  \end{gather}
  Thus, we have constructed the tensor product of functions, which are the zero-forms.
\end{intro}

\begin{intro}
  We now extend this concept to alternating multilinear forms. We
  begin with one-forms on $\R^1$. They form the one-dimensional vector
  space $\Alt^1\R^1$ and we call the only basis element $dx$. We
  assume that for the basis vector $e$ of $\R^1$ there holds $dx(e) = 1$.

  Let $\omega_1 = \alpha \in \Alt^0\R^1$ and
  $\omega_2 = \beta dx \in\Alt^1\R^1$. Then,
  \begin{gather}
    \begin{split}
          \omega_1\otimes\omega_2 (x) &= \alpha\beta dx(\pi_2x) \in\Alt^1\R^2,
          \\
          \omega_2\otimes\omega_1 (x) &= \alpha\beta dx(\pi_1x) \in\Alt^1\R^2.
    \end{split}
  \end{gather}
  Again, $\pi_i$ projects on the $i$th factor of the space
  $\R^1\times \R^1$. Let now $\omega_3 = \gamma dx$ be another 1-form
  on $\R^1$. Then,
  \begin{gather}
    \phi(x,y) = \omega_2\times \omega_3(x,y) = \beta\gamma (dx(\pi_1 x)
    dx(\pi_2 y) - dx(\pi_2 x) dx(\pi_1)y) \in \Alt^2\R^2.
  \end{gather}
  Here, we chose the antisymmetric product, such that $\phi$ is
  alternating. Thus, we have a recipe for the construction of
  alternating forms on higher dimensional spaces from those of lower
  dimensional spaces.
\end{intro}


\begin{Definition}{wedge-product}
  The \define{wedge product} of two alternating forms
  $\omega\in \Alt^k\R^d$ and $\eta\in\Alt^l\R^d$ is a form in $\Alt^{k+l}\R^d$ such that for
  $v_1,\dots,v_{k+l}\in\R^d$ holds
  \begin{gather}
    (\omega\wedge\eta)(v_1,\dots,v_{k+l})
    = \sum_{\sigma} \operatorname{sgn}(\sigma) \omega(v_{\sigma_1},\dots,v_{\sigma_k})
    \eta(v_{\sigma_{k+1}},\dots,v_{\sigma_{k+l}}),
  \end{gather}
  where the sum is taken over all permutations $\sigma$ such that
  \begin{gather}
    \begin{split}
      \sigma_1 &< \dots < \sigma_k,\\
      \sigma_{k+1} & < \dots < \sigma_{k+l}.
    \end{split}
  \end{gather}
\end{Definition}

\begin{Notation}{dx-basis}
  Let $V$ be a vector space with basis $e_1,\dots,e_d$. Then, we denote by $dx_i$ the dual basis.
\end{Notation}
