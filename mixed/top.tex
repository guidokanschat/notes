\tikzset{shape veloxy/.style={color=black,draw,fill=red,thick}}
\tikzset{shape pressure/.style={color=black,draw,fill=cyan,thick}}

\maketitle

\section*{Preface}
%These notes are a short presentation of the material presented in my
lecture. They follow the notes by Rannacher (Numerik 1 in German)
as well as the books by Hairer, Nørsett, and
Wanner~\cite{HairerNorsettWanner93} and Hairer and
Wanner~\cite{HairerWanner10}. Furthermore, I used the book by
Deuflhard and Hohmann~\cite{DeuflhardBornemann08}. Historical remarks
are in part taken from the article by Butcher~\cite{Butcher96}.

I am always thankful for hints and errata. But please verify that you
have the latest version, which is available on github.

My thanks go to David Stronczek, Markus Schubert, and Dörte Jando for their help
writing and translating these notes.

\clearpage
\section*{Index for shortcuts}
\begin{tabular}{ll}
  IVP & Initial value problem, s. definition~\ref{Definition:awa} on page~\pageref{Definition:awa}\\
  BDF & Backward differencing formula, s. example~\ref{ex:lmm:3} on page~\pageref{ex:lmm:3}\\
  ODE & Ordinary differential equation\\
  DIRK & Diagonal implicit Runge-Kutta method\\
  ERK & Explicit Runge-Kutta method\\
  IRK & Implicit Runge-Kutta method\\
  LMM & Linear multistep method, s. Definition~\ref{Definition:lmm} on
  page~\pageref{ex:lmm:1}\\
  VIE & Volterra integral equation, s. Remark~\vref{remark:volterra}
\end{tabular}

\section*{Index for symbols}
\begin{tabular}{lp{10cm}}
  $\C$ & The set of complex numbers\\
  $e_i$ & The unit vector of $\C^d$ in direction $d$\\
  $\Re$ & Real part of a complex number\\
  $\R$ & The set of real numbers\\
  $\R^d$ & The $d$-dimensional vectorspace of the real $d$-tuple\\
  $u$ & The exact solution of an ODE or IVP\\
  $u_k$ & The exact solution at time step $t_k$\\
  $y_k$ & The discrete solution at time step $t_k$\\
  $\scal(x,y)$ & The Euclidean scalar product in the space $\R^d$ or
  $\C^d$\\
  $\abs{x}$ & The absolute value of a real number, the modulus of a
  complex number, or the Euclidean norm in $\R^d$ or $\C^d$, depending
  on its argument\\
  $\norm{u}$ & A norm in a vector space (with exception of the special
  cases covered by $\abs{x}$)
\end{tabular}

\cleardoublepage

%%% Local Variables: 
%%% mode: latex
%%% TeX-master: "notes"
%%% End: 


These notes compile information from various sources, which should be
consulted for more details on particular subjects. These are in
particular the books by Daniele Boffi, Franco Brezzi, and Michel
Fortin~\cite{BoffiBrezziFortin13} and by Alexandre Ern and Jean-Luc
Guermond~\cite{ErnGuermond04} on the general theory of mixed finite
elements, the books by Dietrich Braess~\cite{Braess97,Braess13} and
Philippe Ciarlet~\cite{Ciarlet88} on elasticity, the book by Vivette
Girault and Pièrre-Arnaud Raviart~\cite{GiraultRaviart86} on
incompressible fluids, the book by Peter Monk~\cite{Monk03}, and the
works by Douglas Arnold, Richard Falk, and Ragnar
Winther~\cite{ArnoldFalkWinther06acta,ArnoldFalkWinther10} on the
finite element exterior calculus.

Special thanks go to Dr.~Daniel Arndt for contributing problems,
solutions, and improving the notes at many points. Furthermore, I
thank Vladislav Olkhovskiy and Saurabh Mehta for valuable corrections
and additions.
\thispagestyle{empty}
\setcounter{page}{0}


\tableofcontents

\chapter{Introduction: from elliptic to mixed problems}
We begin our course of mixed finite element methods by studying a
vector-valued elliptic problem. Then, we study its dependence on its
parameters and naturally arrive at a mixed formulation. We derive a
few properties of mixed systems and then turn our attention to the
first example: the Stokes equations of incompressible flow. We close
the chapter by considering mixed systems as first order conditions
for constrained minimization problems.

\section{Linear elasticity}

In this section, we study the simplest mathematical model for elastic
deformation of solids based on Hooke's law. For comparison,
consider~\cite{Braess97}. For the full nonlinear model in all
mathematical detail refer to~\cite{Ciarlet}.

\begin{Notation}{vector-diff-operators}
  Differential operators for vector fields $u:\R^d\to\R^d$
  are defined as follows:
  \begin{xalignat}2
    \nabla u &=
    \begin{pmatrix}
      \d_1 u_1 & \cdots & \d_d u_1\\
      \vdots && \vdots \\
      \d_1 u_d & \cdots & \d_d u_d
    \end{pmatrix}
    &&\text{(gradient)}
    \\
    \div u &= \sum_{i=1^d} \d_iu_i
    &&\text{(divergence)}
  \end{xalignat}

  For a tensor field $\sigma: \R^d\to \R^{d\times d}$, the divergence
  is a vector defined row-wise as
  \begin{gather}
    \div\sigma = \left(\sum_{i=1}^d \d_i \sigma_{ij}\right)_{j=1,\dots,d}
  \end{gather}
\end{Notation}

\begin{intro}
  The deformation of a solid body is described as a mapping $\Phi$
  from the \define{reference configuration} $\domain\subset \R^d$ to a
  deformed configuration $\deformed\domain \subset \R^d$, such that
  each undeformed point $x\in\domain$ is mapped to the point
  $\deformed x$ after deformation. The domain $d$ is 3 for physically
  relevent models, but we investigate two-dimensional problems in
  order to study mathematical properties and numerical methods more
  easily.

  Actually, we are not quite interested in this mapping $\Phi$, since
  it depends on the position of the points $x$. On the other hand, a
  basic principle of physical laws is frame invariance, namely, if we
  change from one Cartesian coordinate system to another, the physical
  law may only change by the same coordinate transformation, not in
  its physical implications. Therefore, only the differences
  $\deformed x-x$ should matter. Thus, we introduce the
  \define{displacement} $u$, such that
  \begin{gather*}
    \Phi = \id + u.
  \end{gather*}
  The symbol $\id$ will refer to all occurrences of identical mappings
  and their matrices.

  So far, by the introduction of $u$, we divide translations of the
  reference configuration out of our model. But, in addition, we have
  to eliminate the influence of solid body rotations. These are
  operations, which leave all distances and angles unchanged. Thus, we
  investigate the change of the distance between $x$ and $x+z$ under
  the mapping $\Phi$. By definition of the derivative, we have
  \begin{align*}
    \abs{\Phi(x+z) - \Phi(x)}^2 &= \norm{\nabla\Phi z} + o(\abs{z})
    \\
                              &= z^T\nabla\Phi^T\nabla\Phi z + o(\abs{z})
    \\
    &= z^T(I + \nabla u^T + \nabla u + \nabla u^T \nabla u) z + o(\abs{z})
    \\
    &= \abs{z}^2 + 2 z^T\strain u z + o(\abs{z}),
  \end{align*}
  where
  \begin{gather}
    \tilde\epsilon(u) = \tfrac12
    \bigl(\nabla u^T + \nabla u + \nabla u^T \nabla u\bigr)
  \end{gather}
  is the \textbf{strain tensor}. From linear algebra, we know that a
  linear mapping which preserves all distances is orthogonal and thus
  also preserves angles. Thus, every deformation with $\strain u=0$
  is a solid body transformation, namely a combination of translation
  and rotation.

  In this class we are concerned only with linear problems, which can
  be justified by the notion of infinitely small deformations
  $u$. Then, we only study first order effects in $u$, which implies
  that we are going to neglect the quadratic term in
  $\strain u$. This is justified by the fact that we obtain a model,
  which is sufficiently accurate for small deformations.
\end{intro}

\begin{Definition}{strain-tensor}
  The linearized \define{strain tensor} or \define{symmetric gradient}
  of $u$ is
  \begin{gather}
    \strain u = \frac{\nabla u + \nabla u^T}2.
  \end{gather}
\end{Definition}

\begin{intro}
  Next, we have to consider the interplay of forces and
  deformations. The basic principle is Newton's axiom of force
  balance. If a body force $f$ acts on a small volume $V$, there have
  to be surface forces acting against $f$ in order to keep $V$ at
  rest. Similarly, if a torque is applied inside this volume, there
  must be tangential forces on the surface balancing this torque. Due
  to Euler, we model these forces as a mapping $t$, which at each
  point $x$ maps a direction vector $n$ to a force vector
  $t(x,n)$. Thus, the balance of forces is written as
  \begin{alignat*}2
    \int_V f \dx &+ \int_{\d V} t(x,n) \ds &=&0\\
    \int_V x\times f \dx &+ \int_{\d V} x\times t(x,n) \ds &=&0.
  \end{alignat*}
  Due to Euler and Cauchy, this mapping $t(x,n)$ can be expressed as
  $\sigma(x)n$ by the \define{stress tensor} $\sigma$. Without proof,
  we note that the balance of torque implies that $\sigma$ is
  symmetric, while the force balance equation after integration by
  parts becomes
  \begin{gather}
    \label{eq:mixedintro:3}
    f + \div \sigma = 0.
  \end{gather}
  What is missing now is a relation between the displacement $u$ and
  the stress $\sigma$, which is not the result of fundamental
  principles, but of material properties.
\end{intro}

\begin{remark}
  At this point, we play again the card of small deformations, such
  that we do not have to distinugish whether equations are formulated
  on the reference domain $\domain$ or on the deformed domain
  $\deformed\domain$. Such a discussion becomes confusing easily and
  thus remains a subject for a more specialized class.
\end{remark}

\begin{Definition}{hooke}
  \define{Hooke's law} states that the stress depends linearly on the
  strain. Together with frame invariance, this implies the relation
  \begin{gather}
    \label{eq:mixedintro:4}
    \sigma = 2\mu \strain u + \lambda \operatorname{tr} \strain u,
  \end{gather}
  where $\lambda\ge 0$ and $\mu> 0$ are material properties called
  \define{Lam\'e-Navier parameters}.
\end{Definition}

\begin{remark}
  The trace of the strain operator is equal to the trace of the
  gradient. Thus, we have
  \begin{gather}
    \label{eq:mixedintro:5}
    \operatorname{tr} \strain u = \div u \,\id.
  \end{gather}
\end{remark}

\begin{intro}
  Equations~\eqref{eq:mixedintro:3} and~\eqref{eq:mixedintro:4}
  together define a second order partial differential equation, for
  which we have to impose boundary conditions. A natural choice, which
  keeps the mathematical analysis simple is the \define{Dirichlet
    boundary condition} $u=0$, corresponding to an elastic body whose
  boundary is fixed. The alternative is the traction free boundary
  condition $\sigma n=0$ with vanishing normal traces. Combinations
  are possible, for instance $u\cdot n=0$ for a boundary that allows
  sliding but no penetration. Constraining ourselves to Dirichlet
  condition on $\Gamma_D\subset \d\domain$ and traction free on
  $\gamma_N = \d\domain\setminus\Gamma_D$, we obtain the
  boundary value problem
  \begin{gather}
    \label{eq:mixedintro:lame-navier-bvp}
    \begin{aligned}
      -\div \sigma(x) &= f(x) & x&\in\domain,\\
      u(x) &= 0 & x&\in\Gamma_D, \\
      \sigma(x)n &=0& x&\in\Gamma_N,
    \end{aligned}
  \end{gather}
  together with the material law~\eqref{eq:mixedintro:4}.  Once we
  test and integrate by parts to obtain our weak formulation, we
  obtain
  \begin{gather*}
    \int_{\domain} -\div \sigma \cdot v\dx
    = \int_{\domain} \sigma\nabla: v\dx
    - \int_{\Gamma_N} \sigma n\cdot v\ds,
  \end{gather*}
  such that traction free is actually the natural boundary condition
  comparable to the Neumann condition for the Laplacian. Note that $:$
  is the double contraction or Frobenius product (see
  Problem~\ref{problem:frobenius} below) of the two tensors.
\end{intro}

\begin{intro}
  We now walk the missing steps to obtain a weak formulation. first,
  we enter Hooke's law for $\sigma$ to obtain:
  \begin{gather*}
    \int_\domain \bigl[2\mu \strain u : \nabla v
    + \lambda \div u \id : \nabla v
    \bigr]\dx = \int_\domain f\cdot v\dx.
  \end{gather*}
  Then, we choose the space
  \begin{gather}
    V = H^1_{\Gamma_D}(\domain; \R^d) = \bigl\{v\in H^1(\domain;\R^d) \big\vert
    v_{|\Gamma_D} = 0 \bigr\}.
  \end{gather}
  We notice for the second term that
  \begin{gather*}
    \id : \nabla v = \sum_{i=1}^d \d_i v_i = \div v.
  \end{gather*}
  Furthermore, we use the result of Problem~\ref{problem:frobenius} to
  obtain
  \begin{gather*}
    \strain u:\nabla v = \strain u : \strain v.
  \end{gather*}
\end{intro}

\begin{Definition}{weak-lame-navier}
  The weak formulation of the Lam\'e-Navier boundary value
  problem
  \begin{gather*}
    \begin{aligned}
      -\div \sigma(x) &= f(x) & x&\in\domain,\\
      u(x) &= 0 & x&\in\Gamma_D, \\
      \sigma(x)n &=0& x&\in\Gamma_N,
    \end{aligned}
  \end{gather*}
  is: find $u\in V = H^1_{\Gamma_D}(\domain;\R^d)$ such that
  \begin{gather}
    a(u,v) \equiv 2\mu\form(\strain u, \strain v)
    + \lambda \form(\div u, \div v)
    = \form(f,v)
    \quad\forall v\in V.
  \end{gather}
\end{Definition}

\begin{Problem}{frobenius}
  Given the vector space of square matrices $X = \R^{d\times d}$ with the
  Frobenius inner product
  \begin{gather}
    \label{eq:mixedintro:frobenius}
    \scal(A,B) = A:B = \sum_{ij} a_{ij}b_{ij}.
  \end{gather}
  Show that the subspaces of symmetric and skew-symmetric matrices,
  respectively, are orthogonal to each other and $X$ is the direct sum
  of those.
\end{Problem}

\begin{intro}
  The form $a(.,.)$ is symmetric and thus semi-definite on $V$. It can
  also be bounded easily by the $H^1$-norm. But, for well-posedness of
  the weak formulation, we also require ellipticity. This question is
  indeed not trivial and rests on the fact that for a function
  $u\in V$, such that $\nabla u$ is skew-symmetric everywhere, there
  holds $\strain u\equiv 0$. Thus, such functions must be excluded by
  the boundary conditions. Note, that in particular for rigid body
  translations and rotations $\strain u = 0$. Therefore, the Dirichlet
  boundary conditions must exclude such solutions.
  
  The condition needed for well-posedness is called Korn's inequality,
  and it will be posed as an assumption. We will give a proof for a
  simple case and refer the readers to a plethora of articles on more
  complicated cases.
\end{intro}

\begin{Assumption}{korn-inequality}
  We assume that the boundary conditions defining the space $V$ in the
  weak formulation of the Lam\'e-Navier equations are such that
  \define{Korn's inequality}
  \begin{gather}
    \label{eq:mixedintro:korn}
    \norm{u}_{H^1(\domain;\R^d)}^2
    \le c_K^2 \norm{u}_{L^2(\Omega;\R^d)}^2
    + \norm{\strain u}_{L^2(\Omega;\R^d)}^2,
  \end{gather}
  holds for all $u\in V$ with a uniform constant $c_K>0$.
\end{Assumption}

\begin{Lemma}{korn}
  Let $V = H^1_0(\domain;\R^d)$ for a Lipschitz domain
  $\domain$. Then, Korn's inequality holds on $V$ with a constant
  $c_K>0$.
\end{Lemma}

\begin{proof}
  
\end{proof}

\begin{Problem}{elasticity-standard}
  Let the space $V=H^1_0(\domain;\R^d)$ be equipped with the inner
  product $\scal(u,v) = a(u,v)$ with the bilinear form of the
  Lam\'e-Navier equations and the corresponding norm $\norm{.}_V$.

  Show using techniques from the standard theory of elliptic partial
  differential equations:
  \begin{enumerate}
  \item The weak formulation has a unique solution for which there holds
    \begin{gather*}
      \norm{u}_V \le \norm{f}_{V^*}.
    \end{gather*}
  \item The ``energy estimate'' for conforming finite element
    approximation with a space $V_h\subset V$
    \begin{gather*}
      \norm{u-u_h}_V = \inf_{v_h\in V_h} \norm{u-v_h}_V.
    \end{gather*}
  \item The $H^1$-error estimate
    \begin{gather}
      \norm{u-u_h}_{H^1}
      \le \frac{2\mu+d\lambda}{2c_K\mu}
      \inf_{v_h\in V_h} \norm{u-v_h}_{H^1}.
    \end{gather}
    Use the fact that the space $V$ can be composed into the space
    $V^0$ of divergence-free functions ($\div v=0$) and its
    complement.
  \item For $\lambda \gg \mu$, the previous estimate is useless. Can
    it be improved easily? In view of the ``energy estimate'', can you
    think of conditions?
  \end{enumerate}
\end{Problem}

\begin{intro}
  As we could see in the preceding problem, approximation of the
  solution to the Lamé-Navier equations becomes difficult, if
  $\lambda \gg \mu$. In this case, the material is called almost
  incompressible, since the divergence measures compression or
  dilation and the dominating divergence term forces the divergence of
  the solution to be small. These cases are important in engineering
  and they initiated a lot of the research that resulted in the topics
  of this class.
\end{intro}


%%% Local Variables: 
%%% mode: latex
%%% TeX-master: "main"
%%% End: 

%%%%%%%%%%%%%%%%%%%%%%%%%%%%%%%%%%%%%%%%%%%%%%%%%%%%%%%%%%%%%%%%%%%%%%
%%%%%%%%%%%%%%%%%%%%%%%%%%%%%%%%%%%%%%%%%%%%%%%%%%%%%%%%%%%%%%%%%%%%%%
\subsection{Relation to constrained minimization}

\begin{intro}
  In this section, we assume that $a(.,.)$ is symmetric and
  $V$-elliptic. Therefore, the minimization problem finding $u\in V$
  such that
  \begin{align*}
    J(u) &=\inf_{v\in V} J(v), \\
    J(v) &= \tfrac12 a(v,v) - f(v),
  \end{align*}
  has a unique solution in a Hilbert space $V$, which then is a
  minimum.

  In the next step, we constrain the solution $u$ to a subspace of
  $V$, that is, we consider the minimization problem
  \begin{gather*}
    J(u) =\min J(v) \quad
    \text{subject to}\quad
    Bu = 0.
  \end{gather*}
\end{intro}

\begin{Theorem}{lagrange-multiplier}
  
\end{Theorem}


%%% Local Variables: 
%%% mode: latex
%%% TeX-master: "main"
%%% End: 


\chapter{Conditions for well-posedness}
\label{sec:mixed-wellposedness}
In this chapter, we will first modify conditions for well-posedness in
finite dimensions from positive definiteness to the general case. In
particular, we will derive a quantitative formulation, which we will
study for infinite dimensional problems in the second section. In the
third section, we derive the inf-sup condition for mixed problems as a
special case.

Let us begin with a simple example, displaying some of the
difficulties we may have.

\begin{Problem}{unbounded-inverse}
  On the space $\ell_2(\R)$ define the operator $A$ by its eigenvalue
  decomposition
  \begin{align}
    A: \ell_2(\R) &\to \ell_2(\R)\\
    e_k & \mapsto \tfrac1k e_k.
  \end{align}
  Here, $\{e_k\}$ is the orthogonal basis of unit vectors of the form
  \begin{gather}
    \arraycolsep0.1em
    \begin{array}{cccccccc}
      e_k =(0&,\ldots,&0&,&1&,&0&,\ldots)^\transpose.\\
      &&&&\uparrow\\
      &&&&k
    \end{array}
  \end{gather}
  \begin{enumerate}
  \item Show that this operator does not have a bounded inverse, albeit
    its eigenvalues are positive.
  \item Show that the range of $A$ is not closed in $\ell_2(\R)$
  \end{enumerate}
\begin{solution}
  \begin{enumerate}
  \item For each $e_k$, the inverse is $A^{-1} e_k = k e_k$. In particular, $A$ is injective.
    On the other hand, it holds
    \begin{gather}
      \lim_{k\to\infty}\frac{\norm{A^{-1}e_k}}{\norm{e_k}}=\lim_{k\to\infty}k=\infty
    \end{gather}
      and the inverse cannot be bounded.
  \item We have to construct a convergent sequence in the range of $A$
    such that the pre-image of the sequence does not converge.
    \begin{enumerate}
    \item Choose
      \begin{gather}
        v_n = \sum_{k=1}^n \frac1k e_k.
      \end{gather}
      \item $v_n$ is a Cauchy-sequence, since
        \begin{gather}
          \norm{v_m-v_n}^2 = \norm*{\sum_{k=m}^n \frac1k e_k}^2
          = \sum_{k=m}^n \frac1{k^2}\norm*{e_k}^2
          \le \frac1{m^2} \sum_{k=1}^\infty \frac1{k^2}
          = \frac{\pi^2}{6} \frac1{m^2}.
        \end{gather}
      \item We conclude that $v=\lim_{n\to\infty}v_n$ exists in the
        closure of the range of $A$.
      \item There holds
        \begin{gather}
          v_n = A \sum_{k=1}^n e_k =: A u_n.
        \end{gather}
      \item Due to the injectivity of $A$ for $v$ to be in the range of $A$,
            $u_n$ has to converge in $\ell_2(\R)$.
      \item The sequence $u_n$ is not a Cauchy sequence, since
        \begin{gather}
          \norm{v_m-v_n}^2 = \norm*{\sum_{k=m}^n e_k}^2
          = \sum_{k=m}^n \norm*{e_k}^2
          = n-m.
        \end{gather}
    \end{enumerate}
  \end{enumerate}
\end{solution}
\end{Problem}

\section{Finite-dimensional problems}
\begin{intro}
  So far, our power horse for well-posedness was the Lax-Milgram
  lemma, which can be applied under the conditions
  \begin{xalignat}2
    a(u,v) &\le M \norm{u}\norm{v} & \forall u,v&\in V\\
    a(u,u) &\ge \ellipa \norm{u}^2 & \forall u&\in V.
  \end{xalignat}
  The second condition can also be rewritten in terms of the
  \putindex{Rayleigh quotient} as
  \begin{gather}
    0 < \ellipa = \inf_{u\in V}\frac{a(u,u)}{\norm{u}^2}.
  \end{gather}
  Restricting this to a finite dimensional space, the notation usually
  changes from
  \begin{gather}
    a(u,v) = f(v)
    \qquad\text{to}\qquad
    v^\transpose A u = v^\transpose f,
  \end{gather}
  where $A\in \R^{n\times n}$ is the matrix associated with the
  bilinear form. The bound for the Rayleigh quotient means nothing but
  that the real parts of all eigenvalues of $A$ are bounded from below
  by $\ellipa$. Thus, a matrix $A$ for which we can apply the
  Lax-Milgram lemma is positive definite. And the statement of the
  lemma in finite dimension is, that a positive definite matrix is
  invertible. We know from linear algebra that this is true, but we
  also know that the condition is all but necessary.
\end{intro}

\begin{intro}
  Why did we replace this clear theorem by the weaker Lax-Milgram
  lemma, when we studied elliptic partial differential equations?  For
  the first condition, it should be noted that spectral properties of
  operators between spaces of infinite dimension are much harder to
  obtain. Further, we do not need information on the whole spectrum,
  but only on the eigenvalue closest to zero. Therefore, we used a
  simple estimate in order to avoid discussing the spectrum at
  all. But, there is an important difference between
  Theorem~\ref{Theorem:la-invertible} and the
  estimate~\eqref{eq:infsup:elliptic}: the assumption of the theorem
  is qualitative, $\lambda \neq 0$, while the assumption of Lax-Milgram
  is quantitative,
  \begin{gather}
    \Re\lambda \ge \ellipa> 0.
  \end{gather}
  The following problem shows why such a change is necessary.
\end{intro}


\begin{Problem}{lax-milgram-not-applicable}
  Find an invertible, symmetric matrix $A\in \R^{2\times 2}$ and a
  vector $v\in \R^2$ such that $v^\transpose A v=0$ and thus the Lax-Milgram
  lemma is inconclusive.
\begin{solution}
  \begin{gather}
    A =
    \begin{pmatrix}
      1 & 0 \\ 0 & -1
    \end{pmatrix}
  \end{gather}
\end{solution}
\end{Problem}

The question of well-posedness in finite dimensions can be answered by:

\begin{Theorem}{la-invertible}
  A matrix $A\in\R^{n\times n}$ is invertible if and only if one of
  the following equivalent conditions holds:
  \begin{enumerate}
  \item all its (possibly complex) eigenvalues are nonzero,
  \item all its singular values are nonzero,
  \item for each nonzero $v\in\R^n$ holds $Av\neq 0$.
  \end{enumerate}
\end{Theorem}

\begin{intro}
  We focus on the second and third conditions, respectively, in
  Theorem~\ref{Theorem:la-invertible}.
  But, the problem above tells us that we
  will run into trouble, if we do not quantify this. Therefore, we
  start our attempt by requiring:
  \begin{gather}
    \norm{Au}^2 \ge \ellipa \norm{u}^2 \qquad\forall u\in V.
  \end{gather}
  But while this is a condition we can easily write down for matrices
  and operators, it does not work that well for bilinear forms. Thus,
  we first look at the singular value decomposition.
\end{intro}

\begin{Notation}{diag}
  Mit $\diag(a_1,a_2,\dots,a_r) \in \R^{m\times n}$ sei allgemein die $m\times n$-Matrix $A$ bezeichnet, deren erste $r$ Diagonalelemente die Werte $a_i$ annehmen. Alle anderen Einträge sind null. Sie hat die Darstellung
  \begin{gather}
    A = \diag(a_1,a_2,\dots,a_r)
    = \left(
      \begin{array}{ccc|c}
        a_1&& \mathbf 0&\\
        &\ddots&& \mathbf 0\\
        \mathbf 0&&a_r&\\\hline
        & \mathbf 0 && \mathbf 0
      \end{array}
      \right),
    \end{gather}
    wobei untere und rechte Nullblöcke leer sein dürfen.
\end{Notation}

\begin{Definition}{svd}
  Die \define{Singulärwertzerlegung} (engl.: \define{singular value decomposition}, \define{SVD}) einer Matrix
  $A\in \R^{m\times n}$ hat die Form
  \begin{gather}
    \label{eq:svd:1}
    A = U \Sigma V^T
  \end{gather}
  mit orthogonalen Matrizen $U\in \R^{m\times m}$ und
  $V\in \R^{n\times n}$. Die Matrix $\Sigma\in\R^{m\times n}$ ist
  \begin{gather}
    \Sigma = \diag(\sigma_1,\dots,\sigma_r)
  \end{gather}
  mit positiven, reellen Einträgen $\sigma_1,\dots,\sigma_r$
  und $r\le \min\{m,n\}$, den \define{Singulärwerten}.
  Die Singulärwerte seien der Größe nach fallend sortiert.
\end{Definition}


\begin{Satz}{svd}
  Jede Matrix $A\in \R^{m\times n}$ besitzt eine Singulärwertzerlegung.
\end{Satz}

\begin{proof}
  Siehe auch~\cite[Satz 4.11]{Rannacher17}.
  Der Beweis läuft induktiv über die Spalten von $U$ und $V$. Wir
  bemerken zunächst, dass es wegen der Stetigkeit der Norm einen
  Vektor $x\in \R^n$ mit $\norm{x}_2 = 1$ gibt, so dass
  \begin{gather}
    \norm{Ax}_2 = \norm{A}_2\norm{x}_2.
  \end{gather}
  Wir definieren $\sigma_1 = \norm{A}$ und es sei $y\in \R^m$ so dass
  $\sigma_1y=Ax$. Wir ergänzen $x$ und $y$ jeweils zu Orthogonalbasen
  und nennen die Matrizen der Basisvektoren $U^{(1)}$ und
  $V^{(1)}$. Es gilt dann
  \begin{gather}
    \left(U^{(1)}\right)^T A^{(1)} V^{(1)} =
    \begin{pmatrix}
      \sigma& w^T\\0 & B
    \end{pmatrix}
  \end{gather}
  mit einem Vektor $w\in \R^{n-1}$ und einer Matrix
  $b\in \R^{m-1\times n-1}$. Da $U$ und $V$ orthogonal sind, gilt
  \begin{gather}
    \norm{A^{(1)}} = \norm{A} = \sigma.
  \end{gather}
  Multipliziert man die Matrix $A^{(1)}$ mit dem Vektor
  $ z=(\sigma,w)^T$, so erhält man
  \begin{gather}
    A^{(1)}z = 
    \begin{pmatrix}
    \sigma&w^T\\0&B
  \end{pmatrix}
  \begin{pmatrix}
    \sigma\\w
  \end{pmatrix}
  =
  \begin{pmatrix}
    \sigma^2+\norm{w}^2\\
    Bw
  \end{pmatrix}
\end{gather}
Daher gilt
\begin{gather}
  \norm{A^{(1)}z}_2^2 = \bigl(\sigma^2+\norm{w}^2\bigr)^2 + \norm{Bw}_2^2
  \ge (\sigma^2+\norm{w}^2) \norm{z}_2^2.
\end{gather}
Daher muss $\norm{w}=0$ und damit $w=0$ gelten. Die Matrix hat also
die Gestalt
  \begin{gather}
    \left(U^{(1)}\right)^T A^{(1)} V^{(1)} =
    \begin{pmatrix}
      \sigma& 0\\0 & B
    \end{pmatrix}.
  \end{gather}
  Nun wenden wir induktiv dasselbe Verfahren auf $B$ an.

  An einem Punkt kann es vorkommen, dass $\norm{B}=0$. In diesem Fall
  können wir die Konstruktion bereits abbrechen, da nun alle weiteren
  Vektoren im Kern der Matrix liegen. Zu diesen können wir beliebige
  Bildvektoren im orthogonalen Komplement des bereits konstruierten
  Raums wählen.
\end{proof}

\begin{Lemma}{svd-eigenschaften}
  Für die Singulärwertzerlegung $A=U\Sigma V^T$ gilt für die Spalten
  von $U$ und $V$
  \begin{gather}
    \label{eq:svd:2}
    A v^{(i)} = \sigma_i u^{(i)}, \qquad
    A^T u^{(i)} = \sigma v^{(i)},\qquad
    i \le \min\{m,n\}.
  \end{gather}
  Sei $\sigma_r\neq0$ der letzte von null verschiedene
  Singulärwert. Dann gilt für den \putindex{Rang} $\operatorname{rg}(A) = r$
  und
  \begin{gather}
    \label{eq:svd:3}
    \begin{aligned}
      \range A &= \span\{u^{(1)},\dots,u^{(r)}\}&&
      \ker A &= \span\{v^{(r+1)},\dots,v^{(n)}\}\\
      \range{A^T} &= \span\{v^{(1)},\dots,v^{(r)}\}&&
      \ker{A^T} &= \span\{u^{(r+1)},\dots,u^{(m)}\}
    \end{aligned}
  \end{gather}
\end{Lemma}

\begin{proof}
  Die Eigenschaften~\eqref{eq:svd:2} liest man direkt der Darstellung
  ab, denn aufgrund der Orthogonalität gilt
  $V^T v^{(i)} = e_i \in \R^n$. Damit gilt
  $\Sigma V^Tv^{(i)} = \sigma_i e_i \in \R^m$. Schliesslich selektiert
  $Ue_i$ den $i$-ten Spaltenvektor von $U$. Für die transponierte
  Matrix gilt dies wegen
  \begin{gather}
    A^T = V \Sigma^T U^T,
  \end{gather}
  wobei bei der Transposition von $\Sigma$ nur die Dimensionen
  getauscht werden, die Diagonalelemente bleiben natürlich gleich.

  Die Aussage~\eqref{eq:svd:3} ist dann eine direkte
  Folge. Insbesondere ist der Rang der Matrix durch den Index des
  letzten positiven Singulärwerts charakterisiert.
\end{proof}

\begin{remark}
  Aus Gleichung~\eqref{eq:svd:3} lesen wir direkt die bekannten
  Beziehungen aus der linearen Algebra ab. Oder umgekehrt, die
  Singulärwertzerlegung erzeugt Orthonormalbasen der Vektorräume, die
  in Kern und orthogonales Komplement zerlegen. Zusätzlich wird der
  (eingeschränkte) Isomorphismus noch diagonalisiert.
\end{remark}

\begin{Satz}{minimalloesung}
  Sei $A\in\R^{m\times n}$ mit Singulärwertzerlegung $A=U\Sigma V^T$
  und Rang $r$. Sei
  \begin{gather}
    \Sigma^+ = \diag\left(\tfrac1{\sigma_1},\tfrac1{\sigma_2},\dots,\tfrac1{\sigma_r}\right) \in \R^{m\times n}.
  \end{gather}
  Dann ist der Vektor $x^*\in \R^n$ mit
  \begin{gather}
    \label{eq:svd:4}
    x^* = V \Sigma^+ U^T b
  \end{gather}
  die eindeutig bestimmte Lösung der Normalengleichungen mit minimaler
  Norm. Für das Residuum gilt
  \begin{gather}
    \label{eq:svd:5}
    \norm{Ax^*-b}_2^2 = \sum_{i=r+1}^m \Bigl(\bigl(u^{(i)}\bigr)^T b\Bigr)^2.
  \end{gather}
\end{Satz}

\begin{proof}
  Sei $x\in \R^n$ und $z=V^T x$ sei seine Koordinatendarstellung in
  der Basis $V$. Dann gilt
  \begin{align}
    \norm{Ax-b}_2^2
    &= \norm{AVV^Tx - b}_2^2\\
    &= \norm{U^TAV z - U^Tb}_2^2\\
    &= \norm{\Sigma z - U^T b}_2^2\\
    &= \sum_{i=1}^r \Bigl(\sigma_iz_i - \bigl(u^{(i)}\bigr)^T b\Bigr)^2
      +\sum_{i=r+1}^m \Bigr(\bigl(u^{(i)}\bigr)^T b\Bigr)^2.
  \end{align}
  Da alle Summanden nichtnegativ sind, wird das Minimum für
  \begin{gather}
    z_i = \frac1{\sigma_i}\bigl(u^{(i)}\bigr)^T b, \qquad i=1,\dots,r
  \end{gather}
  angenommen. Damit verschwindet die erste Summe und~\eqref{eq:svd:5}
  ist für den so bestimmten Vektor $z$ bewiesen. Offensichtlich ist
  die Norm des Vektors $z$ minimal, wenn alle weiteren Komponenten
  verschwinden, also
  \begin{gather}
    z_i=0,\qquad i=r+1,\dots,n.
  \end{gather}
  Damit können wir zusammenfassend schreiben
  \begin{gather}
    z = \Sigma^+ U^T b.
  \end{gather}
  Da $V$ orthogonal ist, überträgt sich die Minimalitätseigenschaft auf $x^*=Vz$
\end{proof}

\begin{remark}
  Für den Fall eine invertierbaren Matrix $A\in\R^{n\times n}$
  entspricht~\eqref{eq:svd:4} gerade der Inversen des Produkts. Im
  Falle das $A$ vollen Rank hat mit $m\ge n$ bekommen wir die
  eindeutige Lösung der Normalengleichungen und die Bedingung \glqq
  mit minimaler Norm\grqq{} entfällt.
\end{remark}

\begin{Definition}{pseudoinverse}
  Die Matrix $A^+ = V\Sigma^+ U^T \in \R^{n\times m}$ ist eine
  Verallgemeinerung der Inversen, die als \define{Pseudoinverse}, auch
  als \define{Moore-Penrose-Inverse} bezeichnet wird. Sie ist für jede
  Matrix $A\in \R^{m\times n}$ definiert.
\end{Definition}

\begin{Satz*}{penrose}{Penrose-Axiome}
  Für die Pseudoinverse $A^+\in \R^{n\times m}$ einer Matrix
  $A\in \R^{m\times n}$ gelten folgende Gleichungen:
  \begin{align}
    \bigl(A^+A\bigr)^T &= A^+A,\\
    \bigl(AA^+\bigr)^T &= AA^+,\\
    \label{eq:svd:6c}
    A^+AA^+ &= A^+,\\
    AA^+A &= A.
  \end{align}
  Insbesondere ist $A^+A$ die orthogonale Projektion auf $\range{A^T}$
  und $AA^+$ die orthogonale Projektion auf $\range A$.
\end{Satz*}

\begin{proof}
  Es gilt
  \begin{gather}
    A^+A = V\Sigma^+U^T U \Sigma V^T = V \Sigma^+\Sigma V^T = V E_r V^T,
  \end{gather}
  wobei $E_r\in \R^{n\times n}$ mit
  \begin{gather}
    E_r = \diag(\underbrace{1,\dots,1}_{r \text{ mal}}).
  \end{gather}
  Daraus folgen sofort Symmetrie und Projektionseigenschaft, sowie aus
  letzterer~\eqref{eq:svd:6c}. Dieseben Argumente bleiben korrekt,
  wenn man $A^+$ und $A$ vertauscht, wobei dann
  $E_r\in \R^{m\times m}$ ist.
\end{proof}

\begin{remark}
  Die Eigenschaft eines Vektors, im Nullraum der Matrix $A$ zu liegen
  ist natürlich nicht invariant unter Störungen von $A$. Im Gegenteil
  wird im Allgemeinen die kleinste Störung dazu führen, dass alle
  Eigenwerte von null verschieden sind. Daher benötigen wir für die
  stabile Zerelgung eines Vektorraums in den Kern und sein
  orthogonales Komplement ein neues Konzept des Nullraums bzw. des
  Rangs einer Matrix.
\end{remark}

\begin{Definition}{eps-rang}
  Der $\epsilon$-Rang einer Matrix $A$ ist definiert als???
\end{Definition}

\begin{Satz}{rang-approximation}
  
\end{Satz}

%%% Local Variables:
%%% mode: latex
%%% TeX-master: "main"
%%% End:


\begin{Definition}{ker-range-rn}
  \index{ker}
  Let $A: V\to W$ be a linear operator. Then, we define
  the \define{kernel} and the \define{range} of $A$ as
  \begin{align}
    \ker A &= \bigl\{ v\in V \big| Av = 0\bigr\} \\
    \range A &= \bigl\{ w\in W \big| \;\exists\,v\in V: Av=w\bigr\}.
  \end{align}
\end{Definition}

\begin{Definition}{orthogonal1}
  Let $V\subset\R^n$ be a subspace. We define the \define{orthogonal
    complement} of $V$ as
  \begin{gather}
    \label{eq:infsup:4}
    \ortho{V} = \bigl\{w\in \R^n \big| \;\forall\,v\in V \scal(w,v) = 0 \bigr\}.
  \end{gather}
\end{Definition}

\begin{Lemma}{ker-coker-rn}
  Let $A\in \R^{m\times n}$ and $A^\transpose$ its transpose. Then, there holds
  \begin{gather}
    \label{eq:infsup:5}
    \begin{split}
      \ker A &= \ortho{\range{A^\transpose}}\\
      \range A &= \ortho{\ker{A^\transpose}}\\
      \ker{A^\transpose} &= \ortho{\range A}\\
      \range{A^\transpose} &= \ortho{\ker{A}}
    \end{split}
  \end{gather}
\end{Lemma}

\begin{proof}
  % First, we note that
  % \begin{gather}
  %   \R^n = \ker A \oplus \ortho{\ker A},
  %   \qquad
  %   \R^m = \ker{A^\transpose} \oplus \ortho{\ker{A^\transpose}}.
  % \end{gather}
  Let $A=U\Sigma V^\transpose$ be the singular value decomposition of $A$ and
  $r$ be the number of nonzero singular values. Then, the first $r$
  vectors of $U$ span the range of $A$ and the last $n-r$ vectors of
  $V$ span its kernel. Furthermore,
  \begin{gather}
    A^\transpose = \bigl(U\Sigma V^\transpose\bigr)^\transpose = V \Sigma U^\transpose.
  \end{gather}
  Therefore, the first $r$ vectors of V span the range of
  $A^\transpose$ and the last $n-r$ vectors of $U$ span its
  kernel. The lemma follows since $U$ and $V$ are orthogonal.
\end{proof}

\begin{Corollary}{ker-coker-iso}
  Let $A\in \R^{m\times n}$ and $A^\transpose$ its transpose. Then, the
  restrictions $A\colon \ortho{\ker A} \to \range A$ and $A^\transpose\colon
  \ortho{\ker{A^\transpose}} \to \range{A^\transpose}$ are isomorphisms.
  
  The linear system $Ax=f\in\R^m$ has at least one solution if and
  only if $f\in \range A$. If $x\in\R^n$ is such a solution, then
  every $y\in\R^n$ with $y-x\in\ker A$ is a solution as well.
\end{Corollary}

\begin{proof}
  We note that $\dim \range A = \dim \range{A^\transpose}$. Thus, by
  \slideref{Lemma}{ker-coker-rn} the dimensions of domain and range of
  each of the restricted operators are equal, say $\dim \range A =
  r$. The singular value decomposition of the operators is
  \begin{gather}
    A = U\Sigma V^\transpose \qquad A^\transpose = V\Sigma U^\transpose,
  \end{gather}
  where all matrices are in $\R^{r\times r}$ and
  \begin{gather}
    \Sigma = \diag(\sigma_1,\dots,\sigma_r),
  \end{gather}
  and all singular values are positive. Thus, $A$ and $A^\transpose$ are invertible.
\end{proof}

\begin{Corollary}{svd-infsup}
  Let $r=\dim \ortho{\ker A}$. Then, for the smallest nonzero singular
  value there holds
  \begin{gather}
    \label{eq:infsup:6}
    \sigma_r
    = \inf_{v\in \ortho{\ker A}} \sup_{w\in \R^m} \frac{w^\transpose A v}{\norm{v}\norm{w}}
    = \inf_{w\in \ortho{\ker{A^\transpose}}} \sup_{v\in \R^n} \frac{w^\transpose A v}{\norm{v}\norm{w}}.
  \end{gather}
\end{Corollary}

\begin{proof}
  Since the Cauchy-Schwarz inequality turns into an equation if and
  only if the two vectors are coaligned, there holds for any $v\in \R^n$:
  \begin{gather}
    \sup_{w\in\R^m}\frac{w^\transpose A v}{\norm{w}} = \frac{v^\transpose A^\transpose A v}{\norm{Av}} = \frac{\norm{Av}^2}{\norm{Av}}.
  \end{gather}
  Therefore,
  \begin{gather}
    \inf_{v\in \ortho{\ker A}} \sup_{w\in \R^m}
    \frac{w^\transpose A v}{\norm{v}\norm{w}}
    = \inf_{v\in \ortho{\ker A}} \frac{\norm{Av}}{\norm{v}}.
  \end{gather}
  Now, let $v = \sum \alpha_i v_i$ where $v_i$ are the columns of $V$
  in the SVD of $A$. Then,
  \begin{gather}
    \norm{Av}^2 = \norm*{A\sum_{i=1}^r \alpha_i v_i}^2
    = \norm*{\sum_{i=1}^r \sigma_i \alpha_i u_i}^2
    = \sum_{i=1}^r \sigma_i^2 \alpha_i^2.
  \end{gather}
  The quotient
  \begin{gather}
    \frac{\sum_{i=1}^r \sigma_i^2 \alpha_i^2}{\sum_{i=1}^r \alpha_i^2}
  \end{gather}
  clearly has its minimum if $\alpha_1 = \dots=\alpha_{r-1} = 0$.
\end{proof}

\begin{Definition}{infsup1}
  A bilinear form $a(\cdot,\cdot)$ on $V\times W$ is said to admit the
  \define{inf-sup condition} or is called \define{inf-sup stable}, if
  there holds
  \begin{gather}
    \label{eq:infsup:1}
    \inf_{u\in V} \sup_{w\in W} \frac{a(u,w)}{\norm{u}_V\norm{w}_W}
    \ge \ellipa > 0.
  \end{gather}
\end{Definition}

\begin{remark}
  In this finite dimensional exposition, is clear that $V$ and $W$
  must have the same dimension, and thus $V=W=\R^n$. This will be
  different, when we consider infinite dimensional spaces and indeed
  consider different spaces $V$ and $W$.
\end{remark}

\begin{Lemma}{infsup2}
  The following statements are equivalent to the inf-sup
  condition~\eqref{eq:infsup:1}:
  \begin{gather}
    \label{eq:infsup:2}
    \forall u\in V \;\exists w\in W \;:\; a(u,w) \ge \ellipa \norm{u}_V\norm{w}_W
  \end{gather}
  \begin{gather}
    \label{eq:infsup:3}
    \forall u\in V
    \;\exists w\in W \;:\;
    \left\{
    \arraycolsep0.3ex
    \begin{array}{rcl}
      \norm{w}_W &\le&\norm{u}_V\\
      a(u,w) &\ge& \ellipa \norm{u}_V^2
    \end{array}
    \right.
  \end{gather}
  \begin{gather}
    \label{eq:infsup:3a}
    \forall u\in V
    \;\exists w\in W \;:\;
    \left\{
    \arraycolsep0.3ex
    \begin{array}{rcl}
      \ellipa \norm{w}_W &\le&\norm{u}_V\\
      Aw &=& u
    \end{array}
    \right.
  \end{gather}
\end{Lemma}

\begin{Problem}{inf-sup-equivalence}
  Prove Lemma 2.1.16.
\begin{solution}
  We have to prove the following statements are equivalent to the inf-sup condition:
  \begin{align}
    \forall u\in V \;\exists w\in W \;:\; a(u,w) \ge \ellipa \norm{u}_V\norm{w}_W     \tag{1}
  \end{align}
  \begin{align}
   \begin{aligned}
    \forall u\in V \;\exists w\in W \;:\;
    \begin{cases}
      \norm{w}_W \le\norm{u}_V\\
      a(u,w) \ge \ellipa \norm{u}_V^2
    \end{cases}
   \end{aligned}
   \tag{2}
  \end{align}
  \begin{align}
   \begin{aligned}
    \forall u\in V \;\exists w\in W \;:\;
    \begin{cases}
      \ellipa \norm{w}_W \le \norm{u}_V\\
      Aw = u
      \end{cases}
  \end{aligned}
      \tag{3}
  \end{align}
  The inf-sup condition reads
  \begin{align}
   \inf_{u\in V} \sup_{w\in W} \frac{a(u,w)}{\norm{u}_V\norm{w}_W} \ge \ellipa \tag{IS}
  \end{align}

  $(IS)\Rightarrow(3)$\\
  The inf-sup condition is equivalent to $A: ker(A)^\perp\to V^*$ being an isomorphism.
  By the Riesz representation theorem there exists for a given $u\in V$ a $w\in W$ such that
  $Aw=J u$ where $J$ is the Riesz map. Hence, it holds
  \begin{align}
a(u,w)=\langle A w, u\rangle = \langle J u, u\rangle = \norm{u}_V^2.
  \end{align}
 Due to $w\in ker(A)^\perp$, $A^{-1}$ is bounded and
 \begin{align}
 \norm{w}_W=\norm{A^{-1}u}_V\leq \frac{1}{\ellipa}\norm{u}_V.
  \end{align}

  $(3)\Rightarrow(2)$ \\
  Define $\tilde{w}=\ellipa w$. Then,
  \begin{align}
a(u,\tilde{w})=\ellipa a(u,w) = \ellipa \norm{u}_V^2
  \end{align}
  and
  \begin{align}
\norm{\tilde{w}}_W=\ellipa \norm{w}_W\leq\norm{u}_V
  \end{align}

  $(2)\Rightarrow(1)$ \\
  \begin{align}
  a(u,w) \ge \ellipa \norm{u}_V^2 \ge \ellipa \norm{u}_V \norm{w}_W
  \end{align}

  $(1)\Rightarrow(IS)$ \\
  \begin{align}
  &\forall u\in V \exists w\in W: a(u,w)\ge \ellipa \norm{u}_V\norm{w}_W\\
  &\Rightarrow \forall u\in V: \sup_{w\in W} \frac{a(u,w)}{\norm{w}_W}\ge \ellipa \norm{u}_V\\
  &\Rightarrow \inf_{u\in V}\sup_{w\in W} \frac{a(u,w)}{\norm{w}_W\norm{u}_V}\ge \ellipa
  \end{align}
\end{solution}
\end{Problem}


%%%%%%%%%%%%%%%%%%%%%%%%%%%%%%%%%%%%%%%%%%%%%%%%%%%%%%%%%%%%%%%%%%%%%%
%%%%%%%%%%%%%%%%%%%%%%%%%%%%%%%%%%%%%%%%%%%%%%%%%%%%%%%%%%%%%%%%%%%%%%
\section{Infinite dimensional Hilbert spaces}
%%%%%%%%%%%%%%%%%%%%%%%%%%%%%%%%%%%%%%%%%%%%%%%%%%%%%%%%%%%%%%%%%%%%%%
%%%%%%%%%%%%%%%%%%%%%%%%%%%%%%%%%%%%%%%%%%%%%%%%%%%%%%%%%%%%%%%%%%%%%%

\begin{intro}
  In the previous section, we derived quantitative conditions to
  ensure the invertibility of a matrix $A$ or its restriction to its
  cokernel $\ortho{\ker A}$. The arguments there have a natural
  extension to infinite dimensional Hilbert spaces, which we will
  derive in this section. We already saw in
  \slideref{Problem}{unbounded-inverse} that we may run into trouble
  if the range of $A$ is not closed. On the other hand, it turns out
  that most notions of linear algebra related to orthogonality can be
  maintained in Hilbert spaces if closed subspaces are considered.
  We begin by citing the most important results.

  The presentation here is neither complete nor self-contained. It
  just highlights some of the important facts. In particular, we
  assume the validity of the \putindex{Riesz representation theorem}
  and the existence of a \putindex{Schauder basis} a priori.
\end{intro}

\begin{Notation}{musical-isomorphisms}
  The \putindex{Riesz representation theorem} induces isomorphisms
  between a Hilbert space $V$ and its dual $V^*$. We call them
  \define{Riesz isomorphism}s. Following custom in differential
  geometry, we denote them as \define{musical isomorphism}s
  \begin{xalignat}2
    I_\flat\colon V&\to V^*
    &I_\sharp\colon V^*&\to V\\
    v&\mapsto v^\flat
    &\phi&\mapsto \phi^\sharp.
  \end{xalignat}
  \index{sharp@$\flat$}\index{flat@$\sharp$}
\end{Notation}

\begin{Problem*}{riesz-h10}{Riesz Isomorphisms}
  Let $V=H^1_0(\domain)$ on the domain $\domain=(0,1)$ be equipped with the
  inner product
  \begin{gather}
    \scal(u,v)_V = \int_0^1 u'v'\dt.
  \end{gather}
  \begin{enumerate}
  \item Show that for any function $f\in L^2(\domain)$ the functional
    $\phi_f$ defined by
    \begin{gather}
      \phi_f(v) = \int_0^1 f v \dt
    \end{gather}
    is in $V^*$, even if $f\not\in V$.
  \item Discuss this for the function $f\equiv 1$ and compute $f^\sharp$.
  \end{enumerate}
\end{Problem*}

\begin{Example}{solution-pressure}
  When we look at computing the pressure in Stokes' equations, we have
  to solve a weak formulation of the form: find $p\in Q$ such that
  \begin{gather}
    \form(\div v,p) = g(v) \qquad \forall v\in V.
  \end{gather}
  Here $g\in V^*$ consists of the parts of the momentum equation not
  containing the pressure. If we define $B^\transpose$ by
  \begin{gather}
    \scal(B^\transpose q,v)_{V^*\times V} = \form(\div v,q)
    \qquad \forall v\in V, q \in Q,
  \end{gather}
  then the operator form of the problem posed in $V^*$ is
  \begin{gather}
    B^\transpose p = g
  \end{gather}
  Since this equation involves two different spaces, we require an
  extension of \slideref{Lemma}{ker-coker-rn}.
\end{Example}

\begin{Definition}{polar-orthogonal}
  Let $W\subset V$ be a subspace of a Hilbert space $V$. We define its
  \define{orthogonal complement} $\ortho{W}\subset V$ and its
  \define{polar space} $\polar{W}\subset V^*$ by
  \begin{gather}
    \label{eq:infsup:7}
    \begin{aligned}
    \ortho{W} &= \bigl\{v\in V &\big|&& \scal(v,w)_{V} &= 0
    &\forall\,w&\in W\bigr\},
    \\
    \polar{W} &= \bigl\{f\in V^* &\big|&& \scal(f,w)_{V^*\times V} &= 0
    &\forall\,w&\in W\bigr\}.
    \end{aligned}
  \end{gather}
  For a subspace $U\subset V^*$, we define its polar space
  \begin{gather}
    \dualpolar{U} = \bigl\{v\in V \quad\big|\quad \scal(u,v)_{V^*\times V} = 0
    \quad\forall\,u\in U\bigr\}
  \end{gather}
\end{Definition}

\begin{Problem}{polar-orthogonal}
  Show that $\polar{W} = I_\flat \ortho{W}$ and $\dualpolar{U} = I_\sharp \ortho{U}$.
\end{Problem}

\begin{Lemma}{orthogonal-closed}
  The polar space $\polar{W}$ and the orthogonal complement $\ortho{W}$ of a
  subspace $W\subset V$ are both closed. So is the polar space $\dualpolar{U}$
  of a subspace $U\subset V^*$.
\end{Lemma}

\begin{proof}
  Consider the mapping
  \begin{align}
    \Phi_w\colon V^* &\to \R,\\
    \phi&\mapsto \scal(\phi,w)_{V^*\times V}.
  \end{align}
  For any $w$, the kernel of $\Phi_w$ is closed as
  the pre-image of a closed set. $\polar{W}$ is closed since it is the
  intersection of these kernels for all $w\in W$.

  The inner product is continuous on $V\times V$. Therefore, the
  mapping
  \begin{align}
    \Psi_w\colon V &\to \R,\\
    v&\mapsto \scal(v,w),
  \end{align}
  is continuous. The argument continues as above. Similar for $\dualpolar{U}$.
\end{proof}

\begin{Theorem}{orthogonal-complement}
  Let $W$ be a subspace of a Hilbert space $V$ and $\ortho{W}$ its
  orthogonal complement. Then, $\ortho{W} =
  \ortho{\overline{W}}$. Furthermore, there holds
  \begin{gather}
    V = W \oplus \ortho{W}
    \qquad\Longleftrightarrow\qquad
    \text{$W$ is closed.}
  \end{gather}
\end{Theorem}

\begin{proof}
  Clearly, $\ortho{\overline{W}} \subset \ortho{W}$ since
  $W\subset\overline{W}$. Let now $u\in \ortho{W}$. Then, $\phi =
  \scal(u,\cdot)$ is a continuous linear functional on $V$. Therefore,
  if a sequence $w_n \subset W$ converges to $w\in \overline{W}$, we
  have
  \begin{gather}
    \scal(u,w) = \lim_{n\to\infty} \scal(u,w_n) = 0.
  \end{gather}
  Hence, $u\in \ortho{\overline{W}}$ and $\ortho{W} = \ortho{\overline{W}}$.

  Now, the ``only if'' follows by the fact, that if $W$ is not
  closed, there is an element $w\in \overline{W}$ but not in $W$ such that
  $\scal(w,u)=0$ for all $u\in \ortho{W}$. Thus, $w\not\in \ortho{W}$ and
  consequently $w\not\in \ortho{W} \oplus W$.

  Let now $W$ be closed. We show that there is a unique decomposition
  \begin{gather}
    \label{eq:infsup:8}
    v = w + u,\qquad w\in W, \;u\in \ortho{W},
  \end{gather}
  which is equivalent to $V = W \oplus \ortho{W}$. Uniqueness follows,
  since
  \begin{gather}
    v = w_1+u_1 = w_2+u_2
  \end{gather}
  implies that for any $y\in V$
  \begin{gather}
    0 = \scal(w_1-w_2+u_1-u_2,y) = \scal(w_1-w_2,y) + \scal(u_1-u_2,y).
  \end{gather}
  Choosing $y=u_1-u_2$ and $w_1-w_2$ in turns, we see that one of the
  inner products vanishes for orthogonality and the other implies that
  the difference is zero.

  If $v\in W$, we choose $w=v$ and $u=0$. For $v\not\in W$, we prove
  existence by considering that due to the closedness of $W$ there holds
  \begin{gather}
    d=\inf_{w\in W} \norm{v-w} >0.
  \end{gather}
  Let $w_n$ be a minimizing sequence. Using the parallelogram identity
  \begin{gather}
    \norm{a+b}^2+\norm{a-b}^2 = 2\norm{a}^2+2\norm{b}^2,
  \end{gather}
  we prove that $\{w_n\}$ is a Cauchy sequence by
  \begin{align}
    \norm{w_m-w_n}^2 &= \norm{(v-w_n)-(v-w_m)}^2\\
    &= 2\norm{v-w_n}^2+2\norm{v-w_m}^2-\norm{2v-w_m-w_n}^2\\
    &= 2\norm{v-w_n}^2+2\norm{v-w_m}^2-4\norm*{v-\frac{w_m+w_n}2}^2\\
    &\le 2\norm{v-w_n}^2+2\norm{v-w_m}^2-4d^2,
  \end{align}
  since $(w_m+w_n)/2\in W$ and $d$ is the infimum. Now we use the
  minimizing property to obtain
  \begin{gather}
    \lim_{m,n\to\infty}\norm{w_m-w_n}^2 = 2d^2-2d^2 -4d^2=0.
  \end{gather}
  By completeness of $V$, $w=\lim w_n$ exists and by the closedness of
  $W$, we have $w\in W$. Let $u=v-w$. By continuity of the norm, we
  have $\norm{u}=d$. It remains to show that $u\in \ortho{W}$. To this
  end, we introduce the variation $w+\epsilon \tilde w$ with $\tilde
  w\in W$ to obtain
  \begin{align}
    d^2 &\le \norm{v-w-\epsilon \tilde w}^2\\
    &= \norm{u}^2-2\epsilon\scal(u,\tilde w)+\epsilon^2 \norm{\tilde w},
  \end{align}
  implying for any $\epsilon>0$
  \begin{gather}
    0\le-2\epsilon\scal(u,\tilde w)+\epsilon^2 \norm{\tilde w},
  \end{gather}
  which requires $\scal(u,\tilde w) = 0$.
\end{proof}

\begin{Definition}{orthogonal-projection}
  Let $V$ be a Hilbert space and $W\subset V$ be a closed
  subspace. For a vector $v\in V$, let $v=w+u$ be the unique
  decomposition with $w\in W$ and $u\in \ortho{W}$. Then we call $w$ and
  $u$ the \define{orthogonal projection}s of $v$ into $W$ and $\ortho{W}$,
  respectively. We write
  \begin{gather}
    \Pi_W v = w, \qquad \Pi_{\ortho{W}} v = u.
  \end{gather}
\end{Definition}

% \begin{Lemma}{polar-orthogonal-hilbert}
%   Let $V$ be a Hilbert space and $W\subset V$ be a closed
%   subspace. Then, the polar space $\polar{W}\subset V^*$ and the orthogonal
%   space $\ortho{W}$ can be isometrically identified by \putindex{Riesz
%     representation}.
% \end{Lemma}

% \begin{proof}
%   For every $f$ in the
%   dual of $\ortho{W}$, define $g\in V^*$ by
%   \begin{gather}
%     \scal(g,v)_{V^*\times V} =
%     \scal(f,\Pi_{\ortho{V}}v)_{(\ortho{V})^*\times \ortho{V}}.
%   \end{gather}
%   Clearly, $g(v)=0$ for $v\in W$, therefore $g\in \polar{W}$.
% \end{proof}

% \begin{Corollary}

% \end{Corollary}


\begin{Theorem*}{closed-range}{Closed Range Theorem}
  Let $V,W$ be Hilbert spaces and $A\colon V\to W$ a continuous linear
  operator. Then, the following statements are equivalent:
  \begin{gather}
    \label{eq:infsup:9}
    \begin{split}
      \range A &\text{ is closed in } W,\\
      \range{A^\transpose} &\text{ is closed in } V^*,\\
      \range A &= \dualpolar{\ker{A^\transpose}},\\
      \range{A^\transpose} &= \polar{\ker A}.
    \end{split}
  \end{gather}
\end{Theorem*}

\begin{remark}
  This is the famous \emph{\putindex{closed range theorem}} by Banach.
  It actually holds under weaker assumptions, for instance $V,W$ only
  Banach spaces. The proof can be found for instance
  in~\cite[p.~205--209]{Yosida80}.
\end{remark}

\begin{Theorem*}{open-mapping}{Open Mapping Theorem}
  Let $A\colon V\to W$ be continuous and surjective. Then, the image
  $A(U)\subset W$ of any open set $U\subset V$ is open.
\end{Theorem*}

\begin{remark}
  This is the \emph{\putindex{open mapping theorem}} by Banach. The
  proof can be found for instance in~\cite[p.75--76]{Yosida80}.
\end{remark}

\begin{Lemma}{closed-infsup}
  Let $A\colon V\to W$ be continuous. Then, $\range A$ is closed in
  $W$ if and only if there exists $\ellipa>0$ such that
  \begin{gather}
    \label{eq:infsup:10}
    \forall w\in \range A\;
    \exists v\in V\quad
    Av = w
    \;\wedge\;
    \ellipa \norm{v}_V \le \norm{w}_W.
  \end{gather}
\end{Lemma}

\begin{proof}
  We first show that the inf-sup condition~\eqref{eq:infsup:10}
  implies $\range A$ closed. To this end, let $\{w_n\}$ be a Cauchy
  sequence in $\range A$ converging to a point $w\in W$. By the assumption,
  there
  is a sequence $\{v_n\}$ in $V$ such that $Av_n = w_n$ and
  $\ellipa \norm{v_n} \le \norm{w_n}$. Hence, using
  \begin{gather}
    \norm{v_m-v_n}_V \le \frac1\ellipa \norm{w_m-w_n}_W,
  \end{gather}
  we realize that $\{v_n\}$ is a Cauchy sequence in $V$. Therefore, $v_n\to v\in
  V$ and due to continuity of $A$ we obtain $Av=w$ and thus $w\in
  \range A$.

  Conversely, let $\range A$ be closed in $W$. Thus, it is a Banach
  space and the \putindex{open mapping theorem} applies to $A\colon
  V\to\range A$. We map the open unit ball $B_1(0)\subset V$ and
  obtain that $A(B_1(0))$ is open in $\range A$, implying that there
  is an open ball $B_\delta(0) \subset A(B_1(0))$. This is sufficient
  to construct $v$:

  Let $w\in\range A$. Then,
  \begin{gather}
    \tilde w = \frac\delta2 \frac{w}{\norm{w}}
    \in B_\delta(0) \subset A(B_1(0)).
  \end{gather}
  Hence, there is $v\in V$ with $\norm{v}<1$ such that $Av=\tilde w$,
  which proves the lemma.
\end{proof}

\begin{Theorem}{infsup-well-equivalence}
  Let $a(\cdot,\cdot)$ on $V\times W$ be a bounded bilinear form % such that
%  \begin{gather}
%    a(v,w) \le M \norm{v}_V \norm{w}_W,
%  \end{gather}
  and $A\colon V\to W^*$ its associated operator.
  Then, the following statements are equivalent:
  \begin{enumerate}
  \item There exists $\ellipa>0$ such that
    \begin{gather}
      \label{eq:infsup:11}
      \inf_{w\in W}\sup_{v\in V}
      \frac{a(v,w)}{\norm{v}_V\norm{w}_W}
      \ge \ellipa.
    \end{gather}
  \item The operator $A^\transpose\colon W\to \polar{\ker A}$ is an isomorphism and
    \begin{gather}
      \label{eq:infsup:12}
      \norm{A^\transpose w}_{V^*} \ge \ellipa \norm{w}_{W} \qquad\forall w\in W.
    \end{gather}
  \item The operator $A\colon \ortho{\ker A}\to W^*$ is an isomorphism
    and
    \begin{gather}
      \label{eq:infsup:13}
      \norm{Av}_{W^*} \ge \ellipa\norm{v}_V\qquad \forall v\in \ortho{\ker A}.
    \end{gather}
  \end{enumerate}
\end{Theorem}

\begin{proof}
  First, we show the equivalence of the first two statements. Let us
  use equivalently to the inf-sup condition~\eqref{eq:infsup:11}
  \begin{gather}
    \norm{A^\transpose w}_{V^*}
    = \sup_{v\in V}\frac{\scal(A^\transpose w,v)}{\norm{v}_V}
    = \sup_{v\in V}\frac{a(v,w)}{\norm{v}_V}
    \ge \ellipa\norm{w} \qquad
    \forall w\in W.
  \end{gather}
  Thus, equations~\eqref{eq:infsup:11} and~\eqref{eq:infsup:12} are
  equivalent and we have already proven that the second statement
  implies the first. It remains to show the $A^\transpose$ is an isomorphism
  from $W$ onto $\polar{\ker A}$. Equation~\eqref{eq:infsup:12} implies that
  $A^\transpose\colon W \to \range{A^\transpose}$ is an isomorphism and its inverse is
  bounded by $1/\ellipa$ (multiply both sides by $A^{-1}$). Using
  \slideref{Lemma}{closed-infsup}, we obtain that $\range{A^\transpose}$ is
  closed in $V^*$ and the \putindex{closed range theorem} settles the
  issue.

  In order to prove equivalence of the second and third statement, we
  use the result of \slideref{Problem}{polar-orthogonal} to isometrically
  identify $(\ortho{\ker A})^*$ with $\polar{\ker A}$. Thus, $A$ is an
  isomorphism from $\ortho{\ker A}$ onto $W^*$ if and only if $A^\transpose$ is an
  isomorphism from $W$ onto $(\ortho{\ker A})^* = \polar{\ker A}$. and
  \begin{gather}
    \norm{A}_{W^*\to \ortho{\ker A}} = \norm{A^\transpose}_{\polar{\ker A}\to W}.
  \end{gather}
\end{proof}

\begin{Corollary}{infsup-well-posedness1}
  Let $a(\cdot,\cdot)$ on $V\times W$ be a bounded bilinear form. %such that
  %\begin{gather}
  %  a(v,w) \le M \norm{v}_V \norm{w}_W.
  %\end{gather}
  Let the inf-sup-condition
  \begin{gather}
    \inf_{w\in W}\sup_{v\in V}
    \frac{a(v,w)}{\norm{v}_V\norm{w}_W}
    \ge \ellipa > 0
  \end{gather}
  hold.  Then, the problem finding $w\in W$ such that
  \begin{gather}
    a(v,w) = f(v) \qquad\forall v\in V,
  \end{gather}
  has a unique solution for $f\in \polar{\ker A}$ and
  \begin{gather}
    \norm{w}_W \le \frac1\ellipa \norm{f}_{V^*}.
  \end{gather}
  The opposite implication holds true.
\end{Corollary}

\begin{remark}
  \slideref{Corollary}{infsup-well-posedness1} exhibits an asymmetry
  between the left and right argument. In particular, we obtain a
  unique solution only for the adjoint operator $A^\transpose$, which is
  exactly what we need, when we compute say a pressure from the
  divergence of a velocity field. In general, we consider the
  restriction of $f$ to the polar set of the kernel in the above
  well-posedness result detrimental and would prefer a result that
  holds for all $f\in V^*$. This on the other hand requires
  $\ker A=\{0\}$, or $\overline{\range{A^\transpose}} = W^*$. Then, on the
  other hand, we see that $\range{A^\transpose}$ is closed since $\range{A}$ is
  closed and the closed range theorem holds. Therefore, we obtain the
  following theorem for the case that we require a unique solution for
  all right hand sides.
\end{remark}

\begin{Theorem}{infsup-well-posedness2}
  Let $a(\cdot,\cdot)$ on $V\times W$ be a bounded bilinear form. % such that
%  \begin{gather}
%    a(v,w) \le M \norm{v}_V \norm{w}_W.
%  \end{gather}
  Let for some $\ellipa>0$ the inf-sup-conditions
  \begin{align}
    \inf_{w\in W}\sup_{v\in V}
    \frac{a(v,w)}{\norm{v}_V\norm{w}_W}
    &\ge \ellipa,\\
    \inf_{v\in V}\sup_{w\in W}
    \frac{a(v,w)}{\norm{v}_V\norm{w}_W}
    &\ge \ellipa
  \end{align}
  hold.  Then, the problem finding $v\in V$ such that
  \begin{gather}
    a(v,w) = f(w) \qquad\forall w\in W,
  \end{gather}
  has a unique solution for $f\in W^*$ and
  \begin{gather}
    \norm{v}_V \le \frac1\ellipa \norm{f}_{W^*}.
  \end{gather}
\end{Theorem}

\begin{Problem}{closed-range}
  Assume for $A\colon V\to W^*$ that $\range A$ is closed. Show
  \begin{enumerate}
  \item $A:V\to W^*$ is surjective if and only if $A^\transpose$ is
    injective.
  \item Show that with all other assumptions unchanged, the second
    inf-sup condition in \slideref{Theorem}{infsup-well-posedness2} is
    equivalent to $A^*$ injective.
  \item \slideref{Theorem}{infsup-well-posedness2} seems excessive if
    we only want on of the statements
    \begin{xalignat}3
      \forall f&\in W^*& \exists u&\in V& a(u,w) &= f(w)\\
      \forall g&\in V^*& \exists u&\in W& a(v,u) &= g(v)
    \end{xalignat}
    Is this true or is there equivalence of both inf-sup conditions
    with only one of these statements?
  \end{enumerate}
\end{Problem}

\begin{remark}
  If we compare \slideref{Theorem}{infsup-well-posedness2} with
  \slideref{Corollary}{infsup-well-posedness1}, we see that the only
  difference lies in the fact that the second inf-sup condition
  ensures surjectivity of $A$ by injectivity of $A^\transpose$. In some cases
  it may be impossible difficult to prove both inf-sup conditions. Then, it is
  sufficient to prove one inf-sup condition, say the first, and then
  only injectivity of $A^\transpose$. Although we verify less than the
  assumptions of \slideref{Theorem}{infsup-well-posedness2}, the
  closed range theorem saves us from the additional work. We further
  note that this notion is symmetric between $A$ and $A^\transpose$, that is,
  it is sufficient to prove inf-sup for either operator and
  injectivity for the other.
\end{remark}

%%%%%%%%%%%%%%%%%%%%%%%%%%%%%%%%%%%%%%%%%%%%%%%%%%%%%%%%%%%%%%%%%%%%%%
%%%%%%%%%%%%%%%%%%%%%%%%%%%%%%%%%%%%%%%%%%%%%%%%%%%%%%%%%%%%%%%%%%%%%%
\section{The inf-sup condition for mixed problems}
%%%%%%%%%%%%%%%%%%%%%%%%%%%%%%%%%%%%%%%%%%%%%%%%%%%%%%%%%%%%%%%%%%%%%%
%%%%%%%%%%%%%%%%%%%%%%%%%%%%%%%%%%%%%%%%%%%%%%%%%%%%%%%%%%%%%%%%%%%%%%

\begin{intro}
  In the previous section, we have developed a framework for
  well-posedness of problems which are not $V$-elliptic. In principle,
  this theory can be applied to the bilinear form
  $\mathcal A((u,p),(v,q))$ as a whole. On the other hand, we can
  formally split the solution of a constrained minimization problem
  into the reduced problem and then computing the Lagrange multiplier,
  which more clearly exhibits the relation of the two spaces $V$ and
  $Q$ involved in the mixed formulation. Here are the resulting
  theorems.
\end{intro}

\begin{Definition}{mixed-weak}
  The abstract saddle-point problem in weak form reads:
  find $(u,p)\in V\times Q$ such that
  \begin{gather}
    \label{eq:saddle-point-weak}
    \arraycolsep.1em
    \begin{matrix}
      a(u,v) &+& b(v,p) &=& f(v) &\quad&\forall v\in V, \\
      b(u,q) && &=& g(q) &&\forall q\in Q.
    \end{matrix}
  \end{gather}
  Here, $V$ and $Q$ are Hilbert spaces chosen, such that $a(.,.)$ and
  $b(.,.)$ are bounded bilinear forms on $V\times V$ and $V\times Q$,
  respectively. The bounds $\bounda$ and $\boundb$ are such that
  \begin{gather}
    \label{eq:saddle-point-bounds}
    \sup_{u,v\in V}\frac{a(u,v)}{\norm{u}_V\norm{v}_V} = \bounda,
    \qquad
    \sup_{\substack{v\in V\\q\in Q}}\frac{b(v,q)}{\norm{v}_V\norm{q}_G} = \boundb.
  \end{gather}
\end{Definition}


\begin{Theorem}{infsup-mixed1}
  Let $V$ and $Q$ be Hilbert spaces. Let $a(\cdot,\cdot)$ and
  $b(\cdot,\cdot)$ be bounded bilinear forms. Then, the weak
  formulation find $u\in V$ and $p\in Q$ such that
  \begin{multline}
    \mathcal A\mixedform(u,p,v,q)
    \equiv a(u,v) + b(v,p) + b(u,q)
    \\
    = \scal(f,v)+\scal(g,q)
    \quad\forall v\in V, q\in Q,
  \end{multline}
  has a unique solution for any $f\in V^*$ and any $g\in Q^*$ if and
  only if there exists $\ellipa>0$ such that
  \begin{gather}
    \label{eq:infsup:system-infsup}
    \inf_{\substack{u\in V\\p\in Q}}
    \sup_{\substack{v\in V\\q\in Q}}
    \quad \frac{\mathcal A\mixedform(u,p,v,q)}{\norm{(u,p)}_{V\times
        Q}\norm{(v,q)}_{V\times Q}} \ge \ellipa.
  \end{gather}
\end{Theorem}

\begin{proof}
  Straight application of \slideref{Theorem}{infsup-well-posedness2}.
\end{proof}

Often, like in the case of Stokes' equations, the properties of the
form $a(\cdot,\cdot)$ or the form $b(\cdot,\cdot)$ are already known
and you want to combine them to a mixed formulation. Thus, the theorem
above is unwieldy, since it requires to do the analysis from
scratch. We thus provide an equivalence theorem, which allows us to
separate the properties of the bilinear forms.

\begin{Theorem}{infsup-mixed2}
  Let $V$ and $Q$ be Hilbert spaces and let
  \begin{gather}
    \begin{split}
      \ker B &= \bigl\{v\in V \big| b(v,q) = 0 \;\forall q\in Q\bigr\}.
    \end{split}
  \end{gather}
  Then, the mixed problem finding $(u,p)\in V\times Q$ such that
  \begin{gather}
    a(u,v) + b(v,p) + b(u,q) = f(v) \quad\forall v\in V, q\in Q,
  \end{gather}
  is well-posed if and only if the reduced problem
  finding $u\in \ker B$ such that
    \begin{gather}
      a(u,v) = f(v) \quad\forall v\in \ker B
    \end{gather}
    is well-posed for any $f\in V^*$ and there is a positive constant
    $\infsupc$ such that
    \begin{gather}
      \inf_{q\in Q}\sup_{v\in V} \frac{b(v,q)}{\norm{v}_V\norm{q}_Q} \ge \infsupc.
    \end{gather}
\end{Theorem}

\begin{proof}
  In order to show the ``if'', we note that by well-posedness of the
  reduced problem, $u\in V$ is well-determined and bounded by the data
  $f\in V^*$ without knowledge of the Lagrange
  multiplier. Furthermore, there holds $b(u,q) = 0$.
  
  Entering this into the mixed formulation, the Lagrange multiplier
  $p$ is determined by
  \begin{gather}
    b(v,p) = f(v) - a(u,v), \qquad\forall v\in V.
  \end{gather}
  Applying \slideref{Corollary}{infsup-well-posedness1} to the
  bilinear form $b(.,.)$, we deduce that this equation has a unique
  solution $p\in Q$ if and only if $f(v)-a(u,v) \in \polar{\ker B}$, which
  is true due to the statement of the reduced problem.

  For the only if, we note that choosing $v=0$ in the mixed problem,
  we see that there holds
  \begin{gather}
    b(u,q) = 0 \qquad \forall q\in Q,
  \end{gather}
  and thus $u\in \ker B$. for such $u$ holds in the mixed formulation
  \begin{gather}
    f(v) = a(u,v) + b(v,p) = a(u,v) \qquad \forall v \in V.
  \end{gather}
  In order to deduce the inf-sup condition, we note that
  well-posedness of the mixed problem implies the inf-sup
  condition~\eqref{eq:infsup:system-infsup}. Since the infimum will
  not decrease by taking a subset, we confine the set to $u=0$ and
  obtain
  \begin{align}
    &\inf_{p\in Q}\sup_{v\in V} \frac{b(v,p)}{\norm{v}_V\norm{q}_Q}
    \\ = &
           \inf_{p\in Q}
           \sup_{\substack{v\in V\\q\in Q}}
    \frac{a(0,v) + b(v,p) + b(0,q)}{\norm{(0,p)}_{V\times Q} \norm{(v,q)}_{V\times Q}}
    \\ \ge&
            \inf_{\substack{u\in U\\p\in Q}}
    \sup_{\substack{v\in V\\q\in Q}}
    \frac{a(u,v) + b(v,p) + b(u,q)}{\norm{(u,p)}_{V\times Q} \norm{(v,q)}_{V\times Q}} \\\ge& \ellipa > 0
  \end{align}
\end{proof}

\begin{Problem}{inhomogeneous-continuity}
  Show that \slideref{Theorem}{infsup-mixed2} can be extended to the
  case with right hand side $f(v)+g(q)$ with $g\in Q^*$.

\begin{solution}
  We want to solve the problem
  \begin{align}
    a(u,v) + b(v,p) + b(u,q) = f(v)+g(q) \quad\forall v\in V, q\in Q,
  \end{align}
  where $b(v,p)$ fulfills a inf-sup condition.

  \begin{enumerate}
  \item Due to \slideref{Theorem}{infsup-well-equivalence}, the
    operator $B: V\to Q^*$ is surjective and thus, there exists
    $u_g\in V$ such that
    \begin{gather}
      b(u_g,q) = q(q) \quad\forall q\in Q.
    \end{gather}
  \item Now consider the function $u_0 = u-u_g$. For $u$ to solve the
    original problem $u_0$ has to solve
    \begin{align}
      a(u_0+u_g,v) + b(v,p) + b(u_0+u_g,q) = f(v)+g(q) \quad\forall v\in V, q\in Q\\
      \Leftrightarrow a(u_0,v) + b(v,p) + b(u_0,q) = f(v)-a(u_g,v) \quad\forall v\in V, q\in Q
    \end{align}
  \item Due to $a(u_g,v) \leq \bounda \norm{u_g}_V\norm{v}_V$,
    the right-hand side $f(\cdot)-a(u_g,\cdot)$ is in $V^*$
    and we are in the setting of
    \slideref{Theorem}{infsup-mixed2}.
  \end{enumerate}
\end{solution}
\end{Problem}

\begin{remark}
  Since $V$ is a Hilbert space, the decomposition
  $V = \ker B \oplus \ker B^\perp$ is uniquely determined and there is
  a corresponding decomposition $V^* = \polar{(\ker B^\perp)} \oplus \polar{\ker B} = I_\flat\ker B \oplus I_\flat \ortho{\ker B} $,
  such that $f = f^0+f^\perp$ above. The way we solve the reduced
  problem first and then compute the Lagrange multiplier implies that
  the solution $u$ only depends on $f^\perp$ only.
\end{remark}

\begin{remark}
  We have imposed well-posedness of the reduced problem only in an
  abstract way. Depending on $a(.,.)$ we can formulate two conditions:
  ellipticity on $\ker B$ or inf-sup stability on $\ker B$. Indeed,
  most problems considered in this class will have symmetric bilinear
  forms $a(.,.)$, such that ellipticity serves as our usual
  assumption.  In these cases, note that $V$-ellipticity already
  implies the well-posedness on $\ker B$.
\end{remark}

\begin{Notation}{v0-kernel}
  Since the kernels of the bilinear form $b(\cdot,\cdot)$ play an
  important role, we abbreviate
  \begin{gather}
    \begin{split}
      V^0 = \ker B &= \bigl\{v\in V \;\big|\; b(v,q) = 0 \quad\forall q\in Q\bigr\},\\
      Q^0 = \ker{B^\transpose} &= \bigl\{q\in Q \;\big|\; b(v,q) = 0 \quad\forall v\in V\bigr\}.
    \end{split}
  \end{gather}
  We als define for $g\in Q^*$ and for $f\in V^*$ the affine spaces
  \begin{gather}
    \begin{split}
      V^g &= \bigl\{v\in V \;\big|\; b(v,q) = g(q) \quad\forall q\in Q\bigr\},\\
      Q^f &= \bigl\{q\in Q \;\big|\; b(v,q) = f(v) \quad\forall v\in V\bigr\}.
    \end{split}
  \end{gather}
\end{Notation}

\begin{intro}
  We summarize the results of this section i theorem for
  well-posedness of the mixed formulation in
  \slideref{Definition}{mixed-weak} with simplified assumptions.  It
  will be the basis for further results in this course. We know from
  the discussion above that this assumption is only sufficient and
  weaker conditions may be imposed on $a(.,.)$. But indeed, it helps
  us through a lot of problems and is a good compromise between
  generality and ease of use.
\end{intro}

\begin{Theorem}{mixed-elliptic}
  Let the bilinear form $a(.,.)$ be positive semi-definite on $V$ and
  elliptic on $V^0 = \ker B$. Let the bilinear form $b(\cdot,\cdot)$
  be inf-sup stable. Thus, there are constants $\ellipa>0$ and
  $\infsupc>0$
  \begin{gather}
    \inf_{v\in V^0} \frac{a(v,v)}{\norm{u}_V^2} = \ellipa,  \qquad
    \inf_{q\in Q}\sup_{v\in V} \frac{b(v,q)}{\norm{v}_V\norm{q}_Q} = \infsupc.
  \end{gather}
  Then, the abstract mixed proble in \slideref{Definition}{mixed-weak}
  has a unique solution.
\end{Theorem}

\begin{Problem}{mixed-elliptic}
  Derive bounds for the solutions $u\in V$ and $p\in Q$ using the
  assumptions of the previous theorem.
\end{Problem}

%%%%%%%%%%%%%%%%%%%%%%%%%%%%%%%%%%%%%%%%%%%%%%%%%%%%%%%%%%%%%%%%%%%%%%
%%%%%%%%%%%%%%%%%%%%%%%%%%%%%%%%%%%%%%%%%%%%%%%%%%%%%%%%%%%%%%%%%%%%%%
\section{Galerkin approximation of mixed problems}
%%%%%%%%%%%%%%%%%%%%%%%%%%%%%%%%%%%%%%%%%%%%%%%%%%%%%%%%%%%%%%%%%%%%%%
%%%%%%%%%%%%%%%%%%%%%%%%%%%%%%%%%%%%%%%%%%%%%%%%%%%%%%%%%%%%%%%%%%%%%%

\begin{intro}
  The Galerkin approximation of mixed problems starts out the same way
  as for elliptic problems, namely, choose discrete subspaces
  $V_h\subset V$ and $Q_h \subset Q$. There is a fundamental
  difference though: the inf-sup condition is not inherited
  automatically on the subspaces like $V$-ellipticity. It actually becomes
  an additional requirement on the choice of $V_h$ and
  $Q_h$. We will thus work our way in several steps towards the final
  result.
\end{intro}

\begin{Definition}{mixed-galerkin}
  Let $V_h\subset V$ and $Q_h\subset Q$. Then, the \define{Galerkin
    approximation} of the mixed problem in
  \slideref{Definition}{mixed-weak} is: find
  $(u_h, p_h)\in V_h\times Q_h$ such that
  \begin{gather}
    \label{eq:galerkin:3}
        \arraycolsep.1em
    \begin{matrix}
      a(u_h,v_h) &+& b(v_h,p_h) &=& f(v_h) &\quad&\forall v_h\in V_h, \\
      b(u_h,q_h) && &=& g(q_h) &&\forall q_h\in Q_h.
    \end{matrix}
  \end{gather}
\end{Definition}

\begin{Definition}{kerbh}
  Let $V_h\subset V$ and $Q_h\subset Q$. Then, we define the subspace
  \begin{gather}
    \label{eq:galerkin:1}
    V_h^0 = \ker{B_h} = \bigl\{v_h\in V_h \big|
    b(v_h, q_h) = 0 \quad\forall q_h\in Q_h
    \bigr\}.
  \end{gather}
  Furthermore, we define the affine space
  \begin{gather}
    \label{eq:galerkin:2}
    V_h^g = \bigl\{v_h\in V_h \big|
    b(v_h, q_h) = g(q) \quad\forall q_h\in Q_h
    \bigr\}.
  \end{gather}
  The discrete reduced problem is: find $u_h\in V_h^g$ such that
  \begin{gather}
    \label{eq:galerkin:4}
    a(u_h, v_h) = f(v_h), \qquad\forall v_h \in \ker{B_h}.
  \end{gather}
\end{Definition}


\begin{Theorem}{galerkin-mixed-u-kerbh}
  Let $V_h^g$ be nonempty and let $a(.,.)$ and $b(.,.)$ be bounded
  with $\bounda$ and $\boundb$, respectively, and let there be constant and $\gamma_h$
  such that
  \begin{gather}
    \label{eq:galerkin:5}
    a(v_h, v_h) \ge \gamma_h \norm{v_h}^2_V,
    \qquad\forall v_h\in \ker{B_h}.
  \end{gather}
  Let furthermore the continuous mixed problem be well-posed with
  solution $(u,p)\in V\times Q$.
  Then, the discrete reduced problem~\eqref{eq:galerkin:4} has a
  unique solution $u_h\in V_h$ and there holds
  \begin{gather}
    \label{eq:galerkin:8}
    \norm{u-u_h}_V \le \left(1+\frac{\bounda}{\gamma_h}\right)
    \inf_{w_h\in V_h^g}\norm{u-w_h}_V
    + \frac{\boundb}{\gamma_h}
    \inf_{q_h\in Q_h}\norm{p-q_h}_Q
  \end{gather}
\end{Theorem}

\begin{proof}
  Let $u_h^g \in V_h^g$ arbitrary. By the ellipticity assumption,
  there is a unique function $u_h^0\in \ker{B_h}$ such that
  \begin{gather}
    a(u_h^0,v_h) = f(v_h) - a(u_h^g,v_h),
    \qquad\forall v_h\in \ker{B_h}.
  \end{gather}
  Hence, $u_h = u_h^g + u_h^0$ is the unique solution
  to~\eqref{eq:galerkin:4}. Choose now $w_h\in V_h^g$
  arbitrarily. Then, $v_h = u_h-w_h\in \ker{B_h}$ and using
  \begin{gather}
    f(v_h) = a(u, v_h) - b(v_h, p),
  \end{gather}
  we obtain
  \begin{gather}
    \begin{split}
      \label{eq:galerkin:9}
      a(v_h, v_h)
      &= f(v_h) - a(u_h-v_h, v_h) \\
      &= a(u-w_h, v_h) - b(v_h, p) \\
      &= a(u-w_h, v_h) - b(v_h, p-q_h)
    \end{split}
  \end{gather}
  for any $q_h\in Q_h$, yielding
  \begin{gather}
    \gamma_h\norm{v_h}^2_V
    \le \abs{a(v_h, v_h)}
    \le \bounda \norm{u-w_h}_V \norm{v_h}_V
      + \boundb \norm{p-q_h}_Q \norm{v_h}_V.
  \end{gather}
  We conclude by the standard triangle inequality argument
  \begin{multline}
    \norm{u-u_h}_V \le \norm{u-w_h}_V + \norm{u_h-w_h}_V
    \\
    \le \norm{u-w_h}_V + \frac{\bounda}{\gamma_h} \norm{u-w_h}_V
    + \frac{\boundb}{\gamma_h} \norm{p-q_h}_Q.
  \end{multline}
  The estimate follows since $w_h\in V_h^g$ and $q_h\in Q_h$ were
  chosen arbitrarily.
\end{proof}

\begin{Corollary}{galerkin-mixed-u-kerb}
  If in addition to the assumptions of
  \slideref{Theorem}{galerkin-mixed-u-kerbh} there holds
  \begin{gather}
    \label{eq:galerkin:6}
    \ker{B_h}\subset \ker B,
  \end{gather}
  then
  \begin{gather}
    \label{eq:galerkin:7}
    \norm{u-u_h}_V \le \left(1+\frac{\bounda}{\gamma_h}\right)
    \inf_{w_h\in V_h^g}\norm{u-w_h}_V
  \end{gather}
\end{Corollary}

\begin{proof}
  Consider that in equation~\eqref{eq:galerkin:9} there holds
  $v_h\in\ker B$.
\end{proof}

\begin{remark}
  Note that we used ellipticity of $a(.,.)$ on the subspace
  $\ker{B_h}$ for the discrete problem and on $\ker B$ for the
  continuous problem. If $\ker{B_h}\not\subset\ker{B}$, then
  ellipticity on $\ker B$ like in \slideref{Theorem}{mixed-elliptic}
  is not sufficient for well-posedness of the discrete problem.
  
  In practice, we will encounter two situations: either ellipticity
  holds on the whole space $V$ (Stokes equations) or
  $\ker{B_h}\subset\ker{B}$ (the other examples in these notes).
\end{remark}

\begin{Theorem}{galerkin-mixed-p}
  Assume in addition the assumptions of
  \slideref{Theorem}{galerkin-mixed-u-kerbh} that there are constants
  $\infsupc_h$, possibly depending on the parameter $h$, such that
  \begin{gather}
    \label{eq:galerkin:10}
    \inf_{q_h\in Q_h} \sup_{v_h\in V_h}
    \frac{b(v_h, q_h)}{\norm{v_h}_V\norm{q_h}_Q}
    \ge \infsupc_h.
  \end{gather}
  Then, there is a unique solution $p_h\in Q_h$ such that $(u_h, p_h)$
  is the unique solution to the discrete mixed
  problem~\eqref{eq:galerkin:3}. There are a constants $c_h^{(i)}$
  only depending on $\bounda$, $\boundb$, $\gamma_h$ and $\infsupc_h$
  such that
  \begin{align}
    \label{eq:galerkin:11}
    \norm{u-u_h}_V
    &\le c_h^{(1)} \inf_{v_h\in V_h}\norm{u-v_h}_V
    + c_h^{(2)} \inf_{q_h\in Q_h}\norm{p-q_h}_Q \\
    \norm{p-p_h}_Q
    &\le c_h^{(3)} \inf_{v_h\in V_h}\norm{u-v_h}_V
    + c_h^{(4)} \inf_{q_h\in Q_h}\norm{p-q_h}_Q.
  \end{align}
\end{Theorem}

\begin{proof}
  Applying \slideref{Theorem}{infsup-mixed2} to the discrete
  problem~\eqref{eq:galerkin:3}, we conclude that there is a unique
  solution $(u_h,p_h)\in V_h\times Q_h$. Let us begin estimating the
  error by establishing the bound
  \begin{gather}
    \label{eq:galerkin:12}
    \inf_{w_h\in V_h^g} \norm{u-w_h}_V
    \le \left(1+\frac{\boundb}{\infsupc_h}\right)
    \inf_{v_h\in V_h} \norm{u-v_h}_V.
  \end{gather}
  By the third condition in
  \slideref{Theorem}{infsup-well-equivalence}, there is a unique
  $z_h\in \ortho{\ker{B_h}}$ such that
  \begin{gather}
    B_h z_h = B_h(u-v_h),\qquad \forall v_h\in V_h,
  \end{gather}
  and
  \begin{gather}
    \norm{z_h}_V \le \frac1{\infsupc_h}\norm{B_h(u-v_h)}_{Q_h^*}
    \le\frac{\boundb}{\infsupc_h} \norm{u-v_h}_V.
  \end{gather}
  Let $w_h = z_h+v_h$. Then, $w_h\in V_h^g$ since
  \begin{gather}
    b(w_h, q_h) = b(u-v_h, q_h) = b(u, q_h) = g(q_h),
    \qquad\forall q_h\in Q_h.
  \end{gather}
  Furthermore,
  \begin{gather}
    \norm{u-w_h}_V \le \norm{u-v_h}_V + \norm{z_h}_V
    \le \left(1+\frac{\boundb}{\infsupc_h}\right)
    \norm{u-v_h}_V.
  \end{gather}
  Since $v_h \in V_h$ was chosen arbitrarily, we have
  proven~\eqref{eq:galerkin:12} and thus by
  \slideref{Theorem}{galerkin-mixed-u-kerbh} the estimate for
  $\norm{u-u_h}_V$ with
  \begin{gather}
    c_h^{(1)} = \left(1+\frac{\boundb}{\infsupc_h}\right)
    \left(1+\frac{\bounda}{\gamma_h}\right),
    \qquad
    c_h^{(2)} = \left(1+\frac{\boundb}{\infsupc_h}\right)
    \frac{\boundb}{\gamma_h}.
  \end{gather}
  It remains to prove the estimate for $\norm{p-p_h}_Q$. Using Galerkin
  orthogonality for the test function $v_h$, we obtain
  \begin{gather}
    \label{eq:galerkin:13}
    a(u-u_h, v_h) + b(v_h, p-p_h) = 0.
  \end{gather}
  Hence, for any $q_h\in Q_h$ there is by the inf-sup condition
  $v_h\in V_h$ with $\norm{v_h}_V=1$ such that
  \begin{align}
    \infsupc_h \norm{p_h-q_h}_Q
    &\le b(v_h, p_h-q_h)\\
    & = a(u-u_h, v_h) + b(v_h, p-q_h)\\
    &\le \bounda \norm{u-u_h}_V + \boundb \norm{p-q_h}_Q.
  \end{align}
  Again, the estimate for $\norm{p-p_h}_Q$ follows by triangle
  inequality.
\end{proof}

\begin{remark}
  Indeed, if~\eqref{eq:galerkin:10} holds, we do not have to require
  that $V_h^g$ is nonempty anymore, since $B_h\colon V_h\to Q_h^*$ is
  surjective.
\end{remark}

\begin{remark}
  We purposely proved the preceding theorems with $\gamma_h$ and
  $\infsupc_h$ depending on the parameter $h$, typically the mesh
  size. This is the minimal condition for well-posedness of the
  discrete problems. Nevertheless, this well-posedness is not uniform
  in $h$, which causes loss of approximation, as the following problem
  shows. Therefore, we will only be satisfied with uniform inf-sup
  constants in applications.
\end{remark}

\begin{Problem}{infsup-uniform}
  Let the following interpolation estimates hold:
  \begin{gather}
    \inf_{v_h\in V_h}\norm{u-v_h}_V = \mathcal O(h^k),
    \qquad
    \inf_{q_h\in Q_h}\norm{p-q_h}_Q = \mathcal O(h^k).
  \end{gather}
  Then, the estimates in \slideref{Theorem}{galerkin-mixed-p}
  are asymptotically optimal if and only if there are
  constants $\tilde \gamma>0$ and $\tilde \infsupc>0$ independent of
  $h$ such that
  \begin{gather}
    \label{eq:galerkin:14}
    \gamma_h\ge \tilde\gamma,\quad\infsupc_h\ge\tilde\infsupc,
  \end{gather}
  independent of $h$.
\begin{solution}
  The estimates in \slideref{Theorem}{galerkin-mixed-p} state
  \begin{align}
    \norm{u-u_h}_V &\le c_h^{(1)} \inf_{v_h\in V_h^g}\norm{u-v_h}_V
	      + c_h^{(2)} \inf_{q_h\in Q_h}\norm{p-q_h}_Q \\
    \norm{p-p_h}_Q &\le c_h^{(3)} \inf_{v_h\in V_h}\norm{u-v_h}_V
	      + c_h^{(4)} \inf_{q_h\in Q_h}\norm{p-q_h}_Q.
  \end{align}
  such that the additional assumptions imply
  \begin{align}
    \norm{u-u_h}_V &\le c_h^{(1)} C h^k + c_h^{(2)} C h^k \\
    \norm{p-p_h}_Q &\le c_h^{(3)} C h^k + c_h^{(4)} C h^k.
  \end{align}
    where the first constants are given by
  \begin{align}
    c_h^{(1)} = \left(1+\frac{\boundb}{\infsupc_h}\right)
    \left(1+\frac{\bounda}{\gamma_h}\right),
    \qquad
    c_h^{(2)} = \left(1+\frac{\boundb}{\infsupc_h}\right)
    \frac{\boundb}{\gamma_h}.
  \end{align}
  For the last two estimates, the relevant estimates read
  \begin{align}
    \infsupc_h \norm{p_h-q_h}_Q
    &\le b(v_h,p_h-q_h)\\
    & = a(u-u_h, v_h) + b(v_h, p-q_h)\\
    &\le \bounda \norm{u-u_h}_V + \boundb \norm{p-q_h}_Q\\
    \Rightarrow \norm{p-p_h}_Q
    &\le \norm{p-q_h} + \norm{p_h-q_h}_Q \\
    &\le \frac{\bounda}{\infsupc_h} \norm{u-u_h}_V + \left(1+\frac{\boundb}{\infsupc_h}\right) \norm{p-q_h}_Q.
  \end{align}
  Therefore, the remaining constants are
  \begin{align}
    c_h^{(3)} = \frac{\bounda}{\infsupc_h}c_h^{(1)}, \qquad
    c_h^{(4)} = \frac{\bounda}{\infsupc_h}c_h^{(2)}+\left(1+\frac{\boundb}{\infsupc}\right).
  \end{align}
\end{solution}
\end{Problem}

\begin{intro}
  As we can see from the form
  \begin{gather}
    \forall q_h\in Q_h \;
    \exists v_h\in V_h\colon
    \quad B_h v_h = q_h
    \quad\wedge\quad
    \norm{v_h}_V \le \norm{q_h}_Q,
  \end{gather}
  the uniform, discrete inf-sup condition introduces a compatibility
  condition between the spaces $V_h$ and $Q_h$. An immediate necessary
  condition is
  \begin{gather}
    \dim V_h \ge \dim Q_h.
  \end{gather}
  We often say that the space $V_h$ is ``rich enough'' to control
  functions in $Q_h$. Obviously, counting dimensions is not
  sufficient, since we could have added basis functions in
  $\ker{B_h}$. Even the condition
  \begin{gather}
    \dim\ortho{\ker{B_h}} = \dim Q_h
  \end{gather}
  is necessary and sufficient only for the existence of an inf-sup
  constant $\infsupc_h$ depending on $h$. Therefore, we need a stronger
  argument in order to ensure compatibility of the discrete
  spaces. Such an argument is the following lemma by Fortin. The
  projection operator $\Pi_{V_h}$ introduced there is usually referred
  to as \define{Fortin projection}.
\end{intro}

\begin{Lemma}{fortin}
  Let the inf-sup condition for the bilinear form $b(.,.)$
  hold on $V\times Q$ with a constant
  $\infsupc>0$. Then, it holds on $V_h\times Q_h$ uniformly with a
  constant $\tilde\infsupc>0$ if and only if there exists a linear
  operator $\Pi_{V_h}\colon V\to V_h$ satisfying for any $v\in V$
  \begin{align}
    \label{eq:galerkin:15}
    b(v-\Pi_{V_h}v,q_h) &= 0,\qquad\forall q_h\in Q_h,
    \\
    \label{eq:galerkin:16}
    \norm{\Pi_{V_h}v}_V \le c \norm{v}_V,
  \end{align}
  with $c$ independent of $h$. There holds $\tilde\infsupc = \infsupc/c$.
\end{Lemma}

\begin{proof}
  Assume first that $\Pi_{V_h}$ exists. then, there holds for any
  $q_h\in Q_h$
  \begin{gather}
    \sup_{v_h\in V_h} \frac{b(v_h, q_h)}{\norm{v_h}_V}
    \ge
    \sup_{v\in V} \frac{b(\Pi_{V_h}v, q_h)}{\norm{\Pi_{V_h} v}_V}
    =
    \sup_{v\in V} \frac{b(v, q_h)}{\norm{\Pi_{V_h} v}_V}
    \ge \frac{\infsupc}{c} \norm{q_h}_Q.
  \end{gather}
  Conversely, we assume the existence of a uniform, discrete inf-sup
  constant $\tilde\infsupc>0$. Then, for any $v\in V$ let
  $g(.) = b(v,.) \in Q_h^*$. By
  \slideref{Theorem}{infsup-well-equivalence}, there is a unique
  element $\Pi_{V_h} v \in \ortho{\ker{B_h}}$ such that
  \begin{gather}
    b(\Pi_{V_h}v,q_h) = b(v, q_h),\qquad\forall q_h\in Q_h
  \end{gather}
  and
  \begin{gather}
    \norm{\Pi_{V_h}v}_V
    \le \frac1{\tilde\infsupc}\norm{B_h v}_{Q_H^*}
    \le \frac{\boundb}{\tilde\infsupc} \norm{v}_V.
  \end{gather}
  Thus, $\Pi_{V_h}$ is bounded and~\eqref{eq:galerkin:15} holds with
  $c=\boundb/\tilde\infsupc$.
\end{proof}


%%% Local Variables:
%%% mode: latex
%%% TeX-master: "main"
%%% End:



%%%%%%%%%%%%%%%%%%%%%%%%%%%%%%%%%%%%%%%%%%%%%%%%%%%%%%%%%%%%%%%%%%%%%%
%%%%%%%%%%%%%%%%%%%%%%%%%%%%%%%%%%%%%%%%%%%%%%%%%%%%%%%%%%%%%%%%%%%%%%
\section{Bringing back $c(p,q)$}
%%%%%%%%%%%%%%%%%%%%%%%%%%%%%%%%%%%%%%%%%%%%%%%%%%%%%%%%%%%%%%%%%%%%%%
%%%%%%%%%%%%%%%%%%%%%%%%%%%%%%%%%%%%%%%%%%%%%%%%%%%%%%%%%%%%%%%%%%%%%%

\begin{intro}
  The key to the mixed analysis which is also underlying our
  quasi-best\-ap\-proxi\-mation result was a splitting of the solution
  process into the reduced problem for $u$ and then applying the
  inf-sup condition for $b(.,.)$ in order to estimate $p$. This way,
  we will be able to obtain estimates for the Stokes problem, but we
  have tacitly abandoned weakly compressible elasticity. Indeed, the
  mixed form of the Lamé-Navier equations is not a constrained
  minimization problem. In this section, we will fill the gap and
  derive estimates for the solution of this problem which are robust
  in $\lambda$.

  In the Lamé-Navier equations, we had
  \begin{gather}
    c(p,q) = -\tfrac1\lambda \scal(q,p)_{L^2(\domain)},
  \end{gather}
  which suggests assuming symmetric and $Q$-elliptic. But, we want
  estimates independent of $\lambda$! Therefore, we should only
  require semi-definite, which on the other hand turns out a bit too
  weak.
\end{intro}

\begin{Assumption}{mixed-elliptic-stabilized}
  In addition to \slideref{Assumption}{mixed-elliptic}, let $c(.,.)$
  be positive semi-definite and elliptic on $\ker{B^\transpose}$,
  \begin{gather}
    c(q,q) \ge u\quad\forall q\in Q,
    \qquad
    c(q,q) \ge \ellipc \norm{q}_Q^2 \quad\forall q\in \ker{B^\transpose}.
  \end{gather}
\end{Assumption}

\begin{remark}
  Again, this assumption is not necessary for the analysis, but it
  yields a convenient and useful theorem which goes far beyond weakly
  compressible elasticity and covers stabilized methods for spaces
  where the inf-sup condition for $b(.,.)$ does not hold for the whole
  space $Q$.
\end{remark}

\begin{Theorem}{mixed-stabilized-well-posed}
  Let \slideref{Assumption}{mixed-elliptic-stabilized} hold and
  let $a(.,.)$ and $c(.,)$ be symmetric. In addition, let there be
  $\infsupc>0$ such that
  \begin{gather}
    \begin{split}
      \inf_{q\in \ortho{\ker{B^\transpose}}}  \sup_{v\in V}
      \frac{b(v,q)}{\norm{v}_V\norm{q}_Q} &\ge \infsupc\\
      \inf_{v\in \ortho{\ker{B}}}  \sup_{q\in Q}
      \frac{b(v,q)}{\norm{v}_V\norm{q}_Q} &\ge \infsupc
    \end{split}
  \end{gather}
  Then, the problem finding $(u,p)\in V\times Q$ such that
  \begin{multline}
    a(u,v) + b(v,p) + b(u,q) - c(p,q) = f(v)+g(q)
    \\
    \forall v\in V, q\in Q
  \end{multline}
  has a unique solution for all $f\in V^*$ and $g\in Q^*$ and there is
  a constant $C$ such that
  \begin{gather}
    \norm{u}_V+\norm{p}_Q
    \le C \bigl(\norm{f}_{V^*} + \norm{g}_{Q^*}\bigr).
  \end{gather}
\end{Theorem}

\begin{proof}
  First, note that the ellipticity assumptions as well as the inf-sup
  conditions are symmetric in $V$ and $Q$. Indeed, replacing the test
  functions and the form $b(.,.)$ by their negatives, we can transform
  the problem into one where $V$ and $Q$ have exchanged their
  roles. Thus, it is sufficient to show well-posedness for $f=0$. The
  same result then holds for $g=0$ and it holds for both nonzero by
  linearity.
  
  We note that by the inf-sup conditions both $\range B$ and
  $\range{B^\transpose}$ are closed. Thus, we can decompose $u=u^0+u^\perp$
  with $u^0\in\ker B$ and $u^\perp$ in its orthogonal
  complement. Assuming $f=0$ and testing with $q=0$ we obtain the
  equation
  \begin{gather}
    \label{eq:infsup:14}
    a(u,v) = -b(v,p) = 0.
  \end{gather}
  In particular, testing with $v=u^0$ yields
  \begin{gather}
    a(u,u^0) = -b(u^0,p) = 0.
  \end{gather}
  Hence,
  \begin{gather}
    \ellipa \norm{u^0}^2_V \le a(u^0,u^0)
    = -a(u^\perp,u^0)
    \le \bounda \norm{u^\perp}\norm{u^0},
  \end{gather}
  which implies
  \begin{gather}
    \label{eq:infsup:15}
    \norm{u^0}_V \le \frac{\bounda}{\ellipa} \norm{u^\perp}_V.
  \end{gather}
  Testing with $v=u$ and $q=-p$ yields
  \begin{gather}
    a(u,u)+c(p,p) \le g(p) = g^0(p^0) + g^\perp(p^\perp),
  \end{gather}
  where $p^0\in \ker{B^\transpose}$, $g^\perp\in \polar{\ker{B^\transpose}}$ and the
  other two are defined by orthogonality in $Q$ and $Q^*$,
  respectively. Let first $g^0=0$. Then, by~\eqref{eq:infsup:14} and
  the inf-sup condition for $p$, there is $v\in V$ with $\norm{v}_V = 1$
  such that
  \begin{gather}
    \infsupc \norm{p^\perp} \le
    \abs{b(v,p^\perp)} = \abs{b(v,p)} = \abs{a(u,v)}
    \le \sqrt{a(u,u)}\sqrt{a(v,v)},
  \end{gather}
  by the Bunyakovsky-Cauchy-Schwarz inequality for symmetric bilinear
  forms.  Therefore, squaring and using the definition of the operator
  norm of $g^\perp$ yields
  \begin{gather}
    \norm{p^\perp}_Q \le \frac{\bounda}{\infsupc^2}\norm{g^\perp}_{Q^*}.
  \end{gather}
  Furthermore, we have
  \begin{gather}
    c(p,p^0) = b(u,p^0) - g^\perp(p^0) = 0.
  \end{gather}
  Hence,
  \begin{gather}
    \ellipc \norm{p^0}_Q^2 \le c(p^0,p^0)
    = c(p^\perp,p^0) \le \norm c \norm{p^0}_Q \norm{p^\perp}_Q,
  \end{gather}
  concluding
  \begin{gather}
    \norm{p^0}_Q
    \le \frac{\bounda \norm c}{\ellipc\infsupc^2}
    \norm{g^\perp}_{Q^*}.
  \end{gather}

  We continue our proof for $g^\perp=0$ and $g^0 \neq 0$. Testing with
  $q=p^0$, we obtain
  \begin{align}
    c(p^0,p^0)
    &= c(p,p^0) - c(p^\perp,p^0) \\
    &= b(u,p^0) - g^0(p^0) - c(p^\perp,p^0) \\
    &\le \norm{g^0}_{Q^*} \norm{p^0}_Q
      + \norm c \norm{p^\perp}_Q \norm{p^0}_Q,
  \end{align}
  yielding
  \begin{gather}
    \norm{p^0}_Q \le \frac1{\ellipc}
    \left(\norm{g^0}_{Q^*} + \norm c \norm{p^\perp}_Q\right).
  \end{gather}
  $p^\perp$ is estimated as before by
  \begin{align}
    \norm{p^\perp}_Q^2
    &\le \frac{\bounda}{\infsupc^2}\norm{g^0}_{Q^*} \norm{p^0}_Q \\
    &\le \frac{\bounda}{\ellipc\infsupc^2} \norm{g^0}_{Q^*}^2
      + \frac{\bounda\norm c}{\ellipc\infsupc^2}
      \norm{g^0}_{Q^*}\norm{p^\perp}_Q\\
    &\le \frac{\bounda}{\ellipc\infsupc^2} \norm{g^0}_{Q^*}^2
      + \frac12 \norm{p^\perp}_Q^2
      + \frac{\bounda^2 \norm c^2}{2{\ellipc}^2\infsupc^4} \norm{g^0}_{Q^*}^2.
  \end{align}
  We conclude that $p^\perp$ and $p^0$ are bounded by $g^0$. Summing
  up, we obtain for $f=0$ and $g\in Q^*$ the estimate
  \begin{gather}
    \norm{p}_Q \le c \norm{g}_{Q^*}.
  \end{gather}

  We estimate $u^0$ by $u^\perp$ using~\eqref{eq:infsup:15} and
  $u^\perp$ by the inf-sup condition, choosing $q\in Q$ with
  $\norm{q}_Q = 1$ such that
  \begin{gather}
    \infsupc\norm{u^\perp} = b(u,q) = c(p,q) + g(q)
    \le \norm{c}\norm{p}_Q +  \norm{g}_{Q^*}.
  \end{gather}
  Thus, we have estimated all components of the solution by the norm
  of $g$, assuming $f=0$. Now we conclude the proof by reverting the
  roles of $u$ and $p$, respectively $f$ and $g$.
\end{proof}

\begin{intro}
  The extension of \slideref{Theorem}{galerkin-mixed-p} to the
  saddle-point problem of \slideref{Definition}{saddle-point-abstract}
  with bilinear form $c(.,.)$ is even more cumbersome than this
  theorem. Nevertheless, the use of residual operators as a technique
  to structure the proof of convergence is instructive and may come
  handy at some point.
\end{intro}

\begin{Definition}{mixed-residual}
  For the saddle-point problem
  \begin{gather}
    a(u,v) + b(v,p) + b(u,q) - c(p,q),
  \end{gather}
  and functions $w_h\in V_h$ and $r_h\in Q_h$ we introduce the the
  residual operators $R_f \in V_h^*$ and $R_g\in Q_h^*$ as
  \begin{gather}
    \begin{split}
      R_f(v_h) &= a(u-w_h, v_h) + b(v_h, p-r_h) \\
      R_g(q_h) &= b(u-w_h, q_h) - c(p-r_h, q_h).
    \end{split}
  \end{gather}
\end{Definition}

\begin{Corollary}{mixed-residual-bounded}
  Under \slideref{Assumption}{mixed-elliptic-stabilized}, we have
  \begin{gather}
    \begin{split}
      \abs{R_f(v_h)}
      &\le \bigl(\bounda \norm{u-w_h}_V + \boundb \norm{p-r_h}\bigr)
      \norm{v_h}_V
      \\
      \abs{R_g(q_h)}
      &\le \bigl(\boundb \norm{u-w_h}_V + \norm c \norm{p-r_h}\bigr)
      \norm{q_h}_Q.
    \end{split}
  \end{gather}
\end{Corollary}

\begin{Lemma}{stabilized-mixed-approximation}
  Let the assumptions of
  \slideref{Theorem}{mixed-stabilized-well-posed} hold. Then, there
  are constants $c_1$ to $c_4$ independent of the solutions $u$, $p$,
  $u_h$, and $p_h$ and the discretization parameter $h$, such that for
  any $v_h\in V_h$ and $q_h\in Q_h$
  \begin{gather}
    \label{eq:infsup:16}
    \begin{split}
      \norm{u_h-v_h} &\le c_1 \norm{R_f}_{V_h^*} + c_2
      \norm{R_g}_{Q_h^*}
      \\
      \norm{p_h-q_h} &\le c_3 \norm{R_f}_{V_h^*} + c_4
      \norm{R_g}_{Q_h^*}.
    \end{split}
  \end{gather}
\end{Lemma}

\begin{proof}
  The proof is lengthy and follows the lines of the proof of
  well-posedness for
  \slideref{Theorem}{mixed-stabilized-well-posed}. It is obtained by
  considering the components $u_h^0-v_h^0$ and $u_h^\perp-v_h^\perp$
  as well as $p_h^0-q_h^0$ and $p_h^\perp-q_h^\perp$ separately.
\end{proof}

\begin{intro}
  In spite of the bad treatment the proof of the previous lemma
  received in these notes, it contains the main parts of the
  convergence proof, and whenever a saddle-point problem including
  $c(.,.)$ is solved, it has to be reproduced. It is just the fact
  that the proof is overwhelmingly technical that led to the decision
  to leave this experience to the first time the reader actually needs
  this result.
\end{intro}

\begin{Corollary}{stabilized-mixed-convergence}
    Let the assumptions of
  \slideref{Theorem}{mixed-stabilized-well-posed} hold. Then, there
  are constants $c_1$ to $c_4$ independent of the solutions $u$, $p$,
  $u_h$, and $p_h$ and the discretization parameter $h$, such that
  \begin{gather}
    \label{eq:infsup:17}
    \begin{split}
      \norm{u-u_h}
      &\le c_1 \inf_{v_h\in V_h}\norm{u-v_h}_V
      + c_2 \inf_{q_h\in Q_h} \norm{p-q_h}_Q
      \\
      \norm{p-p_h}
      &\le c_3 \inf_{v_h\in V_h}\norm{u-v_h}_V
      + c_4 \inf_{q_h\in Q_h} \norm{p-q_h}_Q.
    \end{split}
  \end{gather}
\end{Corollary}

\begin{proof}
  The proof begins with the standard approach with triangle inequality
  \begin{align}
    \norm{u-u_h} &\le \norm{u-v_h} + \norm{v_h-u_h} \\
    \norm{p-p_h} &\le \norm{p-q_h} + \norm{q_h-p_h}.
  \end{align}
  Then, we employ \slideref{Lemma}{stabilized-mixed-approximation} on
  the terms on the right and use the estimate of
  \slideref{Corollary}{mixed-residual-bounded}.
\end{proof}

%%% Local Variables: 
%%% mode: latex
%%% TeX-master: "main"
%%% End:


\chapter{The Stokes problem}
\label{cha:stokes}

\section{Well-posedness of the continuous problem}

\begin{intro}
  We begin our investigation into the Stokes problem by investigating
  the well-posedness of the continuous problem. This is particularly
  simple, since we have
  \begin{gather*}
    a(u,v) = \form(\strain u,\strain v)
  \end{gather*}
  for the original Stokes problem in \blockref{Definition}{stokes-eq1}
  and
  \begin{gather*}
    a(u,v) = \form(\nabla u,\nabla v)
  \end{gather*}
  for the simplified \putindex{Stokes equations} in
  \blockref{Definition}{stokes-eq2}. From the standard theory for the
  Laplacian, we know that the second one is $V$-elliptic on
  $V=H^1_0(\domain;\R^d)$. For the first one, we conclude this by using a
  \putindex{Korn inequality}. Therefore, we can already conclude a
  first result:
\end{intro}

\begin{Lemma}{stokes-a-elliptic}
  Let $V=H^1_0(\domain, \R^d)$ and $V_h\subset V$ a finite dimensional
  subspace. Then, $a(.,.)$ is elliptic on $\ker B$ and on $\ker{B_h}$
  independent of the choices of $Q$ and $Q_h$.
\end{Lemma}

\begin{remark}
  We focus here on no-slip boundary condition on the whole boundary as
  the exemplary case. Other boundary conditions are possible, but as
  soon as the Dirichlet boundary for one velocity component becomes
  too small, the ellipticity of $a(.,.)$ on $V$ must be established by
  new arguments known for instance for Robin boundary conditions. In
  the extreme case of natural boundary conditions all around, $V$ is
  the subspace of $H^1(\domain, \R^d)$ obtained by dividing by the
  space of all translations for the simplified form and by the space
  of all rigid body movements.

  Note that we have established already in
  \blockref{Lemma}{divergence-compatibility} that the condition
  $V=H^1_0(\domain, \R^d)$ implies the reduction of the pressure to
  the space $Q = L^2_0(\domain)$ from
  \blockref{Notation}{pressure-constant}.
\end{remark}

\begin{intro}
  The previous lemma guarantees well-posedness of the
  \putindex{reduced problem} in all possible cases. Therefore, the
  remainder of this section is only concerned with the inf-sup
  condition for the divergence operator. We
  follow~\cite{GiraultRaviart86} in this presentation.
\end{intro}

\begin{Lemma}{stokes-helmholtz}
  Let $V=H^1_0(\domain,\R^d)$. Then, the divergence operator
  $\div\colon V \to L^2(\domain)$ is continuous and the subspace
  \begin{gather*}
    V^0 = \ker \div
    = \bigl\{v\in V \big|
    \div v = 0 \text{ a.e.} \bigr\}
  \end{gather*}
  is closed in $V$ and $V$ admits the orthogonal decomposition
  \begin{gather*}
    V = V^0\oplus V^\perp.
  \end{gather*}
\end{Lemma}

\begin{proof}
  We have that
  \begin{gather*}
    \norm{\div v}_{L^2(\domain)}^2
    = \int_\domain \left(\sum \d_iv_i\right)^2\dx
    \le d \int_\domain \sum \abs{\d_iv_i}^2\dx
    \le d \norm{v}_{H^1(\domain;\R^d)}^2.
  \end{gather*}
  Thus, the divergence operator is a continuous mapping from $V$ to
  $L^2(\domain)$. The definition of $V^0$ is equivalent to the
  definition of zero in $L^2(\domain)$. Finally, since the kernel is
  the pre-image of a closed set under a continuous map, it is
  closed. The existence of the decomposition follows from
  \blockref{Theorem}{orthogonal-complement}.
\end{proof}

\begin{Lemma}{stokes-grad}
  If $f\in V^* = H^{-1}(\domain;\R^d)$ satisfies
  \begin{gather*}
    f(v) = 0 \quad\forall v\in V^0,
  \end{gather*}
  then, there exists $p\in L^2(\domain)$ such that
  \begin{gather*}
    f = \nabla p.
  \end{gather*}
  If $\domain$ is connected, then $p$ is unique up to an additive
  constant.
\end{Lemma}

\begin{proof}
  First, we identify $L^2(\domain)$ with its dual. Then, by
  \begin{gather*}
    \scal(-\nabla p, v)_{V^*\times V}
    = \scal(p, \div v)_{L^2(\domain)},
    \qquad\forall v\in V,
  \end{gather*}
  we see that $-\nabla\colon L^2(\domain)\to V^*$ is the dual to the
  divergence operator. Using the Cauchy-sequence argument, we see that
  $\range{\div}$ is closed in $L^2(\domain)$ and the closed range
  theorem applies. Thus, $\range{-\nabla}$ is closed in $V^*$ and
  \begin{gather*}
    \range{\nabla} = \polar{(V^0)} \cong V^\perp
  \end{gather*}
  is the polar set of the kernel $V^0$. This implies the statement
  that there is a $p$ for every $f$. Uniqueness follows by the fact
  that the only differentiable functions on a connected domain with
  $\nabla p=0$ are the constant functions, and by density of such
  functions in $L^2(\domain)$.
\end{proof}

\begin{Corollary}{stokes-iso}
  Let $\domain$ be connected. Then,
  \begin{enumerate}
  \item $\nabla\colon L^2_0(\domain) \to V^0$ is an isomorphism
  \item $\div\colon V^\perp \to L^2_0(\domain)$ is an isomorphism
  \end{enumerate}
\end{Corollary}

\begin{Theorem}{stokes-infsup}
  Let $\domain\subset \R^d$ be a Lipschitz-domain,
  $V=H^1_0(\domain,\R^d)$ and $Q=L^2_0(\domain)$. Then, there is a
  constant $\beta>0$ depending only on the geometry of $\domain$ such
  that
  \begin{gather}
    \label{eq:stokes:1}
    \inf_{q\in Q}\sup_{v\in V}\frac{\form(\div
      v,q)}{\norm{v}_V\norm{q}_Q} \ge \beta.
  \end{gather}
  Furthermore, the problem finding $(u,p)\in V\times Q$ such that
  \begin{gather}
    \label{eq:stokes:3}
    a(u,v)+\form(\div v,p)+\form(\div u,q) = f(v)+g(q)
    \quad\forall v\in V, q\in Q,
  \end{gather}
  has a unique solution for any right hand side $f\in V^*$ and $g\in
  \range{\div}$.
\end{Theorem}

\section{Stable discretizations}

\begin{intro}
  We begin by application of the generic theory of the previous
  chapter to the Stokes problem in order to obtain a generic error
  estimate based on the concrete choice of norms and a single
  assumption. Guided by this theorem, we spend the remaining part of
  this section exploring different options for the discrete spaces.
\end{intro}

\begin{Theorem}{stokes-convergence}
  Let $V=H^1_0(\domain;\R^d)$ and $Q=L^2_0(\domain)$. Let furthermore
  $V_h\subset V$ and $Q_h\subset Q$ be discrete subspaces such that
  there exists $\beta>0$ independent of $h$ such that
  \begin{gather}
    \label{eq:stokes:2}
    \inf_{q_h\in Q_h}\sup_{v_h\in V_h}\frac{\form(\div
      v_h,q_h)}{\norm{v_h}_V\norm{q_h}_Q} \ge \beta.
  \end{gather}
  Then, the Galerkin approximation of~\eqref{eq:stokes:3} admits a
  unique solution $(u_h, p_h)\in V_h\times Q_h$ with the
  quasi-bestapproximation property
  \begin{gather}
    \label{eq:stokes:4}
    \begin{split}
      \norm{u-u_h}_1
      &\le c_1 \inf_{v_h\in V_h}\norm{u-v_h}_1
      + c_2 \inf_{q_h\in Q_h}\norm{p-q_h}_0
      \\
      \norm{p-p_h}_1
      &\le c_3 \inf_{v_h\in V_h}\norm{u-v_h}_1
      + c_4 \inf_{q_h\in Q_h}\norm{p-q_h}_0.
    \end{split}
  \end{gather}
\end{Theorem}

\begin{Corollary}{stokes-convergence2}
  Under the assumptions of \blockref{Theorem}{stokes-convergence},
  let there be in addition interpolation operators $I_{V_h}$ and
  $I_{Q_h}$ such that
  \begin{gather}
    \label{eq:stokes:5}
    \begin{split}
      \norm{u-I_{V_h} u}_1 &\le c h^k \snorm{u}_{k+1} \\
      \norm{p-I_{Q_h} p}_0 &\le c h^k \snorm{p}_{k}.
    \end{split}
  \end{gather}
  Then, there is a constant $c$ independent of the approximation
  spaces such that
  \begin{gather}
    \label{eq:stokes:6}
    \begin{split}
      \norm{u-u_h}_1 &\le c h^k \bigl(\snorm{u}_{k+1} +
      \snorm{p}_{k}\bigr)
      \\
      \norm{p-p_h}_1 &\le c h^k \bigl(\snorm{u}_{k+1} +
      \snorm{p}_{k}\bigr).
    \end{split}
  \end{gather}
\end{Corollary}
\begin{intro}
  We continue showing that the most natural discretizations
  in two dimensions are not inf-sup stable. This holds for the
  discretization using continuous linear or bilinear elements for both
  velocity components and the pressure as well as for continuous
  linear or bilinear velocity functions combined with piecewise
  constant pressure functions.
\end{intro}

\begin{example}
  Take a patch of four quadrilaterals or triangles meeting in a common
  vertex. Let $\domain$ be the union of these grid cells. Choose
  linear and bilinear shape functions for $V_h$, respectively. Then, $\dim
  V_h = 2$, since we have one interior vertex with one basis function for
  each velocity component. Choose piecewise constant pressure
  functions. Dividing out the global constant, we conclude that $\dim
  Q_h = 3$. Thus, the statement
  \begin{gather*}
    \forall q_h\in Q_h \;\exists v_h\in V_h:
    \quad \norm{v_h}_1 = \norm{q_h}_0
    \;\wedge\; b(v_h, q_h) \ge \beta \norm{q_h}^2
  \end{gather*}
  cannot hold true. Therefore, the inf-sup condition does not hold. In
  fact, $\ker{B_h} = \{0\}$.
  \begin{figure}[tp]
    \centering
    \includegraphics[width=.4\textwidth]{./fig/patch1.tikz}
    \hfill
    \includegraphics[width=.4\textwidth]{./fig/patch2.tikz}
    \caption[Very coarse meshes with Dirichlet boundary.]{Very coarse meshes with Dirichlet boundary. Degrees of freedom for pressure (\tikz\draw[shape pressure] (0,0) circle (1ex);) and for both velocity components(\tikz\draw[shape veloxy] (0,0) circle (1ex);).}
    \label{fig:stokes:example1}
  \end{figure}

  Thus, we conclude that for this combination of shape function
  spaces, there is a mesh such that they are not suited for the
  approximation of the Stokes problem. But, this may be a problem of a
  mesh with too few cells. In fact, asymptotically, a triangular mesh
  contains twice as many vertices as cells, a quadrilateral mesh as
  many. Therefore, $\dim V_h > \dim Q_h$ as soon as the mesh is
  sufficiently fine. Will this be sufficient?
\end{example}

\begin{Problem}{checker-board}
  Let $\domain = (0,1)^2$ be the unit square and let the mesh consist
  of Cartesian squares of side length $1/n$. Choose $V_h \subset V$
  based on bilinear shape functions. Show that the piecewise constant
  pressure function $p_c=\pm 1$ in a checkerboard fashion is in the
  kernel of $B_h^T$, that is
  \begin{gather*}
    b(v_h, p_c) = 0 \quad\forall v_h\in V_h.
  \end{gather*}
\end{Problem}

\section{Nearly incompressible elasticity}

%%% Local Variables:
%%% mode: latex
%%% TeX-master: "main"
%%% End:


\chapter{Mixed formulation of elliptic problems}
\label{cha:darcy}

\section{Modelling diffusion problems}

\begin{intro}
  Diffusion problems arise when a balance law, for instance for mass
  in ground water flow or for energy in temperature conduction is
  coupled with a constitutive equation relating the direction of
  movement to the gradient of the quantity of interest.
\end{intro}

\begin{intro}
  Let $\rho$ be the density of a conserved quantity. Then, for any given
  volume we have the ``mass''
  \begin{gather*}
    m = \int_V \rho\dx.
  \end{gather*}
  Changes of this mass can be due to two processes:
  \begin{enumerate}
  \item Generation of additional mass by a source $g$,
  \item Flow of mass over the boundary of $V$ at a velocity $v$.
  \end{enumerate}
  In formulas, we have
  \begin{gather*}
    \tfrac{d}{dt} m = \int_V g\dx - \oint_{\d V} J\cdot \n \ds,
  \end{gather*}
  also known as \define{Reynolds transport theorem}. Here, $J$ is the
  \define{flux}. The exact form of the flux will be modelled later.
  The formula above is somewhat unwieldy, since it combines volume and
  surface integrals. Therefore, we apply the \putindex{Gauss theorem}
  to obtain
  \begin{gather}
    \label{eq:darcy:2}
    \frac{d}{dt} \int_V \rho\dx = \int_V g\dx - \int_V \div J \dx.
  \end{gather}
  Concentrating and assuming sufficient regularity, we arrive at the
  equation
  \begin{gather}
    \label{eq:darcy:3}
    \d_t \rho + \div J = g.
  \end{gather}
  As before in these notes, we ignore the time dependence and only
  look at stationary limits. In this case, this reduces to
  \begin{gather}
    \label{eq:darcy:4}
    \div J = g.
  \end{gather}
\end{intro}

\begin{example}
  Next we consider constitutive relations between $\rho$ and $J$ such
  that we can complement equation~\eqref{eq:darcy:4} by a second
  equation and obtain a solvable system. To this end, we consider
  thermal diffusion and ground water flow.

  \begin{description}
  \item[Heat conduction:] Here, the conserved quantity is not the
    density $\rho$, but the temperature $T$. \define{Fourier's law}
    states that the flux is proportional to the gradient of the
    temperature, pointing in opposite direction:
    \begin{gather*}
      J = -k\nabla T.
    \end{gather*}
    The constant of proportionality $k$ is the heat conductivity.
  \item[Porous media flow:] The conserved quantity is the amount of
    fluid, represented by the hydraulic head or pressure
    $p$. \define{Darcy's law} says that the flux is the product of the
    hydraulic \define{permeability} of the media and the gradient of the
    pressure:
    \begin{gather*}
      J = -K\nabla p.
    \end{gather*}
    Here, the permeability $K$ is either a positive scalar function or
    a symmetric, positive definite matrix. Note that in the latter
    case, $J$ and $\nabla p$ do not point in the same direction.
    \item[General diffusion processes:] \define{Fick's law} states,
      that the flux of a diffusion process is determined by the
      gradient of the diffusing quantity $p$ by the relation
    \begin{gather*}
      J = -D\nabla p.
    \end{gather*}
    $D$ is the symmetric, positive definite \define{diffusion tensor}.
  \end{description}
\end{example}

\begin{intro}
  From the two equations for $J$, we derive the following system of
  PDE:
  \begin{gather*}
    \arraycolsep2pt
    \begin{matrix}
      K^{-1} J &-& \nabla p &=& 0
      \div J &&&= f.
    \end{matrix}
  \end{gather*}
\end{intro}


% \begin{Definition}{mass-balance}
%   We call a quantity $p$ conserved, if the \define{conservation law}
%   \begin{gather}
%     \label{eq:darcy:1}
%     \d_t p = \div p + g
%   \end{gather}
%   holds with a \define{source} function $g$. In words, the change of
%   the quantity within a control volume is equal to the balance of
%   sources and the flow over the boundary of the volume.
% \end{Definition}


%%% Local Variables:
%%% mode: latex
%%% TeX-master: "main"
%%% End:


\chapter{Divergence conforming discontinuous Galerkin methods}
\label{cha:hdivdg}
\begin{intro}
  In the previous chapter, we studied discretizations with
  $\div V_h = Q_h$ with two advantages. First, due to
  \blockref{Corollary}{galerkin-mixed-u-kerb} the velocity error is
  independent of the pressure. Second, the divergence converges faster
  than the gradient. A natural question arising is whether we can do
  something similar for the Stokes problem. There, the equation
  \begin{gather*}
    \form(\div v_h, q_h) = 0 \qquad\forall q_h\in Q_h,
  \end{gather*}
  would immediately imply $\div v_h=0$, that is, the discrete solution
  is exactly divergence free.
  
  The answer to this question is a current research topic. So far,
  beginning with the element by Scott and Vogelius, several methods
  have been proposed for special mesh geometry or macro meshes. The
  difficulty is balancing the condition $\div V_h = Q_h$ with the
  $H^1$-conformity of the velocity space. All the spaces in the
  previous chapter were only $\Hdiv$-conforming with discontinuous
  tangential components.
  
  A fairly simple solution to this question though can be obtained by
  using discontinuous Galerkin methods. These were introduced to
  obtain formulations \emph{consistent} with $H^1$ while not
  \emph{conforming}. Thus, we can apply them directly to
  Raviart-Thomas oder Brezzi-Douglas-Marini elements to obtain a
  consistent method with divergence free solutions.

  We begin this chapter by a quick review of the interior penalty
  method before diving into divergence conforming methods.
\end{intro}


% \begin{intro}
%   In this section we extend the weakening of continuity, which we
%   explored for boundary values in Section~\ref{sec:nitsches-method}
%   using Nitsche's method to interior interfaces between mesh
%   cells. While the methods obtained may look much more complicated,
%   the mathematical analysis is completely analogue to that
%   section. Thus, we can be fairly brief.
% \end{intro}

\begin{intro}
  We review the basic definitions necessary to describe discontinuous
  Galerkin (DG) methods. In particular, we need the sets of faces
  $\F_h$ of a mesh, discontinuous piecewise polynomial spaces and
  broken integrals.
\end{intro}

\begin{Definition}{dg-faces}
  Let $\T_h$ be a mesh of $\Omega \subset \R^d$ consisting of mesh
  cells $T_i$. For every boundary facet $F\subset \partial T_i$, we
  assume\footnote{This assumption can indeed be relaxed} that either
  $F \subset \partial \Omega$ or $F$ is a boundary facet of another
  cell $T_j$. In the second case, we indicate this relation by
  labeling this facet $F_{ij}$. The set of all facets $F_{ij}$ is the
  set of interior faces $\F_h^i$. The set of facets on the boundary is
  $\F_h^\partial$.
\end{Definition}

\begin{Definition}{dg-spaces}
  The discontinuous finite element space on $\T_h$ is constructed by
  concatenation of all shape function spaces $P_T$ for $T\in \T_h$
  without additional continuity requirements:
  \begin{gather}
    V_h = \bigl\{v\in L^2(\Omega) \big|
    v_{|T} \in P_T \;\forall T\in \T_h\bigr\}.
  \end{gather}
\end{Definition}

\begin{Definition}{broken-integrals}
  For any set of cells $\mesh_h$ or faces $\faces_h$, we define the bilinear
  forms
  \begin{align}
    \form(u,v)_{\mesh_h} &= \sum_{\cell\in\mesh_h} \form(u,v)_\cell, \\
    \forme(u,v)_{\faces_h} &= \sum_{\face\in\faces_h} \forme(u,v)_\face. \\
  \end{align}
\end{Definition}

\begin{intro}
  We start out with the equation
  \begin{gather*}
    -\Delta u = f.
  \end{gather*}
  Integrating by parts on each mesh cell yields
  \begin{gather*}
    \form(-\Delta u,v )_\cell
    = \form(\nabla u, \nabla v)_\cell - \forme(\d_n u, v)_{\d\cell} = \form(f,v)_T.
  \end{gather*}
  We realize that the choice of discontinuous finite element spaces
  introduces a consistency term on the interfaces between cells and on
  the boundary.

  On interior faces, there is the issue that $u$ and
  $\d_n u$ actually have two values on the interface, one from the
  left cell and one from the right. Therefore, we have to consolidate
  these two values into one. To this end, we introduce the concept of
  a numerical flux, which constructs a single value out of these
  two. Thus, we introduce on the interface $\face$ between two cells
  $\cell^+$ and $\cell^-$
  \begin{gather*}
    \mathcal F(\nabla u) = \frac{\nabla u^+ + \nabla u^-}{2} = :
    \mvl{\nabla u}.
  \end{gather*}

  Using $\forme(\d_n u,v) = \forme(\nabla u,v\n)$ we change our point
  of view and instead of integrating over the boundary $\d\cell$, we
  integrate over a face $\face$ between two cells $\cell^+$ and
  $\cell^-$. Adding up integrals from both sides, we obtain the term
  \begin{gather*}
    -\forme(\mvl{\nabla u},v^+\n^+ +v^-\n^-)_{\face}
    = -2\forme(\mvl{\nabla u},\mvl{v\n})_{\face}.
  \end{gather*}
  On boundary faces, we simply get
  \begin{gather*}
    \forme(\d_\n u,v)_{\face}.
  \end{gather*}

  Adding over all cells and faces, we obtain the equation
  \begin{gather*}
    \form(\nabla u,\nabla v)_{\T_h}
    -2\forme(\mvl{\nabla u},\mvl{v\n})_{\F_h^i}
    -\forme(\d_\n u,v)_{\F_h^\d} = \form(f,v)_{\domain}.
  \end{gather*}

  Following the idea of Nitsche, we symmetrize this term
  to obtain
  \begin{multline*}
    \form(\nabla u,\nabla v)_{\T_h}
    -2\forme(\mvl{\nabla u},\mvl{v\n})_{\F_h^i}
    -2\forme(\mvl{u\n},\mvl{\nabla v})_{\F_h^i}
    \\
    -\forme(\d_\n u,v)_{\F_h^\d}
    -\forme(u,\d_\n v)_{\F_h^\d}
    = \form(f,v)_{\domain}
    - \forme(u^o,\d_n v)_{\F_h^\d}.
  \end{multline*}
  Here the second term on the right was introduced for consistency.
  Finally, it turns out that this method is not stable and needs
  stabilization by a jump term. This will be done in
  \blockref{Definition}{ip}. Before, we introduce the notation for
  averaging and jump operators.
\end{intro}

\begin{Notation}{dg-operators}
  Let $\face$ be a face between the cells $\cell^+$ and $\cell^-$. Let
  $\n^+$ and $\n^-=-\n^+$ be the outer normal vectors of the cells at a
  point $x\in \face$. For a function $u\in V_h$, the traces $u^+$ and
  $u_-$ of $u$ on $\face$ taken from the cell $\cell^+$
  and $\cell^-$ are defined as:
  \begin{align*}
    u^+(x) &= \lim_{\epsilon\searrow 0} u(x-\epsilon\n^+), \\
    u^-(x) &= \lim_{\epsilon\searrow 0} u(x-\epsilon\n^-).
  \end{align*}
  We define the \define{averaging operator} $\mvl{.}$ and the
  \define{jump operator} $\jmp{.}$ as
  \begin{gather}
    \label{eq:ip:1}
    \mvl{u} = \frac{u^++u^-}{2},
    \qquad
    \jmp{u} = u^+-u^-.
  \end{gather}
  Not that the sign of the jump of $u$ depends on the choice of the
  cells $\cell^+$ and $\cell^-$. It will only be used in quadratic
  terms.
\end{Notation}

\begin{remark}
  The jump can be denoted as the mean value of the product of a
  function and the normal vector,
  \begin{gather}
    \jmp{u} = 2\mvl{u\n}\cdot\n^+ = -2\mvl{u\n}\cdot\n^-.
  \end{gather}
\end{remark}

\begin{Definition}{ip}
  The \define{interior penalty method}\footnote{Also known as
    symmetric interior penalty (SIPG) or IP-DG.} uses the bilinear
  form
  \begin{multline}
    \label{eq:ip:2}
    a_h(u,v) = \form(\nabla u,\nabla v)_{\mesh_h}
    + \forme(\ipp_h\jmp{u},\jmp{v})_{\faces_h^i}
    + \forme(\ipp_h u,v)_{\faces_h^\d}
    \\
    -2\forme(\mvl{\nabla u},\mvl{v\n})_{\faces_h^i}
    -2\forme(\mvl{u\n},\mvl{\nabla v})_{\faces_h^i}
    \\
    - \forme(\d_n u,v)_{\faces_h^\d}
    - \forme(u,\d_n v)_{\faces_h^\d},
  \end{multline}
  and the linear form
  \begin{gather}
    \label{eq:ip:3}
    f_h(v) = \form(f,v)_{\domain} - \forme(u^D,\d_n v)_{\faces_h^\d}
    + \forme(\ipp_h u,v)_{\faces_h^\d},
  \end{gather}
  where $f$ is the right hand side of the equation and $u^D$ the
  Dirichlet boundary value.
\end{Definition}

\begin{Definition}{ip-norm}
  On the space $V_h$ we define the norm $\norm{.}_{1,h}$ by
  \begin{gather}
    \label{eq:ip:4}
    \norm{v}_{1,h}^2 = \sum_{\cell\in\mesh_h} \norm{\nabla v}_\cell^2
    + \sum_{\face\in\faces_h^i} \norm{\sqrt{\ipp_h}\jmp{v}}_\face^2
    + \sum_{\face\in\faces_h^\d} \norm{\sqrt{\ipp_h}v}_\face^2.
  \end{gather}
\end{Definition}

\begin{Problem}{ip-norm}
  Prove that the norm defined in (\ref{eq:ip:4}) is indeed a norm on $V_h$.
\begin{solution}
\begin{align*}
  0=\norm{v}_{1,h}^2 &= \sum_{\cell\in\mesh_h} \norm{\nabla v}_\cell^2
    + \sum_{\face\in\faces_h^i} \norm{\sigma_h\jmp{v}}_\face^2
    + \sum_{\face\in\faces_h^\d} \norm{\sigma_hv}_\face^2
\end{align*}
implies first of all $v|_T \equiv const.$ for all $T \in \mathbb{T}_h$.
Furthermore, $\norm{\sigma_h\jmp{v}}_\face^2=0$ implies $v \equiv const.$
and the last condition gives $v \equiv 0$.

 We use the trace inequality
 \begin{align*}
  \norm{v}_{0,\partial T}^2
  \lesssim \left( h_T^{-1} \norm{v}_{0,T}^2+\norm{v}_{0,T}\norm{\nabla v}_{0,T}\right)
  \quad \forall v \in H^1(T).
 \end{align*}
 Then, we can estimate
 \begin{align*}
  &\sum_{\face\in\faces_h^i} \norm{\sigma_h\jmp{v}}_\face^2 + \sum_{\face\in\faces_h^\d} \norm{\sigma_hv}_\face^2\\
  &\lesssim \min_{F\in F_h^i\cup F_h^\partial} \sigma_{h,F} \sum_{T \in \mathbb{T}_h}\norm{v}_{0,\partial T}^2 \\
  &\lesssim \min_{F\in F_h^i\cup F_h^\partial} \sigma_{h,F} \sum_{T \in \mathbb{T}_h}\left( h_T^{-1}
  \norm{v}_{0,T}^2+\norm{v}_{0,T}\norm{\nabla v}_{0,T}\right)\\
  &\lesssim \min_{F\in F_h^i\cup F_h^\partial} \sigma_{h,F} \sum_{T \in \mathbb{T}_h}\left( h_T^{-1}
  \norm{v}_{0,T}^2+h_T\norm{\nabla v}_{0,T}\right)
 \end{align*}
 and thus
 \begin{align*}
    \norm{v}_{1,h}^2 &= \sum_{\cell\in\mesh_h} \norm{\nabla v}_\cell^2
    + \sum_{\face\in\faces_h^i} \norm{\sigma_h\jmp{v}}_\face^2
    + \sum_{\face\in\faces_h^\d} \norm{\sigma_hv}_\face^2\\
    &\lesssim \min_{F\in F_h^i\cup F_h^\partial} \sigma_{h,F}
      \sum_{T \in \mathbb{T}_h}\left( h_T^{-1} \norm{v}_{0,T}^2+\norm{\nabla v}_{0,T}\right)\\
  &\lesssim \norm{v}_1^2
  \end{align*}
\end{solution}
\end{Problem}

\begin{Lemma}{ip-stability}
  Let $\T_h$ be shape-regular and chosen on each face $\face$ as
  $\sigma_h = \sigma_0/h_\face$, where $h_T$ is the minimal diameter
  of a cell adjacent to $\face$. Then, there is a $\sigma_0>0$ such
  that there exists a constant $\ellipa>0$, such that independent of
  $h$ there holds
  \begin{gather}
    \label{eq:ip:5}
    a_h(u_h,u_h) \ge \ellipa \norm{u_h}_{1,h}^2 \quad \forall u_h\in V_h.
  \end{gather}
\end{Lemma}

\begin{Problem}{ip-stability}
  Prove \blockref{Lemma}{ip-stability}.
\begin{solution}
We first note the estimate
\begin{align*}
 (\boldsymbol{n}\cdot\nabla v_h)_e^2
 &\leq C (h_K^{-1}\norm{\nabla v_h}_{0,K}^2+\norm{\nabla v_h}_{0,K}\norm{\nabla^2 v_h}_{0,K}) \\
 &\leq C \left(\frac{1}{h_K}+\frac{p_K^2}{h_K}\right) \norm{\nabla v_h}_{0,K}^2\\
 &\leq C \frac{p_K^2}{h_K} \norm{\nabla v_h}_{0,K}^2
 = C\frac{\sigma_h}{\delta} \norm{\nabla v_h}_{0,K}^2
\end{align*}

Testing the bilinear form symmetrically, we obtain
\begin{align*}
 a_h(u_h,u_h) &= \norm{\nabla u_h}_0^2
    + \ipp_h\norm{\jmp{u_h}}_{\faces_h^i}
    + \ipp_h\norm{u_h}_{\faces_h^\d}
    \\
    &-4\forme(\mvl{\nabla u_h},\mvl{u_h\n})_{\faces_h^i}
    -2 \forme(\d_n u_h,u_h)_{\faces_h^\d}.
\end{align*}
and the last two terms can be estimated by
\begin{align*}
 \forme(\mvl{\nabla u_h},\mvl{u_h\n})_{\faces_h^i} &=\forme(n^+\cdot\nabla u_h^+-n^-\cdot\nabla u_h^-, u_h^+-u_h^-)\\
 &\leq \frac{\epsilon}{2}\norm{n^+\{\{\nabla u_h\}\}}_0^2+\frac{1}{2\epsilon}\norm{[[u_h]]}_0^2\\
 &\leq C\frac{\sigma_h\epsilon}{2\delta}\norm{\nabla u_h}_{0,K}^2+\frac{1}{2\epsilon}\norm{[[u_h]]}_0^2
\end{align*}
and therefore
\begin{align*}
 a_h(u_h,u_h) -\gamma \norm{u_h}_{1,h}^2&\geq \norm{\nabla u_h}_0^2 \left(1-\gamma-C\frac{\sigma_h\epsilon}{2\delta}\right)\\&\quad
    + \ipp_h\norm{\jmp{u_h}}_{\faces_h^i}\left(1-\gamma-\frac{1}{2\epsilon\sigma_h}\right).
\end{align*}

Hence, we have to choose $\epsilon, \delta>0$ such that
\begin{align*}
 1-\gamma-C\frac{\sigma_h\epsilon}{2\delta}>0 \\
 1-\gamma-\frac{1}{2\epsilon\sigma_h}>0 .
\end{align*}
This is possible for all $\gamma\in(0,1)$ and in fact we get the lower limits
\begin{align*}
\epsilon&>\frac{1}{2(1-\gamma)\sigma_h}\\
 \delta&>C \frac{\sigma_h \epsilon}{2(1-\gamma)}.
\end{align*}


\end{solution}

\end{Problem}

\begin{Lemma}{ip-consistence}
  Let $f\in L^2(\domain)$ and let the boundary conditions admit that
  for the solution to
  \begin{xalignat*}2
    -\Delta u &= f &\text{in }&\domain, \\
    u &= u^D &\text{on }&\d\domain,
  \end{xalignat*}
  there holds $u\in H^{1+\epsilon}(\domain)$ for a positive
  $\epsilon$. Then, the interior penalty method is consistent, that
  is,
  \begin{gather}
    a_h(u,v_h) = f_h(v_h)\quad\forall v_h\in V_h.
  \end{gather}
\end{Lemma}

\begin{proof}
  From $f\in L^2(\domain)$ we deduce that
  $\nabla u\in \Hdiv(\domain)$. Thus, with the extra regularity, the
  traces of $\d_n u$ on faces are well-defined and coincide from both
  sides. The remainder is integration by parts.
\end{proof}

\begin{Theorem}{ip-convergence}
  For $k\ge 1$ let $\P_k\subset P_\cell$ and $u\in H^{s+1}(\domain)$ with
  $1/2 \le s \le k$. Then, the interior penalty method admits the
  error estimate
  \begin{gather}
    \norm{u-u_h}_{1,h} \le c h^s \snorm{u}_{s+1}.
  \end{gather}
  If furthermore the boundary condition admits \putindex{elliptic
    regularity},
there holds
  \begin{gather}
    \norm{u-u_h}_{0} \le c h^{s+1} \snorm{u}_{s+1}.
  \end{gather}
\end{Theorem}

%%% Local Variables:
%%% mode: latex
%%% TeX-master: "main"
%%% End:






\begin{remark}
  The extension of the interior penalty method to vector-valued
  problems is obvious. Furthermore, since the method generates an
  elliptic bilinear form on the discontinuous space $V_h$, this
  ellipticity is inherited by any subspace of
  $V_h\cap\Hdiv(\domain)$. Thus, we can write down the weak
  formulation of a divergence conforming DG method for the Stokes
  equations. In the following definition, we assume slip or no-slip
  boundary conditions, that is, $v\cdot\n=0$ on the whole boundary.
\end{remark}

\begin{Definition}{hdiv-ip}
  A divergence conforming DG method for the Stokes equations consists
  of a discrete velocity space $V_h\subset \Hdiv_0(\domain)$ and a
  pressure space $Q_h\subset L^2_0(\domain)$ such that
  \begin{gather}
    \label{eq:hdivdg:1}
    \div V_h = Q_h.
  \end{gather}
  Using the interior penalty bilinear form $a_h(.,.)$, we search for
  solutions $(u_h,p_h)\in V_h\times Q_h$ such that for all $(v,q)\in
  V_h\times Q_h$ there holds
  \begin{gather}
    \label{eq:hdivdg:2}
    a_h(u_h,v) +\form(\div v,p_h)+\form(\div u_h,q) = f(v).
  \end{gather}
\end{Definition}

\begin{Lemma}{dg-fortin}
  The \putindex{canonical interpolation} operators of the
  Brezzi-Douglas-Marini and Raviart-Thomas elements admit the bound
  \begin{gather}
    \label{eq:hdivdg:4}
    \norm{v-I_h v}_{1,h} \le c \norm{v}_1
  \end{gather}
\end{Lemma}

\begin{Lemma}{hdivdg-infsup}
  The method in \blockref{Definition}{hdiv-ip}
  admits the inf-sup condition
  \begin{gather}
    \label{eq:hdivdg:3}
    \inf_{q_h\in Q_h} \sup_{v_h\in V_h}
    \frac{\form(\div v_h,q_h)}{\norm{v_h}_{1,h}\norm{q_h}_0} \ge \beta,
  \end{gather}
  with a constant $\beta >0$ independent of $h$.
\end{Lemma}

% \begin{proof}
%   First, we make use of the fact that $q_h\in Q_h \subset Q$ to deduce
%   that 
% \end{proof}

%%% Local Variables: 
%%% mode: latex
%%% TeX-master: "main"
%%% End: 


\chapter{Maxwell's equations and the de Rham complex}
\section{Maxwell's equations}
\label{cha:maxwell}
\begin{Notation}{curl}
  With $\curl u$ we describe the curl of a vector field $u$, which in
  three dimensions is defined as
  \begin{gather}
    \label{eq:maxwell:2}
    \curl u =
    \begin{pmatrix}
      \d_2u_3-\d_3u_2\\\d_3u_1-\d_1u_3\\\d_1u_2-\d_2u_1
    \end{pmatrix}.
  \end{gather}
  In two dimension, we distinguish between the vector curl of a scalar
  function and the scalar curl of a vector function
  \begin{gather}
    \curl u = \d_1u_2-\d_2u_1,
    \qquad
    \curl \phi =
    \begin{pmatrix}
      \d_2 \phi \\ -\d_1\phi
    \end{pmatrix}.
  \end{gather}
\end{Notation}

\begin{remark}
  The scalar curl of a two-dimensional vector field is equal to the
  third component of the extension of this vector field by zero into
  $\R^3$, in formulas,
  \begin{gather*}
    \curl
    \begin{pmatrix}
      u_1\\u_2
    \end{pmatrix}
    =
    \curl
    \begin{pmatrix}
      u_1\\u_2\\0
    \end{pmatrix}_3.
  \end{gather*}
  Similarly, the vector curl of a scalar function $\phi$ in two dimensions
  consistes of the first two components of the curl of a three
  dimensional function in the last component of t he vector,
  \begin{gather*}
    \curl \phi = \curl
    \begin{pmatrix}
      0\\0\\\phi
    \end{pmatrix}_{1,2}.
  \end{gather*}
\end{remark}
\begin{remark}
  A polular error in the literature consists of the following
  argument: since $\div E = 0$, there also holds $\nabla \div E =
  0$. Therefore, we can use the formula
  \begin{gather*}
    \Delta u = \nabla\div u - \curl\curl u,
  \end{gather*}
  and avoid the div-curl-problem alltogether. Unfortunately, this is
  only true, if solutions of~\eqref{eq:maxwell:1} are in
  $H^1(\domain;\R^d)$, which is not true, depending on the boundary
  conditions.
\end{remark}

\begin{Lemma}{curl-green}
  For vector fields $u,v\in C^1(\overline{\domain})$, there holds
  \begin{gather}
    \label{eq:maxwell:3}
    \int_{\domain} \curl u\cdot v\dx = \int_{\domain} u\cdot\curl v\dx
    + \int_{\d\domain} (n\times u) \cdot v \ds.
  \end{gather}
\end{Lemma}

\begin{intro}
  Electromagnetic fields are governed by four laws of nature put together
  by James Clerk Maxwell to a single system. The laws are
  \begin{enumerate}
  \item Gauss' law for the electric field: the electric flux
    through a closed surface equals $1/\epsilon$ times the electric
    charge enclosed by the surface:
    \begin{gather*}
      \int_{\d V} E\cdot \n \ds = \int_V \frac\rho\epsilon \dx.
    \end{gather*}
  \item There are no magnetic monopoles, therefore the magnetic flux
    through any closed surface vanishes:
    \begin{gather*}
      \int_{\d V} B\cdot \n \ds = 0.
    \end{gather*}
  \item Faraday's law of induction: the voltage induced in a closed
    loop is proportional to the rate of change of the magnetic field
    through the surface encloded by the loop:
    \begin{gather*}
      \int_{\d A} E\cdot\ds = -\frac{d}{dt}\int_A B\cdot \n \ds.
    \end{gather*}
  \item Ampère's law: the magnetic field induced in a closed loop is
    proportional to the electric current plus the change of electric
    field through that loop:
    \begin{gather*}
      \int_{\d A} B\cdot\ds
      = \mu \int_A J\cdot \n \ds
      + \mu\epsilon\frac{d}{dt}\int_A E\cdot \n \ds.
    \end{gather*}
  \end{enumerate}
  
  Using the Gauss theorem for the first two and the Stokes theorem for
  the remaining two laws, we obtain the \define{Maxwell equations} of
  electromagnetics
  \begin{xalignat}2
    \div E &= \frac\rho\epsilon
    & \curl E &= -\d_t B,\\
    \div B &= 0
    & \curl B &= \mu J + \mu\epsilon E.
  \end{xalignat}
  They are an hyperbolic system of equations and typically have wave
  solutions. Many simplifications have been developed to suit
  particular purposes.
\end{intro}

\begin{intro}
  An important simplification of the Maxwell equations is obtained by
  assuming an isolating material, that is, the electric current $J$
  vanishes. Additionally, we may assume that there are no electric
  charges, such that $\div E=0$. Then, taking the curl of the equation
  for $\curl E$ and inserting the formula for $\curl B$, we obtain
  \begin{gather}
    \mu\epsilon \d_t^2 E + \curl\curl E = 0
    \qquad \div E=0.
  \end{gather}
  We can even go further and study the stationary limit
  \begin{gather}
    \label{eq:maxwell:1}
    \curl\curl E=0 \qquad \div E=0.
  \end{gather}
  This is the equation we are concerned with most, since its solution
  theory also provides insight into the other forms.
\end{intro}


\begin{Definition}{Maxwell-boundary}
  The Maxwell equation~\eqref{eq:maxwell:1} is complemented with the
  following boundary conditions:
  \begin{itemize}
  \item Perfectly conducting:
    \begin{gather}
      \n\cdot u = 0.
    \end{gather}
    \item Natural:
      \begin{gather}
        n\times \curl u = 0.
      \end{gather}
    \item Impedance:
      \begin{gather}
        n\times \curl u - \alpha  (\n\times u) \times \n = 0.
      \end{gather}
  \end{itemize}
\end{Definition}


\begin{Definition}{curl-traces}
  For $u\in C^1(\overline\domain)$, we define the trace operators
  \begin{gather}
    \label{eq:maxwell:4}
    \begin{split}
      \gamma_{\tau} &= \n\times u_{|\d\domain}, \\
      \gamma_{T} &= \n\times u_{|\d\domain} \times \n.\\
    \end{split}
  \end{gather}
  The second of these is the tangential component of $u$ on the
  boundary. Furthermore, we introduce the space $\Hcurl_0$ as the
  completion of the space of differentiable functions with compact
  support under the norm of $\Hcurl$
  \begin{gather}
    \Hcurl_0 = \overline{C^\infty_{00}(\domain;\R^d)}^{\Hcurl}.
  \end{gather}
\end{Definition}


\begin{Theorem}{curl-traces}
  The trace operator $\gamma_\tau$ can be extended to a continuous,
  surjective operator
  \begin{gather*}
    \gamma_\tau \colon \Hcurl(\domain) \to Y(\d\domain),
  \end{gather*}
  where
  \begin{gather}
    \label{eq:maxwell:5}
    \begin{split}
      Y(\d\domain) &= \bigl\{
      u\in H^{-1/2}_\tau(\d\domain) \big\vert
      \;\nu\cdot(\curl u) \in H^{-1/2}(\d\domain) \bigr\},\\
      H^{-1/2}_\tau(\d\domain) &= \bigl\{
      u\in H^{-1/2}(\d\domain;\R^d) \big\vert 
      u\cdot\n=0 \text{ a.e.}\bigr\}.
    \end{split}
  \end{gather}
  Furthermore, the trace operator $\gamma_T$ can be extended to a
  continuous operator
  \begin{gather*}
    \gamma_T \colon \Hcurl(\domain) \to Y(\d\domain)^*.
  \end{gather*}
\end{Theorem}


\begin{intro}
  The trace theorem indicates, that $\Hcurl_0(\domain)$ is the correct
  space to solve the problem with perfectly conducting boundary
  condition on the whole boundary. It remains now to deal with the
  divergence constraint. First, we note, that the divergence operator
  is not well-defined on $\Hcurl$, and that the subspace of $\Hcurl$
  with divergence in $L^2$ is $H^1$, which must be avoided. Therefore,
  we have to resort to a dual formulation of this constraint, which
  leads to the following weak form of the perfectly conducting Maxwell
  problem.
\end{intro}

\begin{Definition}{Maxwell-mixed-0}
  The Maxwell problem for perfectly conducting boundary conditions in
  weak form reads: find $(u,p)\in V\times Q$, where
  $V=\Hcurl_0(\domain)$ and $Q=H^1_0(\domain)$ such that there holds
    \begin{gather}
      \label{eq:maxwell:6}
    \begin{aligned}
      \form(\curl u, \curl v) &+ \form(v,\nabla p) &=&\form(f,v)
      &\forall v&\in V\\
      \form(u,\nabla q) & &=&0
      &\forall q&\in Q.\\      
    \end{aligned}
  \end{gather}
\end{Definition}

\begin{remark}
  At this point, our task is laid out. We have to prove well-posedness
  of the Maxwell problem in mixed form, then find suitable finite
  element spaces and commuting interpolation operators.  It turns out
  that this can be done in a more general framework, called the de
  Rham complex.
\end{remark}

% \begin{Definition}{Maxwell-mixed-imp}
%   The Maxwell problem for impedance boundary conditions in
%   weak form reads: find $(u,p)\in V\times Q$, where
%   \begin{gather}
    
%   \end{gather}
%   $V=\Hcurl_0(\domain)$ and $Q=H^1_0(\domain)$ such that there holds
%     \begin{gather}
%       \label{eq:maxwell:6}
%     \begin{aligned}
%       \form(\curl u, \curl v) &+ \form(v,\nabla p) &=&\form(f,v)
%       &\forall v&\in V\\
%       \form(u,\nabla q) & &=&0
%       &\forall q&\in Q.\\      
%     \end{aligned}
%   \end{gather}
% \end{Definition}



%%% Local Variables: 
%%% mode: latex
%%% TeX-master: "main"
%%% End: 


\section{The de Rham complex}
\label{cha:derham}
\begin{intro}
  We can embed finite element methods for the Darcy problem, also for
  the Maxwell problem, into a common framework based on the de Rham
  complex. If we wanted to do this in its full mathematical beauty, we
  would have to spend some time introducing the concept and notation
  of differential forms. As an alternative, we can use the concrete
  vector spaces $\Hdiv(\domain)$ and $\Hcurl(\domain)$. The drawback
  is, that we have to prove several particular cases, where the
  abstract theory only knows one common case. Nevertheless, it is
  worthwhile to begin this way, such that the reader has an easier
  task reading the full theory
  in~\cite{ArnoldFalkWinther06acta,ArnoldFalkWinther10}. As a byproduct,
  we will prove in generality some of the properties of polynomial
  spaces in Chapter~\ref{cha:darcy}.
\end{intro}

\begin{intro}
  We now know three differential operators, $\nabla$, $\curl$, and
  $\div$ with the interesting property
  \begin{gather}
    \label{eq:derham:10}
    \curl\nabla \phi = 0
    \qquad \div\curl E=0.
  \end{gather}
  As a consequence, for $\phi\in H^1(\domain)$ we not only have
  $\nabla \phi\in L^2(\domain;\R^3)$, we also have
  $\curl\nabla\phi=0\in L^2(\domain;\R^3)$. This gives rise to the sequence
  \begin{gather}
    \label{eq:derham:11}
    \R
    \overset{\subset}{\longrightarrow} H^1(\domain)
    \overset{\nabla}{\longrightarrow} \Hcurl(\domain)
    \overset{\curl}{\longrightarrow} \Hdiv(\domain)
    \overset{\div}{\longrightarrow} L^2(\domain)
    \longrightarrow 0,
  \end{gather}
  such that the range of an operator is always in the kernel of the
  operator to its right.
\end{intro}

\begin{Notation}{hlambda}
  The notation of exterior calculus of differential forms allows us to
  write this sequence elegantly as
  \begin{gather}\minCDarrowwidth20pt
    \label{eq:derham:9}
    \begin{CD}
      \R
      @>{d}>> H\Lambda^0(\domain)
      @>{d}>> H\Lambda^1(\domain)
      @>{d}>> H\Lambda^2(\domain)
      @>{d}>> H\Lambda^3(\domain)
      @>>> 0
      \\
      @.
      @V{\cong}VV
      @V{\cong}VV
      @V{\cong}VV
      @V{\cong}VV
      \\
      \R
      @>{\subset}>> H^1(\domain)
      @>{\nabla}>> \Hcurl(\domain)
      @>{\curl}>> \Hdiv(\domain)
      @>{\div}>> L^2(\domain)
      @>>> 0,
    \end{CD}
  \end{gather}
  such that $d=d_k\colon H\Lambda^k(\domain) \to H\Lambda^{k+1}(\domain)$ and
  \begin{gather}
    d^2 = d\circ d = d_{k+1} \circ d_k = 0.
  \end{gather}
\end{Notation}

\begin{remark}
  The spaces $H\Lambda^k(\domain)$ are Hilbert spaces with values in
  the spaces of alternating $k$-forms on $\R^d$. From linear algebra,
  we know that all alternating $k$-forms are zero if $k$ exceeds the
  dimension of the vector space.  Therefore, the sequence above is
  only valid in three dimensions, and it must be shorter by one member
  in two dimensions. Changing our view back to differential operators,
  we realize that there are two relevant sequences in two
  dimensions. In the following diagram, the sequence on top can be
  used to formulate Maxwell problems in $\Hcurl$ in two dimensions,
  while the sequence on the bottom relates to the mixed form of the
  Laplacian.

  We introduce the sequences in two dimensions and afterwards will
  focus our arguments on the more general case of three dimensions
  again. Specialization to two dimensions are straight forward.
\end{remark}

\begin{Notation}{hlambda-2d}
  In two dimensions, we consider the de Rham sequences
  \begin{gather}\minCDarrowwidth20pt
    \label{eq:derham:8}
    \begin{CD}
      \R
      @>{\subset}>> H^1(\domain)
      @>{\nabla}>> \Hcurl(\domain)
      @>{\curl}>> L^2(\domain)
      @>>> 0
      \\
      @.
      @A{\cong}AA
      @A{\cong}AA
      @A{\cong}AA
      \\
      \R
      @>{d}>> H\Lambda^0(\domain)
      @>{d}>> H\Lambda^1(\domain)
      @>{d}>> H\Lambda^2(\domain)
      @>>> 0
      \\
      @.
      @V{\cong}VV
      @V{\cong}VV
      @V{\cong}VV
      \\
      \R
      @>{\subset}>> H^1(\domain)
      @>{\curl}>> \Hdiv(\domain)
      @>{\div}>> L^2(\domain)
      @>>> 0,
    \end{CD}
  \end{gather}
\end{Notation}

\begin{Notation}{hlambda-norm}
  The spaces $H\Lambda^k(\domain)$ are Hilbert spaces with the inner product
  \begin{gather}
    \scal(u,v)_{H\Lambda^k} = \scal(u,v)_{L^2} + \scal(d u, d v)_{L^2}.
  \end{gather}
\end{Notation}

The value of this notation lies in the following theorem by de Rham,
which describes the relation between the elements of the sequence. It
is cited here without proof.

\begin{Theorem}{de-rham}
  Assume the domain $\domain$ is Lipschitz.  If $\domain$ is simply
  conntected, the sequences in equations~\eqref{eq:derham:9}
  and~\eqref{eq:derham:9} are exact, that is, there holds
  \begin{gather}
    \label{eq:derham:7}
    \ker {d_{k+1}} = \range{d_k}.
  \end{gather}
  If it is not simply connected, the codimension of $\range{d_k}$ in
  $\ker{d_{k+1}}$ is finite. In particular, in both cases,
  $\range{d_k}$ is closed in $H\Lambda^{k+1}(\domain)$.
\end{Theorem}

So far, we have not considered boundary conditions. The next lemma,
which is again stated without proof, indicates that the properties of
the de Rham complex are inherited, if the appropriate boundary
conditions are applied to each space, namely, function values in
$H^1$, tangential traces in $\Hcurl$, and normal traces in
$\Hdiv$. The last restriction from $L^2$ to $L^2_0$ is not a boundary
condition, but it is the compatibility condition implied by the Gauss
theorem on $\Hdiv$.

\begin{Lemma}{hlambda-0}
  The bounded Hilbert cochain complex
  \begin{gather}\minCDarrowwidth20pt
    \begin{CD}
      0
      @>{d}>> H\Lambda^0_0(\domain)
      @>{d}>> H\Lambda^1_0(\domain)
      @>{d}>> H\Lambda^2_0(\domain)
      @>{d}>> H\Lambda^3_0(\domain)
      @>>> 0
      \\
      @.
      @V{\cong}VV
      @V{\cong}VV
      @V{\cong}VV
      @V{\cong}VV
      \\
      0
      @>>> H^1_0(\domain)
      @>{\nabla}>> \Hcurl_0(\domain)
      @>{\curl}>> \Hdiv_0(\domain)
      @>{\div}>> L^2_0(\domain)
      @>>> 0,
    \end{CD}
  \end{gather}
  has the same properties as stated for the Hilbert complex without
  boundary conditions.
\end{Lemma}

% \begin{Example}{not-simply-connected}
  
% \end{Example}

\begin{remark}
  The complex does not start with $\R$ on the left, but with zero,
  since the constant functions are not members of $H^1_0(\domain)$.

  On the other hand, we could have replaced the right end of the
  complex by
  \begin{gather*}
    L^2(\domain) \xrightarrow{\frac1{\abs{\domain}}\int} \R,
  \end{gather*}
  where the arror is the mean value operator.
\end{remark}

\begin{Theorem}{div-curl-well-posed}
  The Maxwell problem in \blockref{Definition}{Maxwell-mixed-0} is
  well posed.
\end{Theorem}

\begin{proof}
  We have to show the inf-sup condition and the ellipticity of the
  curl-curl bilinear form. Let us introduce
  \begin{gather*}
    a(u,v) = \form(\curl u, \curl v),
    \qquad
    b(v,q) = \form(v,\nabla q).
  \end{gather*}
  From the fact that the de Rham complex starts with zero, we obtain
  that the kernel of the gradient is zero. Thus, for any $q\in
  H^1_0(\domain)$, we have $v = \nabla q \neq 0$ and
  $\norm{v}_{\Hcurl} = \norm{v}_{L^2} \le \norm{q}_{H^1}$. Thus, the
  inf-sup condition holds.

  We show now that $a(.,.)$ is elliptic on $\ker B$. From the
  definition of $b(.,.)$, we deduce that
  $\ker B \perp \nabla H^1_0(\domain) = \ker A$. Thus, $A$ is an
  isomorphism between $\ker B$ and its dual, and consequently
  elliptic.
\end{proof}

\begin{Problem}{darcy-derham}
  Prove well-posedness for the Darcy problem using the de Rham complex
  for proving \blockref{Lemma}{darcy-reduced-wellposed} and
  \blockref{Lemma}{darcy-infsup}.
\end{Problem}

\subsection{A polynomial complex}

\begin{intro}
  We have already seen that adding $x\P_k$ to the space $\P_k^d$, we
  obtain a surjective divergence operator from the Raviart-Thomas
  element to the pressure space $\P_k$. In this section, we see that
  there is a general principle behind this concept and it can be
  extended to the curl and gradient operators.
\end{intro}

\begin{Notation}{pk-complex}
  The homogeneous polynomial spaces $\breve\P_k$ form the cochain complex
  \begin{gather}\minCDarrowwidth15pt
    \begin{CD}
      \R
      @>{d}>> \breve\P_r\Lambda^0
      @>{d}>> \breve\P_{r-1}\Lambda^1
      @>{d}>> \breve\P_{r-2}\Lambda^2
      @>{d}>> \breve\P_{r-3}\Lambda^3
      @>{d}>> 0
      \\
      @.
      @V{\cong}VV
      @V{\cong}VV
      @V{\cong}VV
      @V{\cong}VV
      \\
      \R
      @>{\subset}>> \breve\P_r
      @>{\nabla}>> \breve\P_{r-1}^3
      @>{\curl}>> \breve\P_{r-2}^3
      @>{\div}>> \breve\P_{r-3}
      @>>> 0,
    \end{CD}
  \end{gather}
  and $d_{k+1}\circ d_k = 0$.
\end{Notation}

\begin{remark}
  Since the polynomial space $\P_r$ is the direct sum
  \begin{gather*}
    \P_r = \oplus_{s=0}^r \breve\P_s,
  \end{gather*}
  the homogeneous polynomial complex above can be extended to a
  general polynomial complex in a straightforward way.
\end{remark}

\begin{Definition}{Koszul-complex}
  The \define{Koszul complex} is a polynomial complex of the form
  \begin{gather}\minCDarrowwidth15pt
    \label{eq:derham:12}
    \begin{CD}
      0
      @<<< \P_r\Lambda^0
      @<{\kappa_1}<< \P_{r-1}\Lambda^1
      @<{\kappa_2}<< \P_{r-2}\Lambda^2
      @<{\kappa_3}<< \P_{r-3}\Lambda^3
      @<<< 0
    \end{CD}.
  \end{gather}
  The \define{Koszul differential} is defined such that
  \begin{gather}
    \label{eq:derham:13}
    \begin{aligned}
      \kappa_1\omega &= x\cdot\omega & \omega&\in \P_s\Lambda^1,\\
      \kappa_2\omega &= -x\times\omega & \omega&\in \P_s\Lambda^2,\\
      \kappa_3\omega &= x\omega & \omega&\in \P_s\Lambda^3,
    \end{aligned}
  \end{gather}
  and there holds
  \begin{gather}
    \label{eq:derham:14}
    \kappa\circ\kappa = \kappa_{k+1}\circ\kappa_k = 0.
  \end{gather}
\end{Definition}

Note that the ``Koszul differential'' increases the polynomial order
and lowers the order of the form, thus acts in the opposite way of the
usual differential $d$.

\begin{Lemma}{kd-plus-dk}
  For $\omega\in \breve \P_r\Lambda^k$ there holds
  \begin{gather}
    \label{eq:derham:15}
    \bigl(d\kappa+\kappa d\bigr)\omega = (r+k) \omega.
  \end{gather}
\end{Lemma}

\begin{proof}
  Since we are not using differential form technology, we prove this
  for each $k$ directly. For $k=0$, we have $\kappa\omega = 0$, thus
  we have to show
  \begin{gather*}
    \kappa d\omega = r\omega.
  \end{gather*}
  Due to linearity of $\kappa$ and $d$, it suffices to prove the
  result for $\omega = p=x_1^ax_2^bx_3^c$. We note that $dp/d_{x_1} =
  a/x_1 p$ and $d(x_1 p)/d_{x_1} = (a+1) p$ and analogue for the other
  coordinates.
  \begin{gather*}
    \kappa_1 d_0\omega = x\cdot \nabla p = x\cdot
    \begin{pmatrix}
      a/x_1\\b/x_2\\c/x_3
    \end{pmatrix}p
    = (a+b+c)p.
  \end{gather*}
  The second easy case is $k=3$ such that $d\omega = 0$. Let again
  $\omega = p$ to obtain
  \begin{gather*}
    d_2\kappa_3 \omega = \div(xp) = \div
    \begin{pmatrix}
      x_1 p \\x_2 p \\x_3 p
    \end{pmatrix}
    = (a+1+b+1+c+1) p = (r+3) \omega.
  \end{gather*}
  For the two vector valued cases, we note that it suffices to prove
  the result for $\omega = (p,0,0)^T$ and to note that the results for
  nonzero second and third component follow suite. Thus, for $k=1$
  \begin{multline*}
    \nabla (x\cdot \omega) - x\times \curl \omega
    = \nabla(x_1 p) - x\times
    \begin{pmatrix}
      0\\c/x_3 \\ -b/x_2
    \end{pmatrix}p
    \\
    =
    \begin{pmatrix}
      a+1 \\ bx_1/x_2\\ cx_1/x_3
    \end{pmatrix}p
    +
    \begin{pmatrix}
      b+c \\ -bx_1/x_2\\cx_1/x_3
    \end{pmatrix}p
    =
    \begin{pmatrix}
      a+b+c+1 \\0\\0
    \end{pmatrix}p
    = (r+1)\omega.
  \end{multline*}
  Finally, for $k=2$
  \begin{multline*}
    \curl(-x\times \omega) + x \div \omega
    = \curl
    \begin{pmatrix}
      0 \\ -x_3 \\ x_2
    \end{pmatrix}p
    +
    \begin{pmatrix}
      x_1 a/x_1\\x_2 a/x_1\\x_3 a/x_1\\
    \end{pmatrix}p
    \\=
    \begin{pmatrix}
      b+1+c+1\\-ax_2/x_1 \\ -a x_3/x_1
    \end{pmatrix}p
    +
    \begin{pmatrix}
      a\\ax_2/x_1\\ax_3/x_1
    \end{pmatrix}p
    =
    \begin{pmatrix}
      a+b+c+2\\0\\0
    \end{pmatrix}p
    = (r+2)\omega.
  \end{multline*}
\end{proof}

\begin{Lemma}{d-kappa-injective}
  The restriction of operator $d$ to $\range \kappa$ is injective and
  vice versa, or equivalently for any polynomial form
  $\omega\in \breve \P_r\Lambda^k$ there holds
  \begin{gather}
    \label{eq:derham:16}
    \begin{aligned}
      d\kappa\omega &= 0 &\Longrightarrow&& \kappa\omega &= 0,\\
      \kappa d\omega &= 0 &\Longrightarrow&& d\omega &= 0.
    \end{aligned}
  \end{gather}
\end{Lemma}

\begin{proof}
  If $r=k=0$, then $\kappa\omega = d\omega = 0$, such that the lemma
  holds trivially. For $r+k\neq 0$, we apply $\kappa$ to
  equation~\eqref{eq:derham:15} to obtain
  \begin{gather*}
    \kappa\omega = \frac1{r+k}
    \bigl(\kappa d\kappa\omega + \kappa^2d\omega\bigr)
    = \frac1{r+k}\kappa d\kappa\omega.
  \end{gather*}
  Thus, we have proven $d\kappa\omega=0$ implies $\kappa\omega=0$. The
  second implication is proven by applying $d$ to~\eqref{eq:derham:15}.
\end{proof}

\begin{Theorem}{polynomial-exact}
  The polynomial de Rham complex and the Koszul complex are exact for
  $r>3$. Furthermore for $r+k>0$, there holds
  \begin{gather}
    \label{eq:derham:17}
    \breve \P_r\Lambda^k = \kappa \breve\P_{r-1}\Lambda^{k+1}
    \oplus d\breve\P_{k+1}\Lambda^{k-1}.
  \end{gather}
\end{Theorem}

\begin{proof}
  We already know $\range{\kappa_{k-1}} \subset \ker{\kappa_k}$. Thus,
  it remains to show the opposite inclusion. Let therefore $\omega\in
  \breve \P_r\Lambda^k$ such that $\kappa\omega=0$. Then,
  \begin{gather*}
    \omega = \frac1{r+k} (d\kappa\omega+\kappa d\omega)
    = \frac1{r+k} \kappa d\omega =: \kappa\eta
  \end{gather*}
  with $\eta \in \breve\P_{r-1}\Lambda^{k+1}$.
\end{proof}

\subsection{The complex of tensor products}

%%% Local Variables: 
%%% mode: latex
%%% TeX-master: "main"
%%% End: 


\appendix
\chapter{Appendix}
\section{Notation}

\begin{Notation}{vector-diff-operators}
  Differential operators for vector fields $u:\R^d\to\R^d$
  are defined as follows:
  \begin{xalignat}2
    \nabla \vu &=
    \begin{pmatrix}
      \d_1 u_1 & \cdots & \d_d u_1\\
      \vdots && \vdots \\
      \d_1 u_d & \cdots & \d_d u_d
    \end{pmatrix}
    &&\text{(gradient)}
    \\
    \div \vu &= \sum_{i=1}^d \d_i u_i
    &&\text{(divergence)}
  \end{xalignat}

  For a tensor field $\sigma: \R^d\to \R^{d\times d}$, the divergence
  is a vector defined column-wise as
  \begin{gather}
    \div\sigma = \left(\sum_{i=1}^d \d_i \sigma_{ij}\right)_{j=1,\dots,d}
  \end{gather}
\end{Notation}

\begin{Notation}{sobolev-spaces}
  For a domain $\domain\in\R^d$, we denote by $L^2(\domain)$\index{L2@$L^2(\domain)$} the space
  of square integrable ``functions'' on $\domain$ with its norm\index{norm!Hk@$\norm{\cdot}_{L^2}=\norm{\cdot}_0$}
  \begin{gather*}
    \norm{u} = \norm{u}_0 = \norm{u}_{L^2(\domain)}.
  \end{gather*}
  By $H^k(\domain)$\index{Hk@$H^k(\domain)$} we denote the
  \define{Sobolev space} of square integrable functions on $\domain$
  with square integrable distributional derivatives up to order
  $k$. Its norm is\index{norm!Hk@$\norm{\cdot}_{H^k}=\norm{\cdot}_{k}$}
  \begin{gather*}
    \norm{u}_k = \norm{u}_{H^k(\domain)} = \sum_{\abs{\alpha} \le k}
    \norm{\d^\alpha u}_{L^2(\domain)}.
  \end{gather*}
  We also use the $H^k$-seminorm\index{seminorm!Hk@$\abs{\cdot}_{H^k}$}
  \begin{gather*}
    \abs{u}_k = \norm{u}_{H^k(\domain)} = \sum_{\abs{\alpha} \le k}
    \norm{\d^\alpha u}_{L^2(\domain)}.
  \end{gather*}
  By $H^1_0(\domain)$\index{H10@$H^1_0(\domain)$} we denote the
  completion of $C^\infty_0(\domain)$ with respect to the norm
  $\norm{\cdot}_{H^1(\domain)}$. Similarly,
  $H^1_{\Gamma}(\domain)$\index{H1gamma@$H^1_\Gamma(\domain)$} for any
  $\Gamma\subset\d\domain$ is the completion of all functions in
  $C^\infty(\domain)$ vanishing on $\gamma$.
\end{Notation}

%%% Local Variables: 
%%% mode: latex
%%% TeX-master: "main"
%%% End: 


\section{Derivation of the Lamé-Navier equations}
\label{sec:lame-navier}
In this section, we study the simplest mathematical model for elastic
deformation of solids based on Hooke's law. For comparison,
consider~\cite{Braess97,Braess13}. For the full nonlinear model in all
mathematical detail refer to~\cite{Ciarlet88}.

\begin{intro}
  The deformation of a solid body is described as a mapping $\Phi$
  from the \define{reference configuration} $\domain\subset \R^d$ to a
  deformed configuration $\deformed\domain \subset \R^d$, such that
  each undeformed point $x\in\domain$ is mapped to the point
  $\deformed x$ after deformation. The domain $d$ is 3 for physically
  relevant models, but we investigate two-dimensional problems in
  order to study mathematical properties and numerical methods more
  easily.

  Actually, we are not quite interested in this mapping $\Phi$, since
  it depends on the position of the points $x$. On the other hand, a
  basic principle of physical laws is frame invariance, namely, if we
  change from one Cartesian coordinate system to another, the physical
  law may only change by the same coordinate transformation, not in
  its physical implications. Therefore, only the differences
  $\deformed x-x$ should matter. Thus, we introduce the
  \define{displacement} $u$, such that
  \begin{gather*}
    \Phi = \id + u.
  \end{gather*}
  The symbol $\id$ will refer to all occurrences of identical mappings
  and their matrices.

  So far, by the introduction of $u$, we divide translations of the
  reference configuration out of our model. But, in addition, we have
  to eliminate the influence of rigid body rotations. These are
  operations, which leave all distances and angles unchanged. Thus, we
  investigate the change of the distance between $x$ and $x+z$ under
  the mapping $\Phi$. By definition of the derivative, we have
  \begin{align*}
    \abs{\Phi(x+z) - \Phi(x)}^2 &= \norm{\nabla\Phi z}^2 + o(\abs{z}^4)
    \\
                              &= z^T\nabla\Phi^T\nabla\Phi z + o(\abs{z}^4)
    \\
    &= z^T(I + \nabla u^T + \nabla u + \nabla u^T \nabla u) z + o(\abs{z}^4)
    \\
    &= \abs{z}^2 + 2 z^T\strain u z + o(\abs{z}^4),
  \end{align*}
  where
  \begin{gather}
    \tilde\epsilon(u) = \tfrac12
    \bigl(\nabla u^T + \nabla u + \nabla u^T \nabla u\bigr)
  \end{gather}
  is the \textbf{strain tensor}. From linear algebra, we know that a
  linear mapping which preserves all distances is orthogonal and thus
  also preserves angles. Thus, every deformation with $\strain u=0$
  is a rigid body transformation, namely a combination of translation
  and rotation.

  In this class we are concerned only with linear problems, which can
  be justified by the notion of infinitely small deformations
  $u$. Then, we only study first order effects in $u$, which implies
  that we are going to neglect the quadratic term in
  $\strain u$. This is justified by the fact that we obtain a model,
  which is sufficiently accurate for small deformations.
\end{intro}

\begin{Definition}{strain-tensor}
  The linearized \define{strain tensor} or \define{symmetric gradient}
  of $u$ is
  \begin{gather}
    \strain u = \frac{\nabla u + \nabla u^T}2.
  \end{gather}
\end{Definition}

\begin{intro}
  Next, we have to consider the interplay of forces and
  deformations. The basic principle is Newton's axiom of force
  balance. If a body force $f$ acts on a small volume $V$, there have
  to be surface forces acting against $f$ in order to keep $V$ at
  rest. Similarly, if a torque is applied inside this volume, there
  must be tangential forces on the surface balancing this torque. Due
  to Euler, we model these forces as a mapping $t$, which at each
  point $x$ maps a direction vector $n$ to a force vector
  $t(x,n)$. Thus, the balance of forces is written as
  \begin{alignat*}2
    \int_V f \dx &+ \int_{\d V} t(x,n) \ds &=&0\\
    \int_V x\times f \dx &+ \int_{\d V} x\times t(x,n) \ds &=&0.
  \end{alignat*}
  Due to Euler and Cauchy, this mapping $t(x,n)$ can be expressed as
  $\sigma(x)n$ by the \define{stress tensor} $\sigma$. Without proof,
  we note that the balance of torque implies that $\sigma$ is
  symmetric, while the force balance equation after integration by
  parts becomes
  \begin{gather}
    \label{eq:mixedintro:3}
    f + \div \sigma = 0.
  \end{gather}
  What is missing now is a relation between the displacement $u$ and
  the stress $\sigma$, which is not the result of fundamental
  principles, but of material properties.
\end{intro}

\begin{remark}
  At this point, we play again the card of small deformations, such
  that we do not have to distinguish whether equations are formulated
  on the reference domain $\domain$ or on the deformed domain
  $\deformed\domain$. Such a discussion becomes confusing easily and
  thus remains a subject for a more specialized class.
\end{remark}

\begin{Definition}{hooke}
  \define{Hooke's law} states that the stress depends linearly on the
  strain. Together with frame invariance, this implies the relation
  \begin{gather}
    \label{eq:mixedintro:4}
    \sigma = 2\mu \strain u + \lambda \operatorname{tr} \strain u \id,
  \end{gather}
  where $\lambda\ge 0$ and $\mu> 0$ are material properties called
  \define{Lamé-Navier parameters}.
\end{Definition}

\begin{remark}
  The trace of the strain operator is equal to the trace of the
  gradient. Thus, we have
  \begin{gather}
    \label{eq:mixedintro:5}
    \operatorname{tr} \strain u = \div u \,\id.
  \end{gather}
\end{remark}

\begin{intro}
  Equations~\eqref{eq:mixedintro:3} and~\eqref{eq:mixedintro:4}
  together define a second order partial differential equation, for
  which we have to impose boundary conditions. A natural choice, which
  keeps the mathematical analysis simple is the \define{Dirichlet
    boundary condition} $u=0$, corresponding to an elastic body whose
  boundary is fixed. The alternative is the traction free boundary
  condition $\sigma n=0$ with vanishing normal traces. Combinations
  are possible, for instance $u\cdot n=0$ for a boundary that allows
  sliding but no penetration. Constraining ourselves to Dirichlet
  condition on $\Gamma_D\subset \d\domain$ and traction free on
  $\gamma_N = \d\domain\setminus\Gamma_D$, we obtain the
  boundary value problem
  \begin{gather}
    \label{eq:mixedintro:lame-navier-bvp}
    \begin{aligned}
      -\div \sigma(x) &= f(x) & x&\in\domain,\\
      u(x) &= 0 & x&\in\Gamma_D, \\
      \sigma(x)n &=0& x&\in\Gamma_N,
    \end{aligned}
  \end{gather}
  together with the material law~\eqref{eq:mixedintro:4}.  Once we
  test and integrate by parts to obtain our weak formulation, we
  obtain
  \begin{gather*}
    \int_{\domain} -(\div \sigma) \cdot v\dx
    = \int_{\domain} \sigma : \nabla v\dx
    - \int_{\Gamma_N} \sigma n\cdot v\ds,
  \end{gather*}
  such that traction free is actually the natural boundary condition
  comparable to the Neumann condition for the Laplacian. Note that $:$
  is the double contraction or Frobenius product (see
  Problem~\ref{Problem:frobenius} below) of the two tensors.
\end{intro}

\begin{intro}
  We now walk the missing steps to obtain a weak formulation. first,
  we enter Hooke's law for $\sigma$ to obtain:
  \begin{gather*}
    \int_\domain \bigl[2\mu \strain u : \nabla v
    + \lambda (\div u \id) : \nabla v
    \bigr]\dx = \int_\domain f\cdot v\dx.
  \end{gather*}
  Then, we choose the space
  \begin{gather}
    V = H^1_{\Gamma_D}(\domain; \R^d) = \bigl\{v\in H^1(\domain;\R^d) \big\vert
    v_{|\Gamma_D} = 0 \bigr\}.
  \end{gather}
  We notice for the second term that
  \begin{gather*}
    \id : \nabla v = \sum_{i=1}^d \d_i v_i = \div v.
  \end{gather*}
  Furthermore, we use the result of Problem~\ref{Problem:frobenius} to
  obtain
  \begin{gather*}
    \strain u:\nabla v = \strain u : \strain v.
  \end{gather*}
\end{intro}

\begin{Definition}{lame-navier-equations}
  The \define{Lamé-Navier equations} of \putindex{linear elasticity} read
  \begin{subequations}
    \begin{gather}
      \begin{aligned}
        -\div \sigma(x) &= f(x) & x&\in\domain,\\
        \sigma(x) &= 2\mu \strain u(x) + \lambda \operatorname{tr} \strain u(x) \id.
      \end{aligned}
    \end{gather}
    They are usually complemented with boundary conditions
    \begin{gather}
      \begin{aligned}
        u(x) &= u^D & x&\in\Gamma_D, \\
        \sigma(x)n &=\sigma^N& x&\in\Gamma_N.
      \end{aligned}
    \end{gather}
  \end{subequations}
  Here $\Gamma_D$ is a subset of $\d\domain$ and
  $\Gamma_N = \d\domain\setminus\Gamma_D$.  We refer to the boundary
  conditions on $\Gamma_D$ and $\Gamma_N$ as \define{displacement boundary}
  and \define{traction boundary} conditions, respectively.
  The function $f$ describes \define{body forces}, for instance gravity.
\end{Definition}


%%% Local Variables: 
%%% mode: latex
%%% TeX-master: "main"
%%% End: 


\bibliographystyle{apalike}
\bibliography{all}
\printindex


%%% Local Variables: 
%%% mode: latex
%%% TeX-master: "main"
%%% End: 
