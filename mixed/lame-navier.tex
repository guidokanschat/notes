In this section, we study the simplest mathematical model for elastic
deformation of solids based on Hooke's law. For comparison,
consider~\cite{Braess97,Braess13}. For the full nonlinear model in all
mathematical detail refer to~\cite{Ciarlet88}.

\begin{intro}
  The deformation of a solid body is described as a mapping $\Phi$
  from the \define{reference configuration} $\domain\subset \R^d$ to a
  deformed configuration $\deformed\domain \subset \R^d$, such that
  each undeformed point $x\in\domain$ is mapped to the point
  $\deformed x$ after deformation. The domain $d$ is 3 for physically
  relevant models, but we investigate two-dimensional problems in
  order to study mathematical properties and numerical methods more
  easily.

  Actually, we are not quite interested in this mapping $\Phi$, since
  it depends on the position of the points $x$. On the other hand, a
  basic principle of physical laws is frame invariance, namely, if we
  change from one Cartesian coordinate system to another, the physical
  law may only change by the same coordinate transformation, not in
  its physical implications. Therefore, only the differences
  $\deformed x-x$ should matter. Thus, we introduce the
  \define{displacement} $u$, such that
  \begin{gather*}
    \Phi = \id + u.
  \end{gather*}
  The symbol $\id$ will refer to all occurrences of identical mappings
  and their matrices.

  So far, by the introduction of $u$, we divide translations of the
  reference configuration out of our model. But, in addition, we have
  to eliminate the influence of rigid body rotations. These are
  operations, which leave all distances and angles unchanged. Thus, we
  investigate the change of the distance between $x$ and $x+z$ under
  the mapping $\Phi$. By definition of the derivative, we have
  \begin{align*}
    \abs{\Phi(x+z) - \Phi(x)}^2 &= \norm{\nabla\Phi z}^2 + o(\abs{z}^4)
    \\
                              &= z^T\nabla\Phi^T\nabla\Phi z + o(\abs{z}^4)
    \\
    &= z^T(I + \nabla u^T + \nabla u + \nabla u^T \nabla u) z + o(\abs{z}^4)
    \\
    &= \abs{z}^2 + 2 z^T\strain u z + o(\abs{z}^4),
  \end{align*}
  where
  \begin{gather}
    \tilde\epsilon(u) = \tfrac12
    \bigl(\nabla u^T + \nabla u + \nabla u^T \nabla u\bigr)
  \end{gather}
  is the \textbf{strain tensor}. From linear algebra, we know that a
  linear mapping which preserves all distances is orthogonal and thus
  also preserves angles. Thus, every deformation with $\strain u=0$
  is a rigid body transformation, namely a combination of translation
  and rotation.

  In this class we are concerned only with linear problems, which can
  be justified by the notion of infinitely small deformations
  $u$. Then, we only study first order effects in $u$, which implies
  that we are going to neglect the quadratic term in
  $\strain u$. This is justified by the fact that we obtain a model,
  which is sufficiently accurate for small deformations.
\end{intro}

\begin{Definition}{strain-tensor}
  The linearized \define{strain tensor} or \define{symmetric gradient}
  of $u$ is
  \begin{gather}
    \strain u = \frac{\nabla u + \nabla u^T}2.
  \end{gather}
\end{Definition}

\begin{intro}
  Next, we have to consider the interplay of forces and
  deformations. The basic principle is Newton's axiom of force
  balance. If a body force $f$ acts on a small volume $V$, there have
  to be surface forces acting against $f$ in order to keep $V$ at
  rest. Similarly, if a torque is applied inside this volume, there
  must be tangential forces on the surface balancing this torque. Due
  to Euler, we model these forces as a mapping $t$, which at each
  point $x$ maps a direction vector $n$ to a force vector
  $t(x,n)$. Thus, the balance of forces is written as
  \begin{alignat*}2
    \int_V f \dx &+ \int_{\d V} t(x,n) \ds &=&0\\
    \int_V x\times f \dx &+ \int_{\d V} x\times t(x,n) \ds &=&0.
  \end{alignat*}
  Due to Euler and Cauchy, this mapping $t(x,n)$ can be expressed as
  $\sigma(x)n$ by the \define{stress tensor} $\sigma$. Without proof,
  we note that the balance of torque implies that $\sigma$ is
  symmetric, while the force balance equation after integration by
  parts becomes
  \begin{gather}
    \label{eq:mixedintro:3}
    f + \div \sigma = 0.
  \end{gather}
  What is missing now is a relation between the displacement $u$ and
  the stress $\sigma$, which is not the result of fundamental
  principles, but of material properties.
\end{intro}

\begin{remark}
  At this point, we play again the card of small deformations, such
  that we do not have to distinguish whether equations are formulated
  on the reference domain $\domain$ or on the deformed domain
  $\deformed\domain$. Such a discussion becomes confusing easily and
  thus remains a subject for a more specialized class.
\end{remark}

\begin{Definition}{hooke}
  \define{Hooke's law} states that the stress depends linearly on the
  strain. Together with frame invariance, this implies the relation
  \begin{gather}
    \label{eq:mixedintro:4}
    \sigma = 2\mu \strain u + \lambda \operatorname{tr} \strain u \id,
  \end{gather}
  where $\lambda\ge 0$ and $\mu> 0$ are material properties called
  \define{Lamé-Navier parameters}.
\end{Definition}

\begin{remark}
  The trace of the strain operator is equal to the trace of the
  gradient. Thus, we have
  \begin{gather}
    \label{eq:mixedintro:5}
    \operatorname{tr} \strain u = \div u \,\id.
  \end{gather}
\end{remark}

\begin{intro}
  Equations~\eqref{eq:mixedintro:3} and~\eqref{eq:mixedintro:4}
  together define a second order partial differential equation, for
  which we have to impose boundary conditions. A natural choice, which
  keeps the mathematical analysis simple is the \define{Dirichlet
    boundary condition} $u=0$, corresponding to an elastic body whose
  boundary is fixed. The alternative is the traction free boundary
  condition $\sigma n=0$ with vanishing normal traces. Combinations
  are possible, for instance $u\cdot n=0$ for a boundary that allows
  sliding but no penetration. Constraining ourselves to Dirichlet
  condition on $\Gamma_D\subset \d\domain$ and traction free on
  $\gamma_N = \d\domain\setminus\Gamma_D$, we obtain the
  boundary value problem
  \begin{gather}
    \label{eq:mixedintro:lame-navier-bvp}
    \begin{aligned}
      -\div \sigma(x) &= f(x) & x&\in\domain,\\
      u(x) &= 0 & x&\in\Gamma_D, \\
      \sigma(x)n &=0& x&\in\Gamma_N,
    \end{aligned}
  \end{gather}
  together with the material law~\eqref{eq:mixedintro:4}.  Once we
  test and integrate by parts to obtain our weak formulation, we
  obtain
  \begin{gather*}
    \int_{\domain} -(\div \sigma) \cdot v\dx
    = \int_{\domain} \sigma : \nabla v\dx
    - \int_{\Gamma_N} \sigma n\cdot v\ds,
  \end{gather*}
  such that traction free is actually the natural boundary condition
  comparable to the Neumann condition for the Laplacian. Note that $:$
  is the double contraction or Frobenius product (see
  Problem~\ref{Problem:frobenius} below) of the two tensors.
\end{intro}

\begin{intro}
  We now walk the missing steps to obtain a weak formulation. first,
  we enter Hooke's law for $\sigma$ to obtain:
  \begin{gather*}
    \int_\domain \bigl[2\mu \strain u : \nabla v
    + \lambda (\div u \id) : \nabla v
    \bigr]\dx = \int_\domain f\cdot v\dx.
  \end{gather*}
  Then, we choose the space
  \begin{gather}
    V = H^1_{\Gamma_D}(\domain; \R^d) = \bigl\{v\in H^1(\domain;\R^d) \big\vert
    v_{|\Gamma_D} = 0 \bigr\}.
  \end{gather}
  We notice for the second term that
  \begin{gather*}
    \id : \nabla v = \sum_{i=1}^d \d_i v_i = \div v.
  \end{gather*}
  Furthermore, we use the result of Problem~\ref{Problem:frobenius} to
  obtain
  \begin{gather*}
    \strain u:\nabla v = \strain u : \strain v.
  \end{gather*}
\end{intro}

\begin{Definition}{lame-navier-equations}
  The \define{Lamé-Navier equations} of \putindex{linear elasticity} read
  \begin{subequations}
    \begin{gather}
      \begin{aligned}
        -\div \sigma(x) &= f(x) & x&\in\domain,\\
        \sigma(x) &= 2\mu \strain u(x) + \lambda \operatorname{tr} \strain u(x) \id.
      \end{aligned}
    \end{gather}
    They are usually complemented with boundary conditions
    \begin{gather}
      \begin{aligned}
        u(x) &= u^D & x&\in\Gamma_D, \\
        \sigma(x)n &=\sigma^N& x&\in\Gamma_N.
      \end{aligned}
    \end{gather}
  \end{subequations}
  Here $\Gamma_D$ is a subset of $\d\domain$ and
  $\Gamma_N = \d\domain\setminus\Gamma_D$.  We refer to the boundary
  conditions on $\Gamma_D$ and $\Gamma_N$ as \define{displacement boundary}
  and \define{traction boundary} conditions, respectively.
  The function $f$ describes \define{body forces}, for instance gravity.
\end{Definition}


%%% Local Variables: 
%%% mode: latex
%%% TeX-master: "main"
%%% End: 
