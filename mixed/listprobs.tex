\input{blocks/Problem-frobenius}
\input{blocks/Problem-elasticity-standard}
\clearpage

\input{blocks/Problem-lagrange-multiplier.tex}
\input{blocks/Problem-unbounded-inverse.tex}
\input{blocks/Problem-lax-milgram-not-applicable.tex}
\input {blocks/Problem-inf-sup-equivalence.tex}
\clearpage

\input{blocks/Problem-inhomogeneous-continuity.tex}
\input{blocks/Problem-infsup-uniform.tex}

\begin {blocktheorem}{Problem}(2.4.13)
Let $A\in\mathbb{R}^{n\times n}$, $B\in\mathbb{R}^{k\times n}$, $k\leq n$.
Moreover, assume that $B$ has full rank and that $A$ is symmetric and
positive definite.\\
Consider the problem
\begin{align}
\begin{aligned}
\begin{pmatrix} A & B^* \\ B & 0 \end{pmatrix}
\begin{pmatrix} x \\ y \end{pmatrix}
= \begin{pmatrix} F \\ G \end{pmatrix}
\end{aligned}
\tag{*}
\label{problem9-stokes}
\end{align}
\begin{enumerate}
\item Prove that then $S := BA^{-1}B^*$ is symmetric and positive definite, too.
How can this matrix be used to solve (\ref{problem9-stokes})?
\item Show that
\begin{align*}
 P := I - B^*(BB^*)^{-1}B.
\end{align*}
is a projector on the kernel of $B$ with $\norm{P}_2=1$.
\item Show for the case $G = 0$ that $x$ is a solution of
\begin{align*}
PAPx = PF
\end{align*}
if $(x,y)$ is a solution of (\ref{problem9-stokes}).
\end{enumerate}
\begin{solution}
\begin{enumerate}
 \item Let's first consider the symmetry:
  \begin{align*}
    S^*=(BA^{-1}B^*)^*=B(A^*)^{-1}B^*=BA^{-1}B^*=S
  \end{align*}
  We further observe
  \begin{align*}
   \norm{v}_S^2 &= v^*Sv = v^*BA^{-1}B^*v = (B^*v)^*A^{-1}B^*v \leq 0 \\
   \norm{v}_S^2 &\Longleftrightarrow B^*v = 0 \Longleftrightarrow v=0
  \end{align*}
  since $A$ and $B$ have full rank.

  The first equation implies
  \begin{align*}
   x=A^{-1} (F-B^* y)
  \end{align*}
  and eliminating $x$ in the second equation yields
  \begin{align*}
   BA^{-1} (F-B^* y)=G \Longleftrightarrow BA^{-1} B^* y=BA^{-1} F-G.
   \end{align*}

  \item \begin{align*}P^2&=(I - B^*(BB^*)^{-1}B)(I - B^*(BB^*)^{-1}B)\\
            &=I - 2B^*(BB^*)^{-1}B+ B^*(BB^*)^{-1}BB^*(BB^*)^{-1}B\\
            &= I-B^*(BB^*)^{-1}B=P
        \end{align*}
        Therefore, $\norm{P}_2=1$ if $\operatorname{ker} B\not=\{0\}$.
        Note that $k<n$ implies this.

        Further, $Px=x$ for all $x\in \operatorname{ker} B$ and $BPy=0$ for all $y$.
  \item Let $x$ be a solution of (\ref{problem9-stokes}) for $G=0$.
  \begin{align*}
PAPx &= (I-B^*(BB^*)^{-1}B) A (I-B^*(BB^*)^{-1}B) x\\
&= (I-B^*(BB^*)^{-1}B) Ax\\
&= (I-B^*(BB^*)^{-1}B) (F-B^* y)\\
&= (I-B^*(BB^*)^{-1}B) F = PF
        \end{align*}

\end{enumerate}
\end{solution}
\end{blocktheorem}

\clearpage

\begin {blocktheorem}{Problem}(Unstable Elements I)
Consider on the unit square $\Omega=(0, 1) \times (0, 1)$ the Stokes problem
\begin{align*}
  -\Delta \boldsymbol{u} + \nabla p &= f \quad \text {in} \Omega\\
  \nabla \cdot \boldsymbol{u}       &= 0 \quad \text {in} \Omega\\
                     \boldsymbol{u} &= 0 \quad \text {on} \partial\Omega.
\end{align*}
The domain $\Omega$ is decomposed into $N \times N$ congruent squares where each
of them is again divided into two triangles. The decomposition $\mathcal{T}_h$
is given by these triangles.

For $l \geq 1$ denote by $\mathbb{P}_{l,0}(\mathcal{T}_h)$ the set of globally
continuous functions on $\Omega$ that vanish on $\partial\Omega$ and which
restricted to cell in $\mathcal{T}_h$ are polynomials of maximum degree $l$.
$\mathbb{P}^{-1}(\mathcal{T}_h)$ is the set of piecewise
constant ansatz functions on $\mathcal{T}_h$.

Prove: Choosing $V_h:= [\mathbb{Q}_{1,0}(\mathcal{T}_h)]^2$ and
 $Q_h:=\mathbb{P}_0^{-1}(\mathcal{T}_h)\cap L_0^2(\Omega)$ the element pair
 $(V_h, Q_h)$ is not inf-sup stable.

Hint: Count the number of degrees of freedom for velocity and pressure.

\begin{solution}
The number of degrees of freedom for the velocity is $2(N-1)^2$ and for the
pressure $2N^2-1$. Hence, there is always a pressure ansatz functions such that
\begin{align*}
 (\nabla \cdot \boldsymbol{v}_h, q_h)=0 \quad \forall \boldsymbol{v}_h\in V_h.
\end{align*}
\end{solution}
\end{blocktheorem}
\begin {blocktheorem}{Problem}(Unstable Elements II)
Consider again the Stokes problem from the previous exercise. This time the
decomposition $\mathcal{T}_h$ is given by the $N\times N$ squares.

For $l \geq 1$ denote by $\mathbb{Q}_{l,0}(\mathcal{T}_h)$ the set of globally
continuous functions on $\Omega$ that vanish on $\partial\Omega$ and which
restricted to cell in $\mathcal{T}_h$ are tensor-product polynomials of
maximum degree $l$.

Prove: Choosing $V_h:= [\mathbb{Q}_{1,0}(\mathcal{T}_h)]^2$ and
 $Q_h:=\mathbb{P}^{-1}(\mathcal{T}_h)\cap L_0^2(\Omega)$ the element pair
 $(V_h, Q_h)$ is not inf-sup stable.

Hint: Find a pressure $q_h\in Q_h$ such that
\begin{align*}
 (\nabla \cdot \boldsymbol{v}_h, q_h)=0 \quad \forall \boldsymbol{v}_h\in V_h.
\end{align*}
\begin{solution}
This time there are $2N^2$ for the velocity and $2N^2-1$ ansatz functions for
the pressure. Therefore, we have to look a big deeper.

Denote by $p_{i+\frac12, j+\frac12}$ the pressure constant on cell $K_{i,j}$
($0\leq i,j \leq N-1$). The values of the velocity at the four vertices are
denoted by $u^1_{i,j}$, $u^1_{i,j+1}$, $u^1_{i+1,j}$, $u^1_{i+1,j+1}$ for the
first component and similarly for the second component.

Using this notation we can write the divergence constraint in terms of nodal and cell values:
\begin{align*}
 &(\nabla \cdot \boldsymbol{v}_h, q_h)\\
   &= \sum_{i,j} q_{i+\frac12,j+\frac12} \int_{K_{i,j}} \nabla \cdot \boldsymbol{v}_h \,\mathrm{d}x\\
   &= \sum_{i,j} q_{i+\frac12,j+\frac12} \int_{\partial K_{i,j}} \boldsymbol{n} \cdot \boldsymbol{v}_h \,\mathrm{d}x\\
   &= \sum_{i,j} q_{i+\frac12,j+\frac12} \frac{h}{2}(u^1_{i+1,j}-u^1_{i,j+1}+u^1_{i+1,j}-u^1_{i+1,j+1})\\
      &\quad +\sum_{i,j} q_{i,j} \frac{h}{2}(u^2_{i,j+1}-u^2_{i+1,j}+u^2_{i,j+1}-u^2_{i+1,j+1})\\
   &= \frac{h}{2}\sum_{i,j} u^1_{i,j}(q_{i-\frac12,j+\frac12}+q_{i-\frac12,j-\frac12}-q_{i+\frac12,j+\frac12}-q_{i+\frac12,j-\frac12})\\
      &\quad +\frac{h}{2}\sum_{i,j} u^2_{i,j}(q_{i+\frac12,j-\frac12}+q_{i-\frac12,j-\frac12}-q_{i+\frac12,j+\frac12}-q_{i-\frac12,j+\frac12})
\end{align*}
Thus,
\begin{align*}
(\nabla \cdot \boldsymbol{v}_h, q_h)=0 \quad \forall \boldsymbol{v}_h\in V_h
\end{align*}
implies
\begin{align*}
q_{i-\frac12,j+\frac12}+q_{i-\frac12,j-\frac12}-q_{i+\frac12,j+\frac12}-q_{i+\frac12,j-\frac12} &=0 \\
q_{i+\frac12,j-\frac12}+q_{i-\frac12,j-\frac12}-q_{i+\frac12,j+\frac12}-q_{i-\frac12,j+\frac12} &=0.
\end{align*}
for all ($0\leq i,j \leq N-1$).
These two constraints can be rephrased as
\begin{align*}
q_{i-\frac12,j+\frac12} &= q_{i+\frac12,j-\frac12} \\
q_{i-\frac12,j-\frac12} &= q_{i+\frac12,j+\frac12}
\end{align*}
for all ($0\leq i,j \leq N-1$). Since the mean value of the pressure must be zero,
the set of pressure functions that fulfill
\begin{align}
(\nabla \cdot \boldsymbol{v}_h, q_h)=0 \quad \forall \boldsymbol{v}_h\in V_h
\end{align}
can be described by
\begin{align*}
 p_{i,j} = c \quad \text{if } (i+j) \operatorname{mod} 2 = 0\\
 p_{i,j} = -c \quad \text{if } (i+j) \operatorname{mod} 2 = 1
\end{align*}
where $c1\not=c2$.
\end{solution}
\end{blocktheorem}




%%% Local Variables:
%%% mode: latex
%%% TeX-master: "solutions"
%%% End:
