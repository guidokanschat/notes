\begin{intro}
  We begin the investigation of variational crimes by studying the
  effect of using numerical quadrature instead of exact integration on
  mesh cells. In particular, we investigate approximations of the form
  \begin{gather}
    \int_\cell f(\vx)\dvx
    \approx \sum_{k=1}^{n_q} \omega_k f(\vx_k) =: Q_\cell(f).
  \end{gather}
  First, we observe that $Q_\cell$ is not a bounded operator on
  $L^1(\domain)$, such that $Q_\cell(\nabla u\cdot \nabla v)$ is
  undefined for functions in $H^1(\domain)$. The surprising result of
  this section is, that quadrature is still admissible for the
  implementation of a finite element method.

  We will first set a theoretical framework and then investigate
  quadrature rules in detail. The presentation follows~\cite[Chapter
  4]{Ciarlet78}.
\end{intro}

\begin{Lemma*}{strang-1}{Strang's first lemma}
  Let $a(.,.)$ be a bounded and elliptic bilinear form on the Hilbert
  space $V$. Let $V_n\subset V$ and let $a_h(.,.)$ be a bilinear form,
  bounded and elliptic on $V_n$ with constants $M_n$ and
  $\gamma_n$. Let $f, f_n\in V^*$. If $u \in V$ and
  $u_n\in V_n \subset V$ are solutions to
  \begin{gather*}
    \begin{aligned}
      a(u,v) &= f(v) & \qquad\forall v&\in V,\\
      a_n(u_n,v_n) &= f_n(v_n) & \qquad\forall v_n&\in V_n,
    \end{aligned}
  \end{gather*}
  respectively, there holds
  \begin{multline}
    \norm{u-u_n}_V \le \left(1+\frac{M}{\gamma_n}\right)
    \inf_{v_n\in V_n}\norm{u-v_n}_V
    \\+ \frac1{\gamma_n}\left(
    \norm{a_n(v_n,.)-a(v_n,.)}_{V_h^*}
         + \norm{f_n-f}_{V_h^*}\right).
  \end{multline}
\end{Lemma*}

\begin{proof}
  Using $V_n$-ellipticity of $a_n(.,.)$ yields for arbitrary $v_n\in V_n$
  \begin{align*}
    \gamma_n \norm{u_n-v_n}^2
    \le &a_n(u_n-v_n,u_n-v_n)\\
    =& f_n(u_n-v_n) + a(u-v_n, u_n-v_n)
       - f(u_n-v_n)
    \\ &+ a(v_n,u_n-v_n) - a_n(v_n, u_n-v_n)
  \end{align*}
  We estimate separately
  \begin{align*}
    \frac{a(u-v_n, u_n-v_n)}{\norm{u_n-v_n}}
    &\le M \norm{u_n-v_n},\\
    \frac{\abs{f_n(u_n-v_n) - f(u_n-v_n)}}{\norm{u_n-v_n}}
    &\le \sup_{w_n\in V_n} \frac{\abs{f_n(w_n)-f(w_n)}}{\norm{w_n}},
    \\
    \frac{\abs{a(v_n,u_n-v_n) - a_n(v_n, u_n-v_n)}}{\norm{u_n-v_n}}
    &\le \sup_{w_n\in V_n} \frac{\abs{a_n(v_n, w_n)-a(v_n,w_n)}}{\norm{w_n}}.
  \end{align*}
  Combining all terms, we obtain
  \begin{multline}
    \norm{u-u_n}
    \le \norm{u-v_n} + \norm{u_n-v_n} \\
    \le \norm{u-v_n} + \frac1{\gamma_n}\left(
         M \norm{u-v_n}
         + \norm{a_n(v_n,.)-a(v_n,.)}_{V_h^*}
         + \norm{f_n-f}_{V_h^*}\right).
  \end{multline}
\end{proof}

\begin{remark}
  We will apply Strang's lemma to a family of meshes indexed by mesh
  size $h$ and assess the infimum by an interpolation operator. It is
  clear, that we will only obtain optimal convergence rates compared
  to the interpolation estimate, if there exists $\gamma_0 >0$ such
  that $\gamma_n \ge \gamma_0$ uniformly with respect to $n$. While it
  is not a prerequisite of Strang's lemma, it is our goal for all
  discretizations.
\end{remark}

\begin{remark}
  Quadrature would be infeasible, if we had to devise a quadrature
  rule for every mesh cell $\cell$. Instead, we tabulate quadrature
  formulas for the reference cell $\refcell$ by choosing quadrature
  points $\refvx_k$ and weights $\omega_k$ and write
  \begin{gather}
    Q_{\refcell}(\reference f)
    = \sum_{k=1}^{n_q} \omega_k \reference f(\refvx_k).
  \end{gather}
  We compute integrals over $\cell$ though mapping,
  \begin{gather}
    Q_T(f) = \sum_{k=1}^{n_q} \det\nabla\Phi_\cell(\refvx_k) \omega_k \reference f(\refvx_k).
  \end{gather}
  Thus, $Q_T$ is defined by quadraqture points
  $\vx_k = \Phi(\refvx_k)$ and weights
  $\det\nabla\Phi_\cell(\refvx_k) \omega_k$.

  Quadrature rules on the reference 
\end{remark}

%%% Local Variables: 
%%% mode: latex
%%% TeX-master: "main"
%%% End: 
