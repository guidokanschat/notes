\begin{intro}
  The interior penalty method introduced so far is $V_h$-elliptic and
  consistent, but it is not bounded on $H^1(\domain)$. This was a
  reason, why we could not use standard techniques for the proof of
  the convergence result and afther applying consistency had to
  estimate each term separately.

  In this section, we will introduce a reformulation of the interior
  penalty method, which is equivalent to the original method on $V_h$,
  but is also bounded in $H^1(\domain)$. As an unpleasant side effect,
  it turns out that this method is inconsistent, and we have to
  estimate the consistency error.
  
  The main technique applied here is the use of lifting operators,
  such that the traces of derivatives on faces can be replaced by
  volume terms. Note that the lifting operators, while very useful for
  the analysis of the method, are not actually used in the
  implementation of the interior penalty method.
\end{intro}

\begin{Definition}{dg-lifting}
  Define the auxiliary space 
  \begin{gather}
    \label{eq:ip-lifting:1}
    \Sigma_h = \bigl\{ \tau\in L^2(\domain;\R^d) \big\vert
    \,\forall \cell\in \mesh_h \tau_{|\cell} \in \Sigma_T \bigr\},
  \end{gather}
  where $\nabla V_T \subset \Sigma_T$. Then, we define the \define{lifting
    operator} 
  \begin{gather}
    \label{eq:ip-lifting:2}
    \lifting\colon V+V_h \to \Sigma_h
  \end{gather}
  by
  \begin{gather}
    \label{eq:ip-lifting:3}
    \form(\lifting v,\tau)_{\mesh_h}
    = 2\forme(\mvl{\tau},\mvl{v\n})_{\faces_h^i}
    + \forme(\tau\cdot\n,v)_{\faces_h^\d}.
  \end{gather}
\end{Definition}

\begin{Lemma}{ip-lifting-bounded}
  The lifting operator is a bounded operator from $L^2(\faces_h)$ to
  $\Sigma_h$, such that
  \begin{gather}
    \label{eq:ip-lifting:4}
    \norm{\lifting v}_{L^2(\domain)}
    \le c \norm{\jmp{v}}_{\faces_h^i} + \norm{v}_{\faces_h^\d}.
  \end{gather}
  In particular, it is bounded on $H^1(\domain)$.
\end{Lemma}

\begin{Definition}{ip-lifting}
  The \define{interior penalty method} with lifting operators uses the
  bilinear form
  \begin{multline}
    \label{eq:ip-lifting:5}
    a_h(u,v) = \form(\nabla u,\nabla v)_{\mesh_h}
    - \form(\lifting u, \nabla v)_{\mesh_h}
    - \form(\nabla u, \lifting v)_{\mesh_h}
    \\
    + \forme(\ipp_h\jmp{u},\jmp{v})_{\faces_h^-}
    + \forme(\ipp_h u,v)_{\faces_h^\d}
    .
  \end{multline}
  and the linear form~\eqref{eq:ip:3} of the original interior penalty
  method. Its residual operator is
  \begin{gather}
    \label{eq:ip-lifting:7}
    \Res(u,v) = a_h(u,v) - \form(f,v).
  \end{gather}
\end{Definition}

\begin{Lemma}{ip-equivalence}
  The interior penalty method in flux form (\blockref{Definition}{ip})
  and in lifting form (\blockref{Definition}{ip}) coincide on the
  discrete space $V_h$ if $\Sigma_h$ is chosen such that $\nabla V_h
  \subset \Sigma_h$.
\end{Lemma}

\begin{proof}
  Since $\nabla V_h \subset \Sigma_h$, $\nabla u_h$ and $\nabla v_h$
  are valid test functions in the definition~\eqref{eq:ip-lifting:3}
  of the lifting operator, and  the equality
  \begin{gather*}
    \form(\lifting{u_h},\nabla v_h)_{\mesh_h}
    = 2\forme(\mvl{u_h\n},\mvl{\nabla v_h})_{\faces_h^i}
    + \forme(u_h,\d_n v_h)_{\faces_h^\d}.
  \end{gather*}
\end{proof}



%%% Local Variables:
%%% mode: latex
%%% TeX-master: "main"
%%% End:
