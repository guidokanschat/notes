\begin{Definition}{facets}
  Let $\cell\subset \R^d$ be a polyhedron. We call the lower
  dimensional polyhedra constituting its boundary \define{facet}s. A
  facet of dimension zero is called \define{vertex}, of dimension one
  \define{edge}, and a facet of codimension one is called a
  \define{face}.
\end{Definition}

\begin{Definition}{mesh}
  A \define{mesh} $\mesh$ is a nonoverlapping subdivision of the
  domain $\domain$ into polyhedral \define{cell}s denoted by $\cell$,
  for instance simplices, quadrilaterals, or hexahedra. The
  faces of a cell are denoted by $\face$, the
  vertices by $\vertex$. Cells are typically considered open sets.

  A mesh $\mesh$ is called regular, if each face
  $\face \subset \d\cell$ of the cell $\cell\in\mesh$ is either a
  face of another cell $\cell\prime$, that is,
  $\overline{\face} = \overline{\cell} \cap \overline{\cell\prime}$,
  or a subset of $\d\domain$.
\end{Definition}

\begin{remark}
  For this introduction, we will assume that indeed $\domain$ is the
  union of mesh cells, which means, that its boundary consists of a
  finite union of planar faces. The more general case of a mesh
  approximating the domain will be deferred to later discussion.
\end{remark}

\begin{Definition}{finite-element}
  With a mesh cell $\cell$, we associate a finite dimensional
  \define{shape function} space $\shapespace(\cell)$ of dimension
  $n_\cell$. The term \define{node functional} denotes linear
  functionals on this space.

  A set of node functionals $\{\nodal_\cell^i\}_{i=1,\dots,n_\cell}$ is called
  \define{unisolvent} on $\shapespace(\cell)$ if for any vector
  $\vu = (u_1,\dots,u_{n_\cell})^T$ there exists a unique
  $u\in \shapespace(\cell)$ such that
  \begin{gather}
    \nodal_\cell^i(u) = \vu_i,\quad i=1,\dots,n.
  \end{gather}

  A \define{finite element} is a set of shape function spaces
  $\shapespace(\cell)$ for all $\cell\in\mesh$ together with
  unisolvent set of node functionals.
\end{Definition}

\begin{Notation}{dofs}
  If the node functionals $\nodal^i$ are unisolvent on
  $\shapespace(\cell)$, then, there is a basis $\{p_k\}$ of $\shapespace(\cell)$
  such that
  \begin{gather}
    \nodal^i(p_k) = \delta_{ik}.
  \end{gather}
  We refer to $\{p_k\}$ as \define{shape function basis} and use the
  term \define{degrees of freedom} for both the node functionals and
  the basis functions.
\end{Notation}

\begin{Definition}{node-topology}
  Node functionals can be associated with the cell $\cell$ or with one
  of its lower dimensional boundary facets. We call this association
  the \define{topology} of the finite element.
\end{Definition}

\begin{Definition}{fe-space}
  The \define{finite element space} on the mesh $\mesh$, denoted by
  $V_\mesh$ is a subset of the concatenation of all shape function
  spaces,
  \begin{gather}
    V_\mesh \subset \bigl\{ f\in L^2(\domain) \big|
    f_{\cell} \in \shapespace(\cell) \bigr\}.
  \end{gather}
  The \define{degrees of freedom} of $V_\mesh$ are the union of all
  node functionals, where we identify node functionals associated to
  boundary facets among all cells sharing this facet. The resulting
  dimension is
  \begin{gather}
    n = \dim V_\mesh \le \sum n_\cell.
  \end{gather}
\end{Definition}

\begin{Notation}{global-local}
  When we enumerate the degrees of freedom of $V_\mesh$, we obtain a
  global numbering of degrees of freedom $\nodal^i$ with
  $i=1,\dots,n$. For each mesh cell, we have a local numbering
  $\nodal_\cell^j$ with $j=1,\dots,n$. By construction of the finite
  element space, there is a unique $i$, such that
  $\nodal_\cell^j(f) = \nodal^i(f)$ for all cells $\cell$ and local
  indices $j$. The converse is not true due to the identification
  process.

  We refer to this mapping as the mapping between global and local
  indices. It induces a ``natural'' basis $\{v_i\}$ of $V_\mesh$ by
  \begin{gather}
    v_{i|\cell} = p_{\cell,j},
  \end{gather}
  where $\{p_{\cell,j}\}$ is the shape function basis on $\cell$.
\end{Notation}

\begin{Lemma}{mesh-continuity}
  Let $\mesh$ be a subdivision of $\domain$, and let $u$ be a function
  on $\domain$, such that $u_{|\cell} \in C^1(\cell)$. Then,
  \begin{gather}
    u\in H^1(\domain)
    \quad \Longleftrightarrow\quad
    u\in C(\overline\domain).
  \end{gather}
\end{Lemma}

\begin{Lemma}{nodal-continuity}
  We have $V_\mesh\subset C(\overline{\domain})$ if and only if for
  every facet $F$ of dimension $d_F < d$ there holds that
  \begin{enumerate}
  \item the traces of the spaces $\shapespace(\cell)$ on $F$ coincide
    for all cells $\cell$ having $F$ as a facet,
  \item The node functionals associated to the facet are unisolvent on
    this trace space.
  \end{enumerate}
\end{Lemma}

%%%%%%%%%%%%%%%%%%%%%%%%%%%%%%%%%%%%%%%%%%%%%%%%%%%%%%%%%%%%%%%%%%%%%%
%%%%%%%%%%%%%%%%%%%%%%%%%%%%%%%%%%%%%%%%%%%%%%%%%%%%%%%%%%%%%%%%%%%%%%
\subsection{Shape function spaces on simplices}
%%%%%%%%%%%%%%%%%%%%%%%%%%%%%%%%%%%%%%%%%%%%%%%%%%%%%%%%%%%%%%%%%%%%%%
%%%%%%%%%%%%%%%%%%%%%%%%%%%%%%%%%%%%%%%%%%%%%%%%%%%%%%%%%%%%%%%%%%%%%%

\begin{Definition}{barycentric-coordinates}
  A simplex $\cell\in \R^d$ with vertices $\vertex_0,\dots,\vertex_d$
  is described by a set of $d+1$ \define{barycentric coordinates}
  $\vlambda = (\lambda_0,\dots,\lambda_d)^T$ such that
  \begin{xalignat}2
    0\le\lambda_i &\le 1& i&=0,\dots,d;\\
    \lambda_i(\vertex_j) &= \delta_{ij}& i,j&=0,\dots,d\\
    \sum \lambda_i(\vx) &= 1,
  \end{xalignat}
  and there holds
  \begin{gather}
    T = \Bigl\{x\in\R^d \Big| x = \sum \vertex_k\lambda_k \Bigr\}.
  \end{gather}
\end{Definition}

\begin{Lemma}{barycentric-affine}
  There is a matrix $B_T\in \R^{d+1\times d}$ and a vector
  $b_T\in\R^{d+1}$, such that
  \begin{gather}
    \vlambda = B_T\vx + b_T.
  \end{gather}
\end{Lemma}

\begin{Corollary}{barycentric-interpolation}
  The barycentric coordinates $\lambda_0,\dots,\lambda_d$ are the
  linear Lagrange interpolating functions for the points
  $\vertex_0,\dots,\vertex_d$. In particular, $\lambda_k \equiv 0$ on
  the facet not containing $\vertex_k$.
\end{Corollary}

\begin{example}
  We can use barycentric coordinates to define interpolating polynomials on
  simplicial meshes easily, as in
  Table~\ref{tab:barycentric-shapes}.
  \begin{table}[tp]
    \centering
    \begin{tabular}{|c|l|}
      \hline Degrees of freedom
      & Shape functions \\\hline
      \adjustbox{valign=center,margin=3pt}{\includegraphics[width=2cm]{mixed/fig/p1-p.tikz}}
      &
        {\begin{minipage}[b]{6cm}
          \begin{gather*}
            \phi_i = \lambda_i,
            \quad i=0,1,2
          \end{gather*}
        \end{minipage}}
      \\\hline
      \adjustbox{valign=center,margin=3pt}{\includegraphics[width=2cm]{mixed/fig/p2-p.tikz}}
      &
        {\begin{minipage}[b]{6cm}
          \begin{xalignat*}2
            \phi_{ii} &= 2\lambda_i^2 - \lambda_i,
            &i&=0,1,2\\
            \phi_{ij} &= 4\lambda_i\lambda_j
            &j&\neq i
          \end{xalignat*}
        \end{minipage}}
        \\\hline
      \adjustbox{valign=center,margin=3pt}{\includegraphics[width=2cm]{mixed/fig/p3-p.tikz}}
      &
        {\begin{minipage}[b]{6cm}
          \begin{xalignat*}2
          \phi_{iii} &= \tfrac12 \lambda_i(3\lambda_i-1)(3\lambda_i-2)
          &i&=0,1,2\\
          \phi_{ij} &= \tfrac92\lambda_i\lambda_j(3\lambda_j-1)
          &j&\neq i\\
          \phi_0 &= 27\lambda_0\lambda_1\lambda_2
        \end{xalignat*}
        \end{minipage}}
        \\\hline
    \end{tabular}
    \caption{Degrees of freedom and shape functions of simplicial elements
      in terms of barycentric coordinates}
    \label{tab:barycentric-shapes}
  \end{table}
\end{example}

\begin{remark}
  The functions $\lambda_i(x)$ are the shape functions of the linear
  $P_1$ element on $T$. They allow us to define basis functions on the
  cell $T$ without use of a reference element $\widehat T$.

  Note that $\lambda_i\equiv 0$ on the face opposite to the
  vertex $x_i$.
\end{remark}



%%% Local Variables: 
%%% mode: latex
%%% TeX-master: "main"
%%% End: 
