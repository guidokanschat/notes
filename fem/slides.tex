\documentclass[USEnglish,ignorenonframetext,notheorems,aspectratio=1610]{beamer}
\usetheme[compress]{Madrid}
\usecolortheme{iwr}
\usepackage{tikz,tikzscale}
\usetikzlibrary{snakes}
\tikzset{shape veloxy/.style={color=black,draw,fill=red,thick}}
\tikzset{shape pressure/.style={color=black,draw,fill=cyan,thick}}


\usepackage{../mathsim}
\usepackage{times}
\usepackage{xr}
\externaldocument{awa}
\externaldocument{explicit}
\externaldocument{implicit}
\externaldocument{lmm}
\externaldocument{rwa}
\externaldocument{newton}
\externaldocument{fd}
\externaldocument{appendix}
\usepackage{mfirstuc}
\usepackage{mathtools}  
\mathtoolsset{showonlyrefs}

\def\footnote#1{}

\title{Finite Elements}
\author{Guido Kanschat}
\date{\today}
\begin{document}
\frame{\maketitle}
\frame{\frametitle{Overview}\tableofcontents[hideallsubsections]}
\section{Elliptic PDE and Their Weak Formulation}
%\frame{\tableofcontents[currentsection,hideothersubsections]}

% \frame {\input{blocks/Notation-coordinates.tex}}
% \frame {\input{blocks/Notation-partial-derivative.tex}}
% \frame {\input{blocks/Notation-elim-coord.tex}
%   \input{blocks/Definition-lin-pde-2order.tex}}
% \frame {\input{blocks/Definition-poisson-eqn.tex}}
% \frame {\input{blocks/Definition-domain.tex}}
% \frame {\input{blocks/Definition-boundary-conditions.tex}}
% \frame {\input{blocks/Definition-dirichlet-problem-differential.tex}}
% \frame {\input{blocks/Theorem-Dirichlet-principle.tex}}
% \frame {\input{blocks/Theorem-Dirichlet-variational-principle.tex}
%   \input{blocks/Corollary-Dirichlet-uniqueness.tex}}
% \frame {\input{blocks/Lemma-reduction-to-zero-bc.tex}}
% \frame {\input{blocks/Notation-l2.tex}}
% \frame {\input{blocks/Lemma-Friedrichs-continuous.tex}
%   \input{blocks/Problem-Friedrichs.tex}}
% \frame {\input{blocks/Lemma-h1-norm.tex}}
% \frame {\input{blocks/Lemma-Dirichlet-energy-boundedness.tex}
%   \input{blocks/Lemma-minimizing-sequence.tex}}
% \frame {\input{blocks/Definition-h10.tex}
%   \input{blocks/Lemma-Friedrichs-h1.tex}}
% \frame {\input{blocks/Definition-weak-formulation.tex}
%   \input{blocks/Theorem-weak-unique-solution-1.tex}}

\frame {\input{blocks/Lemma-neumann-weak.tex}}
\frame {\input{blocks/Definition-natural-bc.tex}}
\frame {\input{blocks/Lemma-mixed-bc-weak.tex}}

\section{Hilbert Spaces and Bilinear Forms}
\frame{\tableofcontents[currentsection,hideothersubsections]}

% \frame {\input{blocks/Definition-inner-product.tex}}
% \frame {\input{blocks/Theorem-bcs-inequality.tex}
%   \input{blocks/Lemma-inner-product-norm.tex}}
% \frame {\input{blocks/Definition-complete.tex}}
% \frame {\input{blocks/Definition-Banach-hilbert.tex}}
% \frame {\input{blocks/Definition-orthogonal.tex}}
% \frame {\input{blocks/Definition-orthogonal-complement.tex}}
% \frame {\input{blocks/Lemma-orthogonal-closed.tex}}
% \frame {\input{blocks/Theorem-orthogonal-complement.tex}}
% %\frame {\input{blocks/Corollary-ortho-density.tex}}
% \frame {\input{blocks/Definition-ortho-projection.tex}}
% \frame {\input{blocks/Definition-dual-space.tex}}
% \frame {\input{blocks/Theorem-Riesz-representation.tex}}
% \frame {\input{blocks/Definition-bilinear-form.tex}}
% \frame {\input{blocks/Lemma-pde-bilinear.tex}}
% \frame {\input{blocks/Lemma-lax-milgram.tex}}
% \frame {\input{blocks/Lemma-weak-well-posed.tex}
%   \input{blocks/Definition-elliptic.tex}}

\section{Sobolev Spaces}
\frame{\tableofcontents[currentsection,hideothersubsections]}

% \frame {\input{blocks/Notation-multi-index.tex}}
% \frame {\input{blocks/Definition-distributional-derivative.tex}}
% \frame {\input{blocks/Definition-Wkp.tex}
%   \input{blocks/Corollary-wkp-embedding.tex}}
% \frame {\input{blocks/Definition-hkp.tex}
%   \input{blocks/Theorem-meyers-serrin.tex}}

% \frame {\input{blocks/Definition-boundary-smoothness.tex}}
% \frame {\input{blocks/Definition-continuous-embedding.tex}}
% \frame {\input{blocks/Theorem-sobolev-embedding.tex}}
% \frame {\input{blocks/Lemma-trace-continuous.tex}
%   \input{blocks/Theorem-trace.tex}}
% \frame {\input{blocks/Definition-hoelder-spaces.tex}}
% \frame {\input{blocks/Theorem-hoelder-embedding.tex}
%   \input{blocks/Corollary-sobolev-continuous.tex}}

\section{Properties of Solutions}
\frame{\tableofcontents[currentsection,hideothersubsections]}

% \frame {\input{blocks/Definition-wkp-loc.tex}
%   \input{blocks/Theorem-gt-8-8.tex}
%   \input{blocks/Theorem-gt-8-10.tex}
%   \input{blocks/Corollary-gt-8-10.tex}}
% \frame {\input{blocks/Theorem-gt-8-13.tex}
%   \input{blocks/Corollary-h2-solution-bvp.tex}
%   \input{blocks/Remark-classical-smooth.tex}
%   \input{blocks/Remark-classical-convex.tex}}
% \frame {\input{blocks/Theorem-kondratev.tex}}

%%%%%%%%%%%%%%%%%%%%%%%%%%%%%%%%%%%%%%%%%%%%%%%%%%%%%%%%%%%%%%%%%%%%%%
%%%%%%%%%%%%%%%%%%%%%%%%%%%%%%%%%%%%%%%%%%%%%%%%%%%%%%%%%%%%%%%%%%%%%%
\section{Meshes, shape functions, and degrees of freedom}
\frame{\tableofcontents[currentsection,hideothersubsections]}
%%%%%%%%%%%%%%%%%%%%%%%%%%%%%%%%%%%%%%%%%%%%%%%%%%%%%%%%%%%%%%%%%%%%%%
%%%%%%%%%%%%%%%%%%%%%%%%%%%%%%%%%%%%%%%%%%%%%%%%%%%%%%%%%%%%%%%%%%%%%%
\frame {\input{blocks/Definition-facets.tex}
  \input{blocks/Definition-mesh.tex}}
\frame {\input{blocks/Definition-finite-element.tex}
\input{blocks/Notation-dofs.tex}}
\frame {\input{blocks/Definition-node-topology.tex}
  \input{blocks/Definition-fe-space.tex}}
\frame {\input{blocks/Notation-global-local.tex}}
\frame {\input{blocks/Definition-local-global.tex}}
\frame {\input{blocks/Lemma-fe-support.tex}}
\frame {\input{blocks/Lemma-mesh-continuity.tex}
  \input{blocks/Lemma-nodal-continuity.tex}}

\frame {\input{blocks/Definition-barycentric-coordinates.tex}}
\frame {\input{blocks/Lemma-barycentric-affine.tex}
  \input{blocks/Corollary-barycentric-interpolation.tex}}

\begin{frame}
  \frametitle{The $P_1$ element in barycentric coordinates}
  \begin{columns}
    \begin{column}{.5\textwidth}
      \begin{center}
        \includegraphics[width=.6\textwidth]{mixed/fig/p1-p.tikz}
      \end{center}
    \end{column}
    \begin{column}{.5\textwidth}
      \begin{gather*}
        \phi_i = \lambda_i,
        \quad i=0,1,2
      \end{gather*}
    \end{column}
  \end{columns}
\end{frame}

\begin{frame}
  \frametitle{The $P_2$ element in barycentric coordinates}
  \begin{columns}
    \begin{column}{.5\textwidth}
      \begin{center}
        \includegraphics[width=.6\textwidth]{mixed/fig/p2-p.tikz}
      \end{center}
    \end{column}
    \begin{column}{.5\textwidth}
      \begin{xalignat*}2
        \phi_{ii} &= 2\lambda_i^2 - \lambda_i,
        &i&=0,1,2\\
        \phi_{ij} &= 4\lambda_i\lambda_j
        &j&\neq i
      \end{xalignat*}
    \end{column}
  \end{columns}
\end{frame}

\begin{frame}
  \frametitle{The $P_3$ element in barycentric coordinates}
  \begin{columns}
    \begin{column}{.5\textwidth}
      \begin{center}
        \includegraphics[width=.6\textwidth]{mixed/fig/p3-p.tikz}
      \end{center}
    \end{column}
    \begin{column}{.5\textwidth}
      \begin{xalignat*}2
        \phi_{iii} &= \tfrac12 \lambda_i(3\lambda_i-1)(3\lambda_i-2)
        &i&=0,1,2\\
        \phi_{ij} &= \tfrac92\lambda_i\lambda_j(3\lambda_j-1)
        &j&\neq i\\
        \phi_0 &= 27\lambda_0\lambda_1\lambda_2
      \end{xalignat*}
    \end{column}
  \end{columns}
\end{frame}

\frame {\input{blocks/Definition-galerkin-approximation.tex}}
\frame {\input{blocks/Corollary-galerkin-equations.tex}}
\frame {\input{blocks/Lemma-discrete-lax-milgram.tex}
  \input{blocks/Lemma-cea.tex}}
\frame {\input{blocks/Lemma-fe-matrix.tex}}
\frame {\input{blocks/Algorithm-matrix-assembling.tex}}
\frame {\input{blocks/Definition-mapped-mesh.tex}}
\frame {\input{blocks/Example-mapping-linear.tex}}
\frame {\input{blocks/Example-mapping-bilinear.tex}}
\frame {\input{blocks/Definition-mapped-fe.tex}}
\frame {\input{blocks/Lemma-mapped-norms-affine.tex}}
\frame {\input{blocks/Lemma-shape-regular-transformation.tex}}
\frame {\input{blocks/Assumption-mapping-decomposition.tex}}
\frame {\input{blocks/Lemma-scaling-1.tex}}


\frame{\bibliographystyle{alpha}
\bibliography{all}}
\end{document}

%%% Local Variables:
%%% mode: latex
%%% TeX-master: t
%%% End:
