\usetikzlibrary{svg.path}
\excludecomment{solution}
\tikzset{shape veloxy/.style={color=black,draw,fill=red,thick}}
\tikzset{shape pressure/.style={color=black,draw,fill=cyan,thick}}


\def\constref#1{C_{\text{\ref{#1}}}}
\title{Finite Elements}
\author{Guido Kanschat}
\date{\today}

\def\vecx{\mathbf x}
\begin{document}
\maketitle
\tableofcontents
%\chapter{Elliptic PDE and Their Weak Formulation}
%\section{Elliptic boundary value problems}
%\subsection{Linear second order PDE}

\begin{Notation}{coordinates}
  Dimension of ``physical space'' will be denoted by $d$.  We denote
  coordinates in $\R^d$ as
  \begin{gather*}
    \vx = (x_1,\dots,x_d)^T  .
  \end{gather*}
  In the special cases $d=2,3$ we also write
  \begin{gather*}
  \vx =
  \begin{pmatrix}
    x\\y
  \end{pmatrix},
  \qquad\qquad
  \vx = \begin{pmatrix}
    x\\y\\z
  \end{pmatrix},
  \end{gather*}
  respectively.
  The Euclidean norm on $\R^d$ is denoted as
  \begin{gather*}
     \abs{\vx} = \sqrt{\sum_{i=1}^d x_i^2}.
  \end{gather*}
\end{Notation}

\begin{Notation}{partial-derivative}
  Partial derivatives of a function $u\in C^1(\R^d)$ are denoted by
  \begin{gather*}
    \frac{\d u(\vx)}{\d x_i} = \tfrac{\d}{\d x_i} u(\vx)
    = \d_{x_i} u(\vx) = \d_i u(\vx).
  \end{gather*}

  The \define{gradient} of $u \in C^1$ is the row vector
  \begin{gather*}
    \nabla u = (\d_1u,\dots,\d_du)
  \end{gather*}

  The \define{Laplacian} of a function $u\in C^2(\R^d)$ is
  \begin{gather*}
    \Delta u = \d_1^2 u + \dots + \d_d^2 u = \sum_{i=1}^d \d_i^2 u
  \end{gather*}
\end{Notation}

\begin{Notation}{elim-coord}
  When we write equations, we typically omit the independent variable
  $\vx$. Therefore,
  \begin{gather*}
    \Delta u \equiv \Delta u(\vx).
  \end{gather*}
\end{Notation}

\begin{Definition}{lin-pde-2order}
  A linear PDE of second order in divergence form for a function
  $u\in C^2(\R^d)$ is an equation of the form
  \begin{gather}
    -\sum_{i,j=1}^d \d_i \bigl(a_{ij}(\vx) \d_j u\bigr)
    + \sum_{i=1}^d \bigl(b_i(\vx) \d_i u\bigr) + c(\vx) u = f(x)
  \end{gather}
\end{Definition}

\begin{Definition}{poisson-eqn}
  An important model problem for the equations we are going to study
  is \define{Poisson's equation}
  \begin{gather}
    \label{eq:Poisson}
    -\Delta u = f.
  \end{gather}
\end{Definition}

\begin{intro}
  Already with ordinary differential equations we experience that we
  typically do not search for solutions of the equation itself, but
  that we ``anchor'' the solution by solving an initial value problem,
  fixing the solution at one point on the time axis.

  It does not make sense to speak about an initial point in
  $\R^d$. Instead, it turns out that it is appropriate to consider
  solutions on certain subsets of $\R^d$ and impose conditions at the
  boundary.
\end{intro}

\begin{Definition}{domain}
  A \define{domain} in $\R^d$ is a connected, open set of $\R^d$. We
  typically use the notation $\domain\subset\R^d$.

  The \define{boundary} of a domain $\domain$ is denoted by
  $\d\domain$. To any point $\vx\in\d\domain$, we associate the outer
  unit \define{normal vector} $\vn \equiv \vn(\vx)$.

  The symbol $\d_n u \equiv (\nabla u) \vn$ denotes the \define{normal
    derivative} of a function $u\in C^1(\overline{\domain})$ at a point
  $\vx\in\d\domain$.
\end{Definition}

\begin{Definition}{boundary-conditions}
  We distinguish three types of boundary conditions for Poisson's
  equation, namely for a point $\vx\in\d\domain$ with a given function $g$
  \begin{enumerate}
  \item Dirichlet:
    \begin{gather*}
      u(\vx) = g(\vx)
    \end{gather*}
  \item Neumann:
    \begin{gather*}
      \d_n u(\vx) = g(\vx)
    \end{gather*}
  \item Robin: for some positive function $\alpha$ on $d\domain$
    \begin{gather*}
      \d_n u(\vx) + \alpha(\vx) u(\vx) = g(\vx)
    \end{gather*}
  \end{enumerate}
  While only one of these boundary conditions can hold in a single
  point $\vx$, different boundary conditions can be active on
  different subsets of $\d\domain$. We denote such subsets as
  $\Gamma_D$, $\Gamma_N$, and $\Gamma_R$.
\end{Definition}

\begin{Definition}{dirichlet-problem-differential}
  The \define{Dirichlet problem} for \putindex{Poisson's equation} (in
  differential form) is: find
  $u\in C^2(\domain)\cap C(\overline{\domain})$, such that
  \begin{subequations}
    \begin{xalignat}2
      -\Delta u(\vx) &= f(\vx) & x&\in \domain, \\
      u(\vx) &= g(\vx) & x&\in \d\domain.
    \end{xalignat}
  \end{subequations}
  Here, the functions $f$ on $\domain$ and $g$ on $\d\domain$ are data
  of the problem.

  The Dirichlet problem is called \define{homogeneous}, if $g\equiv 0$.
\end{Definition}

\begin{Theorem*}{Dirichlet-principle}{Dirichlet principle}
  If a function $u\in C^2(\domain)\cap C(\overline{\domain})$ solves
  the \putindex{Dirichlet problem}, then it minimizes the
  \define{Dirichlet energy}
  \begin{gather}
    \label{eq:Dirichlet-energy}
    E(v) = \int_{\domain} \tfrac12 \abs{\nabla v}^2 \dvx - \int_{\domain} f v \dvx,
  \end{gather}
  among all functions $v$ from the set
  \begin{gather}
    V_g = \bigl\{ v\in C^2(\domain)\cap C(\overline{\domain})
    \big| v_{|\d\domain} = g \bigr\}.
  \end{gather}
  This minimizer is unique.
\end{Theorem*}

\begin{proof}
  Using variation of $E$, we will show that
  \begin{gather*}
    \frac{\diffd}{\diffd \varepsilon} E(u+\varepsilon v)
    \Big|_{\varepsilon=0} = 0
  \end{gather*}
  for all $v \in V_0$ since this implies $u + \varepsilon v = g$
  on $\d\domain$. By evaluating the square we have
  \begin{gather*}
    \frac{\diffd}{\diffd \varepsilon} E(u+\varepsilon v)
    = \int_\domain \nabla u \nabla v + \varepsilon \abs{\nabla v}^2 - fv \dx
  \end{gather*}
  Since we are intersted in $E(u)$, we now consider $\varepsilon=0$. We
  get that $u$ minimizes $E(u+\varepsilon v)$ at $\varepsilon = 0$ implies
  \begin{gather*}
    \int_\domain \nabla u \nabla v \dx = \int_\domain fv \dx,
    \qquad\forall v\in V_0.
  \end{gather*}
  By Green's formula
  \begin{gather*}
    \int_\domain \nabla u \nabla v \dx = \int_\domain - \Delta u v \dx
    + \int_{\d\domain} \d_n u v \ds
  \end{gather*}
  we obtain that if $u$ minimizes $E(\cdot)$, then
  \begin{gather*}
    \int_\domain \nabla u \nabla v \dx = \int_\domain fv \dx,
    \qquad\forall v\in V_0,
  \end{gather*}
  since $v \in V_0$ vanishes on $\d\domain$. In summary, we have
  proven so far that if $u$ solves Poisson's Equation, then it is a
  stationary point of $E(\cdot)$. It remains to show that
  $E(u) \le E(u+v)$ for any $v\in V_0$.  Using
  $\int_\domain fv \dx = \int_\domain \nabla u \nabla v$ yields
  \begin{multline*}
    E(u+v) - E(u) = \frac 12 \int_\domain |\nabla (u+v)|^2 - 2 \nabla (u+v) \nabla u
    + |\nabla u|^2 \dx
    = \frac 12 |\nabla v|^2 \dx \ge 0.
  \end{multline*}
  This also proves uniqueness.
\end{proof}

\begin{Lemma}{Dirichlet-Cauchy}
  A minimizing sequence for the Dirichlet energy exists and it is a
  Cauchy sequence.
\end{Lemma}

\begin{proof}
  The Dirichlet energy $E(\cdot)$ is bounded from
  below and hence an infinum exists. Thus, there also exists a series
  $\{u^{(n)}\}_{n \in \mathbb{N}}$ converging to this infinum, i.e.
  \begin{gather*}
    \lim_{n \to \infty} E(u^{(n)}) = \inf_{v \in V_0} E(v).
  \end{gather*}
  Second, we show that $\{u^{(n)}\}_n$ is a Cauchy sequence.
  
  For the first part we use Friedrich's inequality
  \begin{gather*}
    \norm{v}_{L^2(\domain)} \le \lambda(\Omega)
    \norm{\nabla v}_{L^2(\domain)} \qquad v \in V_0.
  \end{gather*}
  The proof of this result will be given later. Using Hölder's inequality
  we obtain
  \begin{gather*}
    E(v) = \frac 12 \norm{\nabla v}^2 _{L^2(\domain} - \int_\domain fv \dx
    \ge \frac 12 \norm{\nabla v}^2 _{L^2(\domain}
    - \norm{f}_{L^2(\domain} \norm{v}_{L^2(\domain}
  \end{gather*}
  Applying Friedrich's inquality yields that the above expression is
  greater or equal than
  \begin{gather*}
    \frac 12 \norm{\nabla v}^2 _{L^2(\domain}
    - \norm{\nabla v}_{L^2(\domain} \frac 1{\lambda (\domain)}
    \norm{f}_{L^2(\domain}.
  \end{gather*}
  Finally, we apply Young's inequality $ab \le \nicefrac 12 (a^2 + b^2)$ to obtain
  \begin{gather*}
    \frac 12 \norm{\nabla v}^2 _{L^2(\domain}
    - \norm{\nabla v}_{L^2(\domain}^2 - \frac 1{2\lambda (\domain)} \norm{f}_{L^2(\domain}^2
  \end{gather*}
  which yields $E(v) \ge - \frac 1{2\lambda (\domain)^2} \norm{f}_{L^2(\domain}^2$
  as a lower bound independent of $v$. To prove the second part,
  we use the parallelogram identity $\abs{v+w}^2 + \abs{v-w}^2 = 2\abs{v}^2 + 2 \abs{w}^2$.
  Let $m, n$ be natural numbers, then
  \begin{align*}
    \snorm{u^{(n)} - u^{(m)}}^2 _1 =& 2 \snorm{u^{(n)}}^2 _1
                                      + 2 \snorm{u^{(m)}}^2_1- 4 \snorm{\nicefrac 12 (u^{(n)} + u^{(m)}}^2_1 \\
    =& 4 E(u^{(n)}) + 4\int f u^{(n)} \dx + 4 E(u^{(m)}) + 4\int f u^{(m)} \dx \\
                                    &- 8 E(\nicefrac 12 (u^{(n)} + u^{(m)}) - 8 \int \nicefrac 12 f(u^{(n)} + u^{(m)}) \\
    =& 4 E(u^{(n)}) + 4 E(u^{(m)}) - 8 E(\nicefrac 12 f(u^{(n)} + u^{(m)}))
  \end{align*}
  Taking the limit $m,n\to \infty$ yields $4 E(u^{(n)}) + 4 E(u^{(m)})
  \to 8 \inf_{v \in V_0} E(v)$. Lastly, $-E(\nicefrac 12 f(u^{(n)} + u^{(m)}))$ can
  be bounded by $inf_{v \in V_0} E(v)$. It follows that $\limsup_{m,n\to\infty}
  \snorm{u^{(n)}-u^{(m)}}^2_1 \le 0$ and consequently as desired
  \begin{gather*}
    \lim_{m,n\to\infty} \snorm{u^{(n)}-u^{(m)}}^2_1 = 0.
  \end{gather*}    
\end{proof}

\begin{notes}{Dirichlet-proof}
  Dirichlet's principle proved essential for the development of a
  rigorous solution theory for Poisson's equation.  Its proof will be
  deferred to the next theorem.
\end{notes}

\subsection{Variational principle and weak formulation}
\begin{Theorem}{Dirichlet-variational-principle}
  A function $u\in V_g$ minimizes the Dirichlet energy, if and only if
  there holds
  \begin{gather}
    \int_{\domain} \nabla u\cdot\nabla v \dx
    = \int_{\domain} fv\dx, \qquad\forall v\in V_0.
  \end{gather}
  Moreover, any solution to the Dirichlet problem in
  \slideref{Definition}{dirichlet-problem-differential} solves this
  equation.
\end{Theorem}

\begin{Corollary}{Dirichlet-uniqueness}
  If a minimizer of the Dirichlet energy exists, it is necessarily unique.
\end{Corollary}

\begin{Lemma}{reduction-to-zero-bc}
  A function $u\in V_g$ minimizes the Dirichlet energy admits the
  representation $u = u_g + u_0$, where $u_g\in V_g$ is arbitrary and
  $u_0\in V_0$ solves
  \begin{gather}
    \int_{\domain} \nabla u\cdot\nabla v \dx
    = \int_{\domain} fv\dx
    - \int_{\domain} \nabla u_g\cdot\nabla v \dx,
    \qquad\forall v\in V_0.
  \end{gather}
  The function $u_0$ depends on the choice of $u_g$, but not the minimizer $u$.
\end{Lemma}

\begin{Notation}{l2}
  The inner product of $L^2(\domain)$ is denoted by
  \begin{gather*}
    \form(u,v) \equiv \form(u,v)_{\domain}
    \equiv \form(u,v)_{L^2(\domain)}
    = \int_{\domain} u v \dvx.
  \end{gather*}
  Its norm is
  \begin{gather*}
    \norm{u} \equiv \norm{u}_{\domain} \equiv \norm{u}_{L^2(\domain)}
    \equiv \norm{u}_{L^2} = \sqrt{\form(u,v)_{L^2(\domain)}}.
  \end{gather*}
\end{Notation}
\begin{Lemma*}{Friedrichs-continuous}{Friedrichs inequality}
  For any function in $v\in V_0$ there holds
  \begin{gather}
      \norm{v}_{\domain}
      \le \diam(\domain) \norm{\nabla v}_{\domain}.
  \end{gather}
\end{Lemma*}

\begin{Lemma}{h1-norm}
  The definitions
  \begin{gather}
    \begin{split}
      \abs{v}_1 &= \norm{\nabla v}_{L^2(\domain)},\\
      \norm{v}_1 &= \sqrt{\norm{v}^2_{L^2(\domain)}
        + \abs{v}^2_1},
    \end{split}
  \end{gather}
  both define a norm on $V_0$.
\end{Lemma}

\begin{Problem}{Friedrichs}
  Prove the Friedrichs inequality.
\end{Problem}

\begin{Lemma}{Dirichlet-energy-boundedness}
  The Dirichlet energy with homogeneous boundary conditions is bounded
  from below and thus has an infimum. In particular, there exists a
  \define{minimizing sequence} $\{u^n\}$ such that as $n\to\infty$,
  \begin{gather}
    E(u^n) \to \inf_{v\in V_0} E(v).
  \end{gather}
\end{Lemma}

\begin{Lemma}{minimizing-sequence}
  The minimizing sequence for the Dirichlet energy is a
  \putindex{Cauchy sequence}.
\end{Lemma}

\begin{Definition}{h10}
  The completion of $V_0$ under the norm $\norm{v}_1$ is the
  \define{Sobolev space} $H^1_0(\domain)$.
\end{Definition}

\begin{Lemma*}{Friedrichs-h1}{Friedrichs inequality}
  For any function in $v\in H^1_0$ there holds
  \begin{gather}
      \norm{v}_{\domain}
      \le \diam(\domain) \norm{\nabla v}_{\domain}.
  \end{gather}
\end{Lemma*}

\begin{proof}
  Let $v\in H^1_0(\domain)$.  We make use of the fact, that by
  definition of $H^1_0(\domain)$, there is a sequence $v_n \to v$ with
  $v_n \in V_0$. By~\slideref{Lemma}{Friedrichs-continuous},
  Friedrichs' inequality holds for $v_n$ uniformly in $n$. We conclude
  \begin{align*}
    \norm{v}_\domain &\le \norm{v - v_n}_\domain + \norm{v_n}_\domain \\
    & \le \norm{v - v_n}_\domain + \diam\domain \norm{\nabla v_n}_\domain \\
    & \le \norm{v - v_n}_\domain + \diam\domain
      \bigl(\norm{\nabla v_n - \nabla v}_\domain + \norm{\nabla v}_\domain\bigr)
  \end{align*}
  As $n\to\infty$, the norms of the differences converge to zero, such
  that the desired result holds in the limit.
\end{proof}

\begin{Definition}{weak-formulation}
  The \putindex{Dirichlet problem} for Poisson's equation in weak form
  reads: find $u\in H^1_g(\domain)$ such that
  \begin{gather}
    \int_{\domain} \nabla u\cdot\nabla v \dx
    = \int_{\domain} fv\dx, \qquad\forall v\in H^1_0(\domain).
  \end{gather}
\end{Definition}

\begin{Theorem}{weak-unique-solution-1}
  The weak formulation in \slideref{Definition}{weak-formulation} has
  a unique solution.
\end{Theorem}

\subsection{Boundary conditions in weak form}

\begin{Lemma}{neumann-weak}
  Let $u\in V=H^1(\domain)$ be a solution to the weak formulation
  \begin{gather}
    \int_{\domain} \nabla u\cdot\nabla v \dx
    = \int_{\domain} fv\dx, \qquad\forall v\in V(\domain).
  \end{gather}
  If $u\in C^2(\domain) \cap C^1(\overline{\domain})$ and $\domain$
  has $C^1$-boundary, then $u$ solves the boundary value problem
  \begin{gather}
    \begin{aligned}
      -\Delta u &= f &\qquad \text{in } &\domain\\
      \d_n u &= 0 &\text{on } &\d\domain.
    \end{aligned}
  \end{gather}
\end{Lemma}

\begin{Definition}{natural-bc}
  A boundary condition inherent in the weak formulation and not
  explicitly stated is called \define{natural boundary condition}. If
  boundary values are obtained by constraining the function space it
  is called \define{essential boundary condition}.

  We also call a boundary condition in strong form, if it is a
  constraint on the function space, and in weak form, if it is part of
  the weak formulation.
\end{Definition}

\begin{remark}
  Dirichlet and homogeneous Neumann boundary conditions are examples
  for essential and natural boundary conditions, respectively.
\end{remark}

\begin{Lemma}{mixed-bc-weak}
  The boundary value problem
  \begin{gather}
    \begin{aligned}
      -\Delta u &= f &\qquad \text{in } &\domain\\
      u &= 0 &\text{on } &\Gamma_D \subset \d\domain\\
      \d_n u + \alpha u &= g &\text{on } &\Gamma_R \subset \d\domain,
    \end{aligned}
  \end{gather}
  has the weak form: find $u\in V$ such that
  \begin{gather}
    \int_{\domain} \nabla u\cdot\nabla v \dx
    + \int_{\Gamma_R}\alpha u v \ds
    = \int_{\domain} f v\dx
    + \int_{\Gamma_R} g v \ds, \qquad\forall v\in V(\domain).    
  \end{gather}
\end{Lemma}

%%% Local Variables: 
%%% mode: latex
%%% TeX-master: "main"
%%% End: 


%\section{Hilbert Spaces and Bilinear Forms}
%\begin{intro}
 In this section we discuss important properties of Hilbert spaces and
 solvability of PDEs in Hilbert spaces. This will lead us to
  \slideref{Lemma}{weak-well-posed},
  which describes elliptic PDEs.
  As we will only consider elliptic PDEs in this lecture, this is
  one of the most important results.
\end{intro}

\begin{Definition}{inner-product}
  Let $V$ be a vector space over $\R$. An \define{inner product} on $V$ is a mapping
  $\scal(.,.): V\times V \to \R$ with the properties
  \begin{xalignat}2
    \label{eq:inner-product:1}
    \scal(\alpha x+y,z) &= \alpha \scal(x,z) + \scal(y,z)
    && \forall x,y,z \in V; \alpha \in \mathbb K\\
    \label{eq:inner-product:2}
    \scal(x,y) &= \scal(y,x) && \forall x,y \in V \\
    \label{eq:inner-product:3}
    \scal(x,x) & \ge 0 \quad\forall x\in V &&\text{and} \\
    \label{eq:inner-product:4}
    \scal(x,x) & =0 \Leftrightarrow x=0,
  \end{xalignat}
  usually referred to as (bi-)linearity, symmetry, and
  positive definiteness. We note that linearity in the second argument follows
  immediately by symmetry.
\end{Definition}

\begin{Theorem*}{bcs-inequality}{Bunyakovsky-Cauchy-Schwarz inequality}
  For every \putindex{inner product} there holds the inequality
  \begin{gather}
    \label{eq:hilbert:6}
    \scal(v,w) \le \sqrt{\scal(v,v)} \sqrt{\scal(w,w)}.
  \end{gather}
  Equality holds if and only if $v$ and $w$ are collinear.
\end{Theorem*}

\begin{proof}
  The case $w = 0$ is trivial. Without loss of generality we can therefore
  assume that $w \not = 0$. Define $\lambda \in \R$ as $\lambda =
  \frac{\scal(v,w)}{\scal(w,w)}$. By (\ref{eq:inner-product:3}) we have
  \begin{gather*}
  0 \le \scal(v - \lambda w, v - \lambda w) \label{eq:bcs:1}
  \end{gather*}
  and by ~(\ref{eq:inner-product:2}) the right-hand side extends to
  \begin{gather*}
  \scal(v, v) - \scal(v, \lambda w) - \scal(\lambda w, v) +
    \scal(\lambda w, \lambda w) \\
  = \scal(v, v) - \overline{\lambda} \scal(v, w) - \overline{\lambda} \scal(v, w)
    + \lambda \overline{\lambda}\scal(w, w).
  \end{gather*}
  Evaluating $\lambda$ yields the inequality
  \begin{gather*}
  0 \le \scal(v, v) - \frac{\overline{\scal(v, w)} \scal(v, w)}{\scal(w, w)}
    - \frac{\overline{\scal(v, w)} \scal(v, w)}{\scal(w, w)}
    + \frac{\overline{\scal(v, w)} \scal(v, w)\scal(w, w)}{\scal(w, w)^2}.
  \end{gather*}
  The result follows from multiplication with $\scal(w, w)$ and arranging
  the summands.
  
  For the second part let $v, \, w$ be colinear, i.e. there is a $\lambda \in
  \mathbb K$ such that $v = \lambda w$. Then deducing the equality is trivial.
  Now let equality hold for (\ref{eq:hilbert:6}). We immediately get that the equality
  must also hold for
  \begin{gather*}
  0 = \scal(v - \lambda w, v - \lambda w).
  \end{gather*}
  However, by ~(\ref{eq:inner-product:4}) this implies
  \begin{gather*}
  0 = v - \lambda w.
  \end{gather*}
  Thus, $v$ and $w$ are colinear.
\end{proof}

\begin{Lemma}{inner-product-norm}
  Every inner product defines a norm by
  \begin{gather}
    \label{eq:hilbert:7}
    \norm{v} = \sqrt{\scal(v,v)}.
  \end{gather}
\end{Lemma}

\begin{proof}
  Definiteness and homogeneity follow from the properties of the inner
  product. It remains to show the triangle inequality
  \begin{gather*}
    \norm{u+v} \le \norm{u}+\norm{v}.
  \end{gather*}
  Squaring the left hand side yields with the
  Bunyakovsky-Cauchy-Schwarz inequality
  \begin{multline*}
    \norm{u+v}^2
    = \scal(u+v,u+v)
    = \norm{u}^2 + 2 \scal(u,v) + \norm{v}^2
    \\
    \le \norm{u}^2 + 2 \norm u \norm v + \norm{v}^2
    = \bigl(\norm u + \norm v\bigr)^2.
  \end{multline*}
\end{proof}

\begin{Definition}{complete}
  A space $V$ with is \define{complete} with respect to a norm, if all
  \putindex{Cauchy sequence}s with elements in $V$ have their limit in
  $V$. A subspace $W\subset V$ is \define{closed} if it is complete in
  the topology of $V$.

  The \define{completion} of a space $V$ with respect to a norm
  consists of the space $V$ and the limits of all Cauchy sequences in
  $V$. We denote the completion of a space $V$ by
  \begin{gather}
    \label{eq:hilbert:8}
    \overline{V} = \overline{V}^{\norm{\cdot}_V}.
  \end{gather}
\end{Definition}

\begin{Definition}{Banach-hilbert}
  A \define{normed vector space} is a vector space $V$ with a norm
  $\norm\cdot$. We may also write $\norm{\cdot}_V$ to highlight the
  connection.

  A normed vector space $V$ which is complete with respect to its norm is
  called a \define{Banach space}.
  
  A vector space $V$ equipped with an inner product $\scal(.,.)$ is called
  an \define{inner product space} or \define{pre-Hilbert space}. A
  \define{Hilbert space} is a pre-Hilbert space which is also \putindex{complete}.
\end{Definition}

\begin{Definition}{orthogonal}
  Let $V$ be an inner product space over a field $\mathbb K$. Two
  vectors $x,y\in V$ are called \define{orthogonal} if $\scal(x,y) = 0$. We
  write $x\perp y$. Let $W$ be a subspace of $V$. We say that a vector $v$
  is orthogonal to the subspace $W$, if it is orthogonal to every vector in
  $W$.

  A set of nonzero mutually orthogonal vectors
  $\{x_i\} \subset V$ is called \define{orthogonal set}. If
  additionally $\norm{x_i} = 1$ for all vectors, it is called an
  \define{orthonormal set}. These notions transfer directly from
  finite to countable sets.
\end{Definition}

\begin{Definition}{orthogonal-complement}
  Let $W\subset V$ be a subspace of a Hilbert space $V$. We define its
  \define{orthogonal complement} $\ortho W\subset V$ by
  \begin{gather}
    \label{eq:infsup:7}
    \ortho W = \bigl\{v\in V \big| \scal(v,w)_{V} = 0
    \;\forall\,w\in W\bigr\}.
  \end{gather}
\end{Definition}

\begin{Lemma}{orthogonal-closed}
  The orthogonal complement $\ortho W$ of a subspace $W\subset V$
  is closed in the sense of ~\slideref{Definition}{complete}.
\end{Lemma}

\begin{proof}
  By the \putindex{Bunyakovsky-Cauchy-Schwarz inequality}, the inner
  product is continuous on $V\times V$. Therefore, the mapping
  \begin{align*}
    \phi_w\colon V &\to \R,\\
    v&\mapsto \scal(v,w),
  \end{align*}
  is continuous. For any $w\in W$, the kernel of $\phi_w$ is closed as
  the pre-image of the closed set $\{0\}$. Since
  \begin{gather*}
    \ortho W = \bigcap_{w\in W} \ker{\phi_w},
  \end{gather*}
  it is closed as the intersection of closed sets.
\end{proof}

\begin{Theorem}{orthogonal-complement}
  Let $W$ be a subspace of a Hilbert space $V$ and $W^\perp$ its
  orthogonal complement. Then, $W^\perp = \overline{W}^\perp$. Further,
  $V = W \oplus W^\perp$ if and only if $W$ is closed.
\end{Theorem}

\begin{proof}
  Clearly, $\overline{W}^\perp \subset W^\perp$ since
  $W\subset\overline{W}$. Let now $u\in W^\perp$. Then, $\phi =
  \scal(u,\cdot)$ is a continuous linear functional on $V$. Therefore,
  if a sequence $w_n \subset W$ converges to $w\in \overline{W}$, we
  have
  \begin{gather*}
    \scal(u,w) = \lim_{n\to\infty} \scal(u,w_n) = 0,
  \end{gather*}
  since $u \in W^\perp$. Hence, $u\in \overline{W}^\perp$ and
  $W^\perp = \overline{W}^\perp$.

  Now, the ``only if'' follows by the fact, that if $W$ is not
  closed, there is an element $w\in \overline{W}$ but not in $W$ such that
  $\scal(w,u)=0$ for all $u\in W^\perp$. Thus, $w\not\in W^\perp$ and
  consequently $w\not\in W^\perp \oplus W$.

  Let now $W$ be closed. We show that for all $v \in V$ there is a unique
  decomposition
  \begin{gather}
    \label{eq:infsup:8}
    v = w + u,\qquad \text{with} \qquad w\in W, \;u\in W^\perp.
  \end{gather}
  This is equivalent to $V = W \oplus W^\perp$. Uniqueness follows,
  since
  \begin{gather*}
    v = w_1+u_1 = w_2+u_2
  \end{gather*}
  implies that for any $y\in V$
  \begin{gather*}
    0 = \scal(w_1-w_2+u_1-u_2,y) = \scal(w_1-w_2,y) + \scal(u_1-u_2,y).
  \end{gather*}
  Choosing $y=u_1-u_2$ and $w_1-w_2$ in turns, we see that one of the
  inner products vanishes for orthogonality and the other implies that
  the difference is zero.

  If $v\in W$, we choose $w=v$ and $u=0$. For $v\not\in W$, we prove
  existence by considering that due to the closedness of $W$ there holds
  \begin{gather*}
    d=\inf_{w' \in W} \norm{v-w'} >0.
  \end{gather*}
  Let $w_n$ be a minimizing sequence. Using the parallelogram identity
  \begin{gather*}
    \norm{a+b}^2+\norm{a-b}^2 = 2\norm{a}^2+2\norm{b}^2,
  \end{gather*}
  we prove that $\{w_n\}$ is a Cauchy sequence by
  \begin{align*}
    \norm{w_m-w_n}^2 &= \norm{(v-w_n)-(v-w_m)}^2\\
    &= 2\norm{v-w_n}^2+2\norm{v-w_m}^2-\norm{2v-w_m-w_n}^2\\
    &= 2\norm{v-w_n}^2+2\norm{v-w_m}^2-4\norm*{v-\frac{w_m+w_n}2}^2\\
    &\le 2\norm{v-w_n}^2+2\norm{v-w_m}^2-4d^2,
  \end{align*}
  since $(w_m+w_n)/2\in W$ and $d$ is the infimum. Now we use the
  minimizing property to obtain
  \begin{gather*}
    \lim_{m,n\to\infty}\norm{w_m-w_n}^2 = 2d^2+2d^2 -4d^2=0.
  \end{gather*}
  Since $V$ is given as a Hilbert space and as such complete, $w=\lim w_n$
  exists and by the closedness of $W$, we have $w\in W$. Let $u=v-w$.
  By continuity of the norm, we have $\norm{u}=d$. It remains to show
  that $u\in W^\perp$. To this end, we introduce the variation
  $w+\epsilon \tilde w$ with $\tilde w \in W$ to obtain
  \begin{align*}
    d^2 &\le \norm{v-w-\epsilon \tilde w}^2\\
    &= \norm{u}^2-2\epsilon\scal(u,\tilde w)+\epsilon^2 \norm{\tilde w},
  \end{align*}
  implying for any $\epsilon>0$
  \begin{gather*}
    0\le-2\epsilon\scal(u,\tilde w)+\epsilon^2 \norm{\tilde w},
  \end{gather*}
  which requires $\scal(u,\tilde w) = 0$. Since $\tilde w \in W$ was chosen
  arbitrarily, we have $u \in W^\perp$.
\end{proof}

% \begin{Corollary}{ortho-density}
%   A subspace $W$ of a Hilbert space $V$ is dense in $V$ if and only if
%   $\ortho W = \{0\}$.
% \end{Corollary}

% \begin{proof}
%   The ``only if'' is an immediate application of
%   \slideref{Theorem}{orthogonal-complement}. For the opposite
%   direction, assume $\overline W \neq V$. Choose $v\in V$ such that
%   $v\not\in \overline W$. By
%   \slideref{Theorem}{orthogonal-complement}, there are unique elements
%   $w\in W$ and $u\in \ortho W$, such that $v=w+u$. In particular,
%   $u\neq 0$.
% \end{proof}

\begin{Definition}{ortho-projection}
  Let $W$ be a closed subspace of the Hilbert space $V$ and $\ortho W$
  be its orthogonal complement. Then, the
  \define{orthogonal projection} operators
  \begin{gather}
    \label{eq:lafa:9}
    \begin{split}
      \Pi_W &\colon V\to W\\
      \Pi_{\ortho W} &\colon V\to \ortho W\\
    \end{split}
  \end{gather}
  are defined by the unique decomposition
  \begin{gather}
    \label{eq:lafa:10}
    v = \Pi_W v + \Pi_{\ortho W} v.
  \end{gather}
\end{Definition}

\begin{Definition}{dual-space}
  A \define{linear functional} on a vector space $V$ is a
  \putindex{linear mapping} from $V$ to $\mathbb K$.

  The \define{dual space} $V^*$ of a vector space $V$, also called the
  \define{normed dual}, is the space of all bounded linear functionals
  on $V$ equipped with the norm
  \begin{gather}
    \norm{\phi}_{V^*} = \sup_{v\in V} \frac{\phi(v)}{\norm{v}_V}.
  \end{gather}
\end{Definition}

\begin{Theorem*}{Riesz-representation}{Riesz representation theorem}
  Let $V$ be a Hilbert space. Then, $V$ is isometrically isomorphic to
  $V^*$. In particular, there is an isomorphism
  \begin{gather}
    \label{eq:hilbert:1}
    \begin{split}
      \varrho\colon V & \to V^*, \\
      y & \mapsto f,
    \end{split}
  \end{gather}
  such that
  \begin{gather}
    \label{eq:hilbert:2}
    \begin{split}
    \scal(x,y) &= f(x) \qquad \forall x\in V,\\
    \norm{y}_V &= \norm{f}_{V^*}.
    \end{split}
  \end{gather}
  We refer to $\varrho$ as \define{Riesz isomorphism}.
\end{Theorem*}

\begin{proof}
  The proof is constructive and makes use of the orthogonal
  complement.

  First, it is clear that for any $y\in V$ a linear functional
  $f\in V^*$ is defined by $f(\cdot) = \scal(\cdot,y)$. Furthermore,
  $\varrho$ is injective, since
  \begin{gather*}
    \scal(x,y) = 0 \qquad\forall x\in V
  \end{gather*}
  implies $y\in \ortho V = \{0\}$. By the
  \putindex{Bunyakovsky-Cauchy-Schwarz inequality}, we have
  \begin{gather*}
    \norm{f}_{V^*} = \sup_{x\in V}\frac{\abs{f(x)}}{\norm{x}_V}
    = \sup_{x\in V}\frac{\abs{\scal(x,y)}}{\norm{x}_V}
      \le \norm{y}_V,
  \end{gather*}
  with equality for $x=y$.  It remains to show that $\varrho$ is
  surjective. To this end, let $f\in V^*$ be arbitrary and let
  $N = \ker f$. If $N=V$, we choose $y=0$. If not, choose
  $\ortho y \in \ortho N$ and let
  \begin{gather}
    \label{eq:lafa:13}
    y = \frac{f(\ortho y)}{\norm*{\ortho y}^2} \ortho y \in
    \ortho N,
  \end{gather}
  such that $f(y) = \abs*{f(\ortho y)}^2/\norm*{\ortho y}^2 \neq 0$.
  Let now $x\in V$ be chosen arbitrarily. Then, there holds
  \begin{gather*}
    x = \left(x-\frac{f(x)}{f(y)} y\right)
    + \frac{f(x)}{f(y)} y,
    \\
  \end{gather*}
  where $\frac{f(x)}{f(y)}$ denotes a scalar.
  Since
  \begin{gather*}
      f \left(x-\frac{f(x)}{f(y)} y\right) 
      = \left(f(x)-f(x) \frac{f(y)}{f(y)}\right)
      = 0,
  \end{gather*}
  this decomposition amounts to $x = x^0+\ortho x$ with $x^0\in N$ and
  $\ortho x \in \ortho N$. It is unique according to
  \slideref{Definition}{ortho-projection}. Thus, we have that $\ortho x$ is a
  multiple of $y$, say $\ortho x=\alpha y$ with $\alpha=\frac{f(x)}{f(y)}$ and thus
  \begin{gather*}
    \begin{aligned}
    f(x) &= f(x^0) + f(\ortho x)
    &=& \alpha f(y)
    &=& \alpha \frac{\abs*{f(\ortho y)}^2}{\norm*{\ortho y}^2}
    \\
    \scal(x,y) &= \scal(x^0,y)  + \scal(\ortho x,y)
    &=& \alpha \norm{y}_V^2
    &=& \alpha \frac{\abs*{f(\ortho y)}^2}{\norm*{\ortho y}^4}
    \norm*{\ortho y}^2
    \end{aligned}
  \end{gather*}
  Hence, the two terms are equal and $\varrho$ is surjective.
\end{proof}

\begin{Definition}{bilinear-form}
  A \define{bilinear form} $a(.,.)$ on a Hilbert space $V$ is a
  mapping $a\colon V\times V \to \R$, which is linear in both
  arguments. The bilinear form is \textbf{bounded}, if there is a
  constant $M$ such that
  \begin{gather}
    a(u,v) \le M \norm{u}_V \norm{v}_V, \qquad \forall u,v\in V.
  \end{gather}
  It is called \define{coercive} or \define{elliptic}, if there is a
  constant $\alpha$ such that
  \begin{gather}
    a(u,u) \ge \alpha \norm{u}_V^2 \qquad\forall u\in V.
  \end{gather}
\end{Definition}

\begin{Lemma}{pde-bilinear}
  Let $a_{ij} \in C^1(\domain)$, $b_i, c\in C^0(\domain)$. A solution
  to the Dirichlet problem
  \begin{gather}
    \label{eq:hilbert:div-eq}
    \begin{aligned}
      -\sum_{i,j=1}^d \d_i \bigl(a_{ij} \d_j u\bigr)
      + \sum_{i=1}^d \bigl(b_i \d_i u\bigr) + c u &= f
      & \qquad\text{in }&\domain\\
      u &= 0
      & \qquad\text{on }&\d\domain
    \end{aligned}
  \end{gather}
  solves the weak problem: find $u\in V_0$ such that for all
  $v\in V_0$
  \begin{gather}
    \label{eq:hilbert:weak}
    a(u,v) \equiv \form(\mathbf A \nabla u,\nabla v)
    + \form(\mathbf b\cdot\nabla u,v)
    +\form(cu,v) = \form(f,v),
  \end{gather}
  where $\mathbf A(\vx) = \bigl(a_{ij}(\vx)\bigr)$ is the matrix of
  coefficients of the second order term and
  $\mathbf b(\vx) = \bigl(b_{i}(\vx)\bigr)$ is the vector of
  coefficients of the first order term.

  If additionally $u\in C^2(\domain)$ holds, then the solution to the
  weak problem solves the Dirichlet problem in differential form.
\end{Lemma}

\begin{Lemma*}{lax-milgram}{Lax-Milgram}
  Let $a(.,.)$ be a bounded, coercive bilinear form on a Hilbert space
  $V$ and let $f \in V^*$. Then, there is a unique element $u\in V$
  such that
  \begin{gather}
    \label{eq:hilbert:lax-milgram}
    a(u,v) = f(v) \qquad\forall v\in V.
  \end{gather}
  Furthermore, there holds
  \begin{gather}
    \label{eq:hilbert:lax-milgram-estimate}
    \norm{u}_V \le \frac1\alpha \norm{f}_{V^*}.
  \end{gather}
\end{Lemma*}

\begin{proof}
  To prove Lax-Milgram we first consider uniqueness and then the
  existence of a solution.

  Assume that there are solutions $u_1, u_2\in V$ of \eqref{eq:hilbert:lax-milgram},
  i.\,e. there holds $a(u_1,v)=f(v)$ and $a(u_2,v)=f(v)$ for all $v\in V$.
  Thus, $a(u_1-u_2,v)=0$ for all $v\in V$.
  Now choose $v=u_1-u_2\in V$.
  Since $a(.,.)$ is coercive with $\alpha>0$ there holds
  \begin{gather*}
    0 = a(u_1-u_2,u_1-u_2)\geq \alpha \norm{u_1-u_2}_{V}^2 
  \end{gather*}
  which implies $u_1-u_2=0$. Hence $u_1=u_2$.

  Let us now consider the existence of a solution.
  We will define a linear functional to apply Riesz representation theorem
  and Banach fixed point theorem.
  For all $y\in V$ define
  \begin{gather*}
    \scal(y,\cdot) - \omega\left[a(y,\cdot) -f(\cdot)\right] \in V^*
  \end{gather*}
  with $\omega>0$.
  Due to Riesz representation theorem there exists an isomorphism $\rho:V^*\rightarrow V$
  that maps a given $\scal(y,\cdot) - \omega \left[ a(y,\cdot)-f(\cdot) \right]$
  to $z\in V$ such that
  \begin{gather*}
    \scal(v,z) = \scal(y,v) - \omega\left[a(y,v) - f(v) \right] \qquad \forall v\in V.
  \end{gather*}
  Now we define the operator $T_\omega:V\rightarrow V$ that maps $y\mapsto z$ and
  define $A:V\rightarrow V$ such that
  \begin{gather*}
    \scal(Au,v) = a(u,v) \qquad \forall v\in V 
  \end{gather*}
  with $\norm{Au}_V \leq M \norm{u}_V$ for $M>0$.
  This leads for all $v\in V$ to
  \begin{gather*}
    \scal(T_\omega y,v)
    = \scal(y,v) - \omega\left[a(y,v) \right]
    = \scal(y,v) - \omega \scal(Ay,v).
  \end{gather*}
  Thus, we can conclude that $T_\omega y=y-\omega Ay$.
  Applying the norm and using the fact that it is induced by the inner product of $V$
  we get
  \begin{gather*}
    \begin{aligned}
      \norm{T_\omega y - T_\omega x}_V^2 &= \norm{y-x}_V^2 - 2 \omega \scal(A(y-x),y-x) + \omega^2 \norm{A(y-x)}^2 \\
      &\leq \norm{y-x}^2 - 2\omega a(y-x,y-x) + \omega^2 M^2 \norm{y-x}^2 \\
      &\leq (1-2\omega\alpha + \omega^2 M^2) \norm{y-x}^2.
    \end{aligned}
  \end{gather*}
  As we want to apply Banach fixed point theorem, we need $T_\omega$ to be a contraction.
  Therefore, we need $1-2\omega\alpha + \omega^2 M^2 < 1$.
  Hence, if we choose $\omega\in\left(0,\frac{2\alpha}{M^2}\right)$, then $T_\omega$ is a
  contraction and there exists a $u\in V$ such that $T_\omega u = u$
  and $\scal(T_\omega u,u) = \scal(u,u) - \omega\left[ a(u,v) - f(v) \right]$
  which implies $a(u,v)=f(v)$.
  Thus, there exists a solution $u$.

  Now the stability estimate \eqref{eq:hilbert:lax-milgram-estimate} is left to prove.
  Using the coercivity of our bilinear form yields
  \begin{gather*}
    \alpha \norm{u}_V \leq \frac{a(u,u)}{\norm{u}_V} = \frac{f(u)}{\norm{u}_V} \leq \sup_{u\in V} \frac{\snorm{f(u)}}{\norm{u}_V} = \norm{f}_{V^*}.
  \end{gather*}
\end{proof}

\begin{Lemma}{divergence-equality}
  For $b_i \in C^2(\domain)$ and $u\in H^1_0(\domain)$ there holds
  \begin{align}
    (\vb\cdot\nabla u,u)_{L^2} = -\frac12 (\nabla \cdot\vb,u^2)_{L^2}.
  \end{align}
\end{Lemma}

\begin{Problem}{divergence-equality}
  Prove \slideref{Lemma}{divergence-equality}.
\end{Problem}

%% \begin{proof}
%%   Starting with the right hand side, integration by parts yields
%%   \begin{gather*}
%%     -\frac12 (\nabla\cdot \vb,u^2)_{L^2} = -\frac12 \int_\domain \nabla\cdot\vb~ u^2 \dx
%%     = \frac12 \int_\domain \vb\cdot\nabla(u^2)\dx - \frac12 \int_{\d\domain} \vb~ u^2\dx
%%   \end{gather*}
%%   where the boundary term vanishes as our solution space is $H^1_0(\domain)$.
%%   Then, there holds
%%   \begin{gather*}
%%     \begin{split}
%%       -\frac12 (\nabla\cdot \vb,u^2)_{L^2} &= \frac12 \int_\domain \vb\cdot\nabla(u^2)\dx
%%       = \frac12 \int_\domain \vb\cdot\nabla u~ 2u\dx
%%       \\
%%       &= \int_\domain \vb\cdot\nabla u u\dx = (\vb\cdot\nabla u,u)_{L^2}.
%%     \end{split}
%%   \end{gather*}
%% \end{proof}

\begin{Lemma}{weak-well-posed}
  Let $a_{ij}, c\in L^\infty(\domain)$, $b_i \in C^1(\overline{\domain})$ such
  that there holds for a positive constant $\alpha$
  \begin{gather}
    \label{eq:hilbert:elliptic}
    \begin{split}
    \alpha\abs{\xi}^2 &\le \xi^T \mathbf A(\vx) \xi,
    \qquad \forall \xi \in \R^d,
    \\
    0 &\le c - \frac12 \nabla\cdot \vb.
    \end{split}
  \end{gather}
  Then, the associated bilinear form is coercive and bounded on
  $H^1_0(\domain)$, and thus the weak formulation has a unique
  solution.
\end{Lemma}

\begin{proof}
  We will prove boundedness of the individual terms and coercivity
  for the whole bilinear form.
  
  We start with the boundedness of the $\mathbf A$-term.
  Due to the Bunyakovsky-Cauchy-Schwarz inequality there holds
  \begin{align*}
    (\mathbf A(\vx) \nabla u,\nabla v)_{L^2} &= \int_\domain \mathbf A(\vx) \nabla u \cdot \nabla v \dvx
    \leq \norm{\mathbf A(\vx) \nabla u}_{L^2} \norm{\nabla v}_{L^2}
    \intertext{and as $a_{ij}\in L^\infty(\domain)$ there holds for $M:=\max_{ij} \norm{a_{ij}(\vx)}_\infty>0$}
    &\leq \max_{ij} \norm{a_{ij}(\vx)}_\infty \norm{\nabla u}_{L^2} \norm{\nabla v}_{L^2} = M \abs{u}_1 \abs{v}_1.
  \end{align*}
  
  Let us now consider boundedness for the $\vb$-term.
  For the standard dot product there holds
  $\vx\cdot \vy\leq \abs{\vx}\abs{\vy}$, which yields
  \begin{align*}
    \int_\domain \vb(\vx)\cdot\nabla u v \dvx \le \int_\domain \abs{\vb(\vx)}\abs{\nabla u}v\dvx.
  \end{align*}
  Now we can use $b_i\in C^2(\overline{\domain})$ and the Bunyakovsky-Cauchy-Schwarz
  inequality which leads to
  \begin{align*}
    (\vb(\vx)\cdot\nabla u,v)_{L^2}&\le \int_\domain\abs{\vb(\vx)}\abs{\nabla u}v\dvx \\
    &\le \max_i \norm{b_i(\vx)}_\infty \norm{\nabla u}_{L^2} \norm{v}_{L^2}  \\
    &\le \max_i \norm{b_i(\vx)}_\infty \abs{u}_1 \abs{v}_1.
  \end{align*}
  In the last step we used Friedrich's inequality for $v\in H^1_0$.
  
  Now, we prove boundedness for the $c$-term.
  Using the Bunyakovsky-Cauchy-Schwarz and Friedrich's inequality and
  $M:=\lambda(\domain)^2 \max_{\vx\in\domain}\abs{c(\vx)}>0$, there holds
  \begin{align*}
    (c(\vx)u,v)_{L^2} = \int_\domain c(\vx)uv\dvx \le \abs{c(\vx)} \norm{u}_{L^2} \norm{v}_{L^2} \le M \abs{u}_1 \abs{v}_1.
  \end{align*}
  Hence, our bilinear form is bounded.

  There is only the coercivity of our bilinear
  form left to prove.
  As $c\in L^\infty(\domain)$ and $b_i\in C^1(\overline{\domain})$ the
  expression $C\coloneqq \min_{\vx\in\domain}(c(\vx)-\frac12 \nabla\cdot \vb(\vx))$ is
  well-defined and $C>0$ by assumption.
  Now consider the whole bilinear form, where we already used
  \slideref{Lemma}{divergence-equality}:
  \begin{align*}
    a(u,u) &= (\mathbf A(\vx)\nabla u,\nabla u)_{L^2} - \frac12 (\nabla \cdot \vb(\vx) u ,u)_{L^2} + (c(\vx)u,u)_{L^2} \\
    &\ge \alpha \int_\domain \abs{\nabla u}^2 \d\vx + \int_\domain (c(\vx)-\frac12 \nabla \cdot \vb(\vx))\abs{u}^2 \dvx \\
    &\ge \alpha \int_\domain \abs{\nabla u}^2 \dvx + C \int_\domain \abs{u}^2 \dvx \\
    &\ge \underbrace{\min\{\alpha,C \}}_{\ge 0}\norm{u}_1^2 \ge \min\{\alpha,C \} \abs{u}_1^2
  \end{align*}
  Thus, the bilinear form is coercive.
\end{proof}

\begin{Definition}{elliptic}
  A differential equation of second order in divergence
  form~\eqref{eq:hilbert:div-eq} with associated coercive
  and bounded bilinear form is called \define{elliptic}.
  The lower bound $\alpha$ is the ellipticity constant.
\end{Definition}

%%% Local Variables: 
%%% mode: latex
%%% TeX-master: "main"
%%% End: 


%\section{Fast Facts on Sobolev Spaces} 

% \begin{intro}
%   While this section reviews some of the basic mathematical properties
%   of Sobolev spaces, it suffers a bit from overly abstract
%   mathematical arguments. While the strongly mathematically inclined
%   reader might appreciate this, it is not really necessary for the
%   remainder of this lecture, where we only need the basic results.
% \end{intro}
%\begin{Notation}{multi-index}
  For a multi-index $\alpha = (\alpha_1,\dots,\alpha_d)$ with
  nonnegative integer $\alpha_i$ and a function with sufficient
  differentiability, we define the derivative
  \begin{gather*}
    \d^\alpha f = \d_1^{\alpha_1}\cdots\d_d^{\alpha_d} f.
  \end{gather*}
  The order of $\d^\alpha$ is
  \begin{gather*}
    \abs{\alpha} = \sum \alpha_i.
  \end{gather*}
\end{Notation}

\begin{Definition}{distributional-derivative}
  If for a given function $u$ there exists a function $w$ such that
  \begin{gather}
    \label{eq:hk:1}
     \int_\Omega w \phi \dx
     =
     -\int_\Omega u \partial_i \phi \dx,
     \qquad\forall \phi\in \coo(\domain),
  \end{gather}
  then we define $\partial_i u := w$ as the \define{distributional
    derivative} (partial) of $u$ with respect to $x_i$. Here,
  $\coo(\domain)$ is the space of all functions in $C^\infty(\domain)$
  with compact support in $\domain$.

  Similarly through integration by parts, we define distributional
  directional derivatives, distributional gradients, $\d^\alpha u$,
  etc.

  We call a distributional derivative \define{weak derivative} in
  $L^p$ if it is a function in this space.
\end{Definition}

\begin{remark}
  Formula~\eqref{eq:hk:1} is the usual integration by
  parts. Therefore, whenever $u\in\co^1$ in a neighborhood of $x$, the
  distributional derivative and the usual derivative coincide.
\end{remark}

\begin{example}
  Let $\Omega=\R$ and $u(x) = |x|$. Intuitively,
  it is clear that the distributional derivative, if it exists, must
  be the \define{Heaviside function}
  \begin{gather}
    \label{eq:hk:2}
    w(x) =
    \begin{cases}
      -1 & x<0 \\ 1 & x>0.
    \end{cases}
  \end{gather}
  The proof that this is actually the distributional derivative is
  left to the reader.
\end{example}

\begin{example}
  For the derivative of the \putindex{Heaviside function}
  in~\eqref{eq:hk:2}, we first observe that it must be zero whenever
  $x\neq 0$, since the function is continuously differentiable
  there. Now, we take a test function $\phi\in\co^\infty$ with support
  in the interval $(-\epsilon,\epsilon)$ for some positive
  $\epsilon$. Let $w'(x)$ be the derivative of $w$. Then, by
  integration by parts
  \begin{gather*}
    \int_{-\epsilon}^\epsilon w(x) \phi'(x)\dx
    = -\int_{-\epsilon}^0 w(x)' \phi(x)\dx
    -\int_0^\epsilon w(x)' \phi(x)\dx
    + 2 \phi(0) = 2\phi(0),
  \end{gather*}
  since $w'(x) = 0$ under both integrals. Thus, $w'(x)$ is an object
  which is zero everywhere except at zero, but its integral against a
  test function $\phi$ is nonzero. This contradicts our notion, that
  integrable functions can be changed on a set of measure zero without
  changing the integral. Indeed, $w'$ is not a function in the usual
  sense, and we write $w'(x) = 2 \delta(x)$, where $\delta(x)$ is the
  \define{Dirac $\delta$-distribution}, which is defined by the two
  conditions
  \begin{gather*}
    \begin{alignedat}{2}
      \delta(x) &= 0, & \forall x & \neq 0
      \\
      \int_\R \delta(x) \phi(x)\dx &= \phi(0), \quad & \forall \phi
      &\in \co^0(\R).
    \end{alignedat}
  \end{gather*}
  We stress that $\delta$ is not an integrable function, or a function
  at all.
\end{example}

\begin{Definition}{Wkp}
  The \define{Sobolev space} $W^{k,p}(\domain)$ is the space
  \begin{gather}
    W^{k,p}(\domain) = \bigl\{
    u\in L^p(\domain) \big|
    \d^\alpha u\in L^p(\domain) \forall \abs{\alpha} \le k
    \bigr\},
  \end{gather}
  where the derivatives are understood in weak sense. Its norm is
  defined by
  \begin{gather}
    \norm{v}_{k,p}^p = \norm{v}_{k,p;\domain}^p
    = \sum_{\abs{\alpha} \le k} \norm{\d^\alpha v}_{L^p(\domain)}^p.
  \end{gather}
  The following seminorm will be useful:
  \begin{gather}
    \snorm{v}_{k,p}^p = \snorm{v}_{k,p;\domain}^p
    = \sum_{\abs{\alpha} = k} \norm{\d^\alpha v}_{L^p(\domain)}^p.
  \end{gather}
\end{Definition}

\begin{Notation}{zero-norm}
  We will use the notation
  \begin{gather*}
    \norm{v}_0 = \norm{v}_{0;\domain} = \norm{v}_{L^2(\domain)}.
  \end{gather*}
  Accordingly, $W^{0,p}(\domain) = L^p(\domain)$.
\end{Notation}

\begin{Corollary}{wkp-embedding}
  There holds
  \begin{gather}
    W^{k,p}(\domain) \subset W^{k-1,p}(\domain) \subset \dots \subset
%    W^{1,p}(\domain) \subset
    W^{0,p}(\domain) = L^{p}(\domain)
  \end{gather}
\end{Corollary}

\begin{Definition}{hkp}
  The \define{Sobolev space} $H^{k,p}(\domain)$ is the completion of
  $C^\infty(\domain)$ with respect to the norm $\norm{\cdot}_{k,p}^p$.

  In the case $p=2$, we write $H^k(\domain) = H^{k,2}(\domain)$.
\end{Definition}

\begin{Theorem*}{meyers-serrin}{Meyers-Serrin}
  \begin{gather*}
    H^{k,p}(\domain)\cong W^{k,p}(\domain)
  \end{gather*}
\end{Theorem*}

\begin{example}
  Functions, which are in $W^{k,p}(\domain)$ or not.
  \begin{enumerate}
  \item The function $x/\abs{x}$ is in $H^1(B_1(0))$ if $d=3$, but not
    if $d=2$.
  \end{enumerate}
\end{example}

\begin{Definition}{boundary-smoothness}
  A bounded domain $\domain\subset\R^d$ is said to have $C^k$-boundary
  or to be a $C^k$-domain, if there is a finite covering $\{U_i\}$ of
  its boundary $\d\domain$, such that for each $U_i$ there is
  a mapping $\Phi_i \in C^k(U_i)$ with the following properties:
  \begin{gather}
    \begin{split}
      \Phi_i(\d\domain \cap U_i)
      &\subset \left\{ \vx\in\R^d \mid x_1 = 0 \right\},\\
      \Phi_i(\domain \cap U_i)
      &\subset \left\{ \vx\in\R^d \mid x_1 > 0 \right\}.      
    \end{split}
  \end{gather}
  The domain is called Lipschitz, if such a construction exists with
  Lipschitz-continuous mappings.
\end{Definition}

\begin{Definition}{continuous-embedding}
  We say that a normed vector space $U \subset V$ is
  \define{continuously embedded} in another space $V$, in symbolic
  language
  \begin{gather}
    U \hookrightarrow V,
  \end{gather}
  if the inclusion mapping $U \ni x \mapsto x\in V$ is continuous, that is, there is a constant $c$ such that
  \begin{gather}
    \norm{x}_V \le c \norm{x}_U.
  \end{gather}
  If the spaces $U$ and $V$ consist of equivalence classes, the
  inclusion may involve choosing representatives on the left or on the
  right.
\end{Definition}

\begin{Theorem}{sobolev-embedding}
  Let $\domain\subset\R^d$ be a bounded Lipschitz domain. For the space
  $W^{k,p}(\domain)$ define the number
  \begin{gather}
    s = k-\tfrac dp.
  \end{gather} Assume $k_1 \le k_2$ and $p_1,p_2\in [1,\infty)$.
  Then, if $s_1 \ge s_2$, we have the continuous embedding
  \begin{gather}
    W^{k_1,p_1}(\domain) \hookrightarrow  W^{k_2,p_2}(\domain).
  \end{gather}
\end{Theorem}

\begin{Lemma}{trace-continuous}
  Let $\domain$ be a bounded Lipschitz domain in $\R^d$. Then, there
  exists a constant $c$ only depending on $\domain$, such that every
  function $u\in H^1(\domain) \cap C^1(\overline{\domain})$ admits the
  estimate
  \begin{gather}
    \norm{u}_{L^p(\d\domain)} \le c \norm{u}_{W^{1,p}(\domain)}.
  \end{gather}
\end{Lemma}

\begin{Theorem*}{trace}{Trace theorem}
  Let $\domain$ be a bounded Lipschitz domain in $\R^d$. Then, every
  function $u\in W^{1,p}(\domain)$ has a well defined trace
  $\gamma u \in L^p(\d\domain)$ and there holds
  \begin{gather}
    \norm{\gamma u}_{L^p(\d\domain)} \le c \norm{u}_{W^{1,p}(\domain)},
  \end{gather}
  with the same constant as in the previous lemma. We simply write
  \begin{gather}
    u_{|\d\domain} = \gamma u.
  \end{gather}
\end{Theorem*}

\begin{remark}
  The trace theorem guarantees that the imposition of Dirichlet
  boundary conditions on Sobolev functions is a reasonable
  operation. In particular, it ensures that $H^1(\domain)$ and
  $H^1_0(\domain)$ are indeed different spaces. The same does not
  hold, if we complete $C^1(\domain)$ and $C^1_0(\domain)$ in
  $L^2(\domain)$.

  Remarkably, we set out defining $W^{1,p}(\domain)$ as a subset of
  $L^p(\domain)$, which consists of functions ``defined up to a set of
  measure zero''. Now it turns out, that functions in
  $W^{1,p}(\domain)$ can have well-defined values on certain sets of
  measure zero. From the point of view of subsets of $L^p(\domain)$,
  this is always to be understood by choosing representatives of the
  equivalence class. This is also the reason why we write
  ``$\hookrightarrow$'' instead of ``$\subset$''.
\end{remark}

\begin{Definition}{hoelder-spaces}
  A function $f \in C^0(\domain)$ is Hölder-continuous with exponent
  $\gamma \in (0,1]$, if there is a constant $C_f$ such that
  \begin{gather}
    \abs{f(\vx)-f(\vy)} \le C_f \abs{\vx-\vy}^\gamma
    \qquad\forall \vx,\vy\in\domain.
  \end{gather}
  In particular, for $\gamma = 1$, we obtain Lipschitz-continuity.

  We define the Hölder space $C^{k,\gamma}$ of $k$-times continuouly
  differentiable functions such that all derivatives of order $k$ are
  Hölder-continuous. The norm is
  \begin{gather}
    \norm{u}_{C^{k,\gamma}(\domain)} = \max_{\abs{\alpha}\le k} \sup_{\vx,\vy\in\domain}
    \frac{\abs{\d^\alpha u(\vx)-\d^\alpha u(\vy)}}{\abs{\vx-\vy}^\gamma}
  \end{gather}
\end{Definition}

\begin{Theorem}{hoelder-embedding}
  Let $\domain\subset\R^d$ be a bounded Lipschitz domain.
  If $s = k-\tfrac dp > j+\gamma$, then every function in $W^{k,p}$ has a
  representative in $C^{j,\gamma}$. We write
  \begin{gather}
    W^{k,p}(\domain) \hookrightarrow C^{j,\gamma}(\domain).
  \end{gather}
\end{Theorem}

\begin{Corollary}{sobolev-continuous}
  Elements of Sobolev spaces are continuous if the derivative
  order is sufficiently high. In particular,
  \begin{gather}
    \begin{aligned}
      H^1(\domain) & \hookrightarrow C(\domain) & d&= 1, \\
      H^2(\domain) & \hookrightarrow C(\domain) & d&= 2,3.
    \end{aligned}
  \end{gather}
\end{Corollary}

\begin{Lemma}{weak-well-posed}
  Let $a_{ij},c\in L^\infty(\domain), b_i\in C^1(\overline{\domain})$ such
  that there holds for a positive constant $\alpha$
  \begin{gather}
    \label{eq:hilbert:elliptic}
    \begin{split}
    \alpha\abs{\xi}^2 &\le \xi^T \mathbf A(\vx) \xi,
    \qquad \forall \xi \in \R^d,
    \\
    0 &\le c - \frac12 \nabla\cdot \vb.
    \end{split}
  \end{gather}
  Then, the associated bilinear form is coercive and bounded on
  $H^1_0(\domain)$, and thus the weak formulation has a unique
  solution.
\end{Lemma}

\begin{proof}
  In this proof we do not consider boundedness for the term containig
  $\mathbf A$ as it is exactly the same as in
  \slideref{Lemma}{semi-weak-well-posed}.

  Let us first consider boundedness for the $\vb$-term.
  For the standard dot product there holds
  $\vx\cdot \vy\leq \abs{\vx}\abs{\vy}$, which yields
  \begin{align*}
    \int_\domain \vb\cdot\nabla u v \dx \le \int_\domain \abs{\vb}\abs{\nabla u}v\dx.
  \end{align*}
  Now we can use Hölder's inequality which leads to
  \begin{align*}
    \abs{(\vb\cdot\nabla u,v)_{L^2}}&\le \Big| \int_\domain\abs{\vb}\abs{\nabla u}v\dx \Big| \\
    &\le \Big(\int_\domain \abs{\vb}^4\dx \Big)^{\frac14} \Big(\int_\domain \abs{\nabla u}^2\dx \Big)^{\frac12}
    \Big(\int_\domain \abs{v}^4 \dx\Big)^{\frac14} \\
    &= \norm{\vb}_{L^4} \norm{\nabla u}_{L^2} \norm{v}_{L^4}.
  \end{align*}
  Due to the Sobolev embedding theorem, holds
  \begin{align*}
    H^1 = W^{1,2} \hookrightarrow W^{0,4} = L^4
  \end{align*}
  which gives us $\norm{\vb}_{L^4},\norm{v}_{L^4}<\infty$.
  Additionally, this yields for $w\in H^1$ and $c\ge 0$ that
  \begin{align*}
    \norm{w}_{L^4} \le c \norm{w}_1.
  \end{align*}
  Hence, we obtain
  \begin{align*}
    (\vb\cdot\nabla u,v)_{L^2} \le c \norm{\vb}_{L^4} \abs{u}_1 \norm{v}_1
    \le c \norm{\vb}_{L^4} \abs{u}_1 \abs{v}_1.
  \end{align*}
  In the last step we used Friedrich's inequality for $v\in H^1_0$.
  
  Now, we prove boundedness for the $c$-term.
  Using the Bunyakovsky-Cauchy-Schwarz and Friedrich's inequality and
  $M:=\lambda(\domain)^2 \max_{x\in\domain}\abs{c(x)}>0$, there holds
  \begin{align*}
    (cu,v)_{L^2} = \int_\domain cuv\dx \le \abs{c} \norm{u}_{L^2} \norm{v}_{L^2} \le M \abs{u}_1 \abs{v}_1.
  \end{align*}
  Hence, our bilinear form is bounded.

  There is only the coercivity of our bilinear
  form left to prove.
  Let us first prove the identitiy
  \begin{gather*}
    (\vb\cdot\nabla u,u)_{L^2} = -\frac12 (\nabla\cdot \vb,u^2)_{L^2}.
  \end{gather*}
  Starting with the right hand side, integration by parts yields
  \begin{gather*}
    -\frac12 (\nabla\cdot \vb,u^2)_{L^2} = -\frac12 \int_\domain \nabla\cdot\vb~ u^2 \dx
    = \frac12 \int_\domain \vb\cdot\nabla(u^2)\dx - \frac12 \int_{\d\domain} \vb~ u^2\dx
  \end{gather*}
  where the boundary term vanishes as our solution space is $H^1_0(\domain)$.
  Then, there holds
  \begin{gather*}
    \begin{split}
      -\frac12 (\nabla\cdot \vb,u^2)_{L^2} &= \frac12 \int_\domain \vb\cdot\nabla(u^2)\dx
      = \frac12 \int_\domain \vb\cdot\nabla u~ 2u\dx
      \\
      &= \int_\domain \vb\cdot\nabla u u\dx = (\vb\cdot\nabla u,u)_{L^2}.
    \end{split}
  \end{gather*}
  As $c\in L^\infty(\domain)$ and $b_i\in C^1(\overline{\domain})$ the
  expression $C\coloneqq \min(c-\frac12 \nabla\cdot \vb)$ is
  well-defined and $C>0$ by assumption.
  Now consider the whole bilinear form
  \begin{align*}
    a(u,u) &= (\mathbf A\nabla u,\nabla u)_{L^2} - \frac12 (\nabla \cdot \vb~ u ,u)_{L^2} + (cu,u)_{L^2} \\
    &\ge \alpha \int_\domain \abs{\nabla u}^2 \dx + \int_\domain (c-\frac12 \nabla \cdot \vb)\abs{u}^2 \dx \\
    &\ge \alpha \int_\domain \abs{\nabla u}^2 \dx + C \int_\domain \abs{u}^2 \dx \\
    &\ge \underbrace{\min\{\alpha,C \}}_{\ge 0}\norm{u}_1^2 \ge \min\{\alpha,C \} \abs{u}_1^2.
  \end{align*}
  Thus, the bilinear form is coercive.
\end{proof}



%%% Local Variables: 
%%% mode: latex
%%% TeX-master: "main"
%%% End: 


% \section{Regularity of Weak Solutions}
% \begin{intro}
  So far, we have proven existence and uniqueness of weak
  solutions. We have seen, that these solutions may not even be
  continuous, far from differentiable. In this section, we collect a
  few results from the analysis of elliptic pde which establish higher
  regularity under stronger conditions.
\end{intro}

\begin{Definition}{wkp-loc}
  The space $W^{k,p}_{\text{loc}}(\domain)$ consists of functions $u$
  such that $u\in W^{k,p}(\domain_1)$ for any
  $\domain_1 \subset\subset \domain$, where the latter reads compactly
  embedded, namely $\overline\domain_1\subset\domain$. Similarly, we
  define $H^k_{\text{loc}}$.
\end{Definition}

\begin{Theorem*}{gt-8-8}{\cite[Theorem 8.8]{GilbargTrudinger98}}
  Let $a_{ij} \in C^{0,1}(\overline\domain)$ and
  $b_i, c \in L^\infty(\domain)$. If $u\in H^1(\domain)$ is a solution
  to the elliptic equation and $f\in L^2(\domain)$, then $u\in H^2_{\text{loc}}(\domain)$.
\end{Theorem*}

\begin{Theorem*}{gt-8-10}{Interior regularity}
  Let $a_{ij} \in C^{k,1}(\overline\domain)$ and
  $b_i, c \in C^{k-1,1}(\overline\domain)$. If $u\in H^1(\domain)$ is
  a solution to the elliptic equation and $f\in W^{k,2}(\domain)$,
  then $u\in H^{k+2}_{\text{loc}}(\domain)$.
\end{Theorem*}

\begin{proof}
  \cite[Theorem 8.10]{GilbargTrudinger98}
\end{proof}

\begin{Corollary}{gt-8-10}
  If in the interior regularity theorem $d=2,3$, then $u$ is a
  classical solution of the PDE if $k\ge 2$.

  If $a_{ij}, b_i, c \in C^{\infty}(\overline\domain)$, then
  $u\in C^\infty(\domain)$.
\end{Corollary}

\begin{Theorem*}{gt-8-13}{Global regularity}
  If in addition to the assumptions of the interior regularity theorem
  $\domain$ is a $C^{k+2}$-domain, then the solution
  $u\in H^1_0(\domain)$ to the homogeneous Dirichlet boundary value problem 
  problem is in $H^{k+2}(\domain)$.
\end{Theorem*}

\begin{proof}
  \cite[Theorem 8.13]{GilbargTrudinger98}
\end{proof}

\begin{Corollary}{h2-solution-bvp}
  Let $\domain\subset \R^d$ with $d=2,3$ be a $C^2$-domain,
  $a_{ij} \in C^{0,1}(\overline\domain)$ and
  $b_i, c \in L^\infty(\domain)$. If $u\in H^1_0(\domain)$ is a solution
  to the elliptic equation and $f\in L^2(\domain)$, then
  $u\in H^2(\domain)$.
\end{Corollary}

\begin{Remark}{classical-smooth}
  In order to guarantee a classical solution by these arguments, we
  must require that $\domain$ has $C^4$ boundary.
\end{Remark}

\begin{Remark}{classical-convex}
  The condition $\d\domain\in C^2$ in the previous corollary can be
  replaced by the assumption that $\domain$ is convex.
\end{Remark}

\begin{Theorem*}{kondratev}{Kondratev}
  Let the assumptions of the interior regularity theorem
  \slideref{Theorem}{gt-8-8} hold. Assume further that $\d\domain$ is
  piecewise $C^2$ with finitely many irregular points. Then, the
  solution $u\in H^1_0(\domain)$ of the elliptic PDE admits a
  representation
  \begin{gather}
    u = u_0 + \sum_{i=1}^n u_i,
  \end{gather}
  where $u_0\in H^2(\domain)$ and $u_i$ is a singularity function
  associated with the irregular point $\vx_i$.
\end{Theorem*}

\begin{proof}
  \cite{Kondratev67}
\end{proof}

%%% Local Variables: 
%%% mode: latex
%%% TeX-master: "main"
%%% End: 


\setcounter{chapter}{3}
\chapter{Conforming Finite Element Methods}
\setcounter{section}{2}
\begin{Definition}{mesh}
  A \define{mesh} $\mesh$ is a nonoverlapping subdivision of the
  domain $\domain$ into polyhedral \define{cell}s denoted by $\cell$,
  for instance simplices, quadrilaterals, or hexahedra. The
  $d-1$-dimensional facets of a cell are denoted by $\face$, the
  vertices by $\vertex$. Cells are typically considered open sets.

  A mesh $\mesh$ is called regular, if each facet
  $\face \subset \d\cell$ of the cell $\cell\in\mesh$ is either a
  facet of another cell $\cell\prime$, that is,
  $\overline{\face} = \overline{\cell} \cap \overline{\cell\prime}$,
  or a subset of $\d\domain$.
\end{Definition}

\begin{Lemma}{mesh-continuity}
  Let $\mesh$ be a subdivision of $\domain$, and let $u$ be a function on $\domain$, such that $u_{|\cell} \in C^1(\cell)$. Then,
  \begin{gather}
    u\in H^1(\domain)
    \quad \Longleftrightarrow\quad
    u\in C(\overline\domain).
  \end{gather}
\end{Lemma}

\begin{Definition}{barycentric-coordinates}
  A simplex $\cell\in \R^d$ with vertices $\vertex_0,\dots,\vertex_d$
  is described by a set of $d+1$ \define{barycentric coordinates}
  $\vlambda = (\lambda_0,\dots,\lambda_d)^T$ such that
  \begin{xalignat}2
    0\le\lambda_i &\le 1& i&=0,\dots,d;\\
    \lambda_i(\vertex_j) &= \delta_{ij}& i,j&=0,\dots,d\\
    \sum \lambda_i(\vx) &= 1,
  \end{xalignat}
  and there holds
  \begin{gather}
    T = \Bigl\{x\in\R^d \Big| x = \sum \vertex_k\lambda_k \Bigr\}.
  \end{gather}
\end{Definition}

\begin{Lemma}{barycentric-affine}
  There is a matrix $B_T\in \R^{d+1\times d}$ and a vector
  $b_T\in\R^{d+1}$, such that
  \begin{gather}
    \vlambda = B_T\vx + b_T.
  \end{gather}
\end{Lemma}

\begin{Corollary}{barycentric-interpolation}
  The barycentric coordinates $\lambda_0,\dots,\lambda_d$ are the
  linear Lagrange interpolating functions for the points
  $\vertex_0,\dots,\vertex_d$. In particular, $\lambda_k \equiv 0$ on
  the facet not containing $\vertex_k$.
\end{Corollary}

\begin{example}
  We can use barycentric coordinates to define interpolating polynomials on
  simplicial meshes easily, as in
  Table~\ref{tab:barycentric-shapes}.
  \begin{table}[tp]
    \centering
    \begin{tabular}{|c|l|}
      \hline Degrees of freedom
      & Shape functions \\\hline
      \adjustbox{valign=center,margin=3pt}{\includegraphics[width=2cm]{mixed/fig/p1-p.tikz}}
      &
        {\begin{minipage}[b]{6cm}
          \begin{gather*}
            \phi_i = \lambda_i,
            \quad i=0,1,2
          \end{gather*}
        \end{minipage}}
      \\\hline
      \adjustbox{valign=center,margin=3pt}{\includegraphics[width=2cm]{mixed/fig/p2-p.tikz}}
      &
        {\begin{minipage}[b]{6cm}
          \begin{xalignat*}2
            \phi_{ii} &= 2\lambda_i^2 - \lambda_i,
            &i&=0,1,2\\
            \phi_{ij} &= 4\lambda_i\lambda_j
            &j&\neq i
          \end{xalignat*}
        \end{minipage}}
        \\\hline
      \adjustbox{valign=center,margin=3pt}{\includegraphics[width=2cm]{mixed/fig/p3-p.tikz}}
      &
        {\begin{minipage}[b]{6cm}
          \begin{xalignat*}2
          \phi_{iii} &= \tfrac12 \lambda_i(3\lambda_i-1)(3\lambda_i-2)
          &i&=0,1,2\\
          \phi_{ij} &= \tfrac92\lambda_i\lambda_j(3\lambda_j-1)
          &j&\neq i\\
          \phi_0 &= 27\lambda_0\lambda_1\lambda_2
        \end{xalignat*}
        \end{minipage}}
        \\\hline
    \end{tabular}
    \caption{Degrees of freedom and shape functions of simplicial elements
      in terms of barycentric coordinates}
    \label{tab:barycentric-shapes}
  \end{table}
\end{example}

\begin{remark}
  The functions $\lambda_i(x)$ are the shape functions of the linear
  $P_1$ element on $T$. They allow us to define basis functions on the
  cell $T$ without use of a reference element $\widehat T$.

  Note that $\lambda_i\equiv 0$ on the face opposite to the
  vertex $x_i$.
\end{remark}



%%% Local Variables: 
%%% mode: latex
%%% TeX-master: "main"
%%% End: 


\section{Eigenvalues of finite element matrices}

\begin{intro}
  The linear systems of equations obtained by finite element methods
  on fine meshes are large and sparse. Hence, the application of
  iterative solvers is advisable. For symmetric, positive definite
  matrices like the matrix of the Laplace operator with Dirichlet
  boundary conditions, the conjugate gradient method is the method of
  choice. It has a convergence estimate of the form
  \begin{gather}
    \label{eq:cg:1}
    \norm{\vecx^{(k)} - \vecx}_A \le 2
    \left(\frac{\sqrt\kappa-1}{\sqrt\kappa+1}\right)^k \norm{\vecx^{(0)} - \vecx}_A
    =
    2\left(1-\frac{2}{\sqrt\kappa+1}\right)^k \norm{\vecx^{(0)} - \vecx}_A
  \end{gather}
  Here, $\kappa = \Lambda/\lambda$ is the \putindex{spectral condition
    number} of the matrix $\mata$ defined as the quotient of greatest
  and smallest eigenvalues. Hence, we will estimate these eigenvalues
  in order to make predictions on the performance of the solver.
\end{intro}

\begin{Lemma}{condition-number-mass-matrix}
  \label{lemma:itintro:1}
  Let $\{\phi_i\}$ be the basis of a finite element shape function space
  on a quasi-uniform mesh of mesh size $h$. Let
  $\matm$, the so called \define{mass matrix} be the matrix associated
  with the $L^2$-inner product with entries
  \begin{gather*}
    m_{ij} = \int_\Omega \phi_i(x) \phi_j(x) \dx.
  \end{gather*}
  Then,
  \begin{gather*}
    \Lambda(\matm) \simeq h^{d} \simeq  \lambda(\matm)
  \end{gather*}
  Therefore, the condition number is
  \begin{gather*}
    \kappa(\matm) = \frac{\mathcal O(h^{d})}{\mathcal O(h^{d})} = {\mathcal O(1)}.
  \end{gather*}
\end{Lemma}

\begin{proof}
  % It is easy to verify, that $m_{ii}> 0$, and that there are not more
  % entries in each row as edges of the triangulation meet in one vertex
  % are different from zero. Furthermore, that the size of those entries
  % is of order $h^d$, where $d$ is the space dimension. From these two
  % facts we immediately obtain
  % \begin{gather*}
  %   \Lambda(\matm) = \mathcal O(h^{d}).
  % \end{gather*}
  % The argument for $\lambda(\matm)$ is more subtle. 
  For any mesh cell
  $T$, let $\vecx_T$ be the entries of the vector $\vecx$ which
  belong to node values of the cell $T$. Let $\matm_T$ be the cell
  mass matrix obtained by restricting the $L^2$-inner product to
  $T$. Then,
  \begin{gather*}
    \vecx^T \matm \vecx
    = \sum_{T\in\T_h} \vecx^T_T \matm_T \vecx_T
    \begin{cases}
    \ge \min\limits_{T\in\T_h} \frac{\vecx^T_T \matm_T \vecx_T}{\abs{\vecx_T}} \sum_{T\in\T_h} \abs{\vecx_T}^2 \ge \lambda(\matm_T) \abs{\vecx}^2,\\
    \le \max\limits_{T\in\T_h} \frac{\vecx^T_T \matm_T \vecx_T}{\abs{\vecx_T}} \sum_{T\in\T_h} \abs{\vecx_T}^2 \le c \Lambda(\matm_T) \abs{\vecx}^2,
    \end{cases}
  \end{gather*}
  where the constant in the upper bound is due to degrees of freedom
  shared by different elements. Dividing by $\abs{\vecx}^2$, we
  obtain
  \begin{gather}
    \lambda(\matm_\cell) \le \frac{\vecx^T \matm \vecx}{\abs{\vecx}^2} \le c \Lambda(\matm_\cell).
  \end{gather}
  In order to estimate the eigenvalues of $\matm_T$, we note that for
  a unisolvent element, the norms $\abs{\vecx_T}$ and $\norm{u}_{0,T}$ are
  equivalent on the reference cell, and the $L^2$-norm scales with
  $h^d$ when transforming to the real cell $T$. Thus, we have
  $\lambda(\matm) = \mathcal O(h^{d}) = \Lambda(\matm)$.
\end{proof}

\begin{Corollary}{refined-condition-number}
  Let $\mesh_h$ be a shape-regular mesh with cell sizes ranging
  between the minimum $h$ and the maximum $H$. Then, we have
  \begin{align*}
    \Lambda(\matm) &= \mathcal O(H^{d}) \\
    \lambda(\matm) &= \mathcal O(h^{d}) \\
    \kappa(\matm) &= \mathcal O\left(\left(\frac Hh\right)^{d}\right)
  \end{align*}
\end{Corollary}



\end{document}


\section{A priori error analysis}
\begin{intro}
  In this section, we develop error estimates of the following type
  \begin{verse}
    If the size of mesh cells converges to zero, then the difference
    between the true solution and the finite element solution
    converges to zero with a certain order.
  \end{verse}
  They are thus a prediction, that the solutions actually converge,
  and they measure the asymptotic convergence rate. They do contain unknown constants, such that they are no prediction of the error on a given mesh.
\end{intro}

\subsection{Approximation of Sobolev spaces by finite elements}

\begin{Lemma*}{poincare}{Poincaré inequality}
  Let $\domain$ be a bounded Lipschitz domain. For any function
  $u\in H^1(\domain)$ define
  \begin{gather}
    \overline u = \frac1{\abs{\domain}} \int_\domain u(\vx) \dvx,
  \end{gather}
  where $\abs{\domain}$ denotes the measure of $\domain$. There exists
  a constant $c$ depending on the domain only, such that each of the
  following inequalities hold:
  \begin{align}
    \norm{u-\overline u}_{L^2(\domain)} &\le c \norm{\nabla u}_{L^2(\domain)}\\
    \norm{u}_{L^2(\domain)}^2 &\le c \Bigl(\norm{\nabla u}_{L^2(\domain)}^2 + \overline u^2\Bigr)
  \end{align}
\end{Lemma*}

\begin{proof}
  The proof exceeds the tools we have developed in this class. The
  proof in \cite[Section 7.8]{GilbargTrudinger98} seems elementary and
  direct, but is technical and requires star-shaped domains. The proof
  in \cite[Section 5.8.1]{Evans98} is more elegant, but it uses
  compact embedding and is indirect, such that the constant cannot be
  determined.
\end{proof}

\begin{Lemma*}{bramble-hilbert}{Bramble-Hilbert}
  Let $\cell\subset \R^d$ be a domain with Lipschitz boundary and let
  $s(.)$ be a bounded sublinear functional on $H^{k+1}(\cell)$ with
  the property
  \begin{gather}
    s(p) = 0 \qquad\forall p\in \P_k.
  \end{gather}
  Then, there exists a constant $c$ only dependent on $\cell$ such that
  \begin{gather}
    \abs{s(v)} \le c \snorm{v}_{k+1,\cell}.
  \end{gather}
\end{Lemma*}

\begin{proof}
  Since $s(\cdot)$ is sublinear and vanishes on $\P_k$, we have for
  $v\in H^{k+1}(\cell)$:
  \begin{gather}
    \label{eq:bramble-hilbert-1}
    \abs{s(v)} \le \abs{s(v+p)} + \abs{s(p)} = \abs{s(v+p)}
    \qquad\forall p\in\P_k.
  \end{gather}
  We will construct a polynomial, such that
  \begin{gather}
    \label{eq:bramble-hilbert-2}
    \overline{\d^\alpha(v+p)}
    = \frac1{\abs{\cell}}\int_\cell \d^\alpha (v+p) \dx = 0
    \qquad\forall \abs{\alpha} \le k,
  \end{gather}
  that is, the sum $v+p$ and all its derivatives up to order $k$ are
  mean-value free. Thus, by recursive application of Poincaré
  inequality, we get for $\abs{\alpha}\le k$
  \begin{alignat*}2
    \norm{v+p}_{L^2(\cell)}^2
    &\le c \left[ \norm{\nabla(v+p)}_{L^2(\cell)}^2 + \overline{v+p}^2\right]
      && \le c \snorm{v+p}_{1;\cell}^2\\
    \norm{\d^\alpha(v+p)}_{L^2(\cell)}^2,
    &\le c \left[ \norm{\nabla\d^\alpha(v+p)}_{L^2(\cell)}^2
      + \overline{\d^\alpha(v+p)}^2\right]
      && \le c \snorm{v+p}_{\abs{\alpha}+1;\cell}^2,
  \end{alignat*}
  such that $\norm{v+p}_{k+1;\cell} \le \snorm{v+p}_{k+1;\cell}$.
  Furthermore, since $p\in\P_k$
  \begin{gather*}
    \norm{\d^\alpha(v+p)}_{L^2(\cell)} = \norm{\d^\alpha v}_{L^2(\cell)}
    \qquad\forall \abs{\alpha} = k+1.
  \end{gather*}
  Combining with~\eqref{eq:bramble-hilbert-1}, we obtain
  \begin{gather*}
    \abs{s(v)} \le c \snorm{v+p}_{k+1;\cell} \le c \snorm{v}_{k+1;\cell}.
  \end{gather*}
  It remains to construct the polynomial $p$ with the desired
  properties. To this end, we note that for two multi-indices $\alpha$
  and $\beta$ holds that $\d^\alpha \vx^\beta =0$ as soon as
  $\alpha_i>\beta_i$ for some index $i$. Let
  \begin{gather*}
    p(\vx) = \sum_{\abs{\beta}\le k} a_\beta \vx^\beta.
  \end{gather*}
  Then, for any $\abs{\alpha} = k$ we get
  \begin{gather*}
    \d^\alpha p(\vx) = \alpha! \,\delta_{\alpha\beta},
  \end{gather*}
  where $\alpha! = \alpha_1!\alpha_2!\dots\alpha_d!$. Thus, we can use
  condition~\eqref{eq:bramble-hilbert-2} to fix the coefficients
  $a_\beta$ to
  \begin{gather*}
    a_\beta = \frac1{\beta!\,\abs{\cell}} \int_\cell \d^\beta v\dx
    \qquad \abs{\beta} = k.
  \end{gather*}
  Thus, we have decomposed $p = \tilde p_k + p_{k-1}$, where
  $\tilde p_k$ is known and $p_{k-1}\in \P_{k-1}$. Thus, we can repeat
  the process determining the coefficients of highest order in
  $p_{k-1}$,
  \begin{gather*}
    a_\beta = \frac1{\beta!\,\abs{\cell}} \int_\cell \d^\beta (v-\tilde p_k)\dx
    \qquad \abs{\beta} = k-1.
  \end{gather*}
  Recursion down to $k=0$ yields the polynomial $p$ with the desired property.
\end{proof}

\begin{Corollary}{b-h-projector}
  Let $\Pi\colon H^{k+1}(\cell) \to \P_k$ be a continuous, linear
  projector. For any $m\le k$ there exists a constant $c$ such that
  \begin{gather}
    \norm{u-\Pi u}_{m,\cell} \le c \snorm{u}_{k+1,\cell}.
  \end{gather}
\end{Corollary}

\begin{Definition}{mesh-family}
  Let $\{\mesh_h\}$ for $h>0$ be a family of meshes parametrized by
  the parameter
  \begin{gather}
    h = \max_{\cell\in\mesh_h} h_\cell,
  \end{gather}
  where $h_\cell$ is the characteristic length from the discussion of
  mappings, for instance the diameter of $\cell$. Such a family is
  called \define{shape regular}, if the constants $M_\cell$,
  $m_\cell$, $d_\cell$, and $D_\cell$ in the scaling lemma
  can be chosen independent of the cell $\cell\in\mesh_h$ and of
  $h>0$. The family is called \define{quasi-uniform}, if in addition
  there is a positive constant independent of $h$ such that
  \begin{gather}
    h \le c \min_{\cell\in\mesh_h} h_\cell.
  \end{gather}
\end{Definition}

\begin{Definition}{nodal-interpolation}
  Let $V$ be a function space on $\domain$, and let
  $V_h = V_{\mesh_h}$ be a finite element space on the mesh $\mesh_h$
  on $\domain$ with node functionals $\nodal_i$ and dual basis $p_i$,
  where $i=1,\dots,n_h$. We define the \define{nodal interpolation}
  operator by
  \begin{gather}
    \begin{split}
      I_h\colon V&\to V_h\\
      v &\mapsto \sum \nodal_i(v)p_i.
    \end{split}
  \end{gather}
\end{Definition}

\begin{Lemma}{nodal-interpolation}
  The nodal interpolation operator $I_h$ is a projector. It is
  continuous on $H^2(\domain)$ if the dimension is $d=2,3$ and the
  node functionals are defined as Lagrange interpolation.
\end{Lemma}

\begin{Definition}{broken-sobolev-norm}
  On a mesh $\mesh_h$, we define the \define{broken Sobolev norm} and
  seminorm by
  \begin{gather}
    \begin{split}
      \norm{u}_{k;h}^2 &= \sum_{\cell\in\mesh_h} \norm{u}_{k;\cell}^2\\
      \snorm{u}_{k;h}^2 &= \sum_{\cell\in\mesh_h} \snorm{u}_{k;\cell}^2
    \end{split}
  \end{gather}
\end{Definition}

\begin{Theorem}{fe-interpolation}
  Let $\{\mesh_h\}$ be a shape regular family of meshes.
  Let the finite element spaces $V_h = V_{\mesh_h}$. Let the nodal
  interpolation operator $I_h$ be surjective onto $\P_k$ on every cell
  $\cell\in\mesh_h$ and continuous on $H^{k+1}(\domain)$. Then, there
  is a constant $c$ such that for any $u\in H^{k+1}(\domain)$ and
  $m\le k+1$ there holds
  \begin{gather}
    \norm{u-I_h u}_{m;h} \le c h^{k+1-m} \snorm{u}_{k+1;h}.
  \end{gather}
\end{Theorem}

\begin{proof}
  We have by definition
  \begin{gather*}
    \snorm{u-I_h u}_{m;h} = \sum_{\cell\in\mesh_h} \snorm{u-I_h u}_{m;\cell}.
  \end{gather*}
  Using the scaling lemma, we get
  \begin{gather*}
    \snorm{u-I_h u}_{m;\cell}
    \le c h_\cell^{\nicefrac d2-m} \snorm{\refu - \widehat{I_hu}}.
  \end{gather*}
  On the reference cell, we use the Bramble-Hilbert lemma, more
  precisely, \slideref{Corollary}{b-h-projector} to obtain
  \begin{gather*}
    \snorm{\refu - \widehat{I_hu}} \le c \snorm{\refu}_{k+1;\refcell}.
  \end{gather*}
  Scaling back yields
  \begin{gather*}
    \snorm{\refu}_{k+1;\refcell}
    \le c h_\cell^{k+1-\nicefrac d2} \snorm{u}_{k+1;\cell}.
  \end{gather*}
  Combining, we obtain
  \begin{gather*}
    \snorm{u-I_h u}_{m;\cell} \le c h_\cell^{k+1-\nicefrac d2 - m +
      \nicefrac d2}\snorm{u}_{k+1;\cell}.
  \end{gather*}
  Summing up and pulling the maximum of $h_\cell^{k+1-m}$ out of the
  sum yields the result for $h_\cell \le 1$.
\end{proof}

\begin{remark}
  Strictly speaking, we have only proven the result for $h_T \le
  1$.
  But then, if $\diam\domain = 1$, this condition is always
  true. Therefore, by rescaling the domain before computing, the
  estimate holds in general.
\end{remark}

\begin{Corollary}{fe-approximation}
  Let $a(.,.)$ be a bounded and elliptic bilinear form on
  $V\subset H^1(\domain)$ and let the
  assumptions of the \slideref{Theorem}{fe-interpolation} hold. If
  furthermore the solution $u\in H^{k+1}(\domain)$, the error between
  the exact and the finite element solution to a uniquely solvable
  elliptic boundary value problem, admits the estimate
  \begin{gather}
    \norm{u-u_h}_{1;h} \le c h^{k} \snorm{u}_{k+1;h}.
  \end{gather}  
\end{Corollary}

\begin{intro}
  For 2nd order elliptic problems, we have now derived estimates of
  the $H^1$-norm of the error under the assumption that the solution
  exhibits further regularity. Now, let us drop this assumptionfor an
  assumption on the boundary condition only, to obtain a qualitative
  convergence result.
\end{intro}

\begin{Theorem}{fe-convergence}
  Let $\domain\subset \R^d$ be polyhedral. Let the solution space
  $V\subset H^1(\domain)$ be such that $H^2(\domain)$ is dense in V
  with respect to the $H^1$-norm. Let $\{\mesh_h\}$ be a quasi-uniform
  family of meshes with finite element spaces $V_h\subset V$. Let
  $u\in V$ and $u_h\in V_h$ be solutions to the exact and the finite
  element versions of a 2nd order boundary value problem. Then,
  \begin{gather}
    \lim_{h\searrow 0} \norm{u-u_h}_{1;\domain} = 0.
  \end{gather}
\end{Theorem}

\subsection{Estimates of stronger norms}

\begin{intro}
  So far, we have seen error estimates in a ``natural norm'' defined
  as a norm such that the Lax-Milgram lemma holds for a given bilinear
  form $a(.,.)$. In this subsection, we now consider the question of
  estimates in stronger norms, such that the bilinear form is not
  elliptic with respect to this norm.
\end{intro}

\begin{Definition}{stronger-norm}
  Let $\norm{\cdot}_X$ and $\norm{\cdot}_Y$ be norms on a vector space
  $V$. We call $\norm{\cdot}_X$ a \define{stronger norm} than
  $\norm{\cdot}_Y$, if there is a constant $c$ such that for all
  $v\in V$:
  \begin{gather}
    \norm{v}_Y \le \norm{v}_X.
  \end{gather}
  In this case, $\norm{\cdot}_Y$ is called the \define{weaker
    norm}. If a converse inequality holds, the norms are called
  \define{equivalent}.
\end{Definition}

\begin{example}
  For a bounded domain $\domain$, the norms $\norm{\cdot}_{k+1}$ and
  $\norm{\cdot}_{k}$ are both defined on $V = H^1_0(\domain)$. By the
  Sobolev embedding theorem, there is a constant $c$ such that for any
  $v\in V$
  \begin{gather*}
    \norm{v}_{k} \le c \norm{v}_{k+1}
  \end{gather*}
\end{example}

\begin{Lemma*}{inverse-estimate}{Inverse estimate}
  Let $\cell$ be a mesh cell of size $h_\cell$. Then, there is a
  constant only depending on $k$ and the constants of the scaling
  lemma, such that for every $u\in \P_k$ there holds
  \begin{gather}
    \norm{u}_{1;\cell} \le c h_T^{-1} \norm{u}_{0;\cell}.
  \end{gather}
\end{Lemma*}



\subsection{Estimates of weaker norms and linear functionals}

%%% Local Variables: 
%%% mode: latex
%%% TeX-master: "main"
%%% End: 

\section{A posteriori error analysis}
\begin{Definition}{a-posteriori}
  Let $u\in V$ be the solution to a boundary value in weak form and
  $u_h\in V_h$ be its finite element approximation on the mesh
  $\mesh_h$. We call a quantity $\eta_h(u_h)$ \define{a posteriori
    error estimator},
  \begin{gather}
    \norm{u-u_h} \le c \eta_h(u_h).
  \end{gather}
  The estimator is \define{reliable}, if the constant $c$ is
  computable. It is \define{efficient}, if the converse estimate holds,
  namely
%  with addition of a quantity $\osc_h(f)$
  \begin{gather}
    \eta_h(u_h) \le c \norm{u-u_h}.
  \end{gather}
\end{Definition}

\subsection{Quasi-interpolation in $H^1$}

\begin{intro}
  Interpolation in Sobolev spaces in
  \slideref{Theorem}{fe-interpolation} relies on the nodal
  interpolation operator in
  \slideref{Definition}{nodal-interpolation}, which in turn requires
  point values of the interpolated function. Therefore, it is not
  defined on $H^1(\domain)$ in dimensions greater than one. Since we
  need such interpolation operators in the analysis of a posteriori
  error estimates, we provide them in this section.
\end{intro}

\begin{Definition}{locally-quasi-uniform}
  A shape-regular family of meshes $\{\mesh_h\}$ is called
  \define{locally quasi-uniform}, if there is a constant $c$ such that
  for every pair of cells $\cell_1$ and $\cell_2$ sharing at least one
  vertex there holds
  \begin{gather}
    h_{\cell_1} \le c h_{\cell_2}.
  \end{gather}
\end{Definition}

\begin{Definition}{fem-neighborhood}
  For a vertex or higher-dimensional boundary facet $F$, we define the
  set of cells
  \begin{gather}
    \mesh_{F} = \bigl\{ \cell\in\mesh \big|F \subset\d\cell \bigr\}.
  \end{gather}
  Similarly, the set of cells sharing at least one vertex with $\cell$
  is called $\mesh_{\cell}$. Additionally, we define the subdomains
  \begin{gather}
    \overline\domain_{F} = \bigcup_{\cell\in \mesh_{F}}\overline\cell,
    \qquad
    \overline\domain_{\cell} = \bigcup_{\cell'\in \mesh_{\cell}}\overline\cell'.
  \end{gather}
\end{Definition}

\begin{Theorem*}{clement}{Clément quasi-interpolation}
  Let $\{\mesh_h\}$ be a locally quasi-uniform family of meshes with
  piecewise polynomial finite element spaces
  $V_h\subset H^1(\domain)$. Then, there exist bounded operators
  $\overline I_h\colon H^1(\domain) \to V_h$ such that for every
  function $u\in H^1(\domain)$, every mesh cell $\cell$, every face
  $F$, and for $m=0,1$ there holds
  \begin{align}
    \norm{u-\overline I_h u}_{m;T} &\le c h_\cell^{1-m} \snorm{u}_{1;\domain_\cell} \\
    \norm{u-\overline I_h u}_{0;F} &\le c h_\cell^{\nicefrac{1}{2}} \snorm{u}_{1;\domain_\cell}
  \end{align}
\end{Theorem*}


%%% Local Variables: 
%%% mode: latex
%%% TeX-master: "main"
%%% End: 


\chapter{Variational Crimes}
\begin{intro}
  In the previous chapter, we considered finite element methods
  applying the original bilinear form to a subspace $V_h\subset
  V$. This assumes, that the domain $\domain$ is exactly represented
  by the mesh, and that all integrals are computed exactly. Both
  assumptions reduce the applicability of the finite element method
  considerably. Therefore, we now extend our analysis to cases, where
  we allow $V_h \not\subset V$ and discrete bilinear forms
  $a_h(.,.) \neq a(.,.)$.
\end{intro}

\section{Numerical quadrature}
\begin{intro}
  We begin the investigation of variational crimes by studying the
  effect of using numerical quadrature instead of exact integration on
  mesh cells. In particular, we investigate approximations of the form
  \begin{gather}
    \int_\cell f(\vx)\dvx
    \approx \sum_{k=1}^{n_q} \omega_k f(\vx_k) =: Q_\cell(f).
  \end{gather}
  First, we observe that $Q_\cell$ is not a bounded operator on
  $L^1(\domain)$, such that $Q_\cell(\nabla u\cdot \nabla v)$ is
  undefined for functions in $H^1(\domain)$. The surprising result of
  this section is, that quadrature is still admissible for the
  implementation of a finite element method.

  We will first set a theoretical framework and then investigate
  quadrature rules in detail. The presentation follows~\cite[Chapter
  4]{Ciarlet78}.
\end{intro}

\begin{Lemma*}{strang-1}{Strang's first lemma}
  Let $a(.,.)$ be a bounded and elliptic bilinear form on the Hilbert
  space $V$. Let $V_n\subset V$ and let $a_h(.,.)$ be a bilinear form,
  bounded and elliptic on $V_n$ with constants $M_n$ and
  $\alpha_n$. Let $f, f_n\in V^*$. If $u \in V$ and
  $u_n\in V_n \subset V$ are solutions to
  \begin{gather*}
    \begin{aligned}
      a(u,v) &= f(v) & \qquad\forall v&\in V,\\
      a_n(u_n,v_n) &= f_n(v_n) & \qquad\forall v_n&\in V_n,
    \end{aligned}
  \end{gather*}
  respectively, there holds
  \begin{multline}
    \norm{u-u_n}_V \\
    \le \inf_{v_n\in V_n}\biggl[\left(1+\frac{M}{\alpha_n}\right)
    \norm{u-v_n}_V
    + \frac1{\alpha_n}
    \norm{a_n(v_n,.)-a(v_n,.)}_{V_h^*}\biggr]\\
         + \frac1{\alpha_n}\norm{f_n-f}_{V_h^*}.
  \end{multline}
\end{Lemma*}

\begin{proof}
  Using $V_n$-ellipticity of $a_n(.,.)$ yields for arbitrary $v_n\in V_n$
  \begin{align*}
    \alpha_n \norm{u_n-v_n}^2
    \le &a_n(u_n-v_n,u_n-v_n)\\
    =& f_n(u_n-v_n) + a(u-v_n, u_n-v_n)
       - f(u_n-v_n)
    \\ &+ a(v_n,u_n-v_n) - a_n(v_n, u_n-v_n)
  \end{align*}
  We estimate separately
  \begin{align*}
    \frac{a(u-v_n, u_n-v_n)}{\norm{u_n-v_n}}
    &\le M \norm{u_n-v_n},\\
    \frac{\abs{f_n(u_n-v_n) - f(u_n-v_n)}}{\norm{u_n-v_n}}
    &\le \sup_{w_n\in V_n} \frac{\abs{f_n(w_n)-f(w_n)}}{\norm{w_n}},
    \\
    \frac{\abs{a(v_n,u_n-v_n) - a_n(v_n, u_n-v_n)}}{\norm{u_n-v_n}}
    &\le \sup_{w_n\in V_n} \frac{\abs{a_n(v_n, w_n)-a(v_n,w_n)}}{\norm{w_n}}.
  \end{align*}
  Combining all terms, we obtain
  \begin{multline}
    \norm{u-u_n}
    \le \norm{u-v_n} + \norm{u_n-v_n} \\
    \le \norm{u-v_n} + \frac1{\alpha_n}\left(
         M \norm{u-v_n}
         + \norm{a_n(v_n,.)-a(v_n,.)}_{V_h^*}
         + \norm{f_n-f}_{V_h^*}\right).
  \end{multline}
\end{proof}

\begin{remark}
  Strang's lemma states, that the error can be split into the
  \putindex{approximation error} of the space $V_h$ and the
  \putindex{consistency error} of the approximations $a_h(.,.)$ und
  $f_h(.)$. Both errors are scaled with the stability factor
  $\nicefrac1{\alpha _n}$ of the discrete problem. So far, this is
  consistent with error estimates for instance for Runge-Kutta
  methods. There is an important difference though: the consistency
  error is not evaluated for the exact solution, but only for its
  discrete approximation.
\end{remark}

\begin{remark}
  We will apply Strang's lemma to a family of meshes indexed by mesh
  size $h$ and assess the infimum by an interpolation operator. It is
  clear, that we will only obtain optimal convergence rates compared
  to the interpolation estimate, if there exists $\alpha_0 >0$ such
  that $\alpha_n \ge \alpha_0$ uniformly with respect to $n$. While it
  is not a prerequisite of Strang's lemma, it is our goal for all
  discretizations.
\end{remark}

\begin{remark}
  Quadrature would be infeasible, if we had to devise a quadrature
  rule for every mesh cell $\cell$. Instead, we tabulate quadrature
  formulas for the reference cell $\refcell$ by choosing quadrature
  points $\refvx_k$ and weights $\omega_k$ and write
  \begin{gather}
    Q_{\refcell}(\reference f)
    = \sum_{k=1}^{n_q} \omega_k \reference f(\refvx_k).
  \end{gather}
  We compute integrals over $\cell$ though mapping,
  \begin{gather}
    Q_T(f) = \sum_{k=1}^{n_q} \det\nabla\Phi_\cell(\refvx_k) \omega_k \reference f(\refvx_k).
  \end{gather}
  Thus, $Q_T$ is defined by quadrature points
  $\vx_k = \Phi(\refvx_k)$ and quadrature weights
  $\det\nabla\Phi_\cell(\refvx_k) \omega_k$.

  Quadrature rules on the reference cell $\refcell$ are obtained by
  interpolation after choosing quadrature points by employing the
  properties of orthogonal polynomials.  The construction of such
  quadrature rules for simplices is somewhat complicated, and we refer
  to tables in the cited literature.

  An important consequence of the use of quadrature points as roots of
  orthogonal polynomials is the fact that all weights are positive.
\end{remark}

\begin{Definition}{tensor-product-quadrature}
  Given a one-dimensional quadrature rule $Q_I$ on the interval
  $I=[0,1]$ with
  \begin{gather}
    Q_I(\reference f) = \sum_{k=1}^{n_q} \omega_k \reference f(\refx_k).
  \end{gather}
  Then, a quadrature rule on $\refcell = [0,1]^d$ is defined by
  \begin{gather}
    Q_{\refcell}(\reference f)
    = \sum_{k_1}^{n_q}\dots\sum_{k_d}^{n_q} \omega_{k_1}\cdots\omega_{k_d}
    \reference f(x_{k_1},\dots,x_{k_d}).
  \end{gather}
\end{Definition}

\begin{Lemma}{tensor-product-gauss}
  If $Q_I$ is an $n$-point Gauß formula, the tensor product quadrature
  in the preceding definition is exact on $\Q_{2n-1}$.
\end{Lemma}

\begin{remark}
  When we look at Poisson's equation on simplicial meshes, the integrals
  \begin{gather}
    a_\cell(\phi_j,\phi_i)
    = \int_\cell \nabla\phi_j\cdot\nabla\phi_i\dvx
    = \int_{\refcell} \nabla\Phi^{-T}\refgrad\refphi_j
    \cdot\nabla\Phi^{-T}\refphi_i \det\nabla\Phi\dvxref,
  \end{gather}
  can be computed exactly, since $\nabla\Phi$ is a constant
  matrix. Already on arbitrary quadrialterals, $\nabla\Phi^{-T}$ is a
  rational function, wich is harder to integrate. But, in order to
  justify the use and investigate the properties of approximations by
  numerical quadrature, we are considering a model problem with
  varying coefficients.
\end{remark}

\begin{Assumption}{bilinear-form-quadrature}
  Let in this section the bilinear form be defined as
  \begin{gather}
    a(u,v) = \int_\domain \nabla u A \nabla v^T \dx,
  \end{gather}
  where $A= A(\vx)$ is a matrix with
  \begin{gather}
    \xi^T A(\vx) \xi \ge \alpha \abs{\xi^2}
    \qquad\forall \xi\in\R^d,\quad \forall \vx\in\domain.
  \end{gather}
  The function $u\in V=H^1_0(\domain)$ is the solution to
  \begin{gather}
    a(u,v) = \form(f,v) \qquad\forall v\in V.
  \end{gather}
  We assume that the coefficients $a_{ij}$ and the right hand side $f$
  are functions which are well-defined in all quadrature points on all
  meshes of a family $\{\mesh_h\}$.
\end{Assumption}

\begin{Definition}{bilinear-quadrature}
  The bilinear form $a_h(.,.)$ obtained by numerical quadrature
  $Q_{\refcell}$ on the mesh $\mesh_h$ is defined as
  \begin{gather}
    a_h(u,v) = \sum_{\cell\in\mesh_h} Q_\cell (\nabla u A \nabla v^T),
  \end{gather}
  where $Q_\cell$ is obtained from $Q_{\refcell}$ by the mapping
  $\Phi_\cell$. The right hand side is
  \begin{gather}
    f_h(v) = \sum_{\cell\in\mesh_h} Q_\cell (fv).
  \end{gather}
\end{Definition}

\begin{Lemma}{quadrature-stability}
  Let a finite element method be defined by a mesh $\mesh_h$ and a
  shape function space $\shapespace$ on the reference cell $\refcell$.
  Let $Q_{\refcell}$ be a quadrature formula, such that all quadrature
  weights are positive. Furthermore, assume that one of the following
  holds:
  \begin{enumerate}
  \item The function values in the quadrature points $\refvx_k$ are
    unisolvent on
    \begin{gather}
      \mathscr G_1
      = \bigl\{ \d_i p\big| p\in \shapespace, i=1,\dots,d\bigr\},
    \end{gather}
  \item The quadrature rule is exact on
    \begin{gather}
      \mathscr G_2
      = \bigl\{ \nabla p\cdot \nabla q \big| p,q\in \shapespace \bigr\}.
    \end{gather}
  \end{enumerate}
  Then, there is a constant $\alpha_0>0$ depending on shape regularity
  and the quadrature rule, but independent of $h$, such that
  \begin{gather}
    a_h(u_h,u_h) \ge \alpha_0 \snorm{u_h}_{1;\domain}^2
    \qquad\forall u_h\in V_h.
  \end{gather}
\end{Lemma}

\begin{remark}
  If $\shapespace = \P_k$, then $\mathscr G_1 = \P_{k-1}$ and
  $\mathscr G_2 = \P_{2k-2}$. For $\Q_k$, these relations are more
  complicated, and the simplest we can say is
  $\mathscr G_1 \subset \Q_k$ and $\mathscr G_2 \subset \Q_{2k}$
\end{remark}

\begin{proof}[Proof of \slideref{Lemma}{quadrature-stability}]
  Assume first condition 1. From the poisitivity of quadrature
  weights, we conclude that
  \begin{gather}
    Q_{\refcell}(\nabla p\cdot\nabla p) = 0
    = \sum_{k=1}^{n_q} \sum_{i=1}^d \omega_k (\d_ip(\refvx_k))^2 = 0
  \end{gather}
  implies $\d_ip(\refvx_k) = 0$. Unisolvence implies
  $\d_i p \equiv 0$. Therefore,
  $\sqrt{Q_{\refcell}(\nabla p\cdot\nabla p)}$ defines a norm on the
  space $\shapespace/\R$. Since this space is finite dimensional and
  $\snorm{.}_{1;\refcell}$ is a second norm, we conclude from norm
  equivalence the existence of a constant $c>0$ such that
  \begin{gather}
    c \snorm{p}_{1;\refcell}^2 \le Q_{\refcell}(\nabla p\cdot\nabla p)
    \qquad\forall p\in \shapespace.
  \end{gather}
  On the other hand, condition 2 yields the same estimate with $c=1$.

  It is easily verified, that $\snorm{p}_{1;\refcell}^2$ and
  $Q_{\refcell}(\nabla p\cdot\nabla p)$ scale equally under the
  mapping $\Phi_\cell$, such that the same estimate holds on $\cell$
  with a different constant depending on shape regularity, again
  denoted by $c$. Finally,
  \begin{align*}
    \alpha_0 c \snorm{u_h}_{1;\domain}^2
    &= \alpha_0 c \sum_{\cell\in\mesh_h} \snorm{u_h}_{1;\cell}^2 \\
    &\le \alpha_0 \sum_{\cell\in\mesh_h}Q_{\refcell}(\nabla u_h\cdot\nabla u_h)\\
    & \le \sum_{\cell\in\mesh_h} Q_{\refcell}(\nabla u_hA\nabla u_h^T)
      = a_h(u_h,u_h).
  \end{align*}
\end{proof}

\begin{intro}
  Now that we have established conditions for stability of the
  discrete problem, we can address an estimate of the consistency
  error. In order to avoid additional overhead in an already technical
  argument, we restrict the analysis to the spaces
  $\shapespace_\cell = \P_k$ on simplices. Instead of proving a
  general result for arbitrary quadrature formulas, we first take ask,
  what the order of the quadrature error should be. Indeed, we have
  the interpolation estimate $\norm{u-I_h u}_{1} = \mathcal
  O(h^k)$.
  Therefore, we investigate quadrature rules introducing consistency
  errors of the same order.
\end{intro}

\begin{Theorem}{quadrature-error-bilinear}
  Assume in addition to
  \slideref{Assumption}{bilinear-form-quadrature} for $k\ge1$ that
  for all cells $\shapespace_\cell = \P_k$ and $a_{ij}\in
  W^{k,\infty}(\domain)$.
  Let the quadrature rule $Q_{\refcell}$ be exact for polynomials in
  $\P_{2k-2}$.  Then, there exists a constant $c$ such that for all
  meshes of a quasi-uniform family $\{\mesh_h\}$ of meshes there holds
  \begin{gather}
    \abs{a(v_h, w_h) - a_h(v_h, w_h)} \le c h^k \norm{A}_{k,\infty;\domain}
    \norm{v_h}_{k;h} \snorm{w_h}_{1;h}.
  \end{gather}
\end{Theorem}

\begin{Lemma}{product-sobolev-norm}
  Let $u\in W^{k,p}(\domain)$ and $v\in W^{k,\infty}(\domain)$. Then,
  $uv\in W^{k,p}(\domain)$ and
  \begin{gather}
    \snorm{uv}_{k,p;\domain}
    \le c \sum_{i=0}^{k} \snorm{u}_{i,q;\domain}\,
    \snorm{v}_{k-i,\infty;\domain},
  \end{gather}
  with a constant $c$ depending on $k$ and the space dimension, but
  not on $\domain$.
\end{Lemma}

\begin{proof}
  This is basically the product rule. For a multi-index $\alpha$ with
  $\abs{\alpha} = k$, we have
  \begin{gather*}
    \d^{\alpha}(uv) = \sum_{i=0}^k
    \sum_{\substack{\abs{\beta}=i\\\beta_j\le \alpha_j}} \d^{\beta}u\,\d^{\alpha-\beta} v.
  \end{gather*}
  Then, Hörder inequality is applied to the sums and the integrals.
\end{proof}

\begin{proof}[Proof of \slideref{Theorem}{quadrature-error-bilinear}]
  We first argue on the reference cell and define the quadrature error
  \begin{gather}
    E_{\refcell}(f) = \int_{\refcell} f \dvxref - Q_{\refcell}(f).
  \end{gather}
  Our goal is the estimation of $E_{\refcell}(apq)$ for functions
  $a\in W^{k,\infty}(\refcell)$ and polynomials $p,q\in \P_{k-1}$,
  such that we obtain an estimate for the constituents of
  $E_{\refcell}(\refgrad\refu\reference A \refgrad\refv)$.

  We combine $\phi = ap\in W^{k,\infty}(\refcell)$ Since
  $W^{k,\infty}(\refcell)\hookrightarrow C(\refcell)$, we obtain
  \begin{gather}
    \abs{E_{\refcell}(\phi q)} \le c \norm{\phi q}_{\infty}
      \le c\norm{\phi}_{0,\infty} \norm{q}_{0,\infty}
      \le c \norm{\phi}_{k,\infty} \norm{q}_{L^2(\refcell)}.
  \end{gather}
  For the last inequality, we used the natural bound of the norm in
  $W^{0,\infty}$ by that of $W^{k,\infty}$ and the norm equivalence on
  $\P_{k-1}$. We conclude that $J_q(\phi) = E_{\refcell}(\phi q)$ is a
  bounded linear functional on $W^{k,\infty}$. Furthermore, it
  vanishes for $\phi\in\P_{k-1}$ by the assumption on the quadrature
  rule. Thus by the Bramble-Hilbert lemma (\slideref{Lemma}{bramble-hilbert}),
  \begin{gather}
    \abs{E_{\refcell}(\phi q)}
    \le c \snorm{\phi}_{k,\infty}\norm{w}_{L^2(\refcell)}.
  \end{gather}
  Now, we apply \slideref{Lemma}{product-sobolev-norm}, to obtain
  \begin{gather}
    \begin{split}
    \snorm{\phi}_{k,\infty}
    &= \snorm{ap}_{k,\infty}\\
    &\le c \sum_{i=0}^{k-1} \snorm{a}_{k-i,\infty} \snorm{p}_{i,\infty}\\
    &\le c \sum_{i=0}^{k-1} \snorm{a}_{k-i,\infty} \snorm{p}_{i},
    \end{split}
  \end{gather}
  where we used that $\snorm{p}_k = 0$ and norm equivalence on
  $\P_{k-1}$.  Collecting everything and reintroducing $\reference a$,
  $\reference p$, and $\reference q$ for $a$, $p$, and $q$,
  respectively, we obtain
  \begin{gather}
    \abs{E_{\refcell}(\reference a\reference p\reference q)}
    \le c \left(\sum_{i=1}^{k-1} \snorm{\reference a}_{k-1,\infty;\refcell}
      \snorm{\reference p}_{i;\refcell}\right)
    \norm{\reference q}_{L^2(\refcell)}.
  \end{gather}
  The scaling lemma yields
  \begin{xalignat}2
    \snorm{\reference a}_{k-1,\infty;\refcell}
    &\le c h_\cell^{k-i} \snorm{a}_{k-i,\infty;\cell}
    & 0 & \le i \le k-1,
    \\
    \snorm{\reference p}_{i;\refcell}
    &\le c h_\cell^{i-\nicefrac{d}{2}} \snorm{p}_{i;\cell}
    & 0 & \le i \le k-1,
    \\
    \norm{\reference q}_{L^2(\refcell)}
    &\le c h_\cell^{-\nicefrac{d}{2}} \norm{q}_{L^2(\cell)}.
  \end{xalignat}
  Therefore,
  \begin{gather}
    \begin{split}
    \abs{E_\cell(a p q)}
    &\le c h_\cell^{\nicefrac{d}{2}}
    \abs{E_{\refcell}(\reference a\reference p\reference q)}
    \\ & \le c h_\cell^{k} \left(
      \sum_{i=1}^{k-1}\snorm{a}_{k-i,\infty;\cell}\snorm{p}_{i;\cell}
    \right)
    \norm{q}_{L^2(\cell)}
    \\ & \le c h_\cell^{k} \norm{a}_{k-i,\infty;\cell}\norm{p}_{i;\cell}
    \norm{q}_{L^2(\cell)}.
    \end{split}
  \end{gather}
  Entering the derivatives of $u_h$ and $v_h$ for $p$ and $q$,
  respectively, and summing up over all cells yields the result.
\end{proof}

\begin{remark}
  Note that the assumption of quasi-uniformity is convenient for
  notation, but excessive. Indeed, shape regularity is sufficcient,
  but in that case, the estimate must be localized, that is,
  \begin{gather}
    a(v_h, w_h) - a_h(v_h, w_h) \le c \sum_{\cell\in\mesh_h}
    h_\cell^k \norm{A}_{k,\infty;\cell}
    \norm{v_h}_{k;\cell} \snorm{w_h}_{1;\cell}.
  \end{gather}
\end{remark}

\begin{Theorem}{quadrature-error-rhs}
  Let the finite element use shape function spaces
  $\shapespace_\cell = \P_k$ and let the quadrature rule
  $Q_{\refcell}$ be exact for polynomials in $\P_{2k-2}$. Then, there
  exists a constant $c$ such that for all cells of a quasi-uniform
  family $\{\mesh_h\}$ of meshes there holds
  \begin{gather}
    \abs{E_\cell(fp)} \le c h_\cell^k \norm{f}_{k;\cell}\norm{p}_{1;\cell}
    \qquad\forall f\in H^k(\domain),\quad \forall p\in\shapespace.
  \end{gather}
\end{Theorem}


%%% Local Variables: 
%%% mode: latex
%%% TeX-master: "main"
%%% End: 


\chapter{Solving the Discrete Problem}

\begin{remark}
  This part of the notes deals with preconditioning of symmetric
  operators, or those, which have a dominating symmetric part. The
  theory of preconditioning methods for nonsymmetric and in particular
  non-normal operators is currently not well developed and thus will not
  be covered by these notes. 
\end{remark}

\begin{notation}
  Iterative methods will be considered in a Hilbert space $X$ with
  inner product $\scal(.,.)$.
\end{notation}

\begin{intro}
  The motivation for the use of iterative methods lies in the fact
  that matrices resulting from the discretization of partial
  differential equations tend to be very big, but sparse, that is,
  most of their entries are zero. Let us take for example trilinear
  finite elements on a uniform grid of $(n-1)$ cells in each
  direction. This leeds to $n^3$ degrees of freedom which we number
  lexicographically. The matrix stencil, that is, the distribution of
  nonzero entries, in ome dimension is tridiagonal, that is,
  \begin{gather*}
    S_1 =
    \begin{pmatrix}
      * & * \\ * & * & * \\
      & \ddots & \ddots & \ddots \\
      && * & *
    \end{pmatrix}
  \end{gather*}
  In two dimensions, we obtain the stencil
  \begin{gather*}
    S_2 = S_1 \otimes S_1
    =
    \begin{pmatrix}
      S_1 & S_1 \\
      S_1 & S_1 & S_1 \\
      & \ddots & \ddots & \ddots \\
      &&S_1 & S_1
    \end{pmatrix},
  \end{gather*}
  and in three dimensions
  \begin{gather*}
    S_3 = S_1 \otimes S_1 \otimes S_1
    =
    \begin{pmatrix}
      S_2 & S_2 \\
      S_2 & S_2 & S_2 \\
      & \ddots & \ddots & \ddots \\
      &&S_2 & S_2      
    \end{pmatrix}.
  \end{gather*}
  The corresponding matrix has $N=n^3$ rows and columns. Let us compare
  some solution methods for $n=100$ with respect to memory and a
  hypothetic hardware which executes $10^9$ multiplications per second
  (additions are free).
  \begin{description}
  \item[Gaussian elimination] The effort needed is $\tfrac13
    N^3+\mathcal O(N^2)$,
    leading to the computing time
    \begin{gather*}
      T_{G} = \frac{10^{18}}{3\cdot 10^9} \;\text{sec}
      \approx 3\cdot10^8 \;\text{sec}
      \approx 10.6 \;\text{years}
    \end{gather*}
    The backward substitution is only of order $N^2$ and can be
    neglected. The memory requirement with double precision is
    $8\cdot10^{12}$ Bytes, almost 10 terabytes.
  \item[Banded LU decomposition] Here we make use of the fact that LU
    decomposition can be restricted to the hull of the outermost
    nonzero elements of the matrix, the so called banded or skyline
    version. Let $M$ be the greatest distance of a nonzero matrix
    element $a_{ij}$ from the diagonal, that is, $M=\left|i-j\right|$. Then,
    the leading term of the effort is $\tfrac13 N\cdot M^2$, yielding
    with $M=n^2$ a computing time of
      \begin{gather*}
        T_{BLU} = \frac{10^{6}\cdot 10^{4} \cdot 10^{4}}{3\cdot 10^9} \;\text{sec}
      \approx 3\cdot10^4 \;\text{sec}
      \approx 9 \;\text{hours}
      \end{gather*}
      The storage requirement for $N\cdot M$ double precision numbers
      is $8\cdot 10^{10}$ Bytes, almost 100 gigabytes.
    \item[matrix vector product] For comparison, the multiplication
      with such a matrix, given that there are at most 9 nonzero
      entries per row costs
      \begin{gather*}
        T_{mult} = \frac{9\cdot 10^6}{10^9} \approx 0.01 \;\text{sec},
      \end{gather*}
      that is, we can perform more than $10^6$ steps of an iterative
      method before we reach the effort of the banded LU
      decomposition. The storage requirement is roughly 1 gigabyte and
      can be reduced to almost zero by a smart implementation. 
  \end{description}
\end{intro}

\begin{remark}
  For purposes of analysis we typically choose the space $X =
  L^2(\Omega)$. We admit a small inaccuracy here: when we run the
  algorithms on a computer, we usually employ the Eiclidean inner
  product, thus $X$ should be the space of degrees of freedom. But
  this is a discrete space, where we cannot use theory of function
  spaces easily. Instead, we note that the $L^2$-inner product of
  standard finite element bases yield inner products equivalent to the
  Euclidean up to the local mesh size (see Lemma~\ref{lemma:itintro:1}).
\end{remark}

\begin{example}
  While the methods developed in this chapter are fairly general, we
  introduce a specific model problem as a simple benchmark case. To
  this end, we consider the Dirichlet problem: find $u\in V =
  H^1_0(\Omega)$ such that
  \begin{gather}
    \label{eq:itintro:1}
    a(u,v) \equiv \int_\Omega \nabla u\cdot \nabla v \dx
    = \int_\Omega f v \dx \equiv f(v),
    \qquad \forall v\in V.
  \end{gather}
  The finite dimensional linear systems of equations are derived from
  finite element discretizations on quasi-uniform meshes of cells with
  maximal diameter $h$, yielding a sequence of spaces $V_h$, on which
  linear systems are introduced by the same weak
  form~\eqref{eq:itintro:1}.
\end{example}

\begin{notation}
  With a bilinear form $a(.,.)$ on $X\times X$ we associate the
  operator $A: X\to X$ by
  \begin{gather}
    \label{eq:itintro:2}
    \scal(Au,v) = a(u,v), \quad \forall v\in V,
  \end{gather}
  where now $V = \mathcal D(A)$ is the \define{domain} of $A$, that is, the
  subset of functions $v\in X$, such that $Av$ is defined and in $X$.
  
  We will tacitly assume that operators $A$, $B$, etc.\ are defined by
  equation~\eqref{eq:itintro:2} and the bilinear forms $a(.,.)$,
  $b(.,.)$, etc., respectively, if they are not defined otherwise.
\end{notation}

\begin{Definition}{bilinear-properties}
  We call the bilinear form $a(.,.)$ and its associated operator $A$
  \define{symmetric}, if there holds
  \begin{gather*}
    a(u,v) = a(v,u) \qquad \forall u,v \in V.
  \end{gather*}
  They are called $V$-\define{elliptic}, if for there is a positive number
  $\gamma$ such that
  \begin{gather*}
    a(u,u) \ge \gamma \norm{u}_V^2 \qquad \forall u\in V.
  \end{gather*}
\end{Definition}

\begin{Definition}{spd-condition-number}
  \defindex{Lambdaa@$\Lambda(A)$}
  \defindex{lambdaa@$\lambda(A)$}
  For positive definite, symmetric operators, we obtain the possibly
  infinite bounds of the spectrum
  \begin{gather}
    \label{eq:richardson:8}
    \Lambda(A) = \sup_{u\in V}\frac{a(u,u)}{\norm{u}_X^2},
    \qquad
    \lambda(A) = \inf_{u\in V}\frac{a(u,u)}{\norm{u}_X^2},
  \end{gather}
  \defindex{kappaa@$\kappa(A)$}
  \index{condition number|see {spectral condition number}}
  as well as the possibly infinite \define{spectral condition number}
  \begin{gather*}
   \kappa(A) = \frac{\Lambda(A)}{\lambda(A)}.
  \end{gather*}
\end{Definition}

\begin{remark}
  Note that the spectral condition number depends on the norm of the
  space $X$. It is bounded, if and only if $A$ is bounded with respect
  to this norm.
\end{remark}

\begin{example}
  Let $X=H^1_0(\Omega)$ with the inner product
  \begin{gather*}
    \scal(u,v)_1 = \int_\Omega \nabla u\cdot \nabla v \dx.  
  \end{gather*}
  If $A$ is the operator associated with the bilinear form $a(.,.)$
  in~\eqref{eq:itintro:1}, then 
  \begin{gather*}
    \Lambda(A) = \lambda(A) = \kappa(A) = 1.    
  \end{gather*}
  If on the other hand $X = L^2(\Omega)$ equipped with the usual
  $L^2$-inner product, then $A$ is unbounded and thus $\kappa(A) =
  \infty$. $\lambda(A)$ is the constant in Friedrichs' inequality.
\end{example}

\begin{notation}
  After choosing a basis for a finite dimensional space $X_n$ or a
  Schauder basis for the space $X$ (assuming $X$ separable), say
  $\{\phi_i\}$, we can define a (possibly infinite-dimensional) matrix
  $\mat A$ associated with the bilinear form $a(.,.)$ with the entries
  \begin{gather*}
    a_{i j} = a(\phi_j, \phi_i).
  \end{gather*}
  
  If we restrict the bilinear forms to a finite dimensional subspace
  $X_n$, we denote the matrices $\mat A$ restricted to this subspace
  by $\mat A_n$.
\end{notation}

\begin{Definition}{matrix-condition-number}
  \defindex{Lambdama@$\Lambda(\mat A)$}
  \defindex{lambdama@$\lambda(\mat A)$}
  The two extremal eigenvalues of the matrix $\mat A_n$ can be
  obtained by the maximum and minimum of the \define{Rayleigh
    quotient}
  \begin{gather}
    \label{eq:itintro:3}
    \Lambda(\mat A) = \max_{\vec x\in \R^n}\frac{\vec x^T \mat A \vec x}{\vec x^T\vec x},
    \qquad
    \lambda(\mat A) = \min_{\vec x\in \R^n}\frac{\vec x^T \mat A \vec x}{\vec x^T\vec x}.
  \end{gather}
  \defindex{kappama@$\kappa(\mat A)$}
  The \putindex{spectral condition number} is
  \begin{gather*}
    \kappa_n(\mat A) = \frac{\Lambda(\mat A)}{\lambda(\mat A)}.
  \end{gather*}
\end{Definition}

\begin{remark}
  The spectral condition number of the operator $A$ depends on the
  bilinear form $a(.,.)$ and the choice of the norm in $X$. On the
  other hand, the spectral condition number of the matrix $\mat A$
  depends on the choice of a basis of the space $X_n$.
\end{remark}

\begin{Lemma}{condition-number-mass-matrix}
  \label{lemma:itintro:1}
  Let $\{\phi_i\}$ be the standard, piecewise linear, finite element
  basis on a quasi-uniform triangulation of mesh size $h$. Let $\mat
  M$, the so called \define{mass matrix} be the matrix associated with
  the $L^2$-inner product with entries
  \begin{gather*}
    m_{ij} = \int_\Omega \phi_i(x) \phi_j(x) \dx.
  \end{gather*}
  Then,
  \begin{gather*}
    \Lambda(\mat M) \simeq h^{d} \simeq  \lambda(\mat M)
  \end{gather*}
  Therefore, the condition number is
  \begin{gather*}
    \kappa(\mat M) = \frac{\mathcal O(h^{d})}{\mathcal O(h^{d})} = {\mathcal O(1)}.
  \end{gather*}
\end{Lemma}

\begin{proof}
  It is easy to verify, that $m_{ii}> 0$, and that not more
  entries in each row as edges of the triangulation meet in one vertex
  are different from zero. Furthermore, that the size of those entries
  is of order $h^d$, where $d$ is the space dimension. From these two
  facts we immediately obtain
  \begin{gather*}
    \Lambda(\mat M) = \mathcal O(h^{d}).
  \end{gather*}
  The argument for $\lambda(\mat M)$ is more subtle. For any mesh cell
  $T$, let $\vec x_T$ be the entries of the vector $\vec x$ which
  belong to node values of the cell $T$. Let $\mat M_T$ be the cell
  mass matrix obtained by restricting the $L^2$-inner product to
  $T$. Then,
  \begin{gather*}
    \vec x^T \mat M \vec x
    = \sum_{T\in\T_h} \vec x^T_T \mat M_T \vec x_T
    \ge \min_{T\in\T_h} \frac{\vec x^T_T \mat M_T \vec x_T}{\vec
      x^T_T\vec x_T}
    \sum_{T\in\T_h} |\vec x_T|^2 \ge \lambda(\mat M_T)  |\vec x|^2.
  \end{gather*}
  In order to estimate the eigenvalues of $\mat M_T$, we note that for
  a unisolvent element, the norms $|\vec x_T|$ and $\|u\|_{0,T}$ are
  equivalent on the reference cell, and the $L^2$-norm scales with
  $h^d$ when transforming to the real cell $T$. Thus, we have
  $\lambda(\mat M) = \mathcal O(h^{d})$.
\end{proof}

\begin{Corollary}{refined-condition-number}
  Let $\mesh_h$ be a mesh with cell sizes ranging between the minimum $h$ and
  the maximum $H$. Then, we have
  \begin{align*}
    \Lambda(\mat M) &= \mathcal O(H^{d}) \\
    \lambda(\mat M) &= \mathcal O(h^{d}) \\
    \kappa(\mat M) &= \mathcal O\left(\left(\frac Hh\right)^{d}\right)
  \end{align*}
\end{Corollary}

\begin{todo}
  This can be fixed by using weighted norms in $\R^n$.
\end{todo}

%%% Local Variables: 
%%% mode: latex
%%% TeX-master: "main"
%%% End: 

\svnid{$Id$}

\section{The Richardson iteration}

\begin{intro}
  As a first example and prototype for all other iterative methods we
  consider Richardson's method, which for matrices and vectors in
  $\R^n$ reads
  \begin{gather}
    \label{eq:richardson:1}
    \vec x^{(k+1)}
    = \vec x^{(k)}
    - \omega_k \bigl(\mat A \vec x^{(k)} - \vec b \bigr).
  \end{gather}
  $\omega_k$ is a relaxation parameter, which can be chosen a priori
  or can be changed in every step. We will for simplicity assume
  $\omega_k = \omega$.
  
  It can be shown easily, that if $\mat A$ is symmetric, positive definite,
  the method converges if $\omega$ is sufficiently small. More
  precisely, let
  \begin{gather}
    \label{eq:richardson:3}
    \lambda
    = \min_{x\in \R^n} \frac{\vec x^T\mat A\vec x}{\vec x^T\vec x},
    \qquad\text{and}\qquad
    \Lambda = \max_{x\in \R^n} \frac{\vec x^T\mat A\vec x}{\vec x^T\vec x}.
  \end{gather}
  Then, the method converges for $0 < \omega < 2/\Lambda$. The optimal
  choice is
  \begin{gather}
    \label{eq:richardson:2}
    \omega = \frac{2}{\lambda+\Lambda},
  \end{gather}
  which leads to a contraction rate of
  \begin{gather}
    \label{eq:richardson:4}
    \rho = \frac{\Lambda-\lambda}{\Lambda+\lambda} =
    \frac{\kappa-1}{\kappa+1} = 1 -\frac2\kappa + \mathcal
    O(\kappa^{-2}),
  \end{gather}
  where $\kappa = \Lambda/\lambda$ is the spectral condition number. 
\end{intro}

\begin{intro}
  The analysis of finite element methods shows that it is beneficial
  to give up the focus on finite dimensional spaces and rather use
  theory that applies to separable Hilbert spaces. If results can
  obtained in this context, they can easily be restricted to finite
  dimensional subspaces and thus become uniform with respect to the
  mesh parameter. Thus, we will first reformulate Richardson's method
  for this case and then derive convergence estimates.
\end{intro}

\begin{intro}
  Elements of an abstract Hilbert space $V$ will be denoted by
  $u,v,w$, etc. On the other hand, coefficient vectors in $\R^n$ are
  denoted by letters $\vec x,\vec y,\vec z$, etc.
\end{intro}

\begin{definition}
  Let $V$ be a Hilbert space with inner product $\scal(.,.)_V$. Let
  $a(.,.)$ be a second bilinear form on $V$. Then, for any right hand
  side $f\in V^*$ and any start vector $u^{(0)}\in V$,
  \define{Richardson's method} is defined by the iteration
  \begin{gather}
    \label{eq:richardson:5}
    \scal(u^{(k+1)},v)_V = \scal(u^{(k)},v)_V
    - \omega_k \bigl(a(u^{(k)},v) - f(v)\bigr), \qquad \forall v\in V.
  \end{gather}
  $\omega_k$ is a suitable \putindex{relaxation parameter}, chosen
  such that the method converges.
\end{definition}

\begin{note}
  The scalar products in~\eqref{eq:richardson:5} become necessary,
  since different from the case in $\R^n$, the result of applying the
  bilinear form $a(.,.)$ to $u^{(k)}$ in the first argument yields a
  linear form on $V$. In order to convert this to a vector in $V$, we
  have to apply the isomorphism induced by the \putindex{Riesz
    representation theorem}.
\end{note}

\begin{theorem}
  \label{theorem:richardson:1}
  Let the bilinear form $a(.,.)$ be bounded and elliptic on $V\times
  V$, namely, let there exist positive constants $\Lambda$ and $\lambda$ such
  that for all $u,v\in V$ there holds
  \begin{gather}
    \label{eq:richardson:6}
    a(u,v) \le \Lambda \norm{u}_V \norm{v}_V,
    \qquad
    a(u,u) \ge \lambda \norm{u}_V^2.
  \end{gather}
  Then, the Richardson iteration~\eqref{eq:richardson:5} converges for
  $\omega_k = \omega \in (0, 2 \lambda/\Lambda^2)$. The optimal
  choice is $\omega$ according to~\eqref{eq:richardson:2}, in which
  case the convergence rate is given by~\eqref{eq:richardson:4}.
\end{theorem}

\begin{proof}
  We define the iteration operator $T$ as the solution operator of
  equation~\eqref{eq:richardson:5}, namely $T u{(k)} := u{(k+1)}$. We
  have to prove that $T$ is a contraction on $V$ under the assumptions
  of the theorem.

  For two arbitrary vectors $u^1, u^2 \in V$, let $w = u^1-u^2$ be
  their difference. Due to linearity, we have $T w = T u^1-T u^2$ and
  \begin{gather*}
    \scal(T w,v)_V = \scal(w,v)_V - \omega a(w,v) = \scal(w-\omega A w,v)_V.
  \end{gather*}
  Using $v=Tw$ as a test function, we obtain
  \begin{align*}
    \norm{Tw}_V^2
    & = \scal(w-\omega A w,w-\omega A w)_V \\
    &= \norm{w}_V^2 - 2\omega a(w,w) + \omega^2 \norm{Aw}_V^2\\
    & \le \norm{w}_V^2 - 2\lambda\omega \norm{w}_V^2
    +  \Lambda^2 \omega^2\norm{w}_V^2\\
    & = \underbrace{\bigl(1-2\lambda\omega
      + \Lambda^2\omega^2\bigr)}_{=:\rho(\omega)} \norm{w}_V^2.
  \end{align*}
  \begin{todo}
    Check the actual convergence!
  \end{todo}
  The function $\rho(\omega)$ is a parabola open to the top, which
  equals one at zero and $2/\Lambda$.
\end{proof}

\begin{note}
  It is clear that the boundedness and ellipticity
  estimates~\eqref{eq:richardson:6} hold for any finite dimensional
  subspace $V_n\subset V$, and thus the convergence
  estimate~\eqref{eq:richardson:4} becomes independent of $n$.
\end{note}  

\begin{note}
  We define an operator $B:V\to V^*$ such that $Bu = b(u,.) :=
  \scal(u,.)_V$. By the Riesz representation theorem, there is a
  continuous inverse operator $B^{-1}: V^*\to V$, which is often
  called \define{Riesz isomorphism}.
\end{note}

\begin{definition}
  When we apply Richardson's method as in~\eqref{eq:richardson:5} on a
  computer, each step involves a multiplication with the matrix $\mat A$,
  but an inversion of the matrix $\mat B$, corresponding to the iteration
  \begin{gather*}
    \mat B \vec x^{(k+1)}
    = \mat B \vec x^{(k)}
    - \omega_k \bigl(\mat A \vec x^{(k)} - \vec b \bigr),
  \end{gather*}
  or equivalently,
  \begin{gather}
    \label{eq:richardson:7}
    \vec x^{(k+1)}
    = \vec x^{(k)}
    - \omega_k \mat B^{-1}\bigl(\mat A \vec x^{(k)} - \vec b \bigr).
  \end{gather}
  The iteration in~\eqref{eq:richardson:7} is commonly referred to as
  \define{preconditioned Richardson iteration} and $\mat B^{-1}$ as the
  \define{preconditioner}. Note that by introducing the iteration in
  its weak form~\eqref{eq:richardson:5}, the preconditioner arrives
  naturally and with necessity.
  
  The goal of this chapter is finding preconditioners $\mat B^{-1}$, or
  equivalently inner products $\scal(.,.)_V$, such that the bilinear
  form $a(.,.)$ is bounded and the condition number
  $\kappa = \Lambda/\lambda$ is small.
  
  In order to reduce (or increase) confusion, we will refer to the
  inner product that we search in order to bund the condition number
  as $b(.,.)$ instead of $\scal(.,.)_V$, this way separating the
  Hilbert space $V$ more clearly from the task of
  preconditioning. Thus, the operator $B$ and the matrix $\mat B$ will
  be associated with a bilinear form $b(.,.)$ and the final version of
  the preconditioned Richardson iteration in the space $V$ is
  \begin{gather}
    \label{eq:richardson:10}
    b(u^{(k+1)},v)_V = b(u^{(k)},v)_V
    - \omega_k \bigl(a(u^{(k)},v) - f(v)\bigr), \qquad \forall v\in V,
  \end{gather}
  or in operator form
  \begin{gather}
    \label{eq:richardson:11}
    u^{(k+1)} = u^{(k)} - \omega_k B^{-1} (A u^{(k)} - f).
  \end{gather}
\end{definition}

\begin{corollary}
  Let the symmetric bilinear forms $a(.,.)$ and $b(.,.)$ in the
  Richardson iteration~\eqref{eq:richardson:10} fulfill the
  \define{spectral equivalence} relation
  \begin{gather}
    \label{eq:richardson:12}
    \lambda b(u,u) \le a(u,u) \le \Lambda b(u,u), \quad \forall u\in V.
  \end{gather}
  Then, if $\omega_k \equiv \omega \in (0,2\Lambda)$, the iteration is
  a contraction on $V$. The optimal contraction number is $\rho$
  according to equation~\eqref{eq:richardson:4} for $\omega$ chosen as
  in~\eqref{eq:richardson:2}.
\end{corollary}

\begin{proof}
  This corollary is equivalent to Theorem~\ref{theorem:richardson:1}
  if the inner product $\scal(.,.)_V$ is replaced by the bilinear form
  $b(.,.)$.
\end{proof}

\begin{notation}
  \index{lambdaBA@$\lambda(B,A)$}
  \index{Lambdaba@$\Lambda(B,A)$}
  In order to distinguish different preconditioners, we will also us
  the notation $\lambda(B, A)$ and $\Lambda(B,A)$ to refer to the
  constants in the norm equivalence~\eqref{eq:richardson:12}.
\end{notation}


\begin{example}
  Let us take the example~\eqref{eq:main:1}.
  By the Poincaré-Friedrichs inequality, $a(.,.)$ is an inner product
  on $V$ and thus we can choose $\scal(.,.)_V = a(.,.)$. In
  particular, $\lambda = \Lambda = 1$ and the optimal choice is
  $\omega = 1$. Then, Richardson's iteration becomes
  \begin{gather*}
    a(u^{(k+1)},v) = a(u^{(k)},v)
    - \bigl(a(u^{(k)},v) - f(v)\bigr) =  f(v), \qquad \forall v\in V,
  \end{gather*}
  which converges in a single step, but we have to solve the original
  equation for $u$. Thus, either the inversion of the matrix $A_n$ is
  trivial on each finite dimensional subspace $V_n$, or the method is
  useless. With usual finite element bases, the latter is true.
\end{example}

\begin{example}
  In the other extreme, we would like to use the $\R^n$ or $L^2$
  inner product on $V_n$ or $V$, such that the Riesz isomorphism is
  easily computable. But then, the bilinear form $a(.,.)$ is unbounded
  on $V$. Thus, while for each finite $n$, the condition number
  $\kappa_n = \Lambda_n/\lambda_n$ exists, it converges to infinity if
  $n\to\infty$.
\end{example}

\begin{todo}
  Show that the condition number grows like $1/h^2$ for finite element
  methods.
\end{todo}

%%% Local Variables: 
%%% mode: latex
%%% TeX-master: "main"
%%% End: 

\svnid{$Id$}

\section{The conjugate gradient method}

\begin{intro}
  Relying on Hilbert space structure more than Richardson's iteration
  is the \define{conjugate gradient method} (cg), since it uses
  orthogonal search directions. Nevertheless, it also relies on
  constructing search directions from residuals, such that a
  \putindex{Riesz isomorphism} enters the same way as before and can
  then be used for preconditioning.
  
  The beauty of the conjugate gradient method is, that it is parameter
  and tuning free, and it converges considerably faster than a linear
  iteration method.
\end{intro}

\begin{definition}
  Let $V$ be a Hilbert space and $V^*$ its dual. The conjugate
  gradient method for an iteration vector $x^{(k)} \in V$ involves the
  residuals $r^{(k)} \in V^*$ as well as the update direction $p^{(k)}
  \in V$ and the auxiliary vector $z^{(k)} \in V$. It consists of the
  steps
  \begin{enumerate}
  \item Initialization: for $f$ and $x^{(0)}$ given, compute
    \begin{xalignat*}{2}
      r^{(0)} &= f- a(x^{(0)},.) \\
      \scal(z^{(0)}, v) &= r^{(0)}(v) & \forall v &\in V \\
      p^{(0)} &= z^{(0)}.
    \end{xalignat*}
    \item Iteration step: for $x^{(k)}$, $r^{(k)}$, $z^{(k)}$, and
      $p^{(k)}$ given, compute
      \begin{xalignat*}2
        \alpha_k &= \frac{r^{(k)}(z^{(k)})}{a(p^{(k)},p^{(k)})} \\
        x^{(k+1)} &= x^{(k)} + \alpha_k p^{(k)} \\
        r^{(k+1)} &= r^{(k)} - \alpha_k a(p^{(k)},.) \\
      \scal(z^{(k+1)}, v) &= r^{(k+1)}(v) & \forall v &\in V \\
      \beta_k &= \frac{r^{(k+1)}(z^{(k+1)})}{r^{(k)}(z^{(k)})}\\
      p^{(k+1)} &= z^{(k+1)} + \beta_k p^{(k)}
      \end{xalignat*}
  \end{enumerate}
\end{definition}

%%% Local Variables: 
%%% mode: latex
%%% TeX-master: "main"
%%% End: 


\section{Multigrid methods}
\def\restrict{r}
\def\prolongate{p}
\begin{example}
  Smoothing property of the Gauss-Seidel method for the one-dimensional Laplacian.
  \begin{figure}[tp]
    \begin{center}
      \includegraphics[width=.49\textwidth]{../graph/smoothing/gauss-seidel-0.tikz}
      \includegraphics[width=.49\textwidth]{../graph/smoothing/gauss-seidel-3.tikz}
      \includegraphics[width=.49\textwidth]{../graph/smoothing/gauss-seidel-1.tikz}
      \includegraphics[width=.49\textwidth]{../graph/smoothing/gauss-seidel-4.tikz}
      \includegraphics[width=.49\textwidth]{../graph/smoothing/gauss-seidel-2.tikz}
      \hbox to .49\textwidth{}
    \end{center}
    \caption{Development of the error of the Gauss-Seidel iteration. Random initial error and errors after one and two steps (left). Errors after 3, 4, and 5 steps (top right) and after 10, 20, 40, 80, 160, and 320 steps (lower right).}
  \end{figure}
\end{example}
\begin{Algorithm*}{mgm}{The Multigrid iteration MGM($\ell, u_\ell^{k}, b_\ell$)}
  \index{multigrid iteration}
  \begin{enumerate}
  \item Pre-smoothing: let $u_\ell^{k,0} = u_\ell^{k}$ and for
    $i=0,\dots,\nu_1-1$ do
    \begin{gather}
      u_\ell^{k,i+1} = u_\ell^{k,i}
      + \omega\bigl(b_\ell - A_\ell u_\ell^{k,i}\bigr).
    \end{gather}
  \item Coarse grid correction: compute
    $b_{\ell-1} = \restrict(b_\ell - A_\ell u_\ell^{k,\nu_1})$ and
    \begin{enumerate}
    \item If $\ell=1$ solve $A_0 u_0^{\mu} = b_0$ exactly
    \item If $\ell>1$, let $u_{\ell-1}^0 = 0$ and for $i=0,\dots,\mu-1$ do
      \begin{gather*}
        u_{\ell-1}^{i+1} = \text{MGM}(\ell-1, u_{\ell-1}^{i},b_{\ell-1}).
      \end{gather*}
      Then, add
      $u_\ell^{k,\nu_1+1} = u_\ell^{k,\nu_1} + \prolongate u_{\ell-1}^{\mu}$.
    \end{enumerate}
  \item Post-smoothing: for $i=\nu_1+1,\dots,\nu_1+\nu_2$ do
    \begin{gather}
      u_\ell^{k,i+1} = u_\ell^{k,i}
      + \omega\bigl(b_\ell - A_\ell u_\ell^{k,i}\bigr).
    \end{gather}
    and let
    $u_\ell^{k+1}=u_\ell^{k,\nu_1+\nu_2+1}$.
  \end{enumerate}
\end{Algorithm*}

\begin{remark}
  The multigrid method for $\mu=1$ is called the V-cycle, for $\mu=2$
  it is the W-cycle. The generalization where $\mu$ depends on $\ell$,
  typically in the form $2^{L-\ell}$, where $L$ is the finest level,
  is called the variable V-cycle.
\end{remark}

\begin{remark}
  The Richardson method in pre- and post-smoothing can be replaced by
  other relaxation methods like Jacobi, Gauss-Seidel, SOR, or
  SSOR. Even more complex operations may be considered.

  If the smoother, like SOR, is not symmetric, it is possible to apply
  the transpose (backward SOR instead of forward) in post-smoothing to
  obtain a method which is symmetric.
\end{remark}

\begin{Lemma}{mgm-error}
  Let $S_\ell$ be the error propagation operator of the
  smoother. Then, the error propagation operator of a single step of
  the multigrid iteration is defined recursively by
  \begin{gather}
    I - M_\ell = S_\ell^{\nu_2}
    (I-\prolongate M_{\ell-1}^{\mu}\restrict A)
    S_\ell^{\nu_1},
  \end{gather}
  and $M_0 = A^{-1}$.
\end{Lemma}

\begin{proof}
  We prove for $k=0$, which proves for every step and omit the superscript $k$.
  First, observe that for the exact solution $u_\ell$ of
  $A_\ell u_\ell = b_\ell$ there holds
  \begin{gather*}
    u_\ell - u_\ell^{i+1}
    = S \left(u_\ell - u_\ell^{i+1}\right)
    \qquad i=i=0,\dots,\nu_1-1,\nu_1+1,\dots,\nu_1+\nu_2.
  \end{gather*}
  Thus,
  \begin{gather*}
    u_\ell - u_\ell^{\nu_1} = S^{\nu_1}(u_\ell - u_\ell^{0}),
    \qquad
    u_\ell - u_\ell^{\nu_1+\nu_2+1}
    = S^{\nu_2}(u_\ell - u_\ell^{\nu_1+1}),
  \end{gather*}
  We continue by induction. For $\ell=1$, we have
  \begin{gather*}
    u_1-u_1^{\nu_1+1}
    = u_1-u_1^{\nu_1} - \prolongate u_{0}^{\mu}
    = u_1-u_1^{\nu_1} - \prolongate A_0^{-1}
     \restrict A_1(u_1-u_1^{\nu_1}).
   \end{gather*}
   Combining with the error propagation of pre and post smoothing, we
   get
  \begin{gather*}
    u_1 -u_1^{\nu_1+\nu_2+1}
    = S_\ell^{\nu_2}
    (I-\prolongate A_0^{-1} \restrict A_1)
    S_\ell^{\nu_1} \left(u_1 -u_1^{0}\right).
  \end{gather*}
\end{proof}

%%% Local Variables:
%%% mode: latex
%%% TeX-master: "main"
%%% End:


\chapter{Discontinuous Galerkin methods}
\section{Nitsche's method}
\label{sec:nitsches-method}
\section{Nitsche's method}

\begin{intro}
Let us consider the inhomogeneous Dirichlet boundary value problem
\begin{gather}
  \label{eq:nitsche:1}
    \arraycolsep1pt
  \begin{array}{rclrl}
    -\Delta u &=& f
    &\text{ in }&\Omega\\
    u &=& u^D &\text{ on }&\partial\Omega.
  \end{array}
\end{gather}

We already know that for $u^d\equiv 0$ we can discretize this problem
by choosing a finite element space $V_h \subset H^1_0(\Omega)$. For
general $u^D$, there are two options:
\begin{enumerate}
\item Compute an arbitrary ``lifting'' $u^D\in H^1(\Omega)$ such that
  it is equal to $u^D$ on the boundary and compute
  $w=u-u^D \in H^1_0(\Omega)$ as the solution to the weak formulation
  \begin{gather}
    \label{eq:nitsche:1a}
    \int_\Omega \nabla w\cdot\nabla v\dx
    = \int_\Omega f v\dx
    + \int_\Omega \nabla u^D\cdot\nabla v\dx.
  \end{gather}
\item Compute an interpolation or projection $u^D_h$ of the boundary
  data $u^D$. Then, eliminate each row of the matrix corresponding to
  a degree of freedom on the boundary. In particular, let $i$ be the
  index of such a degree of freedom and let $k$ be an index
  corresponding to an interior degree of freedom not constraint by a
  boundary condition, but such that $a_{ik}\neq 0 \neq a_{ki}$. Then,
  replace the rows
  \begin{gather}
    \arraycolsep1pt
    \begin{array}{ccccccccccl}
      \cdots &+& a_{ii} u_i &+& \cdots &+& a_{ik} u_k &+& \cdots &=& f_i \\
      \cdots &+& a_{ki} u_i &+& \cdots &+& a_{kk} u_k &+& \cdots &=& f_k
    \end{array}
  \end{gather}
  by the rows
  \begin{gather}
    \arraycolsep1pt
    \begin{array}{ccccccccccl}
      && u_i &&  && &&  &=& u^D_i \\
      \cdots &+& 0 &+& \cdots &+& a_{kk} u_k &+& \cdots &=& f_k - a_{ki} u^D_i
    \end{array}
  \end{gather}
\end{enumerate}

The first option introduces the complication of finding a function in
$H^1(\Omega)$, which cannot be achieved automatically. The second can
be implemented in an automatic way, but it complicates code, in
particular for nonlinear problems.

A completely different approach modifying the bilinear form was first
suggested in the 60s and then perfected by Joachim Nitsche in 1971. In
this section, we motivate this method and derive its error
estimates. Its key feature is the transition from
$V_h \subset H^1_0(\Omega)$ to $V_h \subset H^1(\Omega)$.
\end{intro}

\begin{intro}
  If we simply derive our weak formulation in $H^1(\Omega)$, we end up
  with an additional boundary term
  \begin{gather}
    \int_\Omega \nabla u \cdot \nabla v \dx
    =
    - \int_\Omega \Delta u v \dx
    + \int_{\partial\Omega} \partial_n u v \ds.
  \end{gather}
  Thus, we obtain the natural boundary condition $\partial_n u=0$,
  which is not consistent with the original BVP. The first step for
  deriving Nitsche's method is subtracting this boundary term on both
  sides. The result is the equation
  \begin{gather}
    \label{eq:nitsche:2}
    \int_\Omega \nabla u \cdot \nabla v \dx
    -\int_{\partial\Omega} \partial_n u v \ds
    = \int_\Omega fv\dx
    \qquad\forall v\in H^1(\Omega).
  \end{gather}
  We observe that the left hand side vanishes for any constant
  function $u$. Thus, we do not have unique solvability and we will
  have to fix this problem. Furthermore, the boundary data $u^D$ does
  not appear in this formulation. We enforce $u=u^D$ in our
  formulation by a so called ``penalty term'' with penalty parameter
  $\alpha$, modifying~\eqref{eq:nitsche:2} to
  \begin{multline}
    \label{eq:nitsche:3}
    \int_\Omega \nabla u \cdot \nabla v \dx
    -\int_{\partial\Omega} \partial_n u v \ds
    + \int_{\partial\Omega} \alpha u v \ds
    \\
    = \int_\Omega fv\dx
    + \int_{\partial\Omega} \alpha u^D v \ds
    \qquad\forall v\in H^1(\Omega).
  \end{multline}
  Integrating by parts, we see that $u$ is a solution to this weak
  formulation. Following Nitsche, we make one additional modification
  which restores the symmetry of our form. We obtain the weak
  formulation
  \begin{gather}
    \label{eq:nitsche:5}
    a_h(u,v) = f_h(v) \qquad\forall v\in H^1(\Omega),
  \end{gather}
  where
  \begin{gather}
    \label{eq:nitsche:4}
    \begin{split}
    a_h(u,v) &= \int_\Omega \nabla u \cdot \nabla v \dx
    -\int_{\partial\Omega} \partial_n u v \ds
    -\int_{\partial\Omega} \partial_n v u \ds
    + \int_{\partial\Omega} \alpha u v \ds,
    \\
    f_h(v) &= \int_\Omega fv\dx
    -\int_{\partial\Omega} \partial_n v u^D \ds
    + \int_{\partial\Omega} \alpha u^D v \ds.
  \end{split}
\end{gather}
  We abbreviate this equation to
\end{intro}

\begin{note}
  Unfortunately, the problem~\eqref{eq:nitsche:4} is not well-posed
  for any finite parameter $\alpha$. Thus, it cannot be used to
  determine $u\in H^1(\Omega)$. Nevertheless, we can establish
  well-posedness on discrete spaces $V_h$ in order to compute a
  discrete solution $u_h$ and use the fact that $u$ is already
  determined by the continuous problem. Our immediate goals are thus:
  \begin{enumerate}
  \item Establish the assumptions of the Lax-Milgram theorem on $V_h$,
    which in this case involves a suitable new norm for measuring the
    error.
  \item Establish a relation between the discrete and continuous
    solution replacing Galerkin orthogonality.
  \item Deriving error estimates in suitable norms.
  \end{enumerate}
\end{note}

\begin{notation}
  From now on, we will use the inner product notation
  \begin{gather*}
    \form(u,v) \equiv \int_\Omega uv\dx
    \qquad
    \form(\nabla u,\nabla v) \equiv \int_\Omega \nabla u\cdot\nabla v\dx,
  \end{gather*}
  on $\Omega$ as well as
  \begin{gather*}
    \forme(u,v) \equiv \int_{\partial\Omega} uv\ds,
  \end{gather*}
  on its boundary.
\end{notation}

\begin{definition}
  A discrete problem~\eqref{eq:nitsche:4} is called \define{consistent},
  if for the solution $u$ of the BVP~\eqref{eq:nitsche:1} there holds
  \begin{gather}
    a_h(u,v_h) = f_h(v_h) \qquad \forall v_h\in V_h.
  \end{gather}
\end{definition}

\begin{corollary}
  Let $u \in H^1(\Omega)$ be the weak solution to~\eqref{eq:nitsche:1}
  in the sense of~\eqref{eq:nitsche:1a}. Then, the discrete
  problem~\eqref{eq:nitsche:4} is consistent.
\end{corollary}

\begin{proof}
  Since $u \in H^1(\Omega)$ and $v_h\in V_h$, all boundary terms in
  $a_h(u,v-h)$ and $f_h(v_h)$ are well-defined and consistency follows
  from $u=u^D$ in the sense of $L^2(\partial\Omega)$.
\end{proof}

\begin{definition}
  The problem dependent norm used for the analysis of Nitsche's method
  is defined by
  \begin{gather}
    \label{eq:nitsche:8}
    \tnorm{v}^2 = \form(\nabla v,\nabla v) + \forme(\alpha v,v).
  \end{gather}
\end{definition}

This choice is justified by
\begin{lemma}
  \label{lemma:nitsche:1}
  Let $V_h \subset V = H^1(\Omega)$ be a piecewise polynomial finite
  element space on a shape-regular mesh $\T_h$. Then, if $\alpha$
  sufficiently large, there exist constants $M$ and $\gamma$ such that
  \begin{align*}
    a_h(u_h, v_h) &\le M \tnorm{u_h}\tnorm{v_h}\\
    a_h(u_h,u_h) &\ge \gamma \tnorm{u_h}^2.
  \end{align*}
\end{lemma}

\begin{proof}
  Key for the proof is the inverse trace estimate
  \begin{gather*}
    |v|_{H^1(\partial T)} \le c h^{-1/2} |v|_{H^1(T)},
  \end{gather*}
  which holds with a constant $c$ depending on shape regularity and
  polynomial degree. Thus, for a cell $T$ at the boundary, there holds
  \begin{align*}
    |\forme(\partial_n u_h, v_h)_{\partial T \cap \partial\Omega}|
    & \le c h^{-1/2} |u_h|_{H^1(T)} \norm{v_h}_{L^2(\partial T \cap \partial\Omega)}
    \\
    &\le \tfrac{1}{4} |u_h|_{H^1(T)}^2
    + \tfrac{c^2}{h_T}\norm{v_h}_{L^2(\partial T \cap \partial\Omega)},
  \end{align*}
  We apply this to the lower bound to obtain
  \begin{gather*}
    a_h(u_h, u_h) \ge \left(1-\tfrac{1}{2}\right)
    |u_h|_{H^1(\Omega)}^2
    + \left(\alpha - \tfrac{2c^2}{h_T}\right)
    \norm{u_h}_{L^2(\partial\Omega)}^2.
  \end{gather*}
  Choosing
  \begin{gather}
    \label{eq:nitsche:6}
    \alpha(x) = \frac{\alpha_0}{h(x)} \ge \frac{4c^2}{h(x)}, 
  \end{gather}
  where $h(x)$ is the size of the
  cell such that $x\in\partial T$, we obtain
  \begin{gather*}
    a_h(u_h, u_h) \ge \frac12 \tnorm{u_h}^2.
  \end{gather*}
  The proof of the upper bound follows the same fashion.
\end{proof}

\begin{corollary}
  Let $\alpha$ be chosen according to
  equation~\eqref{eq:nitsche:6}. Then, the discrete
  problem~\eqref{eq:nitsche:5} has a unique solution $u_h\in V_h$.
\end{corollary}

\begin{proof}
  According to the previous lemma, the lemma of Lax-Milgram applies to
  the bilinear form $a_h(.,.)$. For the right hand side $f_h(.,)$, we
  have again because of trace estimates in $H^1(\Omega)$ and inverse
  estimates in $V_h$
  \begin{gather*}
    f_h(v) \le c \left(\norm{f}_{0,\Omega}+
      \norm{u^D}_{H^1(\Omega)}\right)\tnorm{v}.
  \end{gather*}
\end{proof}

\begin{theorem}
\label{theorem:nitsche:1}
  Let $u \in H^{k+1}(\Omega)$ with $k\ge 1$ be the solution
  of~\eqref{eq:nitsche:1} and let $u_h\in V_h$ be the solution
  to~\eqref{eq:nitsche:5} and let the assumptions of
  Lemma~\ref{lemma:nitsche:1} hold. Let furthermore $\{\T_h\}$ be a
  family of quasi-uniform, shape-regular meshes of maximal cell
  diameter $h$, and let the shape function spaces contain the
  polynomial space $P_k$. Then, there holds
  \begin{gather}
    \tnorm{u-u_h} \le c h^k |u|_{k+1,\T_h}.
  \end{gather}
\end{theorem}

\begin{proof}
  We begin with the triangle inequality
  \begin{gather*}
    \tnorm{u-u_h} \le \tnorm{u-I_h u} + \tnorm{I_h u-u_h}.
  \end{gather*}
  The interpolation error can be estimated by
  \begin{gather}
    \label{eq:nitsche:7}
    \begin{split}
      \left| u-I_h u\right|_{1,\Omega}
      &\le h^k \left|u\right|_{2,\Omega},
      \\
      \left| u-I_h u\right|_{0,\partial\Omega}
      & \le h^{3/2} \left|u\right|_{2,\Omega}.
      \\
      \left| u-I_h u\right|_{1,\partial\Omega}
      & \le h^{1/2} \left|u\right|_{2,\Omega}.
    \end{split}
  \end{gather}
  The second and third estimate actually require some deeper arguments from
  functional analysis, which is beyond the scope of this class. It
  involves an intuitive notion of Sobolev spaces with non-integer
  derivatives. Allowing such spaces, the trace estimate becomes
  \begin{gather}
    \norm{u}_{1/2, \partial\Omega} \le c \norm{u}_{1,\Omega}.
  \end{gather}

  For the remaining error term, we use $V_h$-ellipticity of the
  discrete form and consistency to obtain
  \begin{gather}
    \gamma \tnorm{I_h u-u_h}^2
    \le a_h(I_h u-u_h,I_h u-u_h)
    = a_h(I_h u - u, I_h u-u_h).
  \end{gather}
  Using Young's inequality, we estimate the right hand side with
  $\epsilon_h = u-I_h u$ and $\eta_h = I_h u-u_h$ on each boundary
  cell $T$ with boundary edge $E$ by
  \begin{gather}
    \begin{split}
      \left|\form(\nabla \epsilon_h, \nabla \eta_h)_T\right|
      & \le \tfrac{1}{\gamma} \snorm{\epsilon_h}_{1,T}^2 + \tfrac\gamma4
      \snorm{\eta_h}_{1,T}^2,
      \\
      \left|\forme(\alpha \epsilon_h, \eta_h)_E\right|
      & \le \tfrac2\gamma \norm{\sqrt{\alpha} \epsilon_h}_{0,E}^2
      +
      \tfrac\gamma8 \norm{\sqrt{\alpha} \eta_h}_{0,E}^2
      \\
      \left|\forme( \epsilon_h, \partial_n\eta_h)_E\right|
      & \le 
      \tfrac1\gamma \norm{\sqrt{\alpha} \epsilon_h}_{0,E}^2
      + \tfrac\gamma4 \snorm{\eta_h}_{1,T}^2
      + \tfrac{\gamma}{4} \norm{\sqrt{\tfrac{\alpha_0}{h_T}} \eta_h}_{0,E}^2
      \\
      \left|\forme(\partial_n\epsilon_h, \eta_h)_E\right|
      & \le \tfrac{2h_T}{\gamma\alpha_0} \snorm{\epsilon_h}_{0,E}^2
      + \tfrac{\gamma\alpha_0}{8h_T} \norm{\eta_h}_{0,E}^2.
    \end{split}
  \end{gather}
  Adding these over all cells, we obtain
  \begin{multline*}
    \gamma \tnorm{I_h u-u_h}^2
    \\
    \le \tfrac{\gamma}{2} \tnorm{I_h u-u_h}^2
    \left(
      \tfrac1\gamma \snorm{u-I_h u}_{1,\Omega}^2
      + \tfrac3\gamma \norm{\sqrt{\alpha} (u-I_h u)}_{0,E}^2
      + \tfrac{2h_T}{\gamma\alpha_0} \snorm{u-I_hu}_{0,E}^2\right),
  \end{multline*}
  and thus, by the interpolation estimate~\eqref{eq:nitsche:7}
  \begin{gather*}
    \tnorm{I_h u-u_h}^2 \le \tfrac2\gamma
    c h^{2k} \snorm{u}_{k+1,\Omega}^2.
  \end{gather*}
\end{proof}

\begin{theorem}
  Assume in addition to the assumptions of
  Theorem~\ref{theorem:nitsche:1} that the adjoint problem
\begin{gather}
  \arraycolsep1pt
  \begin{array}{rclrl}
    -\Delta z &=& u-u_h
    &\text{ in }&\Omega\\
    z &=& 0 &\text{ on }&\partial\Omega.
  \end{array}
\end{gather}
admits elliptic regularity, namely
\begin{gather}
  \snorm{z}_2 \le c \norm{u-u_h}_0.
\end{gather}
Then, the solutions $u$ and $u_h$ admit the estimate
\begin{gather}
  \norm{u-u_h}_0 \le c h^{k+1} \snorm{u}_{k+1}.
\end{gather}
\end{theorem}

\begin{proof}
  Due to symmetry of the discrete bilinear form, we have adjoint
  consistency:
  \begin{gather}
    a_h(v_h, z) = \form(u-u_h,v_h) \qquad \forall v_h\in V_h.
  \end{gather}
  From here, we proceed like in the continuous case:
  \begin{gather*}
    \norm{u-u_h}_0^2 = a_h(u-u_h,z) = a_h(u-u_h,z-I_h z).
  \end{gather*}
  Using the same derivation as in the previous theorem, we obtain the
  result.
\end{proof}

\begin{remark}
  We chose the norm defined in~\eqref{eq:nitsche:8} for our energy
  norm analysis in Theorem~\ref{theorem:nitsche:1}. We could have done
  the same using the operator norm $\sqrt{a_h(v,v)}$. In fact,
  Lemma~\ref{lemma:nitsche:1} states that both norms are
  equivalent. Therefore, we chose the one involving less terms and
  providing for simpler interpolation estimates.

  The ``triple norm'' notation $\tnorm{.}$ is very common as a
  notation for problem adjusted norms.
\end{remark}

%%% Local Variables: 
%%% mode: latex
%%% TeX-master: "main"
%%% End: 


\section{The interior penalty method}
\label{sec:interior-penalty}

% \begin{intro}
%   In this section we extend the weakening of continuity, which we
%   explored for boundary values in Section~\ref{sec:nitsches-method}
%   using Nitsche's method to interior interfaces between mesh
%   cells. While the methods obtained may look much more complicated,
%   the mathematical analysis is completely analogue to that
%   section. Thus, we can be fairly brief.
% \end{intro}

\begin{intro}
  We review the basic definitions necessary to describe discontinuous
  Galerkin (DG) methods. In particular, we need the sets of faces
  $\F_h$ of a mesh, discontinuous piecewise polynomial spaces and
  broken integrals.
\end{intro}

\begin{Definition}{dg-faces}
  Let $\T_h$ be a mesh of $\Omega \subset \R^d$ consisting of mesh
  cells $T_i$. For every boundary facet $F\subset \partial T_i$, we
  assume\footnote{This assumption can indeed be relaxed} that either
  $F \subset \partial \Omega$ or $F$ is a boundary facet of another
  cell $T_j$. In the second case, we indicate this relation by
  labeling this facet $F_{ij}$. The set of all facets $F_{ij}$ is the
  set of interior faces $\F_h^i$. The set of facets on the boundary is
  $\F_h^\partial$.
\end{Definition}

\begin{Definition}{dg-spaces}
  The discontinuous finite element space on $\T_h$ is constructed by
  concatenation of all shape function spaces $P_T$ for $T\in \T_h$
  without additional continuity requirements:
  \begin{gather}
    V_h = \bigl\{v\in L^2(\Omega) \big|
    v_{|T} \in P_T \;\forall T\in \T_h\bigr\}.
  \end{gather}
\end{Definition}

\begin{Definition}{broken-integrals}
  For any set of cells $\mesh_h$ or faces $\faces_h$, we define the bilinear
  forms
  \begin{align}
    \form(u,v)_{\mesh_h} &= \sum_{\cell\in\mesh_h} \form(u,v)_\cell, \\
    \forme(u,v)_{\faces_h} &= \sum_{\face\in\faces_h} \forme(u,v)_\face. \\
  \end{align}
\end{Definition}

\begin{intro}
  We start out with the equation
  \begin{gather*}
    -\Delta u = f.
  \end{gather*}
  Integrating by parts on each mesh cell yields
  \begin{gather*}
    \form(-\Delta u,v )_\cell
    = \form(\nabla u, \nabla v)_\cell - \forme(\d_n u, v)_{\d\cell} = \form(f,v)_T.
  \end{gather*}
  We realize that the choice of discontinuous finite element spaces
  introduces a consistency term on the interfaces between cells and on
  the boundary.

  On interior faces, there is the issue that $u$ and
  $\d_n u$ actually have two values on the interface, one from the
  left cell and one from the right. Therefore, we have to consolidate
  these two values into one. To this end, we introduce the concept of
  a numerical flux, which constructs a single value out of these
  two. Thus, we introduce on the interface $\face$ between two cells
  $\cell^+$ and $\cell^-$
  \begin{gather*}
    \mathcal F(\nabla u) = \frac{\nabla u^+ + \nabla u^-}{2} = :
    \mvl{\nabla u}.
  \end{gather*}

  Using $\forme(\d_n u,v) = \forme(\nabla u,v\n)$ we change our point
  of view and instead of integrating over the boundary $\d\cell$, we
  integrate over a face $\face$ between two cells $\cell^+$ and
  $\cell^-$. Adding up integrals from both sides, we obtain the term
  \begin{gather*}
    -\forme(\mvl{\nabla u},v^+\n^+ +v^-\n^-)_{\face}
    = -2\forme(\mvl{\nabla u},\mvl{v\n})_{\face}.
  \end{gather*}
  On boundary faces, we simply get
  \begin{gather*}
    \forme(\d_\n u,v)_{\face}.
  \end{gather*}

  Adding over all cells and faces, we obtain the equation
  \begin{gather*}
    \form(\nabla u,\nabla v)_{\T_h}
    -2\forme(\mvl{\nabla u},\mvl{v\n})_{\F_h^i}
    -\forme(\d_\n u,v)_{\F_h^\d} = \form(f,v)_{\domain}.
  \end{gather*}

  Following the idea of Nitsche, we symmetrize this term
  to obtain
  \begin{multline*}
    \form(\nabla u,\nabla v)_{\T_h}
    -2\forme(\mvl{\nabla u},\mvl{v\n})_{\F_h^i}
    -2\forme(\mvl{u\n},\mvl{\nabla v})_{\F_h^i}
    \\
    -\forme(\d_\n u,v)_{\F_h^\d}
    -\forme(u,\d_\n v)_{\F_h^\d}
    = \form(f,v)_{\domain}
    - \forme(u^o,\d_n v)_{\F_h^\d}.
  \end{multline*}
  Here the second term on the right was introduced for consistency.
  Finally, it turns out that this method is not stable and needs
  stabilization by a jump term. This will be done in
  \blockref{Definition}{ip}. Before, we introduce the notation for
  averaging and jump operators.
\end{intro}

\begin{Notation}{dg-operators}
  Let $\face$ be a face between the cells $\cell^+$ and $\cell^-$. Let
  $\n^+$ and $\n^-=-\n^+$ be the outer normal vectors of the cells at a
  point $x\in \face$. For a function $u\in V_h$, the traces $u^+$ and
  $u_-$ of $u$ on $\face$ taken from the cell $\cell^+$
  and $\cell^-$ are defined as:
  \begin{align*}
    u^+(x) &= \lim_{\epsilon\searrow 0} u(x-\epsilon\n^+), \\
    u^-(x) &= \lim_{\epsilon\searrow 0} u(x-\epsilon\n^-).
  \end{align*}
  We define the \define{averaging operator} $\mvl{.}$ and the
  \define{jump operator} $\jmp{.}$ as
  \begin{gather}
    \label{eq:ip:1}
    \mvl{u} = \frac{u^++u^-}{2},
    \qquad
    \jmp{u} = u^+-u^-.
  \end{gather}
  Not that the sign of the jump of $u$ depends on the choice of the
  cells $\cell^+$ and $\cell^-$. It will only be used in quadratic
  terms.
\end{Notation}

\begin{remark}
  The jump can be denoted as the mean value of the product of a
  function and the normal vector,
  \begin{gather}
    \jmp{u} = 2\mvl{u\n}\cdot\n^+ = -2\mvl{u\n}\cdot\n^-.
  \end{gather}
\end{remark}

\begin{Definition}{ip}
  The \define{interior penalty method}\footnote{Also known as
    symmetric interior penalty (SIPG) or IP-DG.} uses the bilinear
  form
  \begin{multline}
    \label{eq:ip:2}
    a_h(u,v) = \form(\nabla u,\nabla v)_{\mesh_h}
    + \forme(\ipp_h\jmp{u},\jmp{v})_{\faces_h^i}
    + \forme(\ipp_h u,v)_{\faces_h^\d}
    \\
    -2\forme(\mvl{\nabla u},\mvl{v\n})_{\faces_h^i}
    -2\forme(\mvl{u\n},\mvl{\nabla v})_{\faces_h^i}
    \\
    - \forme(\d_n u,v)_{\faces_h^\d}
    - \forme(u,\d_n v)_{\faces_h^\d},
  \end{multline}
  and the linear form
  \begin{gather}
    \label{eq:ip:3}
    f_h(v) = \form(f,v)_{\domain} - \forme(u^D,\d_n v)_{\faces_h^\d}
    + \forme(\ipp_h u,v)_{\faces_h^\d},
  \end{gather}
  where $f$ is the right hand side of the equation and $u^D$ the
  Dirichlet boundary value.
\end{Definition}

\begin{Definition}{ip-norm}
  On the space $V_h$ we define the norm $\norm{.}_{1,h}$ by
  \begin{gather}
    \label{eq:ip:4}
    \norm{v}_{1,h}^2 = \sum_{\cell\in\mesh_h} \norm{\nabla v}_\cell^2
    + \sum_{\face\in\faces_h^i} \norm{\sqrt{\ipp_h}\jmp{v}}_\face^2
    + \sum_{\face\in\faces_h^\d} \norm{\sqrt{\ipp_h}v}_\face^2.
  \end{gather}
\end{Definition}

\begin{Problem}{ip-norm}
  Prove that the norm defined in (\ref{eq:ip:4}) is indeed a norm on $V_h$.
\begin{solution}
\begin{align*}
  0=\norm{v}_{1,h}^2 &= \sum_{\cell\in\mesh_h} \norm{\nabla v}_\cell^2
    + \sum_{\face\in\faces_h^i} \norm{\sigma_h\jmp{v}}_\face^2
    + \sum_{\face\in\faces_h^\d} \norm{\sigma_hv}_\face^2
\end{align*}
implies first of all $v|_T \equiv const.$ for all $T \in \mathbb{T}_h$.
Furthermore, $\norm{\sigma_h\jmp{v}}_\face^2=0$ implies $v \equiv const.$
and the last condition gives $v \equiv 0$.

 We use the trace inequality
 \begin{align*}
  \norm{v}_{0,\partial T}^2
  \lesssim \left( h_T^{-1} \norm{v}_{0,T}^2+\norm{v}_{0,T}\norm{\nabla v}_{0,T}\right)
  \quad \forall v \in H^1(T).
 \end{align*}
 Then, we can estimate
 \begin{align*}
  &\sum_{\face\in\faces_h^i} \norm{\sigma_h\jmp{v}}_\face^2 + \sum_{\face\in\faces_h^\d} \norm{\sigma_hv}_\face^2\\
  &\lesssim \min_{F\in F_h^i\cup F_h^\partial} \sigma_{h,F} \sum_{T \in \mathbb{T}_h}\norm{v}_{0,\partial T}^2 \\
  &\lesssim \min_{F\in F_h^i\cup F_h^\partial} \sigma_{h,F} \sum_{T \in \mathbb{T}_h}\left( h_T^{-1}
  \norm{v}_{0,T}^2+\norm{v}_{0,T}\norm{\nabla v}_{0,T}\right)\\
  &\lesssim \min_{F\in F_h^i\cup F_h^\partial} \sigma_{h,F} \sum_{T \in \mathbb{T}_h}\left( h_T^{-1}
  \norm{v}_{0,T}^2+h_T\norm{\nabla v}_{0,T}\right)
 \end{align*}
 and thus
 \begin{align*}
    \norm{v}_{1,h}^2 &= \sum_{\cell\in\mesh_h} \norm{\nabla v}_\cell^2
    + \sum_{\face\in\faces_h^i} \norm{\sigma_h\jmp{v}}_\face^2
    + \sum_{\face\in\faces_h^\d} \norm{\sigma_hv}_\face^2\\
    &\lesssim \min_{F\in F_h^i\cup F_h^\partial} \sigma_{h,F}
      \sum_{T \in \mathbb{T}_h}\left( h_T^{-1} \norm{v}_{0,T}^2+\norm{\nabla v}_{0,T}\right)\\
  &\lesssim \norm{v}_1^2
  \end{align*}
\end{solution}
\end{Problem}

\begin{Lemma}{ip-stability}
  Let $\T_h$ be shape-regular and chosen on each face $\face$ as
  $\sigma_h = \sigma_0/h_\face$, where $h_T$ is the minimal diameter
  of a cell adjacent to $\face$. Then, there is a $\sigma_0>0$ such
  that there exists a constant $\ellipa>0$, such that independent of
  $h$ there holds
  \begin{gather}
    \label{eq:ip:5}
    a_h(u_h,u_h) \ge \ellipa \norm{u_h}_{1,h}^2 \quad \forall u_h\in V_h.
  \end{gather}
\end{Lemma}

\begin{Problem}{ip-stability}
  Prove \blockref{Lemma}{ip-stability}.
\begin{solution}
We first note the estimate
\begin{align*}
 (\boldsymbol{n}\cdot\nabla v_h)_e^2
 &\leq C (h_K^{-1}\norm{\nabla v_h}_{0,K}^2+\norm{\nabla v_h}_{0,K}\norm{\nabla^2 v_h}_{0,K}) \\
 &\leq C \left(\frac{1}{h_K}+\frac{p_K^2}{h_K}\right) \norm{\nabla v_h}_{0,K}^2\\
 &\leq C \frac{p_K^2}{h_K} \norm{\nabla v_h}_{0,K}^2
 = C\frac{\sigma_h}{\delta} \norm{\nabla v_h}_{0,K}^2
\end{align*}

Testing the bilinear form symmetrically, we obtain
\begin{align*}
 a_h(u_h,u_h) &= \norm{\nabla u_h}_0^2
    + \ipp_h\norm{\jmp{u_h}}_{\faces_h^i}
    + \ipp_h\norm{u_h}_{\faces_h^\d}
    \\
    &-4\forme(\mvl{\nabla u_h},\mvl{u_h\n})_{\faces_h^i}
    -2 \forme(\d_n u_h,u_h)_{\faces_h^\d}.
\end{align*}
and the last two terms can be estimated by
\begin{align*}
 \forme(\mvl{\nabla u_h},\mvl{u_h\n})_{\faces_h^i} &=\forme(n^+\cdot\nabla u_h^+-n^-\cdot\nabla u_h^-, u_h^+-u_h^-)\\
 &\leq \frac{\epsilon}{2}\norm{n^+\{\{\nabla u_h\}\}}_0^2+\frac{1}{2\epsilon}\norm{[[u_h]]}_0^2\\
 &\leq C\frac{\sigma_h\epsilon}{2\delta}\norm{\nabla u_h}_{0,K}^2+\frac{1}{2\epsilon}\norm{[[u_h]]}_0^2
\end{align*}
and therefore
\begin{align*}
 a_h(u_h,u_h) -\gamma \norm{u_h}_{1,h}^2&\geq \norm{\nabla u_h}_0^2 \left(1-\gamma-C\frac{\sigma_h\epsilon}{2\delta}\right)\\&\quad
    + \ipp_h\norm{\jmp{u_h}}_{\faces_h^i}\left(1-\gamma-\frac{1}{2\epsilon\sigma_h}\right).
\end{align*}

Hence, we have to choose $\epsilon, \delta>0$ such that
\begin{align*}
 1-\gamma-C\frac{\sigma_h\epsilon}{2\delta}>0 \\
 1-\gamma-\frac{1}{2\epsilon\sigma_h}>0 .
\end{align*}
This is possible for all $\gamma\in(0,1)$ and in fact we get the lower limits
\begin{align*}
\epsilon&>\frac{1}{2(1-\gamma)\sigma_h}\\
 \delta&>C \frac{\sigma_h \epsilon}{2(1-\gamma)}.
\end{align*}


\end{solution}

\end{Problem}

\begin{Lemma}{ip-consistence}
  Let $f\in L^2(\domain)$ and let the boundary conditions admit that
  for the solution to
  \begin{xalignat*}2
    -\Delta u &= f &\text{in }&\domain, \\
    u &= u^D &\text{on }&\d\domain,
  \end{xalignat*}
  there holds $u\in H^{1+\epsilon}(\domain)$ for a positive
  $\epsilon$. Then, the interior penalty method is consistent, that
  is,
  \begin{gather}
    a_h(u,v_h) = f_h(v_h)\quad\forall v_h\in V_h.
  \end{gather}
\end{Lemma}

\begin{proof}
  From $f\in L^2(\domain)$ we deduce that
  $\nabla u\in \Hdiv(\domain)$. Thus, with the extra regularity, the
  traces of $\d_n u$ on faces are well-defined and coincide from both
  sides. The remainder is integration by parts.
\end{proof}

\begin{Theorem}{ip-convergence}
  For $k\ge 1$ let $\P_k\subset P_\cell$ and $u\in H^{s+1}(\domain)$ with
  $1/2 \le s \le k$. Then, the interior penalty method admits the
  error estimate
  \begin{gather}
    \norm{u-u_h}_{1,h} \le c h^s \snorm{u}_{s+1}.
  \end{gather}
  If furthermore the boundary condition admits \putindex{elliptic
    regularity},
there holds
  \begin{gather}
    \norm{u-u_h}_{0} \le c h^{s+1} \snorm{u}_{s+1}.
  \end{gather}
\end{Theorem}

%%% Local Variables:
%%% mode: latex
%%% TeX-master: "main"
%%% End:





\subsection{Bounded formulation in $H^1$}
\begin{intro}
  The interior penalty method introduced so far is $V_h$-elliptic and
  consistent, but it is not bounded on $H^1(\domain)$. This was a
  reason, why we could not use standard techniques for the proof of
  the convergence result and after applying consistency had to
  estimate each term separately.

  In this section, we will introduce a reformulation of the interior
  penalty method, which is equivalent to the original method on $V_h$,
  but is also bounded in $H^1(\domain)$. As an unpleasant side effect,
  it turns out that this method is inconsistent, and we have to
  estimate the consistency error.
  
  The main technique applied here is the use of lifting operators,
  such that the traces of derivatives on faces can be replaced by
  volume terms. Note that the lifting operators, while very useful for
  the analysis of the method, are not actually used in the
  implementation of the interior penalty method.
\end{intro}

\begin{Definition}{dg-lifting}
  Define the auxiliary space 
  \begin{gather}
    \label{eq:ip-lifting:1}
    \Sigma_h = \bigl\{ \tau\in L^2(\domain;\R^d) \big\vert
    \,\forall \cell\in \mesh_h: \tau_{|\cell} \in \Sigma_T \bigr\},
  \end{gather}
  where $\Sigma_T$ is a (possibly mapped) polynomial space chosen such that
  $\nabla V_T \subset \Sigma_T$. Then, we define the \define{lifting
    operator}
  \begin{gather}
    \label{eq:ip-lifting:2}
    \lifting\colon V+V_h \to \Sigma_h
  \end{gather}
  by
  \begin{gather}
    \label{eq:ip-lifting:3}
    \form(\lifting v,\tau)_{\mesh_h}
    = 2\forme(\mvl{\tau},\mvl{v\n})_{\faces_h^i}
    + \forme(\tau\cdot\n,v)_{\faces_h^\d}.
  \end{gather}
\end{Definition}

\begin{Lemma}{ip-lifting-bounded}
  The lifting operator is a bounded operator from $L^2(\faces_h)$ to
  $\Sigma_h$, such that
  \begin{gather}
    \label{eq:ip-lifting:4}
    \norm{\lifting v}_{L^2(\domain)}
    \le c \norm*{\tfrac1{\sqrt{h}}\jmp{v}}_{\faces_h^i}
    + \norm*{\tfrac1{\sqrt{h}} v}_{\faces_h^\d}.
  \end{gather}
  In particular, it is bounded on $H^1(\domain)$.
\end{Lemma}

\begin{proof}
  It is clear, that the operator is bounded on $L^2(\faces_h)$, since
  its definition involves face integrals weighted with polynomial
  functions. The dependence on the mesh size is due to the standard
  scaling argument.
\end{proof}

\begin{Definition}{ip-lifting}
  The \define{interior penalty method} with lifting operators uses the
  bilinear form
  \begin{multline}
    \label{eq:ip-lifting:5}
    a_h(u,v) = \form(\nabla u,\nabla v)_{\mesh_h}
    - \form(\lifting u, \nabla v)_{\mesh_h}
    - \form(\nabla u, \lifting v)_{\mesh_h}
    \\
    + \forme(\ipp_h\jmp{u},\jmp{v})_{\faces_h^-}
    + \forme(\ipp_h u,v)_{\faces_h^\d}
    .
  \end{multline}
  and the linear form~\eqref{eq:ip:3} of the original interior penalty
  method. Its residual operator is
  \begin{gather}
    \label{eq:ip-lifting:7}
    \Res(u,v) = a_h(u,v) - \form(f,v).
  \end{gather}
\end{Definition}

\begin{Lemma}{ip-equivalence}
  The interior penalty method in flux form (\slideref{Definition}{ip})
  and in lifting form (\slideref{Definition}{ip}) coincide on the
  discrete space $V_h$ if $\Sigma_h$ is chosen such that $\nabla V_h
  \subset \Sigma_h$.
  
  The lifting form is bounded on $H^1(\domain)+V_h$, but not consistent. The flux form is consistent for continuos solutions $u\in H^{\nicefrac32}(\domain)$ but not bounded on $H^1(\domain)+V_h$.
\end{Lemma}

\begin{proof}
  Since $\nabla V_h \subset \Sigma_h$, $\nabla u_h$ and $\nabla v_h$
  are valid test functions in the definition~\eqref{eq:ip-lifting:3}
  of the lifting operator, and  the equality
  \begin{gather*}
    \form(\lifting{u_h},\nabla v_h)_{\mesh_h}
    = 2\forme(\mvl{u_h\n},\mvl{\nabla v_h})_{\faces_h^i}
    + \forme(u_h,\d_n v_h)_{\faces_h^\d}.
  \end{gather*}
  
  The remaining statements of the lemma summarize and contextualize our analysis of the method in flux form and the application of \slideref{Lemma}{ip-lifting-bounded}.
\end{proof}

\begin{Definition}{ip-residual}
  Let $V\subset H^1(\domain)$ and let $u,u^*\in V$ solve the primal
  and dual problems
  \begin{gather}
    a(u,v) = f(v),
    \qquad
    a(v,u^*) = \psi(v),
    \qquad
    \forall v\in V,
  \end{gather}
  with a bounded, $V$-elliptic bilinear form $a(.,.)$. For a discrete
  bilinear form $a_h(.,.)$ defined on $V+V_h$, we define the primal
  and dual \define{residual operator}s
  \begin{gather}
    \label{eq:ip-lifting:8}
    \begin{split}
      \Res(u,v) &= a_h(u,v) - f(v), \\
      \Res^*(u^*,v) &= a_h(v,u^*) - \psi(v).
    \end{split}
  \end{gather}
\end{Definition}

% This IS Strang's second lemma!
\begin{Lemma*}{ip-lifting-strang}{Strang's 2nd Lemma}
  Let $a_h(.,.)$ be a bounded bilinear form on $V+V_h$ and elliptic on
  $V_h$ with norm $\norm{.}_{V_h}$ and constant $\ellipa$. Then, the error
  $u-u_h$ admits the estimate
  \begin{gather}
    \label{eq:ip-lifting:9}
    \norm{u-u_h}_{V_h} \le \frac1{\ellipa}
    \norm{\Res(u,.)}_{V_h^*}
    + \left(1+\frac{\norm{a_h}}{\ellipa}\right)
    \inf_{w_h\in V_h}\norm{u-w_h}
  \end{gather}
\end{Lemma}*

\begin{proof}
  First, by the definition of the residual, we have the error equation
  \begin{gather}
    a_h(u-u_h, v_h) = \Res(u,v_h),\qquad\forall v_h\in V_h.
  \end{gather}
  Inserting $w_h-w_h$ for an arbitrary element $w_h\in V_h$, we obtain
  \begin{gather*}
    a_h(w_h-u_h, v_h) = \Res(u,v_h) - a_h(u-w_h, v_h),\qquad\forall v_h\in V_h.
  \end{gather*}
  Using $v_h = w_h-u_h$ and ellipticity, we obtain
  \begin{align*}
    \ellipa \norm{w_h-u_h}_{V_h}^2
    &\le a_h(w_h-u_h,w_h-u_h)\\
    & = \Res(u,w_h-u_h) - a_h(u-w_h, w_h-u_h)\\
    & \le \bigl(\norm{\Res(u,.)}_{V_h^*} + \norm{a_h} \norm{u-w_h}_{V_h}\bigr)
      \norm{w_h-u_h}_{V_h}.
  \end{align*}
  Hence, by triangle inequality
  \begin{gather*}
    \norm{u-u_h}_{V_h} \le \frac1{\ellipa} \norm{\Res(u,.)}_{V_h^*}
    + \left(1+\frac{\norm{a_h}}{\ellipa}\right)
    \inf_{w_h\in V_h}\norm{u-w_h}_{V_h}
  \end{gather*}
\end{proof}

% From Girault/Kanschat/Riviere

\begin{Lemma}{ip-lifting-residual-1}
  Let $u\in V$ be the solution to the Poisson equation
  with right hand side $f\in L^2(\domain)$. Assume $u\in H^s(\domain)$
  with $s>3/2$. Then, we have for $v\in V+V_h$:
  \begin{gather}
    \label{eq:hdivdg:11}
    \form(f,v) = \form(\nabla u,\nabla v)_{\mesh_h}
    - 2\forme(\nabla u,\mvl{v \n})_{\faces_h^i}
    -\forme(\d_n u,v)_{\faces_h^\d}.
  \end{gather}
\end{Lemma}

\begin{proof}
  We set out from the strong form of the Poisson equation and
  integrate by parts.
  \begin{gather*}
    \form(f,v) = \form(-\Delta u, v)
    = \form(\nabla u,\nabla v)_{\mesh_h}
    - \sum_{\cell\in\mesh_h} \forme(\d_n u,v)_{\d\cell}
    .
  \end{gather*}
  Under the regularity assumptions of the lemma, all of these
  integrals make sense at least as duality pairings. In particular,
  $\d_n u\in L^2(\d\cell)$, and thus we can split $\d\cell$ into
  individual faces. Therefore,
  \begin{gather*}
    \sum_{\cell\in\mesh_h} \forme(\d_n u,v)_{\d\cell}
    = 2\forme(\nabla u,\mvl{v \n})_{\faces_h^i}
    + \forme(\d_n u,v)_{\faces_h^\d}.
  \end{gather*}
  The proof concludes by collecting the results.
\end{proof}

\begin{Lemma}{ip-lifting-residual-2}
  Let $k\ge 1$ and let $V_h$ such that $\P_{k-1} \subset \Sigma_T$. Then, if
  $u\in H^{k+1}(\domain)$ and $v\in V+V_h$, there holds
  \begin{gather}
    \label{eq:ip-lifting:10}
    \begin{split}
    \abs{\Res(u,v)}
    &\le c h^{k} \snorm{u}_{k+1}
    \bigl(
    \norm{\sqrt{\ipp_h}\jmp{v}}_{\faces_h^i}
    +
    \norm{\sqrt{\ipp_h}v}_{\faces_h^\d}\bigr)
    \\
    &\le c h^{k} \snorm{u}_{k+1} \norm{v}_{1,h}.
    \end{split}
  \end{gather}
  % Furthermore, for $v_h\in V_h$, there holds
  % \begin{gather}
  %   \label{eq:ip-lifting:11}
  %   \Res(u,v_h) = 0.
  % \end{gather}
\end{Lemma}

\begin{proof}
  First, we observe that by the regularity assumption, $\jmp{u} = 0$
  and thus, $\lifting u=0$. Hence,
  \begin{gather*}
    a_h(u,v) = \form(\nabla u, \nabla v)_{\mesh_h}
    - \form(\nabla u,\lifting v)_{\mesh_h}.
  \end{gather*}
  By \slideref{Lemma}{ip-lifting-residual-1} and regularity of $u$,
  \begin{align*}
    \Res(u,v)
    &= 2\forme(\nabla u,\mvl{v\n})_{\faces_h^i}
      + \forme(\d_n u,v)_{\faces_h^\d}
      - \form(\nabla u,\lifting v)_{\mesh_h}
    \\
    &= 2\forme(\mvl{\nabla u},\mvl{v\n})_{\faces_h^i}
      + \forme(\d_n u,v)_{\faces_h^\d}
      - \form(\nabla u,\lifting v)_{\mesh_h}
    \\
    &= 2\forme(\mvl{\nabla u},\mvl{v\n})_{\faces_h^i}
      + \forme(\d_n u,v)_{\faces_h^\d}
      - \form(\Pi_{\Sigma_h} \nabla u,\lifting v)_{\mesh_h},
  \end{align*}
  where $\Pi_{\Sigma_h}$ is the $L^2$-projection. Now, we can apply
  the definition of the lifting term to obtain
  \begin{multline*}
    \Res(u,v) = 2\forme(\tfrac1{\ipp_h}\mvl{\nabla u -
      \Pi_{\Sigma_h}\nabla u}, \ipp_h\mvl{v\n})_{\faces_h^i}
    \\
    + \forme(\tfrac1{\ipp_h} (\nabla u -\Pi_{\Sigma_h} \nabla
    u)\cdot\n, \ipp_h v)_{\faces_h^\d}.
  \end{multline*}
  Application of standard approximation and trace estimates yields the
  result observing that $\ipp_h = \ipp_0/h$.
\end{proof}

\begin{Theorem}{ip-lifting-h1}
  Let $k\ge 1$ and $V_h$ such that $\P_k \subset V_\cell$. Let
  $u\in H^{k+1}(\domain)$ be the solution to the continuous Poisson
  problem. Let $a_h(.,.)$ be the interior penalty method with lifting
  operators such that $\nabla V_h\subset\Sigma_h$. Then, there holds
  \begin{gather}
    \norm{u-u_h}_{1,h} \le c h^{k} \snorm{u}_{k+1}.
  \end{gather}
\end{Theorem}

\begin{proof}
  Application of \slideref{Lemma}{ip-lifting-strang},
  \slideref{Lemma}{ip-lifting-residual-2}, and standard interpolation
  results.
\end{proof}

\begin{Theorem}{ip-lifting-l2}
  Let the assumptions of \slideref{Theorem}{ip-lifting-h1} hold and in
  addition assume that the problem
  \begin{gather*}
    a(v,u^*) = \psi(v),\qquad\forall v\in V,
  \end{gather*}
  admits the \putindex{elliptic regularity} estimate
  \begin{gather}
    \label{eq:ip-lifting:12}
    \norm{u^*}_{H^2(\domain)} \le c \norm{\psi}_{L^2(\domain)}.
  \end{gather}
  Then, there holds
  \begin{gather}
    \label{eq:ip-lifting:13}
    \norm{u-u_h}_{L^2(\domain)} \le c h^{k+1} \snorm{u}_{H^{k+1}(\domain)}.
  \end{gather}
\end{Theorem}

\begin{proof}
  The proof uses the duality argument by Aubin and Nitsche, which sets
  out solving the auxiliary problem
  \begin{gather*}
    a(v,u^*) = \form(u-u_h,v),\qquad\forall v\in V.
  \end{gather*}
  Using the definition of the dual residual, we obtain the equation
  \begin{gather*}
    \form(u-u_h, v) = a_h(v,u^*) - \Res^*(u^*,v),\qquad\forall v\in V+V_h.
  \end{gather*}
  Testing with $v=u-u_h$ yields
  \begin{gather*}
    \norm{u-u_h}^2 = a_h(u-u_h,u^*) - \Res^*(u^*,u-u_h).
  \end{gather*}
  Additionally, we us the error equation
  \begin{gather*}
    a_h(u-u_h, v_h) = \Res(u,v_h),
  \end{gather*}
  tested with $v_h = I_h u^*$, to obtain
  \begin{multline*}
    \norm{u-u_h}^2 = a_h(u-u_h, u^*-I_h u^*) - \Res^*(u^*,u-u_h)
    + \Res(u,I_h u^*).
  \end{multline*}
  Using the regularity of $u^*$, the first term on the right
  admits the estimate
  \begin{gather*}
    \abs{a_h(u-u_h, u^*-I_h u^*)}
    \le \norm{u-u_h}_{1,h}\norm{u^*-I_hu^*}_{1,h}
    \le c h \norm{u-u_h}_{1,h}.
  \end{gather*}
  For the second term, we use \slideref{Lemma}{ip-lifting-residual-2}
  to obtain
  \begin{gather*}
    \abs{\Res^*(u^*,u-u_h)} \le c h \snorm{u^*}_2 \norm{u-u_h}_{1,h}.
  \end{gather*}
  Finally, using $\jmp{u^*} = 0$, the same lemma yields
  \begin{align*}
    \abs{\Res(u, I_h u^*)}
    &\le c h^k \snorm{u}_{k+1}
      \bigl(\norm{\sqrt{\ipp_h}\jmp{I_h u^*}}_{\faces_h^i}
      + \norm{\sqrt{\ipp_h}I_h u^*}_{\faces_h^\d}\bigr)
    \\
    & = c h^k \snorm{u}_{k+1}
      \bigl(\norm{\sqrt{\ipp_h}\jmp{u^*-I_h u^*}}_{\faces_h^i}
      + \norm{\sqrt{\ipp_h}(u^*-I_h u^*)}_{\faces_h^\d}\bigr)
    \\
    & \le c h^k \snorm{u}_{k+1} h \snorm{u^*}_2
  \end{align*}
  Using the energy estimate in \slideref{Theorem}{ip-lifting-h1} we
  can conclude the prove.
\end{proof}

%%% Local Variables:
%%% mode: latex
%%% TeX-master: "main"
%%% End:


\bibliographystyle{apalike}
\bibliography{all}
\printindex

%%% Local Variables:
%%% mode: latex
%%% TeX-master: "main"
%%% End:
