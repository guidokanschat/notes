%%%%%%%%%%%%%%%%%%%%%%%%%%%%%%%%%%%%%%%%%%%%%%%%%%%%%%%%%%%%%%%%%%%%%%
\subsection{Matlab und Octave}
\frame{\subtoc}

\begin{frame}
  Matlab und Octave sind beides große Programmpakete mit einer auf die
  Lösung mathematischer Aufgaben fokussierten Programmiersprache.

  Beide Programmiersprachen sind weitgehend kompatibel miteinander.

  \begin{columns}
    \begin{column}[t]{.49\textwidth}
      \begin{block}{Matlab}
        \begin{itemize}
        \item Kommerzielle Software
        \item Studierendenlizenzen im URZ
        \item Besser auf Leistung optimiert
        \item Besseres GUI
        \end{itemize}
      \end{block}
    \end{column}
    \begin{column}[t]{.49\textwidth}
      \begin{block}{GNU Octave}
        \begin{itemize}
        \item Freie Software
        \item Sehr gut für uns geeignet
        \item Besserer DGL-Löser?
        \end{itemize}
      \end{block}
    \end{column}
  \end{columns}
\end{frame}

\begin{frame}{Downloading}

  \begin{itemize}
  \item GNU Octave
    \begin{itemize}
    \item The homepage is \url{https://octave.org/}
    \item The direct download link is \url{https://octave.org/download.html}
    \end{itemize}
  \item Matlab
    \begin{itemize}
    \item
      \url{https://www.urz.uni-heidelberg.de/de/service-katalog/software-und-anwendungen/matlab-simulink}
    \end{itemize}
  \end{itemize}
\end{frame}

\begin{frame}{lsode (Octave)}
  Die Funktion \lstinline!lsode(@fun, x0, tinv)! löst
  \begin{gather*}
    x' = \text{fun}(x,t), \qquad x(t_0) = x0.
  \end{gather*}
  Sie hat 3 Parameter:
  \begin{enumerate}
  \item \lstinline!@fun! ist die rechte Seite der Gleichung als Funktion
  \item \lstinline!x0! ist der Startwert, immer als Vektor
  \item \lstinline!tinv! ist ein Vektor mit allen Zeitpunkten, an denen
    die Lösung gewünscht ist
  \end{enumerate}
  Der Rückgabewert ist ein Vektor mit einer Lösung für jeden Eintrag
  von \lstinline!tinv!
\end{frame}

\begin{frame}{Beispielprogramm (lsode)}
  \lstinputlisting{octave/logistic.m}

  \lstinputlisting{octave/simple_lsode.m}
\end{frame}

\begin{frame}{ode45 (Matlab, Octave)}
  Die Funktion \lstinline!ode(@fun, tinv, x0)! löst
  \begin{gather*}
    x' = \text{fun}(t,x), \qquad x(t_0) = x0.
  \end{gather*}
  Sie hat 3 Parameter:
  \begin{enumerate}
  \item \lstinline!@fun! ist die rechte Seite der Gleichung als Funktion
  \item \lstinline!tinv = [t0,tend]! ist das Zeitintervall
  \item \lstinline!x0! ist der Startwert, immer als Vektor
  \end{enumerate}
  Die Funktion gibt einen Vektor mit Zeitpukten und einen mit
  Funktionswerten zurück.
\end{frame}

\begin{frame}{Beispielprogramm (ode45)}
  \lstinputlisting{octave/logistic_ode45.m}

  \lstinputlisting{octave/simple_ode45.m}
\end{frame}



%%%%%%%%%%%%%%%%%%%%%%%%%%%%%%%%%%%%%%%%%%%%%%%%%%%%%%%%%%%%%%%%%%%%%%
\subsection{Fehler und Berechenbarkeit}
\frame{\subtoc}

\begin{frame}{Fehlerquellen einer AWA}
  \begin{enumerate}
  \item Messfehler
    \begin{itemize}
    \item Startwerte
    \item Funktionsparameter
    \end{itemize}
  \item Diskretisierungsfehler
    \begin{itemize}
    \item Ersatz der DGl durch ein Zeitschrittverfahren
    \end{itemize}
  \item Rechenfehler
    \begin{itemize}
    \item Rundung bei der Zahlendarstellung im Computer
    \item Inexakte Berechnungen im Algorithmus
    \end{itemize}
  \end{enumerate}
\end{frame}

%%% Local Variables:
%%% mode: latex
%%% TeX-master: "slides.tex"
%%% End:
