
\begin{intro}
  In this section, we summarize some basic information on finite
  difference methods as the simplest discretization of partial
  differential equations in order to facilitate implementation of the
  5-point and 7-point stencils as example for sparse matrix
  methods. Note that this method serves as a template here, but that
  it is not recommended in general as a discretization method.

  We restrict this exposition to the simplest case of boundary value
  problems on the unit square or cube $[0,1]^d$ for $d=2,3$.
\end{intro}

\begin{Definition}{Laplacian}
  The \define{Laplacian} is a differential operator $\Delta$ defined
  for a function $u\in C^2$ in two dimensions by
  \begin{gather}
    \Delta u = \d_{xx} u + \d_{yy} u,
  \end{gather}
  and in three dimensions by
  \begin{gather}
    \Delta u = \d_{xx} u + \d_{yy} u + \d_{zz} u.
  \end{gather}
  Here, we use the notation for partial derivatives
  \begin{gather}
    \d_{x} = \tfrac{\d}{\d x},
    \qquad \d_{xx} = \tfrac{\d^2}{\d x^2},
    \qquad \d_{y} = \tfrac{\d}{\d y},
    \qquad\text{etc.}
  \end{gather}
\end{Definition}

\begin{Definition*}{poisson}{Poisson equation}
  Stationary diffusion processes are modeled in the simplest case by
  the equation
  \begin{gather}
    -\Delta u = f,
  \end{gather}
  where $u= u(\vx)$ describes the stationary distribution of the
  diffused quantity in space and $f$ is a source.

  In order to obtain a well-posed problem, this equation has to be
  complemented by suitable \putindex{boundary condition}s.
\end{Definition*}

\begin{Example}{Dirichlet-problem}
  On the \putindex{unit square} and \putindex{unit cube}
  $\Omega = [0,1]^d$ for $d=2,3$, the homogeneous \define{Dirichlet
    boundary value problem} for the Poisson equation reads
  \begin{xalignat}2
    -\Delta u &= f & \text{in } & \Omega,\\
    u &= 0 & \text{on } & \d\Omega.
  \end{xalignat}
\end{Example}

\begin{Definition}{fd-grid}
  On the unit square, we define a \define{finite difference grid}, by
  first choosing a number of points $n$ in each coordinate direction
  with \define{grid size} $h = \nicefrac1{n+1}$. We number these points with
  double indices $ij$, such that
  \begin{gather}
    \vx_{ij} =
    \begin{pmatrix}
      x_i\\y_i
    \end{pmatrix}
    =
    \begin{pmatrix}
      ih\\jh
    \end{pmatrix},
    \qquad i,j=1,\dots,n.
  \end{gather}
  These points cover the interior of the domain, since we assume that
  boundary conditions are given and no computation is necessary on the
  boundary. The same construction is possible in three dimensions with
  triple indices $ijk$ and
  \begin{gather}
    \vx_{ijk} =
    \begin{pmatrix}
      x_i\\y_j\\z_k
    \end{pmatrix}
    =
    \begin{pmatrix}
      ih\\jh\\kh
    \end{pmatrix},
    \qquad i,j,k=1,\dots,n.
  \end{gather}  
\end{Definition}

\begin{Definition}{fd-function}
  A \define{finite difference function} $\vu$ is a function with values in
  the points of the finite difference grid, namely
  \begin{gather}
    u_{ij} = u(\vx_{ij}), \qquad u_{ijk} = u(\vx_{ijk}).
  \end{gather}
\end{Definition}

\begin{Definition}{lexicographic}
  Since there are $n^d$ points in a finite difference grid, a finite
  difference function is characterized by $n^d$ values. Hence, it is a
  vector in $\R^{n^d}$. In order to index this vector in the usual
  way, we introduce \define{lexicographic numbering}, mapping the
  multi-indices $ijk$ to a single index $m$ by
  \begin{gather}
    m = i+nj+n^2k,
  \end{gather}
  and in two dimensions without the last term. The inverse is obtained by
  \begin{gather}
    i = m \bmod n,\quad j = (m/n) \bmod n,\quad k= m/n^2.
  \end{gather}
\end{Definition}

\begin{Definition}{fd-operators}
  Approximations to differential operators on finite difference
  functions are first defined for partial derivatives in one
  dimension. There are three commonly used operators for first
  derivatives, namely
  \begin{alignat}2
    &\text{forward difference: } & D_+ u(x_i) &= \frac{u_{i+1} - u_{i}}{h}\\
    &\text{backward difference: } & D_- u(x_i) &= \frac{u_{i} - u_{i-1}}{h}\\
    &\text{symmetric difference: } & D_s u(x_i) &= \frac{u_{i+1} - u_{i-1}}{2h}.
  \end{alignat}
  A second order difference operator can be obtained by
  \begin{gather}
    D^2u(x_i) = D_+D_-u(x_i) = D_-D_+u(x_i) = \frac{-2u_i + u_{i-1}+u_{i+1}}{h^2}.
  \end{gather}
\end{Definition}

\begin{remark}
  The finite difference approximation of the Laplacian on the finite
  difference grids defined in \slideref{Definition}{fd-grid} can be
  found in \slideref{Example}{7-point-stencil}. It is obtained by
  adding the second order difference operators for each coordinate
  direction.
\end{remark}

\begin{remark}
  The finite differenze operators acting on finite difference
  functions are linear operators. Hence, they can be represented by a
  matrix. We will build this matrix for the Dirichlet boundary value
  problem in \slideref{Example}{Dirichlet-problem} dimension by
  dimension.

  First, in one dimension, every point couples to its two neighbors by
  the second order difference operators. Our only problem is the lack
  of neighbors at the two ends of the interval. But, there we can add
  a virtual point on the boundary and employ the boundary
  condition. Thus, we obtain a matrix of dimension $n$ of the form
  \begin{gather}
    -D^2 \vu = \frac1{h^2} \matt_2 u :=
    \begin{pmatrix}
      2 & -1 \\
      -1 & 2 & -1\\
      &\ddots & \ddots & \ddots \\
      &&-1 & 2 & -1\\
      &&& -1 & 2
    \end{pmatrix}.
  \end{gather}

  In two dimensions, we have to add a second difference operator in
  $y$-direction. It has the same entries as the previous one, but the
  location of the off-diagonal entries is different. Using
  lexicographic numbering, every value in the interior couples with
  the ones $n$ before and $n$ after in the vector. Again, the
  couplings at the lower and upper boundary are missing. The result is
  summarized as follows:
  \begin{gather}
    -(D_x^2+D_y^2) \vu = \frac1{h^2} \matb_4 \vu :=
    \begin{pmatrix}
      \matt_4 & \id_n \\
      \id_n & \matt_4 & \id_n \\
      &\ddots & \ddots & \ddots \\
      &&\id_n & \matt_4 & \id_n \\
      &&&\id_n & \matt_4 \\
    \end{pmatrix},
  \end{gather}
  where $\id_n$ is the identity matrix of dimension $n$ and $\matt_4$
  is obtained from $\matt_2$ by replacing the diagonal element 2 by 4.

  This procedure can be extended to three dimensions in the same
  way. Now, the distance to the previous and next element in
  $z$-direction is $n^2$ in lexicographic numbering. Hence
  \begin{gather}
    -(D_x^2+D_y^2+D_z^2) \vu = \frac1{h^2}
    \begin{pmatrix}
      \matb_6 & \id_{n^2} \\
      \id_{n^2} & \matb_6 & \id_{n^2} \\
      &\ddots & \ddots & \ddots \\
      &&\id_{n^2} & \matb_6 & \id_{n^2} \\
      &&&\id_{n^2} & \matb_6 \\
    \end{pmatrix}.
  \end{gather}
  Independent of the dimension, we will call all these matrices
  $-\Delta_h$ and refer to them as the finite difference Laplacian.
\end{remark}

\begin{Example}{Dirichlet-discrete}
  The discrete form of the Dirichlet boundary value problem
  ~\slideref{Example}{Dirichlet-problem} is
  \begin{gather}
    -\Delta_h \vu = \vf,
  \end{gather}
  where $f_m = f(\vx_m)$ with the points $\vx$ in lexicographic ordering.
\end{Example}

\begin{Lemma}{Laplacian-eigenvalues-1d}
  The eigenvalues of the matrix $\matt_2$ are
  \begin{gather}
    \lambda_j = 2 - 2\cos(j\theta) = 4 \sin^2 \frac{j\theta}2,
    \quad j=1,\dots, n
  \end{gather}
  where $\theta = \nicefrac{\pi}{n+1}$. The corresponding eigenvectors are
  \begin{gather}
    \vv_j = \bigl(\sin(j\theta,\sin(2j\theta),\dots,\sin(nj\theta)\bigr)^T.
  \end{gather}
\end{Lemma}

\begin{Problem}{Laplacian-eigenvalues-1d}
  Verify the eigenpairs of \slideref{Lemma}{Laplacian-eigenvalues-1d}.
\end{Problem}

\begin{Lemma}{Laplacian-eigenvalues-2d}
  The eigenvalues of the matrix $\matb_4$ are obtained by taking all
  possible sums of two eigenvalues of $\matt_2$, that is, using double
  indices for the eigenvalues of $\matb_4$ and single indices for
  those of $\matt_2$,
  \begin{gather}
    \lambda_{ij} = \lambda_i+\lambda_j.
  \end{gather}
  The corresponding eigenvectors $\vv_{ij}$ are obtained as tensor
  products of the eigenvectors $\vv_i$ and $\vv_j$, namely
  \begin{gather}
    \vv_{ij}(\vx_{k\ell}) = \vv_i(x_k)\vv_j(y_\ell).
  \end{gather}
\end{Lemma}

%%% Local Variables:
%%% mode: latex
%%% TeX-master: "main"
%%% End:
