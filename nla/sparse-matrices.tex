\begin{intro}
  A simple method to approximate the solution to partial differential
  equations is the finite difference method. We provide a short
  introduction sufficient for the purpose in this class in
  Appendix~\ref{sec:fd}.
\end{intro}

\begin{Example}{7-point-stencil}
  Let there be a sequence of points $x_k$, $k=0,\dots,n$ such that
  $x_k-x_{k-1} = h$ and an approximating function $u_k = u(x_k)$. The
  second derivative of a function $u(x)$ in a point $x_k$ in the
  interior can be approximated by the 3-point stencil
  \begin{gather}
    u''(x_k) = \Delta_h^2 u(x_k) = - \frac{2 u_k - u_{k-1} - u_{k+1}}{h^2}.
  \end{gather}
  This can be generalized to the Laplacian in two dimensions by the \define{5-point stencil}
  \begin{gather}
    \Delta u_{ij} = -\frac{4 u_{ij}- u_{i-1,j} - u_{i+1,j}- u_{i,j-1} - u_{i,j+1}}{h^2},
  \end{gather}
  and to three dimensions by the \define{7-point stencil}
  \begin{multline}
    \Delta u_{ijk} = \tfrac{-1}{h^2}\Bigl(6 u_{ij}
    - u_{i-1,j,k} - u_{i+1,j,k}
    \\
    - u_{i,j-1,k} - u_{i,j+1,k}
    - u_{i,j,k-1} - u_{i,j,k+1}\Bigr).
  \end{multline}
\end{Example}

\begin{Example*}{page-rank}{The PageRank Algorithm}
  The page rank of a web page is computed from the Google Matrix
  \begin{gather}
    \matp = d (\matl+\vw \mathbf 1^T) + \tfrac{(1-d)}{n} \id,
  \end{gather}
  where $d \in (0,1)$ is a damping parameter, $\mathbf 1^T = (1,\dots,1)$,
  \begin{gather}
    l_{ij} =
    \begin{cases}
      \nicefrac1{c_i} & \text{if $i$ links to $j$}\\0&\text{else} 
    \end{cases},
    \qquad
    w_i =
    \begin{cases}
      1 &\text{if } c_i=1\\
      0&\text{else}
    \end{cases},
  \end{gather}
  and $c_i$ is the number of links from page $i$.

  The number of web pages is estimated at 4.86 billion pages\footnote{\url{https://www.worldwidewebsize.com/}}.
\end{Example*}

\begin{intro}
  A simple method to approximate the solution to partial differential
  equations is the finite difference method. We provide a short
  introduction sufficient for the purpose in this class in
  Appendix~\ref{sec:fd}.
\end{intro}

\begin{Example}{7-point-stencil}
  Let there be a sequence of points $x_k$, $k=0,\dots,n$ such that
  $x_k-x_{k-1} = h$ and an approximating function $u_k = u(x_k)$. The
  second derivative of a function $u(x)$ in a point $x_k$ in the
  interior can be approximated by the 3-point stencil
  \begin{gather}
    u''(x_k) = \Delta_h^2 u(x_k) = - \frac{2 u_k - u_{k-1} - u_{k+1}}{h^2}.
  \end{gather}
  This can be generalized to the Laplacian in two dimensions by the \define{5-point stencil}
  \begin{gather}
    \Delta u_{ij} = -\frac{4 u_{ij}- u_{i-1,j} - u_{i+1,j}- u_{i,j-1} - u_{i,j+1}}{h^2},
  \end{gather}
  and to three dimensions by the \define{7-point stencil}
  \begin{multline}
    \Delta u_{ijk} = \tfrac{-1}{h^2}\Bigl(6 u_{ij}
    - u_{i-1,j,k} - u_{i+1,j,k}
    \\
    - u_{i,j-1,k} - u_{i,j+1,k}
    - u_{i,j,k-1} - u_{i,j,k+1}\Bigr).
  \end{multline}
\end{Example}

\begin{Definition}{sparse-matrix}
  We call a matrix \textbf{sparse}\index{sparse matrix} in strict
  sense, if the number of nonzero entries is much less than the total
  number of entries, typically $\bigo(n)$ instead of $\bigo(n^2)$. The
  distribution of nonzero entries is called \define{sparsity pattern}.

  More broadly, a sparse matrix has the property that its application
  to a vector requires considerably less than $n^2$, typically
  $\bigo(n)$ operations. It also can be stored with considerably less
  than $n^2$ floating point numbers.
\end{Definition}

\begin{Remark}{sparse-inverse}
  Warning: neither the inverse of a sparse matrix is typically sparse, nor are
  its LU or QR factorizations!
\end{Remark}

\begin{Example*}{csr}{Compressed row storage (CSR)}
  A sparse matrix can be stored using two integer fields defining the
  sparsity pattern and a field of floating point values for its
  entries. Let \lstinline!n! be the dimension of the matrix and
  \lstinline!n_nonzero! the total number of nonzero entries.
  \begin{lstlisting}[language=Python]
    import numpy as np
    row_start = np.empty(n, dtype=np.uint)
    column = np.empty(n_nonzero, dtype=np.uint)
    entries = np.emtpy(n_nonzero, dtype=np.double)
  \end{lstlisting}

  The operation $\vy=\mata\vx$ is then implemented as
  \begin{lstlisting}[language=Python]
    for i in range(0,n):
    y[i] = 0.
    for j in range (row_start[i],row_start[i+1):
      y[i] += entries[j]*x[column[j]]
  \end{lstlisting}  
\end{Example*}

\begin{Remark}{algorithmic-matrix}
  In cases where the sparsity pattern and the entries of a matrix are
  known at compile time, a compressed stored matrix can be substituted
  by an algorithm performing the multiplication.

  Thus, we start viewing matrices more as linear operators than as
  rectangular schemes of numbers.

  Such algorithms are of high importance on modern hardware, where the
  time and also the energy cost of computations is much lower than of
  moving data from memory.
\end{Remark}

\begin{Example}{sparse-computation-memory}
  The matrix of the 7-point stencil on a grid of $N=n^3$ points has
  dimension $N$. For $n=100$, this is $N=10^6$. The whole matrix has
  $N^2=10^{12}$ entries, but only $7N \approx 10^7$ are nonzero.
  \begin{enumerate}
  \item Storing the full matrix in double precision requires
    \begin{gather}
     8\cdot 10^{12}\text{B} \approx 10\text{TB}. 
    \end{gather}
    Storing relevant information in
    CSR requires
    \begin{gather}
      12\cdot7\cdot 10^6\text{B} \approx 100\text{MB}.
    \end{gather}
  \item On a hypothetical CPU with $10^9$ multiplication/additions per second, multiplying a vector with this matrix takes
    \begin{gather}
      \text{Full: } 1000\text{sec} \approx 20\text{min},
      \qquad
      \text{CSR: } 7\text{msec}.
    \end{gather}
  \end{enumerate}
\end{Example}

\begin{Example}{sparse-computation-factorization}
  Solving a linear system with the matrix of the 7-point stencil on a $100^3$ grid
  \begin{enumerate}
  \item Full Choleski factorization
    \begin{gather}
      \frac16 \left(100^3\right)^3\text{flops}\approx 10^{17}\text{flops}
      \simeq 10^8\text{sec} \approx 3\text{yrs.}
    \end{gather}
%    \pause
  \item Banded Choleski factorization
    \begin{gather}
      \frac16 10^{16} \approx 10^{15}\text{flops}
      \simeq 10^6\text{sec} \approx 10\text{d}.
    \end{gather}
%    \pause
  \item Cramer's Rule with Laplace Expansion: $\approx 10^{65,000}$ years.
  \end{enumerate}
\end{Example}


\begin{Theorem*}{von-Neumann-series}{von Neumann series}
  If $\norm{\mata}<1$, then the right hand side of the following
  equation is convergent and
  \begin{gather}
    (\id-\mata)^{-1} = \sum_{k=0}^{\infty} \mata^k.
  \end{gather}
\end{Theorem*}

\begin{Algorithm}{von-Neumann-series}
  An approximative solution to the problem $\mata\vx = \vb$ can
  be obtained from an initial vector $\vx$ by the iteration
  \begin{algorithmic}
    \State $\vy \gets \vx$
    \Repeat
    \State $\vy \gets (\id-\mata) \vy$
    \State $\vx \gets \vx + \vy$
    \Until convergence
  \end{algorithmic}
  This algorithm converges for any initial vector if $\norm{\id-\mata}< 1$.
\end{Algorithm}

\begin{intro}
  While this algorithm serves as a first example that a linear system
  can be solved by successive multiplication with $\mata$, it
  typically slow. Furthermore, most matrices will not admit the
  condition. It is therefore our goal to find more efficient and
  reliable algorithms.
\end{intro}

%%% Local Variables:
%%% mode: latex
%%% TeX-master: "main"
%%% End:
