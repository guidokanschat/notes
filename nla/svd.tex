\subsection{Singular Value Decomposition (SVD)}

\begin{Definition}{svd}
  The \define{singular value decomposition} (\define{SVD}) of a matrix $\mata\in\C^{m\times n}$ is a facorization
  \begin{gather}
    \mata = \matu\matsigma\matv^*
  \end{gather}
  with unitary matrices $\matu\in\C^{m\times m}$ and $\matv\in\Cnn$ as
  well as a real, diagonal matrix $\matsigma$ with diagonal entries
  \begin{gather}
    \sigma_1 \ge \sigma_2 \ge \dots \ge \sigma_p \ge 0,
  \end{gather}
  where $p = \min\{m,n\}$.
\end{Definition}

\begin{Theorem}{svd}
  Every matrix $\mata\in\C^{m\times n}$ admits a singular value
  decomposition. Every real matrix admits a singular value
  decomposition with orthogonal matrices $\matu$, $\matv$.
\end{Theorem}

\begin{Corollary}{svd-rank}
  Let $\mata = \matu\matsigma\matv^*$ be the SVD of $\mata$ with
  \begin{gather}
    \sigma_1 \ge \dots \ge \sigma_r > \sigma_{r+1} = \dots = \sigma_p = 0.
  \end{gather}
  Then, $\rank \mata = r$
\end{Corollary}

\begin{Corollary}{svd-inverse}
  If $\mata\in\Cnn$ is invertible, then
  \begin{gather}
    \mata^{-1} = \matv\matsigma^{-1}\matu^*,
  \end{gather}
  where
  \begin{gather}
    \matsigma^{-1} = \diag\left(\frac1{\sigma_1},\dots,\frac1{\sigma_n}\right).
  \end{gather}
\end{Corollary}

\begin{Remark}{svd-geometry}
  Let
  \begin{gather}
    E = \bigl\{ \vy\in \R^m \big| \vy=\mata\vx, \norm{\vx}_2 = 1 \bigr\},
  \end{gather}
  be the ellipsoid obtained by mapping the unit sphere though
  $\mata$. Then, the column vectors of $\matu$ and the singular values
  $\sigma_i$ are the directions and lengths of the semi-axes of this
  ellipsoid, respectively.
\end{Remark}

\begin{Lemma}{svd-ata}
  The singular values of $\mata$ are the square roots of the
  eigenvalues of $\mata^*\mata$ and of $\mata\mata^*$, respectively. For $m\ge n$ there holds
  \begin{align}
    \matv^*(\mata^*\mata)\matv &= \diag(\sigma_1^2,\dots,\sigma_n^2)
    &&\in \R^{n\times n}\\
    \matu^*(\mata\mata^*)\matu &= \diag(\sigma_1^2,\dots,\sigma_n^2, 0,\dots,0)
    &&\in \R^{m\times m}
  \end{align}
\end{Lemma}

%%% Local Variables:
%%% mode: latex
%%% TeX-master: "main"
%%% End:
