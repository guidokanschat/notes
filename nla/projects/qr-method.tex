\documentclass[a4paper]{article}
\usepackage{amsmath}
%\usepackage[scale=5]{draftwatermark}
%\usepackage{background}

\addtolength{\textwidth}{2cm}
\addtolength{\oddsidemargin}{-1cm}

\setlength{\parindent}{0pt}
\setlength{\parskip}{1ex plus 1ex}

\begin{document}
\begin{center}
  \textbf{\Large Programming-Exam Numerical Linear Algebra}
  
  \textbf{Winter 2023/24}
\end{center}

\subsection*{Assignment}

Implement the QR-method with shifts and deflation for Hermitian
matrices and document the decisions you made in the implementation.

Verify your program with meaningful tests and document their results.

\subsection*{Formalities}

The program(s), documentation, and results should be submitted
electronically in a single PDF or a PDF plus one file with code by Feb
4th, 2023. You are welcome and encouraged to discuss your work with
your peers, but every student must submit their own, unique work.

As part of the exam, you will be asked to give a short oral
presentation of your work and you should be able to answer questions
about your code.  Oral presentations will be scheduled for the week of
Feb 5th, or later upon request.

Please submit at the oral test a signed declaration: ``I/We have
  prepared the assignment myself/ourselves and I/we have only used the
  sources declared in the submission''.

\subsection*{Guidelines}
\begin{enumerate}
\item The program must run with several example matrices without
  crashing and you must be able to change parameters like the matrix
  size
\item The program must be subdivided into functions of well-defined
  purpose
\item Your code should be well structured and readable
\item You must be able to describe how you verify the correctness of
  your program
\item The target dimension for matrices should be at least 20
\item Follow all suggestions from the notes on how to save memory and operations
\item Avoid complex numbers whenever possible
\item The QR iteration must be able to identify multiple eigenvalues
\item Use appropriate shift strategies and document
\item Document your deflation strategy
\item Bonus if you compute not only eigenvalues but also eigenvectors
\end{enumerate}

\subsection*{Hints}
\begin{enumerate}
\item Jupyter notebooks are a great way of producing this PDF, but
  they are not necessary.
\item The ``[:]'' notation for selecting column vectors or submatrices
  can be very helpful, if used the right way.
\end{enumerate}
\end{document}