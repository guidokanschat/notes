\chapter{Orthogonalization and QR Factorization}
\label{chap:qr}

\begin{intro}
  The facts in this chapter can be found in many textbooks on
  numerical linear algebra, for instance in~\cite[Chapter
  5]{GolubVanLoan83}. In particular, the QR factorization in section
  5.2, Housholder transformations and Givens rotations in 5.1 there.
\end{intro}

\section{The Gram-Schmidt Algorithm}

\begin{Algorithm*}{gram-schmidt}{Gram-Schmidt Orthogonalization}
  Let $\vx_1, \dots,\vx_k\in\C^n$ with $k\le n$ be linearly independent.

  For $j=1,\dots,k$
  \begin{align}
    \label{eq:ortho:gs:1}
    \vw_j &= \vx_j - \sum_{i=1}^{j-1} \scal(\vx_j,\vv_i)\vv_i\\
    \label{eq:ortho:gs:2}
    \vv_j &= \frac{\vw_j}{\norm{\vw_j}_2}.
  \end{align}
\end{Algorithm*}

\begin{Theorem}{gram-schmidt}
  If the input vectors $\vx_1, \dots,\vx_k\in\C^n$ are linearly
  independent, then the Gram-Schmidt orthogonalization generates an
  orthonormal set $\vv_1, \dots,\vv_k\in\C^n$. Furthermore, for
  $j=1,\dots,k$ there holds
  \begin{gather}
    \spann{\vv_1, \dots,\vv_j} = \spann{\vx_1, \dots,\vx_j}.
  \end{gather}
\end{Theorem}

\begin{Algorithm}{gram-schmidt-implementation}
  \slideref{Algorithm}{gram-schmidt} can be implemented in two mathematically equivalent ways:

  \hrulefill
  \vspace*{2mm}

  \begin{minipage}{.49\textwidth}
    \begin{algorithmic}[1]
      \For{$j=1,\dots,k$}
      \State $\vy = 0$
      \For{$i=1,\dots,j-1$}
      \State $r \gets \scal(\vx_j,\vv_i)$
      \State $\vy \gets \vy + r \vv_i$
      \EndFor
      \State $\vv_j = \frac{\vx_j-\vy}{\norm{\vx_j-\vy}_2}$
      \EndFor
    \end{algorithmic}
  \end{minipage}
  \begin{minipage}{.49\textwidth}
    \begin{algorithmic}[1]
      \For{$j=1,\dots,k$}
      \State $\vw \gets \vx_j$
      \For{$i=1,\dots,j-1$}
      \State $r \gets \scal(\vw,\vv_i)$
      \State $\vw \gets \vw- r \vv_i$
      \EndFor
      \State $\vv_j = \frac{\vw}{\norm{\vw}_2}$
      \EndFor
    \end{algorithmic}
  \end{minipage}
\end{Algorithm}


\begin{Example}{gram-schmidt-implementation}
  \begin{enumerate}
  \item Implement both methods in
    \slideref{Algorithm}{gram-schmidt-implementation} with double precision arithmetic.
  \item Apply them to the
    vectors $\vx_1,\dots,\vx_n\in\R^n$, where
    \begin{gather*}
      \vx_j = \left(\frac1j,\frac1{j+1},\dots,\frac1{j+n-1}\right)^T.
    \end{gather*}
  \item Compute the \define{Gramian matrix} $\matg$ with entries
    $g_{ij} = \scal(\vv_i,\vv_j)$ of the resulting vectors.
  \end{enumerate}
\end{Example}

\begin{Example}{hilbert-gram-schmidt}
  The Gramian matrix (rounded absolute values) of the orthonormal system generated by the
  Gram-Schmidt algorithm for $n=7$ is\small
  \begin{gather*}
    \begin{bmatrix}
      1. & 1\cdot 10^{-15} & 1\cdot 10^{-14} & 3\cdot 10^{-13} & 7\cdot 10^{-12} & 3\cdot 10^{-10} & 1\cdot 10^{-08}\\
      1\cdot 10^{-15} & 1. & 7\cdot 10^{-14} & 9\cdot 10^{-13} & 1\cdot 10^{-11} & 1\cdot 10^{-10} & 8\cdot 10^{-11}\\
      1\cdot 10^{-14} & 7\cdot 10^{-14} & 1. & 3\cdot 10^{-11} & 8\cdot 10^{-10} & 1\cdot 10^{-08} & 3\cdot 10^{-07}\\
      3\cdot 10^{-13} & 9\cdot 10^{-13} & 3\cdot 10^{-11} & 1. & 3\cdot 10^{-08} & 1\cdot 10^{-06} & 5\cdot 10^{-05}\\
      7\cdot 10^{-12} & 1\cdot 10^{-11} & 8\cdot 10^{-10} & 3\cdot 10^{-08} & 1. & 9\cdot 10^{-05} & 6\cdot 10^{-03}\\
      3\cdot 10^{-10} & 1\cdot 10^{-10} & 1\cdot 10^{-08} & 1\cdot 10^{-06} & 9\cdot 10^{-05} & 1. & 6\cdot 10^{-01}\\
      1\cdot 10^{-08} & 8\cdot 10^{-11} & 3\cdot 10^{-07} & 5\cdot 10^{-05} & 6\cdot 10^{-03} & 6\cdot 10^{-01} & 1.\\
    \end{bmatrix}
  \end{gather*}
\end{Example}

\begin{Example}{hilbert-modified-gram-schmidt}
  The Gramian matrix (rounded absolute values) of the orthonormal system generated by the
  modified Gram-Schmidt algorithm for $n=7$ is\small
  \begin{gather*}
    \begin{bmatrix}
      1. & 1\cdot 10^{-15} & 2\cdot 10^{-14} & 3\cdot 10^{-13} & 8\cdot 10^{-12} & 3\cdot 10^{-10} & 1\cdot 10^{-08}\\
      1\cdot 10^{-15} & 1. & 2\cdot 10^{-15} & 5\cdot 10^{-14} & 1\cdot 10^{-12} & 6\cdot 10^{-11} & 2\cdot 10^{-09}\\
      2\cdot 10^{-14} & 2\cdot 10^{-15} & 1. & 3\cdot 10^{-15} & 5\cdot 10^{-14} & 4\cdot 10^{-12} & 6\cdot 10^{-10}\\
      3\cdot 10^{-13} & 5\cdot 10^{-14} & 3\cdot 10^{-15} & 1. & 5\cdot 10^{-15} & 6\cdot 10^{-13} & 2\cdot 10^{-11}\\
      8\cdot 10^{-12} & 1\cdot 10^{-12} & 5\cdot 10^{-14} & 5\cdot 10^{-15} & 1. & 8\cdot 10^{-15} & 1\cdot 10^{-12}\\
      3\cdot 10^{-10} & 6\cdot 10^{-11} & 4\cdot 10^{-12} & 6\cdot 10^{-13} & 8\cdot 10^{-15} & 1. & 1\cdot 10^{-14}\\
      1\cdot 10^{-08} & 2\cdot 10^{-09} & 6\cdot 10^{-10} & 2\cdot 10^{-11} & 1\cdot 10^{-12} & 1\cdot 10^{-14} & 1.\\
    \end{bmatrix}
  \end{gather*}
\end{Example}

\begin{Example}{hilbert-modified-gram-schmidt-12}
  The last row of the same matrix for $n=12$ is\small
  \begin{multline*}
    2\cdot 10^{-01} \; 2\cdot 10^{-01} \; 1\cdot 10^{-02} \;
    3\cdot 10^{-03} \; 6\cdot 10^{-04} \; 1\cdot 10^{-05} \\
    3\cdot 10^{-06} \; 1\cdot 10^{-08} \;
    3\cdot 10^{-11} \; 6\cdot 10^{-12} \; 4\cdot 10^{-14} \; 1
  \end{multline*}
\end{Example}

\begin{remark}
  The previous example shows, that the Gram-Schmidt algorithm is
  highly susceptible to round-off errors. This is particularly true to
  the implementation on the left. The one on the right, often called
  \define{modified Gram-Schmidt algorithm}\index{Gram-Schmidt
    Orthogonalization!modified}, albeit it is the more natural
  implementation of the same \slideref{Algorithm}{gram-schmidt}.

  In our examples, we only showed the effect of round-off errors on
  orthogonality. Similarly, the fact that the span of the first $k$
  vectors before and after orthogonalization is affected.

  The vectors in \slideref{Example}{gram-schmidt-implementation} form
  the so-called \define{Hilbert matrix}, which is notoriously
  ill-conditioned. Nevertheless, the example shows that either
  orthogonalization is very hard or we have to search for a better
  algorithm.
\end{remark}

\begin{intro}
  If we give names $r_{ij} = \scal(\vx_j,\vv_j)$ for $i<j$ to the
  coefficients of the Gram-Schmidt algorithm and let
  $r_{jj} = \norm{\vw_j}_2$ in equation~\eqref{eq:ortho:gs:2}, then
  equation~\eqref{eq:ortho:gs:2} becomes
  \begin{gather*}
    \vx_j = r_{jj} \vv_j + \sum_{i=1}^{j-1} r_{ij} \vv_i
    = \sum_{i=1}^{j} r_{ij} \vv_i.
  \end{gather*}
  Moreover, with the matrix notation
  $\matx = (\vx_1,\dots,\vx_k)$ and $\matv=(\vv_1,\dots,\vv_k)$, we
  obtain the equation
  \begin{gather*}
    \matx = \matv \matr,
  \end{gather*}
  where $\matr \in \R^{k\times k}$ is the upper triangular matrix with
  entries $r_{ij}$ as defined above. This gives rise to the definition
  of the QR factorization of a matrix.
\end{intro}

\begin{Definition}{qr-decomposition}
  The \define{QR factorization} of a matrix $\mata\in\C^{m\times n}$
  with $m\ge n$ is the product
  \begin{gather}
    \mata = \matq\matr,
  \end{gather}
  such that $\matq \in\C^{m\times n}$ is unitary and
  $\matr\in \C^{n\times n}$ is upper triangular.
\end{Definition}

\begin{intro}
  The existence of a QR factorization of an invertible matrix $\mata$
  follows constructively from \slideref{Theorem}{gram-schmidt}. For a
  singular matrix, the column vectors are linearly dependent, which
  will result in $r_{jj}=0$ for some index $j$.
\begin{todo} % Indent on todo removed since it failed to compile otherwise
    QR for singular matrices?
\end{todo}
  The following lemma rephrases the same statement we have already
  learned about the Gram-Schmidt orthogonalization.
\end{intro}

\begin{Lemma}{qr-columns}
  Let $\mata = \matq\matr$ and let $\matr$ be invertible. Then, the
  column vectors of $\mata$ and $\matq$ admit the relation
  \begin{gather}
    \va_j = \sum_{i=1}^j r_{ij} \vq_i.
  \end{gather}
  If $r_{ii}\neq 0$ for $i=1,\dots,j$, this relation is uniquely
  invertible. In particular,
  \begin{gather}
    \spann{\vq_1,\dots,\vq_j}
    =
    \spann{\va_1,\dots,\va_j}.
  \end{gather}
\end{Lemma}

% GvL Theorem 5.2.1
\begin{Theorem}{qr-existence}
  Every matrix $\mata\in\C^{m\times n}$ with $m\ge n$
  admits a QR factorization.

  If the matrix $\mata$ has full rank, the columns of the matrix
  $\matq$ and the diagonal entries of $\matr$ are uniquely determined
  up to unit factors.

  The factorization can be made unique by the condition that for all
  $j$ there holds $r_{jj} > 0$.
\end{Theorem}

\begin{intro}
  The QR factorization can be used to solve linear systems. It will
  also be an important component of eigenvalue solvers in
  chapter~\ref{chap:dense-eigen}.

  It is thus important to find ways to compute QR factorizations in a
  numerically stable way. The remainder of this chapter is devoted to
  this question. The key will be to replace the projections of the
  Gram-Schmidt algorithm by orthogonal transformations, namely
  reflections and rotations.

  The construction of such methods starts with the idea that if I find
  an orthogonal transformation of the matrix $\mata$ such that
  $\matq^*\mata=\matr$ is upper triangular, then
  $\mata=\matq\matr$. What is left, is the construction of
  $\matq^*$. Note that by \slideref{Theorem}{qr-existence}, all
  possible matrices $\matq$ of such factorizations are essentially
  equal (columns are equal upto scalar factors of size unity).

  While these methods are often also subjects of introductory courses
  on numerical methods, we discuss them here in detail since we also
  need the complex versions.
\end{intro}

\begin{intro}
  \label{intro:ortho:1}
  The matrix $\matq$ of a QR factorization can be obtained by an
  algorithm traversing the columns of $\mata$. It progresses as
  follows: first, find a matrix $\matq_1$ such that $\matq_1^*\mata$
  has the structure
  \begin{gather}
    \mata_1 = \matq_1^* \mata =
    \begin{pmatrix}
      * & * & * & \dots & * & * \\
      0 & * & * & \dots & * & * \\
      0 & * & * & \dots & * & * \\
      0 & * & * & \dots & * & * \\
      \vdots & \vdots & \vdots & \cdots & \vdots & \vdots \\
      0 & * & * & \dots & * & *
    \end{pmatrix}.
  \end{gather}
  Hence, in the matrix $\mata_1$ all elements of the first column have
  been eliminated. The stars indicate unknown values usually not
  identical with the entries of $\mata$.

  In the next step, find a matrix $\matq_2$ such that
  \begin{gather}
    \mata_2 = \matq_2^* \mata_1
    =
    \begin{pmatrix}
      1 & 0 \\ 0 & \widetilde\matq_2^*
    \end{pmatrix}
    \mata_1
    =
    \begin{pmatrix}
      x & x & x & \cdots & x & x \\
      0 & * & * & \cdots & * & * \\
      0 & 0 & * & \cdots & * & * \\
      0 & 0 & * & \cdots & * & * \\
      \vdots & \vdots & \vdots & \cdots & \vdots & \vdots \\
      0 & 0 & * & \cdots & * & * \\
    \end{pmatrix}.
  \end{gather}
  Since the upper left block of $\matq_2^*$ is the
  identity, the multiplication from the left does not change the first
  row of $\mata_1$, such that the values indicated with $x$ are equal
  to the corresponding entries of $\mata_1$. This algorithm continues with
  \begin{gather}
    \matq_3^* \mata_2 =
    \begin{pmatrix}
      1&&\\
      &1&\\
      &&\widetilde\matq_3^*
    \end{pmatrix}
    \mata_2
        =
    \begin{pmatrix}
      x & x & x & \dots & x & x \\
      0 & x & x & \dots & x & x \\
      0 & 0 & * & \dots & * & * \\
      0 & 0 & 0 & \dots & * & * \\
      \vdots & \vdots & \vdots & \cdots & \vdots & \vdots \\
      0 & 0 & 0 & \dots & * & * \\
    \end{pmatrix}.
    ,
  \end{gather}
  until
  \begin{gather}
    \matq_{n-1}^*\mata_{n-2}
    =
    \begin{pmatrix}
      \id_{n-2}&\\
      &\widetilde\matq_{n-1}^*
    \end{pmatrix}\mata_{n-2}
    =
    \begin{pmatrix}
      x & x & x & \dots & x & x \\
      0 & x & x & \dots & x & x \\
      0 & 0 & x & \dots & x & x \\
      0 & 0 & 0 & \dots & x & x \\
      \vdots & \vdots & \vdots & \cdots & \vdots & \vdots \\
      0 & 0 & 0 & \dots & * & * \\
      0 & 0 & 0 & \dots & 0 & * \\
    \end{pmatrix} = \matr,
  \end{gather}
  where $\widetilde\matq_{n-1}$ is a 2-by-2 matrix.
\end{intro}

\begin{intro}
  \label{sec:ortho:qr:q-product}
  In the previous section, we computed $\matr$ as
  \begin{gather}
    \matr = \matq_{n-1}^* \cdots \matq_{2}^*\matq_{1}^*\mata,
  \end{gather}
  Thus, we get the product representation
  \begin{gather}
    \label{eq:ortho:1}
    \matq = \matq_{1} \matq_{2} \cdots \matq_{n-1}.
  \end{gather}
  Depending on the represention of the factors in this product, this
  product can be either computed as a matrix, or it can be ecaluated
  factor by factor.
\end{intro}

%%%%%%%%%%%%%%%%%%%%%%%%%%%%%%%%%%%%%%%%%%%%%%%%%%%%%%%%%%%%%%%%%%%%%%
\section{Householder transformation}
%%%%%%%%%%%%%%%%%%%%%%%%%%%%%%%%%%%%%%%%%%%%%%%%%%%%%%%%%%%%%%%%%%%%%%

\begin{Lemma}{projection-reflection}
  Given a vector $\vw\in\C^n$ and the hyperplane $V\subset \C$ which is orthogonal to $\vw$, the matrix
  \begin{gather}
    \id - \frac{\vw\vw^*}{\vw^*\vw}
  \end{gather}
  is the \define{projection matrix} onto $V$ and
  \begin{gather}
    \matq_{\vw} = \id - 2\frac{\vw\vw^*}{\vw^*\vw}
  \end{gather}
  is the matrix which computes the \define{reflection} of $\vw$ at $V$
  and is hence called \define{reflection matrix}.
\end{Lemma}

\begin{proof}
  THe projection o$\vy$ f an arbitrary vector $\vx$ onto the subspace
  spanned by $\vw$ is computed by
  \begin{gather}
    \vy = \frac{\scal(\vx,\vw)}{\norm{\vw}_2^2} \vw.
  \end{gather}
  This is a linear operation, which can be cast in matrix form as
  \begin{gather}
    \vy = \tfrac1{\norm{\vw}_2^2} \vw\vw^*.
  \end{gather}
  Since the projection $\vv$ of $\vx$ into $V$ is equal to the
  projection error of this projection, we have
  \begin{gather}
    \vv = \vx - \vy = \left(\id - \tfrac1{\norm{\vw}_2^2} \vw\vw^*\right).
  \end{gather}
  A simple sketch reveals that $\vx-2\vy$ yields the reflection of $\vx$ at $V$.
\end{proof}

\begin{Remark}{householder-inner}
  The proof of \slideref{Lemma}{projection-reflection} reveals that
  while we can write a reflection for mathematical convenience using
  the dyadic product $\vw\vw^*$, its application to a vector
  is achieved by an inner product and a scaling of a vector.
\end{Remark}

\begin{Lemma}{householder-symmetry}
  For any vector $\vw\in\C^n$ the reflection matrix
  $\matq_{\vw}$ is Hermitian and orthogonal, that is,
  \begin{gather}
    \matq_{\vw}^{-1} = \matq_{\vw}^* = \matq_{\vw}.
  \end{gather}
\end{Lemma}

\begin{proof}
  The complex symmetry follows immediately from the rule for the
  adjoints of matrix products. Furthermore,
  \begin{multline}
    \matq_{\vw}^2 = \left(\id - 2\frac{\vw\vw^*}{\vw^*\vw}\right)^2
    = \id - 4\frac{\vw\vw^*}{\vw^*\vw} + 4\frac{\vw\vw^*\vw\vw^*}{\vw^*\vw\vw^*\vw}
    \\
    = \id - 4\frac{\vw\vw^*}{\vw^*\vw} + 4\frac{\vw\vw^*}{\vw^*\vw} = \id.
  \end{multline}
\end{proof}

\begin{Definition*}{householder-transformation}{Householder transformation}
  The (real) \define{Householder transformation} is the reflection
  which transforms the vector $\vx$ to a multiple of the first unit
  vector $\ve_1$, that is, a vector of the form $(x_1,0,\dots,0)^T$.

    It is also called \define{Householder matrix} or, particularly in
  the real case, \define{Householder reflection}, since it reflects
  the vector $\vx$ at the subspace perpendicular to $\vw$.
\end{Definition*}

\begin{Lemma}{householder-qr}
  For any vector $\vx\in\C^n$ there are vectors $\vw_\phi\in\C^n$ such
  that $\matq_{\vw_\phi} \vx$ is a multiple of $\ve_1$.

  The vector of choice for numerical stability is
  \begin{gather}
    \label{eq:ortho:householder:1}
    \vw = \vx + e^{i\phi} \norm{\vx}_2\ve_1,
  \end{gather}
  where $\phi$ is the phase of $x_1$. We call this vector \define{Householder vector} for $\vx$.
\end{Lemma}

\begin{proof}
  The statement of the lemma says that for a suitable vector $\vw$
  there is a complex number $\alpha$ such that
  $\matq_{\vw} \vx = -\alpha \ve_1$, where the sign is chosen for
  later convenience. Since $\matq_{\vw}$ preserves the Euclidean norm,
  we already know
  \begin{gather}
    \abs{\alpha} = \norm{\vx}_2.
  \end{gather}
  We now construct the vector $\vw$. First, we want to achieve
  \begin{gather}
    \label{eq:ortho:householder:2}
    -\alpha\ve_1 = \matq_{\vw} \vx
    = \vx - 2 \frac{\vw\vw^*\vx}{\vw^*\vw}
    = \vx - 2 \frac{\vw^*\vx}{\vw^*\vw}\vw.
  \end{gather}
  Thus, $\vw$ must be in the span of $\vx+\alpha \ve_1$. Since we divide by
  its norm, its length does not matter and we let for some angle $\phi$
  \begin{gather}
    \vw_\phi =
    \begin{pmatrix}
      x_1 + e^{i\phi} \norm{\vx}_2\\x_2\\\vdots\\x_n
    \end{pmatrix}.
  \end{gather}
  If we compare the leftmost and rightmost terms
  in~\eqref{eq:ortho:householder:2}, we see that the quotient of inner
  products must be equal to $\nicefrac12$ such that $x_2,\dots,x_n$
  are canceled. Since
  \begin{align}
    \vw^*\vx &= \norm{\vx}_2^2 - x_1 e^{-i\phi} \norm{\vx}_2
    \\
    \vw^*\vw &= 2\norm{\vx}_2^2 - (x_1^*e^{i\phi} + x_1 e^{-i\phi})\norm{\vx}_2,
  \end{align}
  this can only hold if $x_1 e^{-i\phi}$ is real or equivalently, if
  $\phi$ is the phase of the complex number $x_1$ or of $-x_1$.

  Since the computation of the first component of $\vw_\phi$ is prone
  to loss of significance, we choose $\phi$ as the phase of $x_1$. In
  the real case, this simplifies to $e^{i\phi} = 1$.
\end{proof}

\begin{remark}
  \label{intro:ortho:householder-storage}
  The Householder vector $\vw$ is determined up to its length, which
  means we have the freedom to normalize it to some criterion when we
  compute it. Here are some options, which have been pursued in the
  literature
  \begin{enumerate}
  \item Use the implicit normalization which is obtained by the proof
    of its existence.
  \item Normalize its first entry to 1. This has been used to develop
    a compact storage scheme for the QR factorization.
  \item Normalize its norm to $\norm{\vw}_2=1$, such that the
    reflection matrix simplifies to $\matq_{\vw} =
    \id-2\vw\vw^*$. This can be improved even more by normalizing
    $\norm{\vw}_2=\sqrt2$ so that $\matq_{\vw} = \id-\vw\vw^*$.
  \end{enumerate}
  All of those may be more or less helpful depending of the circumstances. 
\end{remark}

\begin{Problem}{householder-compute}
  \begin{enumerate}
  \item Write a function \textsc{householder\_compute} which takes a
    vector $\vx$ and returns the normalized Householder vector
    $\vw$ with $\norm{\vw}_2=1$.
  \item Apply the function to the vector $\vx=(10,9,8,7,6,5,4,3,2,1)^T$.
  \item Test this function by implementing
    \slideref{Algorithm}{householder-multiplication} and applying the
    multiplication to the vector $\vx$ with $\beta = 2$.
  \end{enumerate}
\end{Problem}

\begin{Algorithm}{householder-multiplication}
  Computation of $\matq_{\vw}\vx$. The result is updated in the vector
  $\vx$.

  \hrulefill
  \vspace*{2mm}
  \begin{algorithmic}[1]
    \Function{householder}{$\vx,\beta,\vw$}
    \If{$\beta \neq 0$}
    \State $\omega \gets \beta \bigl(\vw^*\vx\bigr)$
    \State $\vx \gets \vx - \omega \vw$
    \EndIf
    \EndFunction
  \end{algorithmic}
   \hrulefill
  %\vspace*{2mm}

   This algorithm requires $2n+1$ multiplications and typically $2n$ additions
   compared to $n^2$ for a general matrix vector product.
\end{Algorithm}

\begin{Problem}{householder-product}
  According to section~\ref{sec:ortho:qr:q-product}, the matrix
  $\matq$ of the QR factorization is a product of Householder
  matrices. Is it better to compute $\matq$ as a matrix or is it
  better to store and apply its factors separately? To answer this
  question, consider the follwoing steps for an $n\times n$ matrix:
  \begin{enumerate}
  \item How much storage do you need for the factors $\matq_1$ to $\matq_{n-1}$?
  \item How many operations does it require to apply the product
    $\matq_1\cdots\matq_{n-1}$ to a vector?
  \item How many operations do you need to compute $\matq$ as a matrix?
  \item Compare the two options.
  \end{enumerate}
  Note in these calculations, that the Householder vector $\vw_k$ used
  for $\matq_k$ only requires $n-k$ entries.
\end{Problem}

% \begin{intro}
%   When applying the Householder transformation to the first column of
%   the matrix $\mata$, we already know that the subdiagonal entries of
%   the first column of the result will be zero. Therefore, we can make
%   use of this memory to store the Householder vector
%   $\vw(2:)$ right when we compute it. $\beta$ we have to store separately.

%   Then, we can apply $\matq_1$ to all other columns of $\mata$,
%   effectively computing $\matq\mata$. This scheme can be continued
%   towards the right and bottom of the matrix. In the end, the upper
%   triangle of $\mata$ contains $\matr$ and the strict lower triangle
%   contains the entries of the Householder vectors
%   $\vw_j$. Grahphically, the algorithm proceeds like this, where
%   $w_{j;i}$ is the $i$-th entry of the Householder vector $\vw_j$:
%   \begin{multline}
%     \begin{pmatrix}
%       a_{11} & \cdots &a_{1n}\\
%       \vdots&&\vdots\\
%       a_{n1}&\cdots&a_{nn}
%     \end{pmatrix}
%     \to
%     \begin{pmatrix}
%       r_{11} &r_{12} &r_{13} &r_{14} \\
%       w_{1;2} & * & * & * \\
%       w_{1;3} & * & *& * \\
%       w_{1;4} & * & *& * \\
%     \end{pmatrix}
%     \\
%     \to
%     \begin{pmatrix}
%       r_{11} &r_{12} &r_{13} &r_{14} \\
%       w_{1;2} & r_{22} & r_{23} & r_{24} \\
%       w_{1;3} & w_{2;2} & * & * \\
%       w_{1;4} & w_{2;3} & *& * \\
%     \end{pmatrix}
%     \to
%     \begin{pmatrix}
%       r_{11} &r_{12} &r_{13} &r_{14} \\
%       w_{1;2} & r_{22} & r_{23} & r_{24} \\
%       w_{1;3} & w_{2;2} & r_{33} & r_{34} \\
%       w_{1;4} & w_{2;3} & w_{3;2} & r_{44} \\
%     \end{pmatrix}
%   \end{multline}
% \end{intro}

%%%%%%%%%%%%%%%%%%%%%%%%%%%%%%%%%%%%%%%%%%%%%%%%%%%%%%%%%%%%%%%%%%%%%%
\section{Givens rotation}
%%%%%%%%%%%%%%%%%%%%%%%%%%%%%%%%%%%%%%%%%%%%%%%%%%%%%%%%%%%%%%%%%%%%%%

\begin{todo}
  Separate genaral rotation in a coordinate plane from Givens rotation
\end{todo}
\begin{Definition}{givens}
  The real \textbf{Givens rotation}\index{Givens rotation!real}
  $\givens_{jk}$ for $j<k$ is a matrix of the form
  \begin{gather}
      \givens_{jk} =
    \begin{pmatrix}
      \id \\
      &c&\cdots&s\\
      &\vdots&\id &\vdots\\
      &-s&\cdots&c\\
      &&&&\id
    \end{pmatrix}
    \in\Rnn.
  \end{gather}
  where $c^2+s^2 = 1$. A particular form is the rotation
  $\givens_{jk}[a,b]$, where $c$ and $s$ are chosen such that
  $cb+sa=0$. It is used such that $\givens_{jk}[x_j,x_k]^* \vx$ yields a
  vector with $x_k=0$.
\end{Definition}

\begin{remark}
  The Givens matrix acts as a rotation in the plane spanned by $\ve_j$
  and $\ve_k$ by the angle $\theta$ where $c=\cos\theta$ and
  $s=\sin\theta$. Nevertheless, the angle $\theta$ does not appear in
  applications.

  The entries $c$ and $s$ of the rotation matrix are in rows and
  columns $j$ and $k$.  The identity matrices $\id$ are of
  corresponding dimensions.

  Applying $\givens_{jk}$ or $\givens_{jk}^*$ to a matrix $\mata$ from
  the left modifies rows $j$ and $k$ of $\mata$. Thus, it is a row
  operation similar to Gauss elimination.

  The action of $\givens_{jk}$ on a vector corresponds to the rotation
  in the plane spanned by $\ve_j$ und $\ve_k$. Thus, it is sufficient
  to investigate $2\times2$-matrices.

  We often abbreviate $\givens_{jk} = \givens_{jk}[x_j,x_k]$.
\end{remark}

\begin{Remark}{givens-storage}
  The corresponding mapping $\givens_{jk}\colon x\mapsto y$ is defined by
  \begin{gather}
    y_i =
    \begin{cases}
      c x_j + s x_k & i=j\\
      -s x_j + c x_k & i=k\\
      x_i &\text{else}
    \end{cases}.
  \end{gather}
  Hence, we never create the Givens matrix in code, but only store the
  values of $c$, $s$, and the two indices.

  There is a compressed way of storing $c$ and $s$ of a Givens
  rotation in a single number, see~\cite[5.1.11]{GolubVanLoan83}.
\end{Remark}

\begin{Notation}{givens-specific}
  In the more general case, that a Givens rotation $\givens_{jk}^T$ is
  not chosen to eliminate $x_k$ using $x_j$, but more generally
  eliminating a number $b$ using a number $a$, we write more
  specifically
  \begin{gather}
    \givens_{jk}[a,b].
  \end{gather}
\end{Notation}

\begin{Lemma}{givens-computation}
  The real Givens rotation $\givens_{jk}^T$ eliminates the second
  component of the vector $(x_j,x_k)^T$ by choosing
  \begin{gather}
    r = \sqrt{x_j^2+x_k^2},\qquad
    c = \frac{x_j}r,\quad s = -\frac{x_k}r.
  \end{gather}
  We obtain
  \begin{gather}
    \begin{pmatrix}
      c & s \\ -s & c
    \end{pmatrix}^T
    \begin{pmatrix}
      x_j\\x_k
    \end{pmatrix}
    =
    \begin{pmatrix}
      r\\0
    \end{pmatrix}
    .
  \end{gather}
\end{Lemma}
\begin{remark}
  Some authors choose
  \begin{gather}
    c = -\frac{x_j}r,\quad s = \frac{x_k}r
    \qquad \Rightarrow \qquad
    \begin{pmatrix}
      c & s \\ -s & c
    \end{pmatrix}^T
    \begin{pmatrix}
      x_j\\x_k
    \end{pmatrix}
    =
    \begin{pmatrix}
      -r\\0
    \end{pmatrix}
    .
  \end{gather}
\end{remark}

\begin{remark}
  Computation of $r$ in the previous lemma is prone to numerical
  overflow due to computation of $x_j^2$ or $x_k^2$, even if $r$
  itself is within the numerical range. For the implementation, we can
  use the function \lstinline!hypot!. It computes the hypothenuse of a
  right-angled triangle without overflow.

  An alternative implementation avoids the overflow, but changes the
  sign in some cases.
\end{remark}

\begin{Algorithm}{givens-computation}
  \begin{algorithmic}[1]
    \Function{compute\_givens}{$a$,$b$}
    \If{$b=0$}
    \State $c=1$, $s=0$
    \Else
    \If{$\abs{b}>\abs{a}$}
    \State $\tau=-\frac ab$, $s=\frac1{\sqrt{1+\tau^2}}$, $c=\tau s$
    \Else
    \State $\tau=-\frac ba$, $c=\frac1{\sqrt{1+\tau^2}}$, $s=\tau c$
    \EndIf
    \EndIf
    \State \Return $(c,s)$
    \EndFunction
  \end{algorithmic}
\end{Algorithm}

\begin{Definition}{givens-complex}
  The complex \textbf{Givens-Rotation}\index{Givens rotation!complex} $\givens_{jk}$ for $j<k$ with angles $\varphi,\theta$ is the matrix
  \begin{gather}
      \givens_{jk} =
    \begin{pmatrix}
      \id \\
      &c&\cdots&s\\
      &\vdots&\id &\vdots\\
      &-\overline s&\cdots&c\\
      &&&&\id
    \end{pmatrix}
    \in\Cnn.
  \end{gather}
    where $c = \cos\theta$ and $s = e^{i\varphi}\sin\theta$.
\end{Definition}

\begin{Lemma}{givens-computation-complex}
  Let $u,v\in \C$ such that $u=u_1+iu_2$ and $v=v_1+iv_2$. Then, the second component of $(u,v)^T$ can be eliminated, such that
  \begin{gather}
    \begin{pmatrix}
      c & s\\-\overline s&c
    \end{pmatrix}^*
    \begin{pmatrix}
      u\\v
    \end{pmatrix}
    =
    \begin{pmatrix}
      r\\0
    \end{pmatrix}
  \end{gather}
  by three real Givens rotations
  \begin{gather}
    \begin{pmatrix}
      c_\alpha & s_\alpha\\-s_\alpha & c_\alpha
    \end{pmatrix}^T
    \begin{pmatrix}
      u_1\\u_2
    \end{pmatrix}
    =
    \begin{pmatrix}
      r_u\\0
    \end{pmatrix}
    ,\quad
    \begin{pmatrix}
      c_\beta & s_\beta\\-s_\beta & c_\beta
    \end{pmatrix}^T
    \begin{pmatrix}
      v_1\\v_2
    \end{pmatrix}
    =
    \begin{pmatrix}
      r_v\\0
    \end{pmatrix},
    \\
    \begin{pmatrix}
      c_\theta & s_\theta \\-s_\theta  & c_\theta
    \end{pmatrix}^T
    \begin{pmatrix}
      r_u\\r_v
    \end{pmatrix}
    =
    \begin{pmatrix}
      r\\0
    \end{pmatrix}
    ,
  \end{gather}
  choosing $\varphi = \beta-\alpha$ and $c=c_\theta$,
  $s=s_\theta e^{i\phi}$.
\end{Lemma}

\begin{proof}
  Note that the first two rotations map a complex number to a real number, hence
  \begin{gather}
    u = r_u e^{-i\alpha}, \qquad v = r_v e^{-i\beta}.
  \end{gather}
  Let now $\varphi = \beta-\alpha$, then,
  \begin{gather}
    \begin{aligned}
      c u - s v
      &= c_\theta e^{-i\alpha} r_u - s_\theta e^{i\phi} e^{-i\beta} r_v
      &&= e^{-i\alpha}\bigl(c_\theta r_u - s_\theta r_v\bigr) = r,
      \\
      c v + \overline s u
      &= c_\theta e^{-i\beta} r_v + s_\theta e^{-i\phi} e^{-i\alpha} r_u
      &&= e^{-i\beta}\bigl(c_\theta r_v + s_\theta r_u\bigr) = 0.
    \end{aligned}
  \end{gather}
\end{proof}

\begin{Lemma}{givens-computation-complex-2}
  The complex Givens rotation $\givens_{jk}^*$ eliminates the second
  component of the vector $(x_j,x_k)^T$ by choosing
  \begin{gather}
    r = \sqrt{\abs{x_j}^2+\abs{x_k}^2},\qquad
    c = \frac{|x_j|}r,\quad s = - \frac{x_j}{|x_j|}\frac{\overline{x}_k}r.
  \end{gather}
  We obtain
  \begin{gather}
    \begin{pmatrix}
      c & s \\ -\overline{s} & c
    \end{pmatrix}^*
    \begin{pmatrix}
      x_j\\x_k
    \end{pmatrix}
    =
    \begin{pmatrix}
      r\\0
    \end{pmatrix}
    .
  \end{gather}
\end{Lemma}

\begin{Algorithm}{givens-multiplication}
  The multiplication
  \begin{gather}
    \givens_{jk}^*\mata\qquad\qquad\qquad\qquad \mata\givens_{jk}
  \end{gather}
  affects only two rows/columns of $\mata\in\C^{m\times n}$ and can be
  implemented with $6n/6m$ multiplications and additions.

  \hrulefill
  \vspace*{2mm}

  \begin{minipage}{.49\textwidth}
    \begin{algorithmic}[1]
      \Require $c,s$ Givens rotation
      \For{$i=1,\dots,n$}
      \State $\alpha \gets a_{ji}$
      \State $\beta \gets a_{ki}$
      \State $a_{ji} \gets c\alpha-s\beta$
      \State $a_{ki} \gets \overline{s}\alpha+c\beta$
      \EndFor
    \end{algorithmic}
  \end{minipage}
  \begin{minipage}{.49\textwidth}
    \begin{algorithmic}[1]
      \Require $c,s$ Givens rotation
      \For{$i=1,\dots,m$}
      \State $\alpha \gets a_{ij}$
      \State $\beta \gets a_{ik}$
      \State $a_{ij} \gets c\alpha-\overline{s}\beta$
      \State $a_{ik} \gets s\alpha+c\beta$
      \EndFor
    \end{algorithmic}
  \end{minipage}
\end{Algorithm}

%%% Local Variables:
%%% mode: latex
%%% TeX-master: "main"
%%% End:
