%%%%%%%%%%%%%%%%%%%%%%%%%%%%%%%%%%%%%%%%%%%%%%%%%%%%%%%%%%%%%%%%%%%%%%

\begin{Problem}{intro-problem-vector-iterations}
	Consider the following matrix
	\begin{gather*}
	\mata =
	\begin{pmatrix}
	\cos\phi & -\sin\phi\\
	\sin\phi &  \cos\phi
	\end{pmatrix}^T
	\begin{pmatrix}
	1 & \\
	& c
	\end{pmatrix}
	\begin{pmatrix}
	\cos\phi & -\sin\phi\\
	\sin\phi &  \cos\phi
	\end{pmatrix}
	\end{gather*}
	with parameters $\phi\in[0,2\pi]$ and $c\in(0,1)$.
	\begin{enumerate}
		\item Determine (think, don't compute!) the eigenvalues and eigenvectors of $\mata$.
		\item (Programming) Write a program which computes the sequence
		$\vx^{(n)}\in\R^2$ defined as
		\begin{align*}
		\vx^{(n)} &= \mata \vx^{(n-1)}, \\
		\vx^{(0)} &= \vx^{*},
		\end{align*}
		for $\vx^{*} = (1,\ 0)^T$, $c = 0.1$, and
		$\phi=\frac\pi4$. Try playing with different values of those
		parameters.
		\item Is there a limit of $\vx^{(n)}$? What is about the case
		$c=1$?
		\item Compute the limit: $\lim_{n\to\infty}\mata^n$.
	\end{enumerate}
\end{Problem}


%%%%%%%%%%%%%%%%%%%%%%%%%%%%%%%%%%%%%%%%%%%%%%%%%%%%%%%%%%%%%%%%%%%%%%
\subsection{Simple iterations}
%%%%%%%%%%%%%%%%%%%%%%%%%%%%%%%%%%%%%%%%%%%%%%%%%%%%%%%%%%%%%%%%%%%%%%

\begin{intro}
  The eigenspace of a dominant eigenvalue seems to get amplified
  whenever we multiply with the matrix $\mata$. Hence, we should be a
  able to approximate this eigenspace by a simple iteration.
\end{intro}

\begin{Algorithm*}{vector-iteration}{Vector iteration (power method)}
  \begin{algorithmic}[1]
    \Require $\mata\in\Cnn$, initial vector $\vx^{(0)}\in\C^n$ with $\norm{\vx}=1$
    \For{$k=1,\dots$ until convergence}
    \State $\vy^{(k)} \gets \mata \vx^{(k-1)}$
    \State $\vx^{(k)} = \frac{\vy^{(k)}}{\norm{\vy^{(k)}}}$
    \State $\alpha_k = \frac{\bigl(\vy^{(k)}\bigr)_i}{\bigl(\vx^{(k)}\bigr)_i}$ where $\abs*{\bigl(\vx^{(k)}\bigr)_i} = \norm*{\vx^{(k)}}_\infty $
    \EndFor
%    \State $\alpha = \frac{\vy^{(k_\max +1)}}{\vv^{(k_{\max})}_i}$
%    where $\abs{\vv^{(k_{\max})}_i}$ is maximal
  \end{algorithmic}
\end{Algorithm*}

\begin{Theorem}{vector-iteration}
  Let $\mata\in\Cnn$ be diagonalizable such that $\lambda_1$ is the
  unique dominant eigenvalue. Let furthermore the component of
  $\vx^{(0)}$ in direction of the dominant eigenvector $\vv_1$ be
  nonzero. Then, the factors $\alpha_k$ and vectors $\vx^{(k)}$ of the
  vector iteration converge to the eigenvalue $\lambda_1$ and its
  associated eigenvector $\vv_1$. Moreover, there holds
  \begin{align}
    \abs{\alpha_{k+1}-\lambda_1}
    &\le \frac{\abs{\lambda_2}}{\abs{\lambda_1}} \abs{\alpha_{k}-\lambda_1}\\
    \norm{\vx^{(k+1)}-\vv_1}
    &\le \frac{\abs{\lambda_2}}{\abs{\lambda_1}} \norm{\vx^{(k)}-\vv_1}
  \end{align}
\end{Theorem}

\begin{remark}
  The previous theorem actually holds eveon for matrices, which are
  not diagonalizable, but then the proof becomes more involved. Since
  we are not interested in eigenvalues, which are not semi-simple, we
  decided for the simpler version.

  The proof can be easily improved to cover the case that $\lambda_1$
  is not a single eigenvalue. In exact arithmetic, there will be no
  change of the result. When computing with floating point numbers,
  there will be a small glitch. It is left to the reader to find it.
\end{remark}

\begin{remark}
  The algorithm can be improved in several ways. First, we saw in the
  proof that it generates a converging sequence of vectors. Therefore,
  we can save the effort of computing eigenvalues inside the iteration
  and postpone it until the end.

  Second, if $\mata$ is a Hermitian matrix, the computation of the
  eigenvalue can be improved by using the Rayleigh quotient.

  Using both statements, we obtain an optimized version of the vector
  iteration.
\end{remark}

\begin{Algorithm*}{vector-iteration-rayleigh}{Power method (optimized for Hermitian matrices)}
  \begin{algorithmic}[1]
    \Require $\mata\in\Cnn$ Hermitian, initial vector $\vv^{(0)}\in\C^n$ with $\norm{\vv}=1$
    \For{$k=1,\dots$ until convergence}
    \State $\vy^{(k)} \gets \mata \vv^{(k-1)}$
    \State $\vv^{(k)} = \frac{\vy^{(k)}}{\norm{\vy^{(k)}}}$
    \EndFor
    \State $\lambda_1 \approx R_\mata(\vv^{(k_{\max})} = \scal(\mata\vv^{(k_{\max})},\vv^{(k_{\max})})$
  \end{algorithmic}
\end{Algorithm*}

\begin{Remark}{vector-iteration}
  The result does not actually require that $\mata$ is diagonalizable,
  as long as $\lambda_1$ is single and of largest modulus. But this
  just complicates the proof and we want to focus on diagonalizable
  matrices anyway.
\end{Remark}

\begin{Remark}{vector-iteration-roundoff}
  The convergence proof of the power iteration requires that the
  initial vector has a component in the direction of the dominant
  eigenvector. Nevertheless, convergence to the dominant eigenvector
  is still observed if this condition does not hold. This is due to
  round-off errors and the exponential growth of rate
  $\nicefrac{\abs{\lambda_2}}{\abs{\lambda_1}}$.
\end{Remark}

\begin{Algorithm*}{shifted-vector-iteration}{Shifted vector iteration}
  The vector iteration can be applied to the matrix $\mata-\sigma\id$
  for some $\sigma\in\C$.

  Then, $\alpha_k$ converges to the eigenvalue $\lambda$ such that
  $\lambda-\sigma$ has largest modulus. $v^{(k)}$ converges to an
  eigenvector for this eigenvalue.
\end{Algorithm*}

%\begin{intro}
%  \begin{todo}
%    Remove!
%  \end{todo}
%\end{intro}

\begin{Definition*}{shifted-matrix-polynomial}{Matrix Polynomial}
  Let \(\mata \in \C^{n \times n}\) be a square matrix.
  The \define{matrix polynomial} of degree \(k\), with shift parameters \(\sigma_1, \ldots, \sigma_k\) is defined by
  \begin{gather}
    p(\mata) := \left(\mata - \sigma_1 \id \right) \cdots \left(\mata - \sigma_k \id \right).
  \end{gather}
\end{Definition*}

\begin{Lemma}{matrix-polynomial}
  Let $\mata$ be diagonalizable and $p(\mata)$ be a matrix
  polynomial. Then, $p(\mata)$ is diagonalizable. Moreover, a vector
  $\vv$ is an eigenvector for $p(\mata)$ and eigenvalue $p(\lambda)$,
  if and only if it is an eigenvector for $\mata$ and eigenvalue $\lambda$.
\end{Lemma}

\begin{Lemma*}{polynomial-filtering}{Polynomial filtering}
  Let $\mu_1,\dots,\mu_k$ be approximations to \putindex{unwanted
    eigenvalues} $\lambda_1,\dots,\lambda_k$ of the matrix
  $\mata$. Define the filtering polynomial
  \begin{gather}
    p(x) = (x-\mu_1)\cdots(x-\mu_k).
  \end{gather}
  Then, the corresponding eigenvalues
  $p(\lambda_1),\dots,p(\lambda_k)$ of the matrix polynomial
  $p(\mata)$ will be small and the power method converges to other
  eigenvalues.
\end{Lemma*}

\begin{Example}{polynomial-filtering}
  Let
  \begin{gather}
    \mata =
    \begin{pmatrix}
      1&&\\ &2& \\ &&3
    \end{pmatrix}.
  \end{gather}
  \begin{enumerate}
  \item The eigenvalue $\lambda=3$ can be computed by the power method.
  \item The eigenvalue $\lambda=1$ can be computed by the shifted power method with $\sigma>2$.
  \item The eigenvalue $\lambda=2$ cannot be computed by the shifted power method, but we can use
    \begin{gather}
      p(\mata) = (\mata-\id)(\mata-3\id),
    \end{gather}
    which has eigenvalues 0 and -1. Since -1 is dominant, the power
    method converges to its eigenvector in a single step.
  \end{enumerate}
\end{Example}

\begin{Algorithm*}{polynomial-filtering}{Power method with polynomial filtering}
  \begin{algorithmic}[1]
    \Require $\mata\in\Cnn$ Hermitian, initial vector $\vv^{(0)}\in\C^n$ with $\norm{\vv}=1$
    \For{$k=1,\dots$ until convergence}
    \State $\vy^{(k)} \gets p(\mata) \vv^{(k-1)}$
    \State $\vv^{(k)} = \frac{\vy^{(k)}}{\norm{\vy^{(k)}}}$
    \EndFor
    \State $\lambda_1 \approx R_\mata(\vv^{(k_{\max})} = \scal(\mata\vv^{(k_{\max})},\vv^{(k_{\max})})$
  \end{algorithmic}
\end{Algorithm*}

\begin{Remark}{polynomial-filtering}
  Each step of the power method with polynomial filtering costs as
  much work as $m$ steps of the shifted power method, if we do not
  count the work for checking convergence. Hence, it can also be seen
  as a power method with variable shifts, cycling through the linear
  factors of the polynomial.
\end{Remark}

\begin{Problem}{polynomial-filtering}
  Devise an algorithm using the power method and matrix polynomial
  which computes all distinct eigenvalues of a matrix, one after the
  other.

  Discuss pros and cons of this algorithm.
\end{Problem}

%%%%%%%%%%%%%%%%%%%%%%%%%%%%%%%%%%%%%%%%%%%%%%%%%%%%%%%%%%%%%%%%%%%%%%
\subsection{Subspace iterations}
%%%%%%%%%%%%%%%%%%%%%%%%%%%%%%%%%%%%%%%%%%%%%%%%%%%%%%%%%%%%%%%%%%%%%%

\begin{intro}
  Instead of approximating eigenvalues consecutively by filtering
  previously converged eigenvalues in the power method, we can
  consider iterating not only a single vector, but instead iterating a
  higher dimensional subspace.

  If we do so, starting with a linearly independent set of $m$ vectors
  $\matx^{(0)} \in \C^{m\times m}$, we can produce a sequence
  $\matx^{(k)}$, for $k=1,\dots$ Note that if we normalize each column
  vector separately, they will all converge to the dominant
  eigenvector. Thus, we have to put in a mechanism such that these
  vectors cannot become parallel. The solution is iterating over an
  orthonormal set of vectors $\matq^{(k)}$.
\end{intro}

\begin{Algorithm*}{subspace-iteration}{Orthogonal subspace iteration}
  \begin{algorithmic}[1]
    \Require $\mata\in\Cnn$ and choose $\matq_0 \in \C^{n\times m}$ orthogonal.
    \For {$k=1,\ldots$ until convergence}
    \State $\maty_{k-1} \gets \mata \matq_{k-1}$
    \State $\matq_k\matr_k \gets \maty_{k-1}$ \Comment{QR factorization}
    \EndFor
    \State {$\lambda_i \approx \vq_{k_{\max},i}^* \mata \vq_{k_{\max},i},\qquad i=1,\dots,m$}\
    \Comment{Rayleigh quotient}
  \end{algorithmic}
\end{Algorithm*}

\begin{Theorem}{convergence-subspace-iteration}
  Let $\mata\in\Cnn$ and
  \begin{gather}
    \abs{\lambda_1} \ge
    \abs{\lambda_2} \ge \dots \ge
    \abs{\lambda_m}>\abs{\lambda_{m+1}} \ge \dots \ge \abs{\lambda_n}.
  \end{gather}
  Choose $\matq_0 \in \C^{n\times m}$ such that
  $\dist(D_m(\mata),\ran{\matq^{(0)}}) < 1$, where $D_m(\mata)$ is the
  \putindex{dominant invariant subspace} of $\mata$ of dimension
  $m$. Then,
  \begin{gather}
    \dist\left(D_m(\mata),\ran{\matq^{(k)}}\right)
    = \bigo \left(\abs*{\frac{\lambda_{m+1}}{\lambda_m}}^k\right).
  \end{gather}
  The constant depends on the separation of the eigenvalues and the deviation of $\mata$ from normality.
%   Then, the $j$-th column of $\matx_k$ converges to $\vq_j$ for $j=1,\dots,m$ up to a factor $e^{i\phi}$.
\end{Theorem}

\begin{proof}
  See~\cite[Theorem 7.3-1]{GolubVanLoan83}.
\end{proof}

\begin{Corollary}{convergence-subspace-iteration}
  The orthogonal subspace iteration converges whenever the first $m$
  eigenvalues are dominant. Moreover, whenever the first $j<m$
  eigenvalues are dominant, then the first $j$ columns of
  $\matq^{(k)}$ converge to the \putindex{dominant invariant subspace}
  of dimension $j$. The latter statement holds even if $m=n$.
\end{Corollary}

%%% Local Variables:
%%% mode: latex
%%% TeX-master: "main"
%%% End:
