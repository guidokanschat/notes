\subsection{Excursion to finite difference methods}

\frame {\input {blocks/Definition-Laplacian.tex}}
\frame {\input {blocks/Definition-poisson.tex}}
\frame {\input {blocks/Example-Dirichlet-problem.tex}}
\frame {\input {blocks/Definition-fd-grid.tex}}
\frame {\input {blocks/Definition-fd-function.tex}}
\frame {\input {blocks/Definition-lexicographic.tex}}
\frame {\input {blocks/Definition-fd-operators.tex}}
\frame {\input {blocks/Example-Dirichlet-discrete.tex}}

\begin{frame}{The Laplacian in 1D}
  \begin{block}{The 3-point stencil}
    \begin{gather}
      -D^2 \vu = \frac1{h^2} \matt_2 u :=
      \begin{pmatrix}
        2 & -1 \\
        -1 & 2 & -1\\
        &\ddots & \ddots & \ddots \\
        &&-1 & 2 & -1\\
        &&& -1 & 2
      \end{pmatrix}.
    \end{gather}
  \end{block}
\end{frame}

\frame {\input {blocks/Example-7-point-stencil.tex}}

\begin{frame}{The Laplacian in 2D}
  \begin{block}{5-point stencil}
      \begin{gather}
    -(D_x^2+D_y^2) \vu = \frac1{h^2} \matb_4 \vu :=
    \begin{pmatrix}
      \matt_4 & \id_n \\
      \id_n & \matt_4 & \id_n \\
      &\ddots & \ddots & \ddots \\
      &&\id_n & \matt_4 & \id_n \\
      &&&\id_n & \matt_4 \\
    \end{pmatrix}.
  \end{gather}
  \end{block}
\end{frame}

\begin{frame}{The Laplacian in 2D}
  \begin{block}{7-point stencil}
  \begin{gather}
    -(D_x^2+D_y^2+D_z^2) \vu = \frac1{h^2}
    \begin{pmatrix}
      \matb_6 & \id_{n^2} \\
      \id_{n^2} & \matb_6 & \id_{n^2} \\
      &\ddots & \ddots & \ddots \\
      &&\id_{n^2} & \matb_6 & \id_{n^2} \\
      &&&\id_{n^2} & \matb_6 \\
    \end{pmatrix}.
  \end{gather}
  \end{block}
\end{frame}

\frame {\input {blocks/Lemma-Laplacian-eigenvalues-1d.tex}
  \pause
  \input {blocks/Problem-Laplacian-eigenvalues-1d.tex}}
\frame {\input {blocks/Lemma-Laplacian-eigenvalues-2d.tex}}

\subsection{Motivation and sparse matrices}
\frame{\subtoc}
\frame {\input {blocks/Example-page-rank.tex}}
\frame {\input {blocks/Definition-sparse-matrix.tex}
  \pause
  \input {blocks/Remark-sparse-inverse.tex}}
\frame {\input {blocks/Example-csr.tex}}
\frame {\input {blocks/Remark-algorithmic-matrix.tex}}
\frame {\input {blocks/Example-sparse-computation-memory.tex}}
\frame {\input {blocks/Example-sparse-computation-factorization.tex}}
\frame {\input {blocks/Theorem-von-Neumann-series.tex}
  \pause
  \input {blocks/Algorithm-von-Neumann-series.tex}}

\subsection{Basic iterations}
\frame{\subtoc}
\frame {\input {blocks/Definition-jacobi.tex}
  \input {blocks/Definition-gauss-seidel.tex}}
\frame {\input {blocks/Definition-richardson-iteration.tex}}
\frame {\input {blocks/Definition-matrix-iteration.tex}}
\frame {\input {blocks/Lemma-Jacobi-gs-matrices.tex}}

\frame {\input {blocks/Theorem-bfpt.tex}}
\frame {\input {blocks/Corollary-bfp-estimates.tex}}
\frame {\input {blocks/Corollary-matrix-norm-convergence.tex}}
\frame {\input {blocks/Example-convergence-row-sum.tex}}
\frame {\input {blocks/Example-matrix-norm-convergence.tex}}

\frame {\input {blocks/Definition-spectral-radius.tex}
  \pause
  \input {blocks/Lemma-spectral-radius.tex}}
\frame {\input {blocks/Theorem-matrix-radius-convergence.tex}}
\frame {\input {blocks/Remark-contraction-vs-convergence.tex}}
\frame {\input {blocks/Theorem-richardson-convergence.tex}}
\frame {\input {blocks/Remark-richardson-convergence.tex}}
\frame {\input {blocks/Definition-convergence-rate-logarithmic.tex}}
\frame {\input {blocks/Corollary-convergence-rate-logarithmic.tex}}
\frame {\input {blocks/Definition-convergence-rate-observed.tex}}


%%% Local Variables:
%%% mode: latex
%%% TeX-master: "slides"
%%% End:
