\usepackage{algpseudocode}
\lstset{language=Python,numbers=left,resetmargins=true,xleftmargin=8pt,basicstyle=\small,numberstyle=\scriptsize}
\usetikzlibrary{svg.path}
\excludecomment{solution}
\tikzset{shape veloxy/.style={color=black,draw,fill=red,thick}}
\tikzset{shape pressure/.style={color=black,draw,fill=cyan,thick}}


\title{Numerical Linear Algebra}
\author{Guido Kanschat}
\date{\today}

\def\esp#1{V_{#1}}

\begin{document}
\maketitle
\tableofcontents
\chapter{Dense Algebraic Eigenvalue Problems}
\begin{intro}
  We refer to problems in linear algebra which allow storing the
  complete matrix as \define{dense linear algebra}. They are typically
  characterized by dimensions into the hundreds, possibly thousands,
  and any entry in the matrix may have a nonzero value. Such matrices
  are typically stored as a rectangular or quadratic array of numbers,
  and we can perform manipulations based on the matrix entry.

  In contrast, we will turn to \define{sparse linear algebra} in the
  later chapters, where dimensions got up to millions and billions
  ($10^9$). Since currently no computer on earth can store a matrix
  with $10^{18}$ entries, those matrices will be characterized by the
  fact that each row only contains very few nonzero entries, or that
  the matrix is not stored, but only available algorithmically in the
  form of a function performing the action $\vx\mapsto \mata
  \vx$. Thus, access to and manipulation of matrix entries is not
  possible, and we have to focus on methods only using the properties
  of the matrix as a linear mapping.
\end{intro}

\section{Mathematical background}
\subsection{Definition of Eigenvalue Problems}
\begin{intro}
  The following results can be found in any book on linear
  algeba. Thus, we will just keep the arguments short. There will be a
  focus on normal matrices justified by results on conditioning of
  eigenvalue problems later on.

  Thus, spectral theory based on module theory will not be needed in
  this class. The spectral theorem for normal matrices on the other
  hand is fairly simple and can be proved without too much overhead.
\end{intro}

\begin{Definition}{eigenvalue}
  An \define{eigenvalue} of a matrix $\mata\in \C^{n\times n}$ is a
  complex number $\lambda$ such that the matrix
  \begin{gather}
   \mata-\lambda\id 
  \end{gather}
  is singular.

  The set of all eigenvalues of $\mata$ is called the
  \define{spectrum} $\sigma(\mata)$.

  The \define{eigenspace} for $\lambda$ is the kernel of
  $A-\lambda\id$, that is, the set
\begin{gather}
    \esp{\lambda} = \bigl\{
    \vv \in \C^n \;\big\vert\;
    \mata\vv = \lambda\vv \bigr\}.
\end{gather}
The \define{geometric multiplicity} of $\lambda$ is the dimension of
$\esp{\lambda}$.


An \define{eigenvector} for $\lambda$ is a (normed) vector in
$\esp\lambda$. We refer to an eigenvector $\lambda$ and a
corresponding eigenvector $\vv$ as \define{eigenpair}.
\end{Definition}

\begin{Definition}{eigenvalue-algebraic}
  An \define{eigenvalue} of a matrix $\mata\in \C^{n\times n}$ is a root of the characteristic polynomial $\chi(\lambda) = \det(\mata-\lambda\id)$.
  
  The \define{algebraic multiplicity} of an eigenvalue is the multiplicity of the corresponding root of the characteristic polynomial.
\end{Definition}

\begin{Lemma}{eigenvalue-equivalent}
  The two definitions of an eigenvalue are consistent.
\end{Lemma}

\begin{Theorem}{eigenvalue-count}
  Every matrix in $\C^{n\times n}$ has at most $n$ eigenvalues. The algebraic multiplicities of all eigenvalues add up to $n$.
\end{Theorem}

\begin{proof}
  The ``at most'' follows from the fact that a polynomial contains
  linear factors $x-\lambda_i$ for each of its roots
  $\lambda_i$. Thus, if the characteristic polynomial has $k$ roots it
  has at least degree $k$. On the other hand, the characteristic
  polynomial has degree $n$, such that $k\le n$.

  The second statement is due to the fact that every polynomial over
  $\C$ is a product of linear factors.
\end{proof}

\begin{remark}
  The last theorem is not true in $\R$, as it is not algebraically
  closed. Thus, even a real matrix may have complex eigenvalues and
  eigenvectors. Therefore, all results in this chapter will be on
  complex matrices, but some simplifications for real matrices will be
  pointed out.
\end{remark}

\begin{Definition}{eigenvalue-simple}
  An eigenvalue is \define{simple}, if its algebraic and geometric multiplicity are one. It is \define{semi-simple}, if its algebraic and geometric multiplicities are equal.
\end{Definition}

\begin{Definition}{right-left-ev}
  Refining \slideref{Definition}{eigenvalue}, we distinguish between a right eigenvector $\vv$ such that
  \begin{gather*}
    \mata \vv = \lambda \vv,
  \end{gather*}
  and a left eigenvector $\vu$ such that
  \begin{gather*}
    \vu^T \mata = \lambda \vu^T.
  \end{gather*}
  
\end{Definition}

\begin{Lemma}{eigenvalues-conjugate}
  Every eigenvalue $\lambda$ of $\mata\in\C^{n\times n}$ is also an
  eigenvalue of $\mata^T$. Furthermore, a left eigenvector of $\mata$
  is a right eigenvector of $\mata^T$ for the same eigenvalue and vice
  versa.
\end{Lemma}

\begin{proof}
  The determinant does not change when the matrix is transposed, therefore
  \begin{gather}
    \chi(\mata^T)
    = \det(\mata^T-\lambda \id)
    = \det(\mata-\lambda \id)
    = \chi(\mata).
  \end{gather}
  Thus, the eigenvalues of $\mata$ and of $\mata^T$ coincide.
\end{proof}

\subsection{Normal and Hermitian matrices}

\begin{Definition}{normal-Hermitian}
  A matrix $\mata\in\C^{n\times n}$ is called \define{normal} if there holds
  \begin{gather}
      A^*A = AA^*.
  \end{gather}
  It is called \define{Hermitian} or \define{complex symmetric}, if there holds
  \begin{gather}
      A=A^*.
  \end{gather}
\end{Definition}

\begin{Lemma}{Hermitian-eigenvalues-real}
  All eigenvalues of a Hermitian matrix are real.
\end{Lemma}

\begin{Theorem*}{Hermitian-diagonalizable}{Spectral theorem for Hermitian matrices}
  A Hermitian matrix $\mata\in\C^{n\times n}$ is diagonalizable with
  an orthogonal basis of eigenvectors and real eigenvalues. That is,
  there is a real, diagonal matrix $\matlambda$ and a unitary matrix
  $\matq$ such that
  \begin{gather}
    \mata = \matq^T \matlambda\matq.
  \end{gather}
\end{Theorem*}

%\begin{proof}
%\end{proof}

\begin{Corollary}{symmetric-diagonalizable}
  A symmetric matrix $\mata\in\R^{n\times n}$ is diagonalizable with
  an orthogonal basis of eigenvectors and real eigenvalues.
\end{Corollary}

\begin{Theorem*}{normal-diagonalizable}{Spectral theorem for normal matrices}
  A matrix $\mata\in\C^{n\times n}$ is normal if and only if it is diagonalizable by a unitary matrix.
  
  It is normal if and only if there exists an orthonormal basis of eigenvectors.
\end{Theorem*}

\begin{proof}
  
\end{proof}



%%% Local Variables:
%%% mode: latex
%%% TeX-master: "main"
%%% End:

\section{Well-posedness of the EVP and bounds on eigenvalues}
\subsection{Bounds on eigenvalues}

\begin{Lemma}{bound-by-norm}
  Let $\norm{\cdot}$ be a vector norm and denote by the same symbol
  a consistent norm for matrices. Then, for any matrix $\mata\in\Cnn$
  and for any eigenvalue $\lambda\in\sigma(\mata)$ there holds
  \begin{gather}
    \abs{\lambda} \le \norm{\mata}.
  \end{gather}
\end{Lemma}

\begin{Lemma}{pre-gershgorin}
  Let $\mata,\matb\in\Cnn$ and let $\norm{\cdot}$ be an operator norm
  on the space of matrices corresponding to a vector norm denoted by
  $\norm{\cdot}$ as well. Then, for any eigenvalue
  $\lambda\in\sigma(\mata)$ such that $\lambda\not\in\sigma(\matb)$
  there holds
  \begin{gather}
    \norm*{(\lambda\id-\matb)^{-1}(A-B)} \ge 1.
  \end{gather}
\end{Lemma}


\begin{Theorem*}{gershgorin}{Gershgorin circle theorem}
  All eigenvalues of a matrix $\mata\in\Cnn$ are contained in the
  union of the \define{Gershgorin Circle}s
  \begin{gather}
    G_j = \left\{ z\in \C \middle| \abs{z-a_{jj}} \le \sum_{k\neq j} \abs{a_{jk}}\right\}. 
  \end{gather}
  Furthermore, if there is a subset of $m$ circles disjoint from the
  other circles, then this subset contains $m$ eigenvalues.
\end{Theorem*}

\subsection{The Rayleigh quotient}

\begin{Definition}{rayleigh-quotient}
  For a matrix $\mata\in\Cnn$ and a vector $\vx\in\C^n$, the
  \define{Rayleigh quotient} is defined as
  \begin{gather}
    R_\mata(\vx) = \frac{\scal(\mata\vx,\vx)}{\scal(\vx,\vx)}.
  \end{gather}
\end{Definition}

\begin{Theorem*}{minmax}{Courant-Fischer min-max theorem}
  Let $\mata\in\Cnn$ be Hermitian with eigenvalues
  $\lambda_1 \le \lambda_2\le\dots\le \lambda_n$. Then, for $k=1,\dots,n$
  \begin{align}
    \lambda_k
    &= \min_{\substack{V \subset \C^n\\\dim V = k}} \max_{\vx\in V} R_\mata(\vx),\\
    &= \max_{\substack{V \subset \C^n\\\dim V = n-k+1}} \min_{\vx\in V} R_\mata(\vx).
  \end{align}
  In particular,
  \begin{gather}
    \lambda_{\min}(\mata) = \min_{\vx\in\C^n} R_\mata(\vx),
    \qquad
    \lambda_{\max}(\mata) = \max_{\vx\in\C^n} R_\mata(\vx).
  \end{gather}
\end{Theorem*}

\subsection{Conditioning of the eigenvalue problem}
In this section, we study the conditioning of finding eigenvalues and
eigenvectors. While we will not cover the full theory, we will provide
examples for ill-posed problems as well as exemplary proofs for
well-posedness.

In all cases, we will investigate the change of eigenvalues or
eigenvectors when the matrix $\mata$ is perturbed by a small matrix
$\mate$ of norm $\epsilon$.

\begin{Example}{characteristic-polynomial}
  Take a matrix of dimension 20 with eigenvalues $1,2,\ldots,20$. Its
  characteristic polynomial is
  \begin{gather}
    \chi(\lambda) = (\lambda-1)\dots(\lambda-20).
  \end{gather}
  The coefficient in front of $\lambda^{20}$ is one, the constant term is $20! > 10^{19}$.
  We perturbe it in the form
  \begin{gather}
    \tilde \chi(\lambda) = \chi(\lambda) - 10^{-23}\lambda^{19}.
  \end{gather}
  Their greatest roots are
  \begin{gather}
    \begin{array}{l@{\qquad}l@{\,}c@{\,}l}
      \multicolumn{1}{c}{\chi}&
      \multicolumn{3}{c}{\tilde \chi}\\
      20&20.847\\
      19,18&19.502&\pm&1.940i\\
      17,16&16.731&\pm&2.813i\\
      15,14&13.992&\pm&2.519i\\
    \end{array}
  \end{gather}
  {\tiny Source: \cite{DeuflhardHohmann08}}
\end{Example}

\begin{Example}{conditioning-Jordan-block}
  Consider the matrix
  \begin{gather}
  \mata_\epsilon =
      \begin{pmatrix}
        0&1\\
        &0&1\\
        &&\ddots&\ddots\\
        &&&0&1\\
        \epsilon &&&&0
      \end{pmatrix}
      \in\C^{n\times n},
  \end{gather}
  For $\epsilon=0$, it has a single eigenvalue of geometric multiplicity one and algebraic multiplicity $n$.
  
  For $\epsilon>0$, it has $n$ simple eigenvalues
  \begin{gather}
      \lambda_j = \sqrt[n]{\epsilon} \,e^{2\frac jni\pi}
  \end{gather}
\end{Example}

\begin{proof}
  For $\epsilon=0$, the matrix is the generic Jordan-block of an eigenvalue which is not semi-simple, thus the ill-posedness of this example implies the ill-posedness for not semi-simple eigenvalues in the general case. Note that this statement holds notwithstanding that special perturbations may be benign.
  
  The characteristic polynomial of this matrix is
  \begin{gather}
      \chi(\lambda) = \det(\mata-\lambda\id)
      = \det\begin{pmatrix}
      -\lambda&1\\  
        &\ddots&\ddots\\
        &&-\lambda&1\\
        \epsilon &&&-\lambda
      \end{pmatrix}.
  \end{gather}
  Applying Laplace expansion to the first column yields
  \begin{gather}
      \chi(\lambda)
      = -\lambda \det\begin{pmatrix}
        -\lambda&1\\
        &\ddots&\ddots\\
        &&-\lambda&1\\
        &&&-\lambda
      \end{pmatrix}
      + (-1)^{n+1} \epsilon\det\begin{pmatrix}
        1 \\
        -\lambda &1\\
        &\ddots&\ddots\\
        &&-\lambda&1
      \end{pmatrix},
  \end{gather}
  where both matrices are of dimension $n-1$. Since they are triangular, recursion of Laplace expansion is particularly simple and yields the product of the diagonal elements. Thus
  \begin{gather}
      \chi(\lambda) = (-1)^n \lambda^n
      + (-1)^{n+1} \epsilon.
  \end{gather}
  Its roots are determined by the condition
  \begin{gather}
      \lambda^n = \epsilon.
  \end{gather}
  Thus, $\lambda$ can be computed as an $n$th root of unity times the (real) $n$th root of $\epsilon$.
\end{proof}

\begin{Theorem}{Jordan-block-ill-conditioned}
  The eigenvalue problem for eigenvalues which are not semi-simple is
  in general ill-posed.
\end{Theorem}

\begin{proof}
  The analysis in \slideref{Example}{conditioning-Jordan-block} is
  generic in the sense that it applies to nonzero eigenvalues and also
  to matrices which are similar to such a block. Thus, we can conclude
  that for every matrix $\mata$ which is similar to a matrix with a
  nontrivial Jordan block for eigenvalue $\lambda$, there is a
  perturbation $\mate$ such that the derivative of the function
  $\lambda(\epsilon) = \lambda(A+\epsilon\mate)$ at zero is unbounded.
\end{proof}

\begin{Theorem*}{bauer-fike}{Bauer-Fike}
  Let $\mata\in \Cnn$ be diagonalizable with matrix of eigenvectors
  $\matv \in \Cnn$ and diagonal matrix
  $\matlambda = \diag(\lambda_1\dots,\lambda_n)$. Let $\mata+\mate$ be
  a perturbation of $\mata$. Then, for any eigenvalue $\mu$ of
  $\mata+\mate$, there is an eigenvalue $\lambda_i$ of $\mata$ such
  that
  \begin{gather}
    \abs{\mu-\lambda_i} \le \cond_2(\matv) \norm{\mate}_2.
  \end{gather}
\end{Theorem*}

\begin{proof}
  Wikipedia
\end{proof}

\begin{Corollary}{conditioning-eigenvalues-normal}
  The eigenvalue problem of a normal matrix $\mata\in\Cnn$ is
  well-conditioned in the sense that for every eigenvalue $\mu$ of the
  perturbed matrix $\mata+\mate$, there is an eigenvalue $\lambda$ of
  $\mata$ such that
  \begin{gather}
    \abs{\mu-\lambda} \le \norm{E}_2.
  \end{gather}
\end{Corollary}

The Bauer-Fike theorem provides a general estimate for diagonalizable
matrices in terms of the condition number of the matrix of
eigenvectors. The following theorem is less general, since it only
applies to simple eigenvalues, but it provides geometric intuition of
the issue.

\begin{Theorem}{conditioning-eigenvalue-single}
  Let $\mata_\epsilon = \mata+\epsilon\mate\in\Cnn$ be a perturbation
  of $\mata\in\Cnn$. Let $\lambda(0)$ and let $\lambda(0)$ be a single
  eigenvalue of $\mata$. Then, there exists a uniquely defined
  differentiable continuation $\lambda(\epsilon)$ for small $\epsilon$
  such that $\lambda(\epsilon) \in \sigma(\mata_\epsilon)$ and with
  its left and right eigenvectors $u$, and $w$, respectively, there
  holds
  \begin{gather}
    \abs*{\tfrac{d}{d\epsilon} \lambda(0)}
    \le \norm{E}_2\frac{\norm{u}_2\norm{w}_2}{\abs{\scal(u,w)}}
    = \norm{E}_2 \frac1{\cos\angle(u,w)}.
  \end{gather}
\end{Theorem}

\subsection{Conditioning of eigenvectors and eigenspaces}

\begin{intro}
  Positive results on the conditioning of eigenvectors require
  additional tools which go beyond the exposition planned for this
  class. We will thus only discuss this question at hand of an example
  and conclude a rule of thumb.
\end{intro}

\begin{Example}{conditioning-eigenvectors}
  Consider the two matrices
  \begin{gather}
    A =
    \begin{pmatrix}
      1-\epsilon & 0\\ 0 & 1+\epsilon
    \end{pmatrix},
    \qquad
    B =
    \begin{pmatrix}
      1&\epsilon\\\epsilon&1
    \end{pmatrix}.
  \end{gather}
  Their eigenvalues are $1-\epsilon$ and $1+\epsilon$, but their
  eigenvectors differ by an angle of $\pi/4$ independent of
  $\epsilon$.
\end{Example}

\begin{Remark}{conditioning-eigenvectors}
  The problem of finding eigenvectors for tight clusters of
  eigenvalues is ill-posed. Nevertheless, finding the invariant
  subspace associated to all eigenvalues in such a cluster is
  well-posed.

  Conditioning of the eigenvector problem depends on the separation of
  eigenvalues.
\end{Remark}

%%% Local Variables:
%%% mode: latex
%%% TeX-master: "main"
%%% End:


\section{Vector iterations}

\subsection{Simple iterations}

\begin{Algorithm*}{vector-iteration}{Vector iteration (power method)}
  \begin{algorithmic}[1]
    \Require $\mata\in\Cnn$, initial vector $\vv^{(0)}\in\C^n$
    \For{$k=1,\dots$ until convergence}
    \State $\vy \gets \mata \vv^{(k-1)}$
    \State $\alpha_k = \frac{\vy_i}{\vv^{(k-1)}_i}$
    where $\abs{\vv^{(k)}_i}$ is maximal
    \State $\vv^{(k)} = \frac{\vy}{\norm{\vy}}$
    \EndFor
  \end{algorithmic}
\end{Algorithm*}

\begin{Theorem}{vector-iteration}
  Lat $\mata\in\Cnn$ be diagonalizable such that $\lambda_1$ is the
  unique eigenvalue with maximal modulus. Let furthermore the
  component of $v^{(0)}$ in direction of the first eigenvector be
  nonzero. Then, the factors $\alpha_k$ and vectors $v^{(k)}$ of the
  vector iteration converge to the eigenvalue $\lambda_1$ and its
  associated eigenvector. Moreover, there holds
  \begin{align}
    \abs{\alpha_{k+1}-\lambda_1}
    &\le \frac{\abs{\lambda_1}}{\abs{\lambda_2}} \abs{\alpha_{k}-\lambda_1}\\
    \norm{v^{(k+1)}-u_1}
    &\le \frac{\abs{\lambda_1}}{\abs{\lambda_2}} \norm{v^{(k)}-u_1}
  \end{align}
\end{Theorem}

\begin{Remark}{vector-iteration}
  The proof actually requires, that the entry defining $\alpha_k$
  remains the same during the iteration, at least during the steps
  used for detecting convergence.

  The result does not actually require that $\mata$ is diagonalizable,
  as long as $\lambda_1$ is single and of largest modulus.
\end{Remark}

\begin{Lemma}{Rayleigh-approximation}
  Let $(\lambda,\vv)$ be an eigenpair of the Hermitian matrix
  $\mata\in\Cnn$, and let $\vw\i\C^n$. Then, there is a constant $C$ depending only on the matrix $\mata$ such that there holds
  \begin{gather}
    \abs{R_\mata(\vw)-\lambda} \le C \norm{\vw-\vv}^2,
  \end{gather}
  where $R_\mata(\vw)$ is the \putindex{Rayleigh quotient} from
  \slideref{Definition}{rayleigh-quotient}.
\end{Lemma}

\begin{Algorithm*}{shifted-vector-iteration}{Shifted vector iteration}
  The vector iteration can be applied to the matrix $\mata-\sigma\id$
  for some $\sigma\in\C$.

  Then, $\alpha_k$ converges to the eigenvalue $\lambda$ such that
  $\lambda-\sigma$ has largest modulus. $v^{(k)}$ converges to an
  eigenvector for this eigenvalue.
\end{Algorithm*}

\begin{Algorithm*}{inverse-iteration}{The inverse power method}
    \begin{algorithmic}[1]
    \Require $\mata\in\Cnn$, $\sigma\in\C$, initial vector $\vv^{(0)}\in\C^n$
    \For{$k=1,\dots$ until convergence}
    \State Solve $(\mata-\sigma\id) \vy = \vv^{(k-1)}$
    \State $\vv^{(k)} = \frac{\vy}{\norm{\vy}}$
    \State $\alpha_k = R_\mata(\vv^{(k)})$
    \EndFor
  \end{algorithmic}
\end{Algorithm*}

\begin{Algorithm*}{Rayleigh-iteration}{The Rayleigh quotient iteration}
    \begin{algorithmic}[1]
    \Require $\mata\in\Cnn$, $\sigma_0\in\C$, initial vector $\vv^{(0)}\in\C^n$
    \For{$k=1,\dots$ until convergence}
    \State Solve $(\mata-\sigma_{k-1}\id) \vy = \vv^{(k-1)}$
    \State $\vv^{(k)} = \frac{\vy}{\norm{\vy}}$
    \State $\sigma_k = R_\mata(\vv^{(k)})$
    \EndFor
  \end{algorithmic}  
\end{Algorithm*}


%%% Local Variables:
%%% mode: latex
%%% TeX-master: "main"
%%% End:


\section{Subspace iterations and the QR method}
\subsection{Derivation from orthogonal subspace iteration}

\begin{todo}
  rename vectors in OSI
\end{todo}

\begin{intro}
  \label{par:qr:intro}
  In this section, we call the orthogonal vectors of the orthogonal subspace iteration $\matu^{(k)}$ to distinguish them from the new orthogonal basis constructed here.
  
  If the \putindex{orthogonal subspace iteration} converges, then
  there is an unitary matrix $\matu$ such that
  $\lim_{k \to \infty} \matu_k = \matu$.  From the two assignments of
  the algorithm, we get
  \begin{gather}
    \lim_{k \to \infty} \matu_k^* \mata \matu_k = \lim_{k \to \infty} \matu_k^* \maty_k = \lim_{k \to \infty}
    \matu_k^*\matu_{k+1}\matr_{k+1} = \matr,
  \end{gather}
  where $\matr\leftarrow\matr_k$. Hence, the matrices
  \begin{gather}
    \label{eq:qr:intro:1}
    \mata_k = \matu_k^*\mata\matu_k
  \end{gather}
  converge to an upper triangular matrix with the converged
  eigenvalues on the diagonal. Our next goal ist the modification of
  the orthogonal subspace iteration, such that we compute the sequence
  $\mata_k$ directly, without the intermediate $\maty_k$. To this end,
  we require $m=n$, that is, we compute the whole spectrum of $\mata$.
\end{intro}

\begin{Lemma}{qr-iteration-derivation}
  The matrices $\mata_k = \matu_k^*\mata\matu_k$ obtained from the
  orthogonal subspace iteration follow the recursion formula
  \begin{gather}
    \label{eq:qr:iteration-derivation:2}
    \mata_k = \matq_k^*\mata_{k-1}\matq_k,
  \end{gather}
  where $\matq_k = \matu_{k-1}^*\matu_k$ for $k=1,\dots,n-1$. There holds
  \begin{gather}
    \label{eq:qr:iteration-derivation:1}
    \matu_k=\matu_0\matq_1\dots\matq_k.
  \end{gather}
  Furthermore, $\mata_k$ can be
  computed from the QR factorization $\mata_{k-1}= \matq_k\matr_k$ as
  \begin{gather}
    \mata_k = \matr_k\matq_k.
  \end{gather}
\end{Lemma}

\begin{proof}
  By the definition of $\mata_{k-1}$ and
  \slideref{Algorithm}{subspace-iteration}, we have
  \begin{gather*}
    \mata_{k-1} = \matu_{k-1}^*\mata\matu_{k-1}
    = \matu_{k-1}^* \maty_{k-1}
    = \left(\matu_{k-1}^*\matu_{k}\right)\matr_k = \matq_k\matr_k,
  \end{gather*}
  where we see that on the right we have obtained the QR factorization of $\mata_{k-1}$.
  On the other hand,
  \begin{align*}
    \mata_k
    &= \matu_k^*\mata\matu_k\\
    &= \matu_k^*\mata\matu_{k-1}\matu_{k-1}^*\matu_k\\
    &= \matu_k^*\maty_{k-1} \matu_{k-1}^*\matu_k\\
    &= \matr_k \left(\matu_{k-1}^*\matu_k\right) = \matr_k\matq_k.
  \end{align*}
  Thus, we see that $\mata_k$ is obtained by multiplying the QR
  factorization of $\mata_{k-1}$ in reverse order.  We have already
  replaced $\matq_k = \matu_{k-1}^*\matu_k$ above. Mutliplying from
  the left by $\matu_{k-1}$ and using $\matu_0=\matq_0=\id$ yields the
  formula for $\matu_k$ by induction.

  Now, let $\matq_0=\id$ such that
  $\mata_0 = \matq_0^*\mata\matq_0=\mata$. Then, for $k\ge 1$ we
  obtain the recursion
  \begin{gather}
    \mata_{k} = \matr_k\matq_k = \matq_k^* \matq_k \matr_k \matq_k = \matq_k^* \mata_{k-1} \matq_k.
  \end{gather}
\end{proof}

\begin{Algorithm*}{qr-iteration}{QR iteration}
  \begin{algorithmic}[1]
    \Require $\mata_0\in\Cnn$.
    \For{$k=1,\dots$ until convergence}
    \State $\matq_k\matr_k \gets \mata_{k-1}$ \Comment{QR factorization}
    \State $\mata_{k} = \matr_k\matq_k$
    \EndFor
  \end{algorithmic}
\end{Algorithm*}

\begin{Lemma}{qr-Schur}
  If convergent, the QR iteration converges to the Schur canonical
  form of the matrix $\mata$ with eigenvalues sorted according to
  their modulus.
\end{Lemma}

\begin{proof}
  See~\ref{par:qr:intro}.
\end{proof}

\begin{Lemma}{qr-1}
  The matrices $\mata_k$ of the QR-iteration with $\mata_0 = \mata$
  have the following properties:
  \begin{enumerate}
  \item If $\matq_0=\id$, $\mata_{k} = \matq_k^*\mata_{k-1}\matq_k = \matq_k^*\dots\matq_0^*\mata\matq_0\dots\matq_k$.
  \item $\mata^k=\matq_1\dots\matq_k\matr_k\dots\matr_1$.
  \item If $\mata$ is normal, so is $\mata_k$ for any $k$.
  \item If $\mata$ is complex symmetric, so is $\mata_k$ for any $k$.
  \end{enumerate}
\end{Lemma}

\begin{proof}
  The first relation follows from~\eqref{eq:qr:intro:1}
  and~\eqref{eq:qr:iteration-derivation:1} letting $\matu_0=\id$,
  which corresponds to $\mata_0 = \matu_0^*\mata\matu_0 = \matq_0^*\mata\matq_0 = \mata$.
  
  The second, we prove by induction, where
  $\mata = \mata_0 = \matq_1\matr_1$ follows directly from the first
  step of the algorithm. For the induction step, we abbreviate
  \begin{gather}
    \matu_k = \matq_1\dots\matq_k,\qquad \mats_k = \matr_k\dots\matr_1.
  \end{gather}
  Assuming $\mata^k = \matu_k\mats_k$, we obtain
  \begin{gather}
    \mata^{k+1} = \mata\matu_k\mats_k = \matu_k\mata_{k+1}\mats_k
    = \matu_k\matq_{k+1}\matr_{k+1}\mats_k = \matu_{k+1}\mats_{k+1}.
  \end{gather}
\end{proof}

%%%%%%%%%%%%%%%%%%%%%%%%%%%%%%%%%%%%%%%%%%%%%%%%%%%%%%%%%%%%%%%%%%%%%%
\subsection{Hessenberg matrices}
%%%%%%%%%%%%%%%%%%%%%%%%%%%%%%%%%%%%%%%%%%%%%%%%%%%%%%%%%%%%%%%%%%%%%%
\begin{intro}
  In each step of the QR-iteration, a QR factorization of the matrix
  is needed, which requires $\bigo(n^3)$ operations. Thus, the
  complexity of the iteration is highly unfavorable. The following
  discussion will provide us with means to reduce the complexity of
  the QR factorization to $\bigo(n^2)$, in the symmetric case even to
  $\bigo(n)$.
\end{intro}

\begin{Definition}{hessenberg}
  A matrix is in \define{Hessenberg form} or is a \define{Hessenberg
    matrix}, if all its entries below the first subdiagonal are zero. Visually,
  \begin{gather}
    H =
    \begin{pNiceMatrix}
      *&\Cdots&\Cdots&\Cdots&*\\
      *&\Ddots&&&\Vdots\\
      &\Ddots&\Ddots&&\Vdots\\
      &&\Ddots&\Ddots&\Vdots\\
      &&&*&*
    \end{pNiceMatrix}
  \end{gather}
  A symmetric or Hermitian Hessenberg matrix is \define{tridiagonal}.
\end{Definition}

\begin{Algorithm*}{Hessenberg-qr-1}{Explicit Hessenberg QR step}
  \index{Hessenberg QR!explicit}
  \begin{algorithmic}[1]
    \Require $\matH\in\Cnn$ in Hessenberg form
    \For{$k=1,\dots,n-1$}
    \Comment{factorization $\matq\matr = \matH$, $\matr$ stored in $\matH$}
    \State $\givens_{k,k+1} \gets$ Givens rotation for $h_{kk},h_{k+1,k}$
    \State $\matH\gets \givens^*_{k,k+1}\matH$
    \EndFor
    \For{$k=1,\dots,n-1$}
    \Comment{$\matH = \matr\matq$}
    \State $\matH\gets \matH\givens_{k,k+1}$
    \EndFor
  \end{algorithmic}
\end{Algorithm*}

\begin{example}
  At the example of a 5-by-5-matrix, we show how this algorithm works.
  \begin{enumerate}
  \item Apply a Givens rotation from the left which eliminates the value $Y$ from the matrix. It affects the two top rows.
    \begin{gather*}
      \left(\begin{NiceArray}[margin,hvlines,corners=SW]{ccccc}
          \CodeBefore
          \rectanglecolor{yellow}{1-2}{2-5}
          \Body
          \cellcolor{green}x&\Block{1-4}{}*&*&*&*\\
          \cellcolor{green}y&\Block{1-4}{}*&*&*&*\\
          &\Block{1-4}{}*&*&*&*\\
          &&\Block{1-3}{}*&*&*\\
          &&&\Block{1-2}{}*&*
        \end{NiceArray}\right)
      \underrightarrow{\quad\matg_{12}^*\matH\quad}
      \left(\begin{NiceArray}[margin,hvlines,corners=SW]{ccccc}
          \Block{1-5}{}*&*&*&*&*\\
          &\Block{1-4}{}*&*&*&*\\
          &\Block{1-4}{}*&*&*&*\\
          &&\Block{1-3}{}*&*&*\\
          &&&\Block{1-2}{}*&*
        \end{NiceArray}\right)    
    \end{gather*}

  \item Do the same with the following row
    \begin{gather*}
      \left(\begin{NiceArray}[margin,hvlines,corners=SW]{ccccc}
          \CodeBefore
          \rectanglecolor{yellow}{2-3}{3-5}
          \Body
          \Block{1-5}{}*&*&*&*&*\\
          &\cellcolor{green}x&\Block{1-3}{}*&*&*\\
          &\cellcolor{green}y&\Block{1-3}{}*&*&*\\
          &&\Block{1-3}{}*&*&*\\
          &&&\Block{1-2}{}*&*
        \end{NiceArray}\right)
      \underrightarrow{\quad\matg_{23}^*\matH\quad}
      \left(\begin{NiceArray}[margin,hvlines,corners=SW]{ccccc}
          \Block{1-5}{}*&*&*&*&*\\
          &\Block{1-4}{}*&*&*&*\\
          &&\Block{1-3}{}*&*&*\\
          &&\Block{1-3}{}*&*&*\\
          &&&\Block{1-2}{}*&*
        \end{NiceArray}\right)    
    \end{gather*}
    to the last pair
    \begin{gather*}
      \left(\begin{NiceArray}[margin,hvlines,corners=SW]{ccccc}
          \CodeBefore
          \rectanglecolor{yellow}{4-5}{5-5}
          \Body
          \Block{1-5}{}*&*&*&*&*\\
          &\Block{1-4}{}*&*&*&*\\
          &&\Block{1-3}{}*&*&*\\
          &&&\cellcolor{green}x&*\\
          &&&\cellcolor{green}y&*\\
        \end{NiceArray}\right)
      \underrightarrow{\quad\matg_{45}^*\matH\quad}
      \left(\begin{NiceArray}[margin,hvlines,corners=SW]{ccccc}
          \Block{1-5}{}*&*&*&*&*\\
          &\Block{1-4}{}*&*&*&*\\
          &&\Block{1-3}{}*&*&*\\
          &&&\Block{1-2}{}*&*\\
          &&&&*
        \end{NiceArray}\right)    
    \end{gather*}
    Note that columns with two zero entries remain unchanged and will not have to be processed.
  \item Now the matrix is upper triangular and the transformation was
    \begin{gather*}
      \matq^* = \givens_{45}^*\givens_{34}^*\givens_{23}^*\givens_{12}^*.
    \end{gather*}

  \item When we go back, we apply Givens rotations from the right, thus affecting columns of the matrices.
    \begin{multline*}\small
      \left(\begin{NiceArray}[margin,hvlines,corners=SW]{ccccc}
          \CodeBefore
          \rectanglecolor{yellow}{1-1}{1-1}
          \rectanglecolor{yellow}{1-2}{2-2}
          \Body
          *&\Block{2-1}{}*&\Block{3-1}{}*&\Block{4-1}{}*&\Block{5-1}{}*\\
          &*&*&*&*\\
          &&*&*&*\\
          &&&*&*\\
          &&&&*
        \end{NiceArray}\right)
      \underrightarrow{\quad\matH\matg_{12}\quad}
      \left(\begin{NiceArray}[margin,hvlines,corners=SW]{ccccc}
          \CodeBefore
          \rectanglecolor{yellow}{1-2}{2-2}
          \rectanglecolor{yellow}{1-3}{3-3}
          \Body
          \Block{2-1}{}*&\Block{2-1}{}*&\Block{3-1}{}*&\Block{4-1}{}*&\Block{5-1}{}*\\
          *&*&*&*&*\\
          &&*&*&*\\
          &&&*&*\\
          &&&&*
        \end{NiceArray}\right)
      \underrightarrow{\quad\matH\matg_{23}\quad}
      \cdots\\\small
      \underrightarrow{\quad\matH\matg_{34}\quad}
      \left(\begin{NiceArray}[margin,hvlines,corners=SW]{ccccc}
          \CodeBefore
          \rectanglecolor{yellow}{1-4}{4-4}
          \rectanglecolor{yellow}{1-5}{5-5}
          \Body
          \Block{2-1}{}*&\Block{3-1}{}*&\Block{4-1}{}*&\Block{4-1}{}*&\Block{5-1}{}*\\
          *&*&*&*&*\\
          &*&*&*&*\\
          &&*&*&*\\
          &&&&*
        \end{NiceArray}\right)
      \underrightarrow{\quad\matH\matg_{45}\quad}
      \left(\begin{NiceArray}[margin,hvlines,corners=SW]{ccccc}
          \Block{2-1}{}*&\Block{3-1}{}*&\Block{4-1}{}*&\Block{5-1}{}*&\Block{5-1}{}*\\
          *&*&*&*&*\\
          &*&*&*&*\\
          &&*&*&*\\
          &&&*&*
        \end{NiceArray}\right)
    \end{multline*}    
  \end{enumerate}
\end{example}

\begin{remark}
  While the previous algorithmdoes the job, it has some
  disadvantages. For once, the Givens rotations accumulated in the
  first part must be stored to be applied in the second. Then, if
  applied to a tridiagonal matrix, the result of the first step has no
  subdiagonal, but two superdiagonals. And finally, if applied to a
  symmetric matrix, all intermediate results are nonsymmetric.

  These drawbacks have led to a second algorithm, where each Givens
  rotation is applied from the left and the right at the same time and
  $\matr$ is never explicitly computed.

  After introducing this algorithm, we will obviously have to make
  sure that its result is useful.
\end{remark}

\begin{Algorithm*}{Hessenberg-qr-2}{Implicit Hessenberg QR step}
  \begin{algorithmic}[1]
    \Require $\matH\in\Cnn$ in Hessenberg form
    \State $\givens_{1,2} \gets$ Givens rotation for $h_{11},h_{21}$
    \State $\matH \gets \givens^*_{1,2}\matH \givens_{1,2}$
    \For{$k=2,\dots,n-1$}
    \State $\givens_{k,k+1}\gets \givens_{k,k+1}[h_{k,k-1},h_{k+1,k-1}]$ Givens rotation
    \State $\matH \gets \givens^*_{k,k+1}\matH \givens_{k,k+1}$
    \EndFor
  \end{algorithmic}
\end{Algorithm*}

\begin{example}
  At the example of a 5-by-5-matrix, we show how this algorithm works.
  \begin{enumerate}
  \item Apply a Givens rotation from the left which eliminates the value $Y$ from the matrix. It affects the two top rows. By
    applying the Givens rotation from the right an additional non zero entry below the subdiagonal is created.
    \begin{gather*}\small
      \left(\begin{NiceArray}[margin,hvlines,corners=SW]{ccccc}
          \CodeBefore
          \rectanglecolor{yellow}{1-2}{2-5}
          \Body
          \cellcolor{green}x&\Block{1-4}{}*&*&*&*\\
          \cellcolor{green}y&\Block{1-4}{}*&*&*&*\\
          &\Block{1-4}{}*&*&*&*\\
          &&\Block{1-3}{}*&*&*\\
          &&&\Block{1-2}{}*&*
        \end{NiceArray}\right)
      \to%\underrightarrow{\quad\matg_{12}^*\matH\quad}
      \left(\begin{NiceArray}[margin,hvlines,corners=SW]{ccccc}
          \CodeBefore
          \rectanglecolor{yellow}{1-1}{3-2}
          \Body
          *&\Block{3-1}{}*&\Block{4-1}{}*&\Block{5-1}{}*&\Block{5-1}{}*\\
          &*&*&*&*\\
          &*&*&*&*\\
          &&*&*&*\\
          &&&*&*
        \end{NiceArray}\right)
      \to%\underrightarrow{\quad\matg_{12}^*\matH\matg_{12}\quad}
      \left(\begin{NiceArray}[margin,hvlines,corners=SW]{ccccc}
          \Block{3-1}{}*&\Block{3-1}{}*&\Block{4-1}{}*&\Block{5-1}{}*&\Block{5-1}{}*\\
          *&*&*&*&*\\
          \cellcolor{red}*&*&*&*&*\\
          &&*&*&*\\
          &&&*&*
        \end{NiceArray}\right)    
    \end{gather*}
\item Apply a Givens rotation from the left to restore Hessenberg form of the first column.
    \begin{gather*}\small
      \left(\begin{NiceArray}[margin,hvlines,corners=SW]{ccccc}
          \CodeBefore
          \rectanglecolor{yellow}{2-2}{3-5}
          \Body
          \Block{1-5}{}*&*&*&*&*\\
          \cellcolor{green}x&\Block{1-4}{}*&*&*&*\\
          \cellcolor{green}y&\Block{1-4}{}*&*&*&*\\
          &&\Block{1-3}{}*&*&*\\
          &&&\Block{1-2}{}*&*
        \end{NiceArray}\right)
      \to%\underrightarrow{\quad\matg_{12}^*\matH\quad}
      \left(\begin{NiceArray}[margin,hvlines,corners=SW]{ccccc}
          \CodeBefore
          \rectanglecolor{yellow}{1-2}{4-3}
          \Body
          \Block{2-1}{}*&\Block{3-1}{}*&\Block{4-1}{}*&\Block{5-1}{}*&\Block{5-1}{}*\\
          *&*&*&*&*\\
          &*&*&*&*\\
          &&*&*&*\\
          &&&*&*
        \end{NiceArray}\right)    
      \to%\underrightarrow{\quad\matH\matg_{12}\quad}
      \left(\begin{NiceArray}[margin,hvlines,corners=SW]{ccccc}
          \Block{2-1}{}*&\Block{4-1}{}*&\Block{4-1}{}*&\Block{5-1}{}*&\Block{5-1}{}*\\
          *&*&*&*&*\\
          &*&*&*&*\\
          &\cellcolor{red}*&*&*&*\\
          &&&*&*
        \end{NiceArray}\right)    
    \end{gather*}
    
  \item Do the same with the second column.
    
    \begin{gather*}\small
      \left(\begin{NiceArray}[margin,hvlines,corners=SW]{ccccc}
          \CodeBefore
          \rectanglecolor{yellow}{3-3}{4-5}
          \Body
          \Block{1-5}{}*&*&*&*&*\\
          \Block{1-5}{}*&*&*&*&*\\
          &\cellcolor{green}x&\Block{1-3}{}*&*&*\\
          &\cellcolor{green}y&\Block{1-3}{}*&*&*\\
          &&&\Block{1-2}{}*&*
        \end{NiceArray}\right)
      \to%\underrightarrow{\quad\matg_{23}^*\matH\quad}
      \left(\begin{NiceArray}[margin,hvlines,corners=SW]{ccccc}
          \CodeBefore
          \rectanglecolor{yellow}{1-3}{5-4}
          \Body
          \Block{2-1}{}*&\Block{3-1}{}*&\Block{4-1}{}*&\Block{5-1}{}*&\Block{5-1}{}*\\
          *&*&*&*&*\\
          &*&*&*&*\\
          &&*&*&*\\
          &&&*&*
        \end{NiceArray}\right)
      \to%\underrightarrow{\quad\matH\matg_{23}\quad}
      \left(\begin{NiceArray}[margin,hvlines,corners=SW]{ccccc}
          \Block{2-1}{}*&\Block{3-1}{}*&\Block{5-1}{}*&\Block{5-1}{}*&\Block{5-1}{}*\\
          *&*&*&*&*\\
          &*&*&*&*\\
          &&*&*&*\\
          &&\cellcolor{red}*&*&*
        \end{NiceArray}\right)    
    \end{gather*}
  
  \item Finally with the last row
    
    \begin{gather*}\small
      \left(\begin{NiceArray}[margin,hvlines,corners=SW]{ccccc}
          \CodeBefore
          \rectanglecolor{yellow}{4-4}{5-5}
          \Body
          \Block{1-5}{}*&*&*&*&*\\
          \Block{1-5}{}*&*&*&*&*\\
          &\Block{1-4}{}*&*&*&*\\
          &&\cellcolor{green}x&\Block{1-2}{}*&*\\
          &&\cellcolor{green}y&\Block{1-2}{}*&*
        \end{NiceArray}\right)
      \to%\underrightarrow{\quad\matg_{23}^*\matH\quad}
      \left(\begin{NiceArray}[margin,hvlines,corners=SW]{ccccc}
          \CodeBefore
          \rectanglecolor{yellow}{1-4}{5-5}
          \Body
          \Block{2-1}{}*&\Block{3-1}{}*&\Block{4-1}{}*&\Block{5-1}{}*&\Block{5-1}{}*\\
          *&*&*&*&*\\
          &*&*&*&*\\
          &&*&*&*\\
          &&&*&*
        \end{NiceArray}\right)
      \to%\underrightarrow{\quad\matH\matg_{23}\quad}
      \left(\begin{NiceArray}[margin,hvlines,corners=SW]{ccccc}
          \Block{2-1}{}*&\Block{3-1}{}*&\Block{4-1}{}*&\Block{5-1}{}*&\Block{5-1}{}*\\
          *&*&*&*&*\\
          &*&*&*&*\\
          &&*&*&*\\
          &&&*&*
        \end{NiceArray}\right)    
    \end{gather*}

    Note that columns with two zero entries remain unchanged and will not have to be processed.
  \end{enumerate}
\end{example}

\begin{Remark}{bulge-chasing}
  This algorithm is called \define{bulge chasing} with the following
  image in mind. After the application of the first rotation, there is
  a bulge protruding down from the Hessenberg form in the furst
  column. This bulge is then ``chased'' down row by row until it
  leaves the matrix at the bottom.
    \begin{gather}\arraycolsep0pt
    \begin{pNiceMatrix}
      *&\Cdots&\Cdots&\Cdots&\Cdots&*\\[-5pt]
      *&\Ddots&&&&\Vdots\\[-5pt]
      &\Ddots&\Ddots&&&\Vdots\\[-5pt]
      &*&\Ddots&\Ddots&&\Vdots\\[-5pt]
      &&&\Ddots&\Ddots&\Vdots\\[-5pt]
      &&&&*&*
    \end{pNiceMatrix}
    \to
    \begin{pNiceMatrix}
      *&\Cdots&\Cdots&\Cdots&\Cdots&*\\[-5pt]
      *&\Ddots&&&&\Vdots\\[-5pt]
      &\Ddots&\Ddots&&&\Vdots\\[-5pt]
      &&\Ddots&\Ddots&&\Vdots\\[-5pt]
      &&*&\Ddots&\Ddots&\Vdots\\[-5pt]
      &&&&*&*
    \end{pNiceMatrix}
  \end{gather}
\end{Remark}

\begin{Definition}{hessenberg-unreduced}
  A Hessenberg matrix is called \define{unreduced} if all entries on
  the first subdiagonal are nonzero. It is called \define{reduced}
  otherwise.
\end{Definition}

\begin{Theorem*}{implicit-Q}{Implicit Q Theorem}
  Let $\mata\in\Cnn$ arbitrary, let $\matq,\matv\in\Cnn$ unitary such that
  \begin{gather}
    \matq^*\mata\matq = \matH,\qquad \matv^*\mata\matv = \matg,
  \end{gather}
  where $\matH$ and $\matg$ are Hessenberg matrices.
  Let $k$ denote
  the smallest integer such that $h_{k+1,k} = 0$, or $k=n$ if $\matH$
  is unreduced. Assume $\vv_1 \parallel \vq_1$. Then,
  $\vv_j\parallel \vq_j$ and $\abs{h_{j+1,j}} = \abs{g_{j+1,j}}$
  for $j=1,\dots,k-1$. If $k<n$, then $g_{k+1,k} = 0$.
\end{Theorem*}

\begin{proof}
  We define the matrix $\matw = \matv^*\matq$, which is unitary as the product of unitary matrices. There holds
  \begin{gather}
    \matg\matw = \matv^*\mata\matv\matv^*\matq = \matv^*\mata\matq
    = \matv^*\matq\matq^*\mata\matq = \matw\matH.
  \end{gather}
  Spelling out column $j$ of this product, we obtain
  \begin{gather}
    \label{eq:qr:implicitq-1}
    \matg\vw_j = \sum_{k=1}^{j+1} h_{kj} \vw_{k}.
  \end{gather}
  We use this equality to show by induction over the columns that the
  entries $w_{ij}$ of $\matw$ are zero for $i>j$. In other words,
  $\matw$ is upper triangular.  For $j=1$, we obtain from the
  unitarity of $\matq$ and $\matv$ and the parallelity of $\vq_1$
  and $\vv_1$ that $\vw_1 = e^{i\phi} \ve_1$ for some argument $\phi$.

  Now let the statement be proven for all columns of $\matw$ up to
  column $j$. Then, from~\eqref{eq:qr:implicitq-1} we obtain
  \begin{gather}
    h_{j+1,j}\vw_{j+1} = \matg\vw_j - \sum_{\nu=1}^j h_{\nu j} \vw_{\nu}.
  \end{gather}
  For each vector in the sum, there holds $(\vw_{\nu})_i = 0$ for
  $i>j$. Since $\matg$ is Hessenberg and applied to $\matw_j$, the
  last possibly nonzero entry of the product is in position $j+1$,
  what we wanted to show.

  Since every unitary matrix is normal and due to
  \slideref{Problem}{normal-triangular-diagonal} every triangular
  normal matrix is diagonal, the matrix $(\vw_1,\dots,\vw_k)$ is
  diagonal with diaognal entries of the form $w_{jj} = e^{i\phi_j}$
  with some arguments $\phi_j$. Hence, $\vv_j = e^{-i\phi_j} \vq_j$
  for $j=1,\dots,k$ and thus $\vv_j$ is parallelto $\vq_j$.

  There holds for $j<k$
  \begin{gather}
    \abs{h_{j+1,j}} = \abs{\ve_{j+1}^*\matH\ve_j} = \abs{\vq_{j+1}^*\mata\vq_j}
    = \abs{\vv_{j+1}^*\mata\vv_j} = \abs{g_{j+1,j}}.
  \end{gather}

  If $k<n$, that is, $h_{k+1,k}=0$, it remains to show that
  \begin{multline}
    g_{k+1,k} = \ve_{k+1}^*\matg\ve_k = e^{i\phi_k}\ve_{k+1}^*\matg\matw\ve_k
    =  e^{i\phi_k}\ve_{k+1}^*\matw\matH\ve_k\\
    =  e^{i\phi_k}\ve_{k+1}^* \sum_{j=1}^k h_{jk} \matw\ve_j
    =  \sum_{j=1}^k e^{i\phi_k} h_{jk} \ve_{k+1}^* \ve_j = 0.
  \end{multline}
\end{proof}

\begin{Definition}{essentially-equal}
  The \putindex{Implicit Q Theorem} says that two Hessenberg forms of $\mata$ with the same initial reduction vector are \define{essentially equal} in the sense that they only differ by the diagonal scaling $\matg = \matd^{-1}\matH\matd$ where $\matd=\diag(d_1,\dots,d_n)$ matrix with $\abs{d_{i}} = 1$.
\end{Definition}

\begin{Corollary}{Hessenberg-qr-equivalence}
  The two versions of the Hessenberg QR step are essentially equal.
\end{Corollary}

\begin{Problem}{Hessenberg-qr-effort}
  \begin{enumerate}
  \item How many operations do the two versions of the Hessenberg QR step require?
  \item Show that if $\matH$ is Hermitian, the result of the
    Hessenberg QR step is Hermitian as well.
  \end{enumerate}
\end{Problem}

\begin{Corollary}{Hessenberg-qr}
  The complexity of each step of the implicit QR-iteration for Hessenberg matrices is $\bigo(n^2)$. For tridiagonal (complex) symmetric matrices, it is $\bigo(n)$.
\end{Corollary}

\begin{Theorem}{Hessenberg-householder}
  Every matrix $\mata\in\Cnn$ is unitarily similar to a Hessenberg matrix $\matH$, that is,
  \begin{gather}
    \matH = \matq^* \mata \matq.
  \end{gather}
  The matrix $\matq$ can be obtained by $n-2$ \putindex{Householder
    reflection}s.
\end{Theorem}

\begin{proof}
  The proof is constructive and relies on Householder
  transformations. We begin by partitioning the matrix $\mata$ as
  \begin{gather}
    \mata =
    \left(\begin{NiceArray}[margin,hvlines]{ccw{c}{4em}c}
        \Block{1-4}{}*&\Cdots&\Cdots&*\\
        \Block{4-1}{\vv_1}&\Block{4-3}<\huge>{*}&&\\
        &&&\\
        &&&\\
        &&&
      \end{NiceArray}\right)
  \end{gather}
  Now we find the Householder vector $\tilde\vw_1\in\C^{n-1}$ which transforms $\vv_1$ to a multiple of $\ve_1\in \C^{n-1}$ and let
  \begin{gather}
    \vw_1 =
    \begin{pmatrix}
      0\\ \tilde\vw_1
    \end{pmatrix},
    \qquad
    \matq_1 = \id - 2 \frac{\vw_1\vw_1^*}{\vw_1^*\vw_1}.
  \end{gather}
  Note that the multiplication $\matq_1\mata$ leaves the first row of
  $\mata$ unchanged, while $\mata\matq_1$ leaves the first column
  unchanged. Hence,
  \begin{gather}
    \matq_1 \mata =
    \left(\begin{NiceArray}[margin,hvlines]{ccw{c}{4em}c}
        \Block{1-4}{}*&\Cdots&\Cdots&*\\
        *&\Block{4-3}<\huge>{*}&&\\
        \Block{3-1}{0}&&&\\
        &&&\\
        &&&
      \end{NiceArray}\right), 
  \end{gather}
  and this structure does not change by multiplication with $\matq_1$
  from the right. Now partition
  \begin{gather}
    \matq_1\mata\matq_1 =
    \left(\begin{NiceArray}[margin,hvlines]{cccw{c}{4em}c}
        \Block{1-5}{}*&\Cdots&\Cdots&\Cdots&*\\
        *&\Block{1-4}{}*&\Cdots&\Cdots&*\\
        \Block{4-1}{0}&\Block{4-1}{\vv_2}&\Block{4-3}<\huge>{*}&&\\
        &&&&\\
        &&&&\\
        &&&&
      \end{NiceArray}\right),     
  \end{gather}
  and choose the Householder vector $\tilde\vw_2\in\C^{n-2}$ which
  maps $\vv_2$ to a multiple of $\ve_1\in\C^{n-2}$. Let
  \begin{gather}
    \vw_2 =
    \begin{pmatrix}
      0\\0\\ \tilde\vw_2
    \end{pmatrix},
    \qquad
    \matq_2 = \id - 2 \frac{\vw_2\vw_2^*}{\vw_2^*\vw_2},
  \end{gather}
  and observe that multiplication with $\matq_2$ from left and right leaves the first two rows and columns untouched, respectively. Hence,
  \begin{gather}
    \matq_2\matq_1\mata\matq_1\matq_2 =
    \left(\begin{NiceArray}[margin,hvlines]{cccw{c}{4em}c}
        \Block{1-5}{}*&\Cdots&\Cdots&\Cdots&*\\
        *&\Block{1-4}{}*&\Cdots&\Cdots&*\\
        \Block{4-1}{0}&*&\Block{4-3}<\huge>{*}&&\\
        &\Block{3-1}{0}&&&\\
        &&&&\\
        &&&&
      \end{NiceArray}\right).
  \end{gather}
  This algorithm can be continued until the third last entry in the
  last row is set to zero. Note that the operations from the left
  mimic the QR factorization, but start to operate one row below the
  diagonal.
\end{proof}

\begin{Problem}{Hermitian-tridiagonal}
  Show that every (complex) Hermitian matrix is unitarily similar
  to a symmetric tridiagonal matrix with real entries.
\end{Problem}

\begin{Algorithm*}{qr-method}{The Hessenberg QR-Method}
  Compute the spectrum of a matrix $\mata\in\Cnn$ by
  \begin{enumerate}
  \item Use $n-2$ Householder transformations to transform $\mata$ to
    Hessenberg form
    \begin{gather}
     \matH_0 = \matq^*\mata\matq.
   \end{gather}
 \item QR-iteration: perform the implicit Hessenberg QR step until convergence
 \item Store Householder vectors as well as $r$ and $c$ for each
   Givens rotation \textbf{only} if the eigenvectors are desired in the end.
  \end{enumerate}
\end{Algorithm*}

\begin{remark}
  In this algorithm, we focus on computing eigenvalues. The
  eigenvectors are neglected. The could be computed by storing the
  $n-2$ Householder vectors of the transformation to Hessenberg form
  and all parameters of the Givens rotations of the iteration. In the
  end, the vectors can be computed by applying all these unitary
  matrices in the right order to an identity matrix.

  Usually, this is not done, since it results in an inefficient
  algorithm. Furthermore, we will modify the algorithm to deal with
  reduced Hessenberg matrices, which may appear anytime in the
  iteration. Actually, we should point out here that the implicit QR
  step should not be continued in this case since the Implicit Q
  Theorem dos not apply anymore.
\end{remark}

\begin{Lemma}{Hessenberg-rank}
  The rank of an unreduced Hessenberg matrix of dimension $n$ is at
  least $n-1$. In particular, the geometric multiplicity of any
  eigenvalue of such a matrix is one.
\end{Lemma}

\begin{proof}
  If $\mata$ is in Hessenberg form and unreduced, the $k+1$-st entry
  of the $k$-th column vector $\va_k$ for $k\le n-1$ is nonzero, while
  the same entry of all previous column vectors is zero. Hence, the
  first $n-1$ column vectors are linearly independent and the rank of
  $\mata$ is at least $n-1$. Note that this argument does not apply to
  the last column.

  If $\mata$ is unreduced Hessenberg, so is $\mata-\lambda\id$, which
  proves the second statement.
\end{proof}

\subsection{Deflation and shifts}

\begin{intro}
  The goal of this section is the development and justification of a
  method which accelerates convergence of the QR-iteration and
  reducing the effort at the same time. It is based on shifts, like
  for the simple or inverse power method. But, shifts are much more
  powerful here, since we compute not only ``converging subspace'',
  but also its complement. The presentation follows
  mostly~\cite{GolubVanLoan83}.
\end{intro}

\begin{Theorem}{qr-reduction}
  Let the matrix $\matH^{(k)}\in\Cnn$ in the QR iteration be of the
  form
  \begin{gather}
    \matH^{(k)} =
    \begin{pmatrix}
      \matH_{11} & \mata_{12}\\0 & \matH_{22}
    \end{pmatrix}
  \end{gather}
  with Hessenberg matrices $\matH_{11}\in\C^{p\times p}$,
  $\matH_{22}\in \C^{n-p\times n-p}$ and an arbitrary matrix
  $\mata_{12}\in \C^{p\times n-p}$. Then, the matrix $\matq^{(k)}$
  decouples into two diagonal blocks and $\matH^{(k+1)}$ has the same
  form. Thus, the iteration decouples into two separate iterations. If
  $p=n-1$, then $h_{nn}$ approximates an eigenvalue.
\end{Theorem}

\begin{proof}
  We consider the algorithm in its explicit form.  The Givens
  transformation $\givens_{p,p+1}$ can be chosen as identity, since
  the subdiagonal entry in column $p$ is already zero. Hence, the
  product of all givens transformations is
  \begin{gather}
    \matq = \givens_{12}\dots\givens_{p-1,p}\givens_{p+1,p+2}\dots\givens_{n-1,n},
  \end{gather}
  where the first $p-1$ only act on the first $p$ rows/columns and the
  remaining ones on th last $n-p$. Therefore, $\matr\matq$ has the
  same block structure as $\matH^{(k)}$.
\end{proof}


\begin{Algorithm*}{qr-deflation}{Deflation}
  After each step of the shifted QR-iteration monitor the subdiagonal
  elements of $\matH^{(k)}$. Whenever
  \begin{gather}
    \abs{h_{j,j-1}} \le \eps \bigl(\abs{h_{j-1,j-1}}+\abs{h_{jj}}\bigr)
  \end{gather}
  set $h_{j,j-1}=0$.

  If this happens in the last row, consider $h_{nn}=\lambda_n$
  converged and proceed with a matrix of dimension $n-1\times n-1$.

  If this happens in the center of the matrix, proceed with both
  remaining diagonal blocks separately.

\end{Algorithm*}

\begin{Remark}{qr-deflation}
  Deflation changes the matrix in a ``nonorthogonal'' way and thus
  changes the eigenvalues. Their accuracy will be determined by the
  parameter $\eps$ in the end.
\end{Remark}

\begin{remark}
  The purpose of deflation is removing Schur vectors from the
  iteration. Thus, if one of the Schur vectors for a multiple
  eigenvalue has converged, the remaining iterations will deal with
  reduced multiplicity.

  Deflation by itself will not help us to deal with the
  requirement that all eigenvalues must have different modulus, but
  this is solved below in combination with shifts.
\end{remark}

\begin{Algorithm*}{shifted-qr-iteration}{QR iteration with shift}
  \begin{algorithmic}[1]
    \Require $\matH_0 \in\Cnn$, Hessenberg, unreduced
    \For {$k=1,\ldots$ until convergence}
    \State $\matq_k\matr_k = \matH_{k-1} - \sigma_k\id$\Comment{QR factorization}
    \State $\matH_{k} = \matr_k\matq_k + \sigma_k\id$
    \EndFor
  \end{algorithmic}
  There is an implicit form of the shifted QR step which follows
  exactly the version outlined for the unshifted case.
\end{Algorithm*}

\begin{Lemma}{shifted-qr-similarity}
  The matrices $\matH_k$ generated by the QR iteration with shifts
  admit the recurrence relation
  \begin{gather}
    \matH_k = \matq_k^*\matH_{k-1}\matq_k.
  \end{gather}
\end{Lemma}  

\begin{proof}
  The proof is almost identical to \slideref{Lemma}{qr-1}. There holds
  \begin{multline}
    \matH_k = \matr_k\matq_k + \sigma_k \id
    = \matq_k^*\matq_k\matr_k\matq_k + \sigma_k \id\\
    =\matq_k^*\left(\matH_{k-1}-\sigma_k\id\right)\matq_k + \sigma_k \id
    =\matq_k^*\matH_{k-1}\matq_k.
  \end{multline}
\end{proof}

\begin{Lemma*}{perfect-shift}{Perfect shift}
  Let $\matH\in\Cnn$ be an unreduced Hessenberg matrix with eigenvalue
  $\sigma$. Let $\matq\matr = \matH - \sigma\id$ be a QR factorization
  and $\widetilde\matH = \matr\matq+\sigma\id$. Then,
  $\tilde h_{n,n-1}=0$ and $\tilde h_{nn} =\sigma$.
\end{Lemma*}

\begin{proof}
  See also~\cite[Theorem 7.5.1]{GolubVanLoan83}.  Since $\matH$ is
  unreduced, its first $n-1$ columns are linearly independent. Hence,
  if $\matq\matr=\matH-\sigma\id$ is a QR factorization, then
  $r_{ii} \neq 0$ for $i=1,\dots,n-1$.

  Since $\matH-\sigma\id$ is singular, we conclude $r_{nn}=0$. Thus,
  the last row of $\matr\matq$ is zero and the statement holds.
\end{proof}

\begin{intro}
  Obviously, if we knew an eigenvalue, we could deflate right
  away. Thus, the next step in the development of the algorithm is the
  determination of a \define{shift strategy} which drives $h_{n,n-1}$
  to zero by approximating the last eigenvalue.

  Such a shift strategy selects a new shift parameter $\sigma_k$ in
  every step of the algorithm. The shift strategies differ in the
  approximation of the eigenvalue which is closest to $h_{nn}$.
\end{intro}

\begin{Example*}{rayleigh-shift}{Rayleigh shift}
  The Rayleigh quotient for the smallest eigenvalue by magnitude
  converges to $h_{nn}$, as
  \begin{gather}
    \ve_n^* H^{(k)} \ve_n = h_{nn}^{(k)}
  \end{gather}
  and $\vq_n$ is orthogonal to all eigenvectors for eigenvalues of
  greater magnitude. Therefore, using $\sigma_k = h_{nn}^{(k)}$ seems
  a good idea, and often is. But it is not reliable, as in the example
  \begin{gather}
    H =
    \begin{pmatrix}
      0 & 1 \\ 1 & 0
    \end{pmatrix}.
  \end{gather}
\end{Example*}

\begin{Definition*}{wilkinson-shift}{Wilkinson shift}
  Let
  \begin{gather}
    \matm =
    \begin{pmatrix}
      h_{n-1,n-1}^{(k)}&h_{n-1,n}^{(k)}\\h_{n,n-1}^{(k)}&h_{nn}^{(k)}
    \end{pmatrix}.
  \end{gather}
  Then, for $\sigma_k$ use the eigenvalue of $\matm$ which is closer
  to $h_{nn}^{(k)}$.
\end{Definition*}

\begin{Example}{wilkinson-failure}
  Consider the orthogonal matrix
  \begin{gather}
    \mata =
    \begin{pmatrix}
      0&0&1\\1&0&0\\0&1&0
    \end{pmatrix}.
  \end{gather}
  The lower right block has a single eigenvalue zero, such that the
  Wilkinson shift and the Rayleigh shift for this matrix are zero. The eigenvalues of $\mata$ are
  \begin{gather}
    \sigma(\mata) = \left\{1, -\tfrac12 \pm \sqrt{\tfrac34}i\right\},
  \end{gather}
  which all have the same modulus. Thus, the algorithm will not converge with either shift.
\end{Example}

\begin{remark}
  The structure of this example for the Wilkinson shift suggests, that
  for every simple shift strategy relying on submatrices, we will find
  a matrix which defeats it. This could be either because the
  iteration stagnates or since it runs in a loop. Hence, we have to
  break this situatuin, which can be achieved by applying a random
  shift.

  A notable exception from this rule are symmetric tridiagonal
  matrices, where we actually have a proof of convergence, see
  \slideref{Theorem}{wilkinson-convergence}.
\end{remark}

\begin{Algorithm}{exceptional-shift}
  If no deflation has ocurred for a given number of iteration steps of
  the shifted QR iteration, the chosen shift strategy has failed.

  In this case, perform a single step with a random shift parameter, a
  so-called \define{exceptional shift}.
\end{Algorithm}

\begin{remark}
  From the necessity to introduce exceptional shifts, we realize that
  a convergence result for the shifted QR iteration is hard to
  obtain. Nevertheless, from the result for the orthogonal subspace
  iteration, we have
  \begin{gather}
    h_{j+1,j}^{(k)} = \bigo \left(\abs*{\frac{\lambda_{j+1}-\sigma}{\lambda_j-\sigma}}^k\right).
  \end{gather}
  Hence, the situation is not hopeless and typically, the algorithm
  converges again after an exceptional shift.
\end{remark}

\begin{Algorithm*}{qr-step-deflation}{QR step with deflation}
  In each step of the QR iteration, first set
  \begin{gather}
    h_{i,i-1} = 0, \qquad \text{where}\quad
    \abs{h_{i,i-1}} \le \eps \left(\abs{h_{i,i}}+\abs{h_{i-1,i-1}}_{\vphantom{g}}\right).
  \end{gather}

  Then, partition the matrix
  $\matH$ as
  \begin{gather}
    \matH =
    \begin{pmatrix}
      \matH_{11} & \matH_{12} & \matH_{13} \\
      & \matH_{22} & \matH_{23}\\
      && \matH_{33}
    \end{pmatrix},
  \end{gather}
  where $\matH_{22}\in\R^{q\times q}$ and
  $\matH_{33}\in\R^{p\times p}$ are chosen maximal such that
  $\matH_{33}$ is upper \putindex{triangular} and $\matH_{22}$
  is unreduced.

  The shifted QR step is then applied to $\matH_{22}$ only.
\end{Algorithm*}

\begin{remark}
  The three diagonal blocks in the preceding algorithm represent different stages in the algorithm:
  \begin{itemize}
  \item $\matH_{33}$ is converged to an upper triangular matrix and
    hence corresponds to a converged Schur form with eigenvalues on
    the diagonal.
  \item $\matH_{22}$ is an unreduced Hessenberg matrix. The shifted QR
    step operates on this block with the goal of driving at least one
    subdiagonal element to zero.
  \item $\matH_{11}$ is the part of the matrix waiting to be handled
    by the algorithm later.
  \end{itemize}
  If a subdiagonal entry of $\matH_{22}$ is in the last row, this last
  row is transferred to the block $\matH_{33}$ in the next step, hence
  reducing the dimension of $\matH_{22}$ by one.

  If such a subdiagonal entry is in an earlier row, we split the
  matrix $\matH_{22}$ at this point. The lower diagonal block becomes
  the new $\matH_{22}$, while the upper one is becoming a part of
  $\matH_{11}$.

  We might save additional iterations by allowing two-by-two blocks on
  the diagonal of $\matH_{33}$. Their eigenvalues can still be
  computed easily by computing the roots of a quadratic polynomial.
  
  We do not compute eigenvectors with this algorithm, such that the
  transformations of $\matH_{13}$ and $\matH_{23}$ can be avoided.
\end{remark}

\begin{remark}
  When implementing \slideref{Algorithm}{qr-step-deflation}, we have
  to be able to run a QR step on a submatrix. Allocating new memory
  and copying the submatrix should be avoided since it comes at
  considerable cost.

  This means on the other hand, that the matrix $\matH_{22}$ will not
  be stored as a consecutive array of $q\times q$ numbers. Depending
  on whether the entries are sorted in row-major or column-major
  order, there will either be a gap between each consecutive element
  of a column or between the last element of one column and the first
  element of the next.

  Thus, the QR step operations must allow for a \define{stride}
  between rows or columns. This can be achieved either by storing the
  matrix as a sequence of column vectors, or by using strided versions
  of the algorithms.

  In FORTRAN, the function DAXPY of the basic linear algebra
  subprograms (BLAS) library has explicit parameters for this
  stride. The submatrix objects of the Armadillo library implement
  this in a transparent way.
\end{remark}

%%% Local Variables:
%%% mode: latex
%%% TeX-master: "main"
%%% End:


%\section{Polynomräume}

% \begin{Satz}{nullstellen}
%   Ein reelles Polynom vom Grad $n$ hat höchstens $n$ Nullstellen oder es ist das Nullpolynom.
% \end{Satz}

% \begin{proof}
%   Für $n=1$ handelt es sich um ein lineares Polynom und die Aussage
%   des Satzes ist unmittelbar klar. Sei nun $p$ ein Polynom strikt vom
%   Grad $n>1$ mit Nullstelle $x_0$. Dann gibt es nach dem euklidischen
%   Algorithmus zur Division mit Rest ein Polynom $q$ vom Grad $n-1$ und
%   eine Konstante $c$, so dass
%   \begin{gather}
%     p(x) = (x-x_0)q(x)+c.
%   \end{gather}
%   Daraus folgt $p(x_0) = c$, so dass folgt $c=0$. Wir können dieses
%   Verfahren für alle weiteren Nullstellen $x_1,\dots,x_m$ wiederholen
%   und erhalten
%   \begin{gather}
%     p(x) =  r(x) \prod_{k=0}^m (x-x_i),
%   \end{gather}
%   wobei $r(x)$ ein Polynom vom Grad $n-m$ sein muss, da $p$ vom Grad
%   $n$ ist. Insbesondere muss gelten $m\le n$.
% \end{proof}

% \begin{Korollar}{polynome-identisch}
%   Zwei reelle Polynome vom Grad $n$ sind identisch, wenn sie in
%   mindestens $n+1$ Punkten übereinstimmen. 
% \end{Korollar}


\begin{Lemma}{monome-linear-unabhaengig}
  Die Menge der Monome $\{x^0, x^1,\dots,x^n\}$ ist linear unabhängig.
\end{Lemma}

\begin{proof}
  Sei $p$ ein Polynom vom Grad $n$, also
  \begin{gather}
     p(x) = a_nx^n+a_{n-1}x^{n-1}+\dots+a_1x+a_0
   \end{gather}
   $p$ ist also gerade eine Linearkombination der Monome.  Zu zeigen
   ist, dass aus der Eigenschaft $p \equiv 0$ folgt, dass alle
   Koeffizienten verschwinden, also
  \begin{gather}
    p(x) \equiv 0
    \quad\Rightarrow\quad a_n = \dots = a_0 = 0.
  \end{gather}
  Zu diesem Zweck berechnen wir die $n$-te Ableitung von $p$ und
  erhalten, da mit $p$ auch alle seine Ableitungen identisch
  verschwinden,
  \begin{gather}
    n! a_n = 0.
  \end{gather}
  Daraus schließen wir $a_n = 0$. Nun gilt für die $(n-1)$-te Ableitung
  \begin{gather}
    n! a_n x + (n-1)! a_{n-1} = (n-1)! a_{n-1} = 0.
  \end{gather}
  Auf diese Weise schließen wir rekursiv bis $a_0$, dass alle Koeffizienten verschwinden. Damit ist das Lemma bewiesen.
\end{proof}

\begin{Satz}{polynomraum}
  Die Polynome vom maximalen Grad $n$ bilden einen Vektorraum der
  Dimension $n+1$.  Wir bezeichnen ihn mit $\P_n$.
\end{Satz}

\begin{proof}
  Es ist leicht nachzurechnen, dass sowohl die Summe, als auch reelle
  Vielfache von Polynomen wieder Polynome sind. Insbesondere erhöhen
  beide Operationen den Grad nicht. Damit ist $\P_n$ ein
  Vektorraum. Er wird per definitionem von den Monomen vom Grad bis zu
  $n$ erzeugt. Da diese nach
  \slideref{Lemma}{monome-linear-unabhaengig} linear unabhängig sind,
  bilden sie eine Basis und die Dimension von $\P_n$ ist $n+1$.
\end{proof}

\begin{Quiz}{Polynomräume}
  Gegeben beliebige Werte $x_j\in\R$ mit $j=1,\dots,n$. Die Menge der
  Polynome $p_i$ definiert durch
  \begin{align*}
    p_0(x) &= 1\\
    p_i(x) &= \prod_{j=1}^i (x-x_j),\qquad i=1,\dots,n
  \end{align*}
  \begin{enumerate}[A]
  \item ist linear unabhängig
  \item ist linear abhängig
  \item ist ein Erzeugendensystem für $\P_n$
  \item ist eine Basis von $\P_n$
  \end{enumerate}
\end{Quiz}
\section{Skalarprodukt und Orthogonalität}
\begin{Definition}{skalarprodukt}
  Sei $V$ ein reeller Vektorraum. Eine Abbildung
  $a\colon V \times V \to \R$ heißt \define{Bilinearform}, wenn für
  $u,v,w\in V$ und $\lambda,\mu\in \R$ gilt
  \begin{align}
    a(\lambda u + \mu v,w) &= \lambda a(u,w) + \mu a(v,w)\\
    a(w,\lambda u + \mu v) &= \lambda (w,u) + \mu a(w,v).
  \end{align}
  Eine Bilinearform heißt \define{symmetrisch}, wenn für $u,v\in V$ gilt
  \begin{gather}
    a(u,v) = a(v,u).
  \end{gather}
  Sie heißt \define{positiv semi-definit}, wenn $a(u,u) \ge 0$ für alle
  $u\in V$ und \define{positiv definit}, wenn zusätzlich
  \begin{gather}
    a(u,u) = 0 \quad \Longrightarrow \quad u=0.
  \end{gather}
  Eine symmetrische, positiv definite Bilinearform heißt
  \define{Skalarprodukt}, in der Regel notiert als $\scal(\cdot,\cdot)$.
\end{Definition}

%%%%%%%%%%%%%%%%%%%%%%%%%%%%%%%%%%%%%%%%%%%%%%%%%%%%%%%%%%%%%%%%%%%%%%
\begin{Lemma*}{bcs}{Bunjakowski-Cauchy-Schwarzsche Ungleichung}
  Sei $\scal(\cdot,\cdot)$ ein Skalarprodukt auf $V$.  Für zwei beliebige Elemente $u,v\in V$ gilt
  \begin{gather}
    \abs{\scal(u,v)} \le \sqrt{\scal(u,u)} \, \sqrt{\scal(v,v)}.
  \end{gather}
  Gleichheit gilt genau dann, wenn $u$ und $v$ kollinear sind, also
  $v=\alpha u$ mit einem skalaren Faktor $\alpha$.
\end{Lemma*}

\begin{proof}
  Zunächst zeigen wir nur die Ungleichung: Für $v=0\in V$ ist sie
  offensichtlich.
  
  Seien nun $v,u \in V$ keine Nullvektoren. Für beliebige $\mu, \lambda \in \R$
  gilt wegen der Bilinearität 
  \begin{gather}
   0 \le \scal(\lambda u + \mu v,\lambda u +  \mu v)
    = \lambda^{2} \scal(u,u)+2 \mu \lambda \scal(u,v) +\mu^{2} \scal(v,v)
  \end{gather}
  Setze $\lambda := \scal(v,v) \neq 0$
  \begin{gather}
   0 \le \scal(v,v)^{2} \scal(u,u) + 2\mu \scal(v,v)\scal(u,v) +\mu^{2}\scal(v,v)
  \end{gather}
  Dividiere durch$\scal(v,v)$
  \begin{gather}
   0 \le \scal(v,v) \scal(u,u) + 2\mu \scal(u,v) +\mu^{2}
  \end{gather}
  Setze nun $\mu := -\scal(u,v)$
  \begin{gather}
    0 \le \scal(v,v) \scal(u,u) -2\scal(u,v)^{2}+\scal(u,v)^{2}
  \end{gather}
  Daraus folgt
  \begin{gather}
    \scal(u,v)^{2} \le \scal(u,u) \scal(v,v)
  \end{gather}
  und mit der Monotonie der Quadratfunktion die Ungleichung.

  Nun bleibt die Äquivalenz für die Gleichheit zu zeigen.
  Für $v=0$ ist dies wieder trivial erfüllt. Seien zunächst $u,v$ linear abhängig, also zum Beispiel $u=av$.
  Dann gilt mit der Abkürzung $f(v) = \sqrt{\scal(v,v)}$
  \begin{gather}
    \abs{\scal(u,v)} = \abs{\scal(av,v)}
    = \abs{a} \cdot f(v) \cdot f(v)
    = f(av) \cdot f(v) =f(u) \cdot f(v).
  \end{gather}

  Gelte nun umgekehrt $\scal(u,v) = \sqrt{\scal(u,u)}\sqrt{\scal(v,v)}$.
  Es folgt
  \begin{gather}
     \scal(v,v) \scal(u,u) -2\scal(u,v)^{2}+\scal(u,v)^{2} = 0.
  \end{gather}
  Setze $\mu = \scal(u,v)\neq 0 $ und
  $\lambda = \scal(v,v)\neq 0$. Dann erhält man
  \begin{gather}
    \lambda \scal(u,u) - 2 \mu \scal(u,v) + \mu^2 = 0.
  \end{gather}
  Multipliplikation mit $\scal(v,v)$ ergibt
  \begin{gather}
   \lambda^2 \scal(u,u)+2\mu \scal(u,v)\scal(v,v) +\mu^{2}\scal(v,v) = 0 = \scal(\lambda u-\mu v,\lambda u-\mu v).
  \end{gather}
  
  Wegen der Definitheit folgt nun
  $\lambda u + \mu v = 0$ und da $\mu$ und $\lambda$ ungleich Null sind gilt,
  dass $ u,v$ linear abhängig sind
\end{proof}

%%%%%%%%%%%%%%%%%%%%%%%%%%%%%%%%%%%%%%%%%%%%%%%%%%%%%%%%%%%%%%%%%%%%%%
\begin{Lemma}{hilbertnorm}
  Sei $V$ ein reeller Vektorraum mit Skalarprodukt
  $\scal(\cdot,\cdot)$. Dann ist durch
  \begin{gather}
    \norm{u} = \sqrt{\scal(u,u)}
  \end{gather}
  auf $V$ eine Norm definiert. Ein reeller Vektorraum $V$ mit
  Skalarprodukt und zugehöriger Norm heißt \define{euklidischer
    Vektorraum}.
\end{Lemma}

\begin{proof}
  Das Skalarprodukt ist nicht negativ, daher ist die Abbildung $\norm{\cdot}\colon V \to \R$ wohldefiniert.
  Wir müssen nun die Normeigenschaften nachrechnen. Sei dazu $u \in V$. Es gilt
  \begin{enumerate}
  \item Nichtnegativität und Definitheit folgen sofort aus den entsprechenden Eigenschaften des Skalarprodukts.
  \item Homogenität
  \begin{gather}
    \norm{\lambda u} = \sqrt{\scal(\lambda u,\lambda u)}
    =\sqrt{\lambda^{2}\scal(u,u)}
    = \abs{\lambda}\sqrt{\scal(u,u)}
    =\abs{\lambda}\norm{u}
  \end{gather}
  \item Deiecksungleichung
  \begin{gather}
    \begin{aligned}
      \norm{u+v}^{2}
      &= \scal(u+v,u+v)\\
      &= \scal(u,u)+ 2 \scal(u,v) + \scal(v,v)\\
      \label{eq:orthopoly:1}
      &\le \scal(u,u)+ 2 \norm{u} \, \norm{v} + \scal(v,v)\\
      &=\norm{u}^{2}+ 2 \norm{u} \, \norm{v}+ \norm{v}^{2}\\
      & =(\norm{u}+\norm{v})^{2}\\
    \end{aligned}
  \end{gather}
  Daraus folgt durch Wurzelziehen auf beiden Seiten $\norm{u+v} \le \norm{u}+\norm{v}$.
  Für die Abschätzung in Zeile~\eqref{eq:orthopoly:1} haben wir die
  Bunyakovsky-Cauchy-Schwarz-Ungleichung aus \slideref{Lemma}{bcs} verwendet.
  \end{enumerate}
\end{proof}

\begin{remark}
  Ein reeller Vektorraum $V$ mit
  Skalarprodukt und zugehöriger Norm heißt \define{euklidischer
    Vektorraum}.
\end{remark}

\begin{Lemma*}{l2-norm}{$L^2$-Skalarprodukt}
  Auf dem Raum $V=\P_n$ der reellen Polynome vom Grad bis zu $n$ ist durch
  \begin{gather}
    \scal(p,q) = \int_{-1}^1 p(x)q(x)\dx
  \end{gather}
  ein Skalarprodukt definiert.
\end{Lemma*}

\begin{proof}
  Hier gilt es zu prüfen, ob die Abbildung auch die vier Eigenschaften eines
  Skalarprodukts erfüllt.\\
  Seien  $p,q,g \in \P_n$ in diesem Beweis.\\
  Da wir schon von einem Skalarprodukt ausgehen, empfiehlt es sich
  zuerst die Symmetrie zu zeigen.
  \begin{gather}
    \scal(p,q) =  \int_{-1}^1 p(x)q(x)\dx = \int_{-1}^1 q(x)p(x)\dx
    =\scal(q,p)
  \end{gather}
  Wenn wir nun zeigen, dass es eine Bilinearform ist müssen wir nur noch eine
  Identität zeigen, da wir schon wissen, dass die Symmetrieeigenschaft
  erfüllt ist.
  \begin{gather}
    \begin{aligned}
    \scal(\lambda p + \mu q, g)
    &= \int_{-1}^1 (\lambda p(x)+ \mu q(x))g(x)\dx\\
   & = \int_{-1}^1 \lambda p(x)g(x)+ \mu q(x)g(x)\dx \\
   &= \int_{-1}^1 \lambda  p(x)g(x)\dx + \int_{-1}^1 \mu q(x)g(x)\dx \\
   &= \lambda \int_{-1}^1 p(x)g(x)\dx + \mu  \int_{-1}^1 q(x)g(x)\dx \\
   &= \lambda \scal(p,g) + \mu \scal(q,g)
    \end{aligned}
  \end{gather}
  Da wir die Symmetrie vorher gezeigt haben, gilt Linearität auch
  im zweiten Argument.\\
  
  Als letztes zeigen wir, dass die Abbildung positiv definit ist.
  \begin{gather}
    0 = \scal(p,p) = \int_{-1}^1 p(x)p(x)\dx =\int_{-1}^1 p(x)^{2}\dx
  \end{gather}
  Aus den Integraleigenschaften folgt
  \begin {gather}
    0 = p(x)^{2} \quad \forall x
  \end{gather}
  Dies kann nur der Fall sein, wenn $p \equiv 0$ ist.\\
  Somit haben wir nachgerechnet, dass es sich um Skalarprodukt handelt.
  \end{proof}
    
  
\begin{Definition}{orthogonal}
  Zwei Vektoren $u,v\in V$ heißen \define{orthogonal}, wenn
  \begin{gather}
    \scal(u,v) = 0.
  \end{gather}
  Ein Vektor $u\in V$ ist orthogonal zum Untervektorraum $W\subset V$, wenn
  \begin{gather}
    \scal(u,v) = 0\quad\forall v\in W.
  \end{gather}
\end{Definition}

\begin{Notation}{euklidischer-vr}
  Von nun an bezeichnet $V$ immer einen endlichdimensionalen, reellen,
  euklidischen Vektorraum.
\end{Notation}

\begin{Lemma*}{pythagoras}{Pythagoras}
  Seien zwei Vektoren $u\in V$ und $v\in V$ orthogonal zueinander. Dann gilt
  \begin{gather}
    \norm{u+v}^{2} = \norm{u}^{2} + \norm{v}^{2}
  \end{gather}
\end{Lemma*}

\begin{proof}
  Seien $u,v \in V$. Es gilt $ 0 = \scal(u,v)$
   \begin{gather}
    \norm{u+v}^{2} = \scal(u+v,u+v)
    %=\scal(u+v,u)+\scal(u+v,v)
    %=\scal(u,u)+\scal(v,u)+\scal(u,v)+\scal(v,v)
    =\norm{u}^{2} + \norm{v}^{2} +2\scal(u,v) = \norm{u}^{2} + \norm{v}^{2}
  \end{gather}
\end{proof}

\section{Bestapproximation und orthogonale Projektion}
\begin{Definition}{bestapproximation}
  Sei $A\subset V$ ein affiner Unterraum eines euklidischen
  Vektorraums. Dann ist die Bestapproximation $u_b\in A$ eines Vektors
  $u\in V$ in $A$ definiert durch die Beziehung
  \begin{gather}
    \norm{u-u_b} = \min_{v\in A} \norm{u-v}.
  \end{gather}
\end{Definition}

\begin{Satz}{bestapproximation}
  Sei $w \in V$ und $W \subset V$.
  Sei $A=w+W$ ein nichtleerer, affiner Unterraum von $V$. Dann
  existiert die Bestapproximation nach
  \slideref{Definition}{bestapproximation} und ist eindeutig
  bestimmt. Es gilt die notwendige und hinreichende Bedingung
  \begin{gather}
    \scal(u-u_b,v) = 0 \quad \forall v\in W.
  \end{gather}
  Das heißt $ u_b$ ist Bestapproximation genau dann wenn $u-u_b$
  orthogonal zu $W$ bzgl. des Skalarprodukts $\scal(\cdot,\cdot)$ ist.
\end{Satz}

\begin{proof}
  Der Beweis gliedert sich in drei Teile. Zuert wird die Äquivalenz
  gezeigt danach zeigen wir die Eindeutigkeit und zum Schluss
  erst die Existenz.\\ \\
 $\glqq \Rightarrow \glqq$
  Sei $ u_b \in A$ die Bestapproximation des Vektors $ u \in V$\\
  Wir defnieren nun eine Funktion:
  \begin{gather}
    F_v(x):= \norm{u-u_b-xv}^{2}, x \in \R,  v\in A
  \end{gather}

  Diese Funktion besitzt ein Minimum bei x=0. Folglich gilt
  \begin{gather}
    \left. \frac{d}{dx} F(x) \right|_{x=0}
    =\left. \frac{d}{dx}\norm{u-u_b-xv}^{2} \right|_{x=0}=0
  \end{gather}
    
  Dies kann weiter umgeformt werden zu
  $\scal(u-u_b-xv,v)|_{x=0}=0 \ \forall v\in A$ und folglich zu
  \begin{gather}
   \scal(u-u_b,v)=0 \ \forall v\in A
  \end{gather}

  $\grqq \Leftarrow \glqq$
  Nun erfüllt $u_b\in A$ die Bedingung.\\
  Dann gilt mit einem beliebigen $v\in A$:
  \begin{gather}
   \norm{u-u_b}^{2}=\scal(u-u_b,u-u_b)\\
   = \scal(u-u_b,u-v)+\scal(u-u_b,v-u_b)\\
   \le \norm{u-u_b}\cdot\norm{u-v}\\
  \end{gather}
  Daraus folgt $\norm{u-u_b} \le \inf_{v\in A}{\norm{u-v}}$\\
  Damit erfüllt $u_b$ eben die Definiton der Bestapproximation\\ \\
  Nun zur Eindeutigkeit:\\
  Seien $u_b$ und $u_d \in$ A zwei Bestapproximationen.
  Dann gilt notwendigerweise
  \begin{gather}
   \scal(u-u_b,v) = 0 = \scal(u-u_d,v) \quad \forall v\in A
  \end{gather}
  Dies wird umgeformt zu
  \begin{gather}
  \scal(u-u_d,v)-\scal(u-u_b,v)=0 \quad  \forall v\in A \\
  \scal(u_b-u_d,v) = 0 \quad \forall v \in A
  \end{gather}
  Wähle nun $v:=u_b-u_d \in A$. Dies ergibt
  $\norm{u_b-u_d}^{2} =0$ und somit folgt $u_b = u_d$\\
  Die Existenz:\\
  Der endliche dimensionale Teilraum A$\subseteq$V besitzt eine Basis
  $(b_1,\dots, b_n)$ mit $n:=dim V$. Die gesuchte Approximation
  $u_b\in A$ lässt sich
  durch die Basis in folgender Form darstellen
  \begin{gather}
   u_b = \sum_{k=1}^n a_k b_k
  \end{gather}
  Dies wird in die notwendige Orthogonalitätsbedingung
  \slideref{Satz}{bestapproximation} eingesetzt.
  \begin{gather}
   \scal(u-\sum_{k=1}^n a_k b_k,v)=\scal(u,v)-\sum_{k=1}^n a_k\scal(b_k,v)=0
   \quad \forall v\in A
   \end{gather}
 Dies ist bei der Wahl von $v:=b_i \quad i=1,\dots,n$ äquivalent zu dem
 linearen $n$x$n$ Gleichungssystem.
 \begin{gather}
   \sum_{k=1}^n\scal(b_k,b_i) a_k= \scal(u,b_i) \quad i=1,\dots,n
 \end{gather}
 Definiere nun $A,x,b$ wie folgt
 \begin{gather}
  A:=(\scal(b_k,b_i))_{i,k=1}^n \quad x:=(a_k)_{k=1}^n\quad b:=(\scal(u,b_i))_{i=1}^n
 \end{gather}
 Dadurch lässt sich das LGS in der Form $Ax=b$ schreiben.
 Betrachte nun folgendes
 \begin{gather}
  x^{T}Ax =\sum_{i,k=1}^n a_i a_k\scal(b_k,b_i)=\scal(u_b,u_b)\ge 0
 \end{gather}
 $A$ ist folglich positiv definit. Das Gleichungssystem $Ax=b$ ist also für
 jede rechte Seite $b$, das heißt für jedes $u \in V$ eindeutig lösbar.
 Folglich bestimmt die Orthogonalitätsbedingung eindeutig ein Element
 $u_b \in A$, welches dann die Bestapproximation von $u$ ist.
 
\end{proof}

\begin{Definition}{komplement-projektion}
  Sei $W \subset V$ ein Untervektorraum. Dann gilt
  $V = W \oplus W^\perp$, wobei das \define{orthogonale Komplement}
  $W^\perp$ eindeutig definiert ist durch
  \begin{gather}
    W^\perp = \bigl\{ v\in V \big| \scal(v,w) = 0 \quad\forall w\in W\bigr\}.
  \end{gather}
  Die Lösung der Bestapproximationsaufgabe bezeichnen wir mit
  \begin{gather}
    \Pi_W u = u_b\in W
  \end{gather}
  und nennen es die \define{orthogonale Projektion} von $u\in V$ auf $W$.
\end{Definition}

\begin{Lemma}{komp-projekt-wohldefiniert}
  Das orthogonale Komplement und die orthogonale Projektion sind wohldefiniert.
\end{Lemma}

\begin{proof}
  \slideref{Satz}{bestapproximation}.
\end{proof}

\begin{Beispiel}{polynom-bestapproximation}
  Die Aufgabe der Gaußschen Ausgleichsrechnung lautet: finde zu einer
  gegebenen Funktion $f$ das Polynom vom Grad höchstens $n$, das auf
  dem Intervall $[-1,1]$ den mittleren quadratischen Abstand
  minimiert, also $p\in \P_n$ mit
  \begin{gather}
    \int_{-1}^1 \bigl(f(x)-p(x)\bigr)^2 \dx
    = \min_{q\in \P_n} \int_{-1}^1 \bigl(f(x)-q(x)\bigr)^2 \dx.
  \end{gather}
  Die Lösung erfüllt
  \begin{gather}
    \int_{-1}^1 p(x)q(x) \dx = \int_{-1}^1 f(x)q(x) \dx
    \qquad\forall q\in \P_n.
  \end{gather}
\end{Beispiel}

\section{Orthogonale Basen}

\begin{Lemma}{gram-system}
  Wählt man eine Basis $\{\phi_i\}$ für $W$, so transformiert wird die
  Orthogonalitätsbedingung in \slideref{Satz}{bestapproximation} zum
  linearen Gleichungssystem
  \begin{gather}
    \matg\vx = \vb.
  \end{gather}
  Hier sind $\vx$ der Koeffizientenvektor der Lösung $u_b$, $\matg$ die
  \define{Gramsche Matrix} und $\vb$ die rechte Seite gegeben durch
\begin{gather}
  g_{ij} = \scal(\phi_i,\phi_j), \qquad
  b_i = \scal(u,\phi_i).
\end{gather}
\end{Lemma}

\begin{remark}
  Das Gleichungssystem hängt nur von der Wahl einer Basis in $W$ ab,
  nicht in $V$.
\end{remark}

\begin{Definition}{ortho-system}
  Eine Menge von Vektoren $\{\phi_1,\dots,\phi_n\}\subset V$ bildet
  ein \define{Orthogonalsystem}, wenn
  \begin{gather*}
    \scal(\phi_i,\phi_j) = 0
    \qquad \forall 1\le i < j \le n.
  \end{gather*}
  Sie ist ein \define{Orthonormalsystem}, wenn zusätzlich
  $\norm{\phi_i} = 1$ für alle Elemente gilt. Ein Orthonormalsystem, das eine Basis bildet, heißt \define{Orthonormalbasis} (\define{ONB}).
\end{Definition}

\begin{Lemma}{ortho-lu}
  Jedes Orthogonalsystem ist linear unabhängig.
\end{Lemma}

\begin{Lemma*}{parseval}{Parsevalsche Gleichung}
  Sei $\{\phi_i\}$ für $i=1,\dots,n$ eine ONB von $V$. dann gilt für
  jedes $v\in V$ mit der Basisdarstellung
  \begin{gather}
    v = \sum_{i=1}^n x_i \phi_i
  \end{gather}
  die Identität
  \begin{gather}
    \norm{v}^2 = \sum_{i=1}^n x_i^2.
  \end{gather}
\end{Lemma*}
\begin{Lemma}{least-squares-orthogonal}
  Bezüglich einer ONB ist die Gramsche Matrix die
  Einheitsmatrix. Damit berechnen sich die Einträge des
  Koeffizientenvektors $\vx$ in \slideref{Lemma}{gram-system} durch
  die einfache Formel
  \begin{gather}
    x_i = b_i = \scal(u,\phi_i).
  \end{gather}
\end{Lemma}

\begin{Theorem*}{gram-schmidt}{Gram-Schmidt-Verfahren}
  Jede linear unabhängige Menge von Vektoren
  $\{v_1,\dots,v_n\}\subset V$ wird mit dem folgenden Verfahren in ein
  Orthonormalsystem $\{\phi_1,\dots,\phi_n\}\subset V$ umgeformt:
  \begin{gather}
    \begin{aligned}
      \phi_1 &= \tfrac1{\norm{v_1}} \,v_1\\
      w_j &= v_j - \sum_{i=1}^{j-1} \scal(v_j,\phi_i)\,\phi_i
      & \quad \phi_j &= \tfrac1{\norm{w_j}}\, w_j
      &\quad j=2,\dots,n
    \end{aligned}
  \end{gather}
  Für alle $1\le k \le n$ gilt
  \begin{gather}
    \operatorname{span}\{\phi_1,\dots,\phi_k\}
    =
    \operatorname{span}\{v_1,\dots,v_k\}
  \end{gather}
\end{Theorem*}

\begin{proof}
  Per Induktion über $n$ zeigen wir Orthogonalität und Normierung.\\

  $Indukionsanfang$ Sei $n=1$.\\
  Wird nur ein Vektor aus dem Raum gewählt, so erfüllt dieser
  die Orthogonalitätsbedingung, da er der einzige Vektor im System ist.
  Wird dieser Vektor zusätzlich normiert erhält man ein Orthonormalsystem.\\
  
  $Induktionsschritt$ Das Verfahren gelte
  für $\{v_1,\dots,v_{n-1}\}$ Vektoren aus V. \\
  $n-1 \rightarrow n$\\
  Sei $(\phi_1,\dots,\phi_{n-1})$ ein Orthonormalsystem\\
  Annahme $w_n$ nicht wohldefiniert. Dann gilt
  \begin{gather}
    w_n = v_n -\sum_{i=1}^{n-1}\scal(v_n,\phi_i)\,\phi_i = 0
  \end{gather}
  In diesem Fall sind $(v_1,\dots,v_n)$ linear abhängig.
  Das ist ein Widerspruch zur Voraussetzung,
  dass $(v_1,\dots,v_n)$ linear unabhängig sind.\\
  $w_n$ wird nun normiert über $\frac{1}{\norm{w_n}} \cdot w_n =\phi_n $.
  Nun zur Orthogonalität:
  \begin{gather}
    \scal(\phi_n,\phi_j)=\scal(v_n,v_j)-
    \sum_{i=1}^{n-1}\scal(v_n,\phi_i)
    \,\underbrace{\scal(\phi_i,\phi_j)}_{=\delta_{ij}}  = 0
    \quad j=1,\dots,n-1
  \end{gather}
\end{proof}

\begin{Algorithmus*}{gram-schmidt}{Gram-Schmidt}
  \lstinputlisting{code/gram-schmidt.py}
\end{Algorithmus*}

\begin{Beispiel}{gram-schmidt}
  Wir wählen für Polynome das $L^2$-Skalarprodukt aus
  \slideref{Lemma}{l2-norm} und die Basis $\{1,x,\dots,x^{n-1}\}$
  für $\P_{n-1}$. Wir verwenden die Iplementation in
  \slideref{Algorithmus}{gram-schmidt} und messen den Erfolg nach der
  Größe der Nebendiagonaleinträge der Gramschen Matrix.
  \begin{center}
    \begin{tabular}{c|c}
      $n$ & $\max_{i\neq j} \abs{g_{ij}}$ \\
      \hline
      5 & $8.9\cdot 10^{-16}$ \\
      10 & $9.1\cdot 10^{-12}$ \\
      15 & $1.2\cdot 10^{-7}$ \\
      20 & $0.23$
    \end{tabular}
  \end{center}
\end{Beispiel}

\begin{Algorithmus*}{mgs}{Modifizierter Gram-Schmidt}
  \lstinputlisting{code/modified-gram-schmidt.py}  
\end{Algorithmus*}

\begin{Beispiel}{gs-mgs}
  In dieser Tabelle wiederholen wir die Zahlen
  $\max_{i\neq j} \abs{g_{ij}}$ aus \slideref{Beispiel}{gram-schmidt}
  und stellen sie den entsprechenden Ergebnissen des modifizierten
  Verfahrens in \slideref{Algorithmus}{mgs} gegenüber.
  \begin{center}
    \begin{tabular}{c|cc}
      $n$ &  Gram-Schmidt & modifiziert\\
      \hline
      5 & $8.9\cdot 10^{-16}$ & $1.3\cdot 10^{-16}$ \\
      10 & $9.1\cdot 10^{-12}$ & $2.9\cdot 10^{-12}$ \\
      15 & $1.2\cdot 10^{-7}$ & $2.7\cdot 10^{-9}$ \\
      20 & $0.23$ & $3.9\cdot 10^{-5}$
    \end{tabular}
  \end{center}
\end{Beispiel}

\begin{remark}
  Wir sehen, dass die Wahl der Implementation eines Rechenverfahrens
  bei mathematischer Äquivalenz durchaus erheblichen Einfluss auf das
  Ergebnis haben kann. Dieses Phänomen werden wir in
  \Cref{sec:stability} näher untersuchen. Zunächst diskutieren wir
  aber eine weitere Variante der Erzeugung orthogonaler Basen in
  Polynomräumen.
\end{remark}

\section{Drei-Term-Rekursion}

\begin{Satz*}{dreiterm}{Dreiterm-Rekursion}
  Zu jedem Skalarprodukt $\scal(\cdot,\cdot)$ auf dem Raum der
  stetigen Funktionen gibt es genau eine Folge von orthogonalen
  Polynomen $p_k\in \P_k$ mit führendem Koeffizienten eins. Sie
  genügen der Dreiterm-Rekursionsformel
  \begin{gather}
    p_k(x) = (x-a_k)p_{k-1}(x) - b_k p_{k-2}(x),
    \qquad k=1,2,\ldots
  \end{gather}
  mit Startwerten $p_{-1} \equiv 0$ und $p_0 \equiv 1$. Die
  Koeffizienten sind
  \begin{gather}
    a_k = \frac{\scal(x p_{k-1},p_{k-1})}{\scal(p_{k-1},p_{k-1})}
    \qquad\text{und}\qquad
    b_k = \frac{\scal(p_{k-1},p_{k-1})}{\scal(p_{k-2},p_{k-2})}.
  \end{gather}
\end{Satz*}

\begin{proof}
  Siehe \cite[Satz 6.2]{DeuflhardHohmann08}
\end{proof}

\begin{Bemerkung}{dreiterm-normierung}
  Der Beweis ergibt, eigentlich die ``Eindeutigkeit einer Orthogonalfolge bis auf Normierung''. Tatsächlich werden in der Literatur immer wieder veschiedene Normierungen benutzt. Beispiele sind:
  \begin{enumerate}
  \item Führender Koeffizient eins, $p_k = x^k + \dots$
  \item $\norm{p_k} = 1$
  \item $p_k(1) = 1$
  \end{enumerate}
\end{Bemerkung}

\begin{Definition}{legendre-polynome}
  Die \define{Legendre-Polynome} $L_k$ sind definiert durch
  die Dreiterm-Rekursion
  \begin{gather}
    L_{k} = \tfrac{2k-1}{k}x L_{k-1}(x) - \tfrac{k-1}{k} L_{k-2}(x).
  \end{gather}
  Sie sind orthogonal bezüglich des $L^2$-Skalarprodukts in
  \slideref{Lemma}{l2-norm}.
\end{Definition}

\begin{Beispiel}{least-squares-legendre}
  Das Problem der Gaußschen Ausgleichsrechnung war: zu einer gegebenen
  Funktion $f$ finde $p\in \P_n$, so dass
  \begin{gather}
    \int_{-1}^1 (f-p)^2 \dx
    = \min_{q\in\P_n} \int_{-1}^1 (f-q)^2 \dx.
  \end{gather}
  Mit Hilfe der Legendre-Polynome können wir nun die Lösung explizit angeben als
  \begin{gather}
    p(x) = \sum_{i=0}^n \alpha_i L_i(x)
    \qquad\text{mit}\qquad
    \alpha_i = \frac1{\norm{L_i}^2}\int_{-1}^1 f L_i(x)\dx.
  \end{gather}
\end{Beispiel}

\begin{Definition}{chebyshev-polynome}
  Die \define{Tschebyscheff-Polynome} $T_k$ sind definiert durch
  die Dreiterm-Rekursion
  \begin{gather}
    T_{k} = 2x T_{k-1}(x) - T_{k-2}(x).
  \end{gather}
  Sie sind orthogonal bezüglich des Skalarprodukts
  \begin{gather}
    \scal(p,q) = \int_{-1}^1 \tfrac1{\sqrt{1-x^2}} \,p(x)q(x)\dx.
  \end{gather}
\end{Definition}

%%% Local Variables:
%%% mode: latex
%%% TeX-master: "main"
%%% End:


\chapter{Solving Large Sparse Linear Systems}

\section{Motivation: discretization of partial differential equations}

\begin{Definition}{sparse-matrix}
  We call a matrix \textbf{sparse}\index{sparse matrix} in strict
  sense, if the number of nonzero entries is much less than the total
  number of entries, typically $\bigo(n)$ instead of $\bigo(n^2)$. The
  distribution of nonzero entries is called \define{sparsity pattern}.

  More broadly, a sparse matrix has the property that its application
  to a vector requires considerably less than $n^2$, typically
  $\bigo(n)$ operations. It also can be stored with considerably less
  than $n^2$ floating point numbers.
\end{Definition}

\begin{Example*}{page-rank}{The PageRank Algorithm}
  The page rank of a web page is computed from the Google Matrix
  \begin{gather}
    P = d (\matl+\vw \mathbf 1^T) + \tfrac{(1-d)}{n} \id,
  \end{gather}
  where $d \in (0,1)$ is a damping parameter, $\mathbf 1^T = (1,\dots,1)$,
  \begin{gather}
    l_{ij} =
    \begin{cases}
      \nicefrac1{c_i} & \text{if $i$ links to $j$}\\0&\text{else} 
    \end{cases},
    \qquad
    w_i =
    \begin{cases}
      1 &\text{if } c_i=1\\
      0&\text{else}
    \end{cases},
  \end{gather}
  and $c_i$ is the number of links from page $i$.
\end{Example*}

\begin{example}
  Discretization of partial differential equations by finite
  difference methods.
\end{example}

\begin{Example*}{csr}{Compressed row storage (CSR)}
  A sparse matrix can be stored using two integer fields defining the
  sparsity pattern and a field of floating point values for its
  entries. Let \lstinline!n! be the dimension of the matrix and
  \lstinline!n_nonzero! the total number of nonzero entries.
  \begin{lstlisting}[language=Python]
    import numpy as np
    row_start = np.empty(n, dtype=np.uint)
    column = np.empty(n_nonzero, dtype=np.uint)
    entries = np.emtpy(n_nonzero, dtype=np.double)
  \end{lstlisting}

  The operation $\vy=\mata\vx$ is then implemented as
  \begin{lstlisting}[language=Python]
    for i in range(0,n):
    y[i] = 0.
    for j in range (row_start[i],row_start[i+1):
      y[i] += entries[j]*x[column[j]]
  \end{lstlisting}  
\end{Example*}

\begin{Remark}{algorithmic-matrix}
  In cases where the sparsity pattern and the entries of a matrix are
  known at compile time, a compressed stored matrix can be substituted
  by an algorithm performing the multiplication.

  Thus, we start viewing matrices more as linear operators than as
  rectangular schemes of numbers.
\end{Remark}
%%% Local Variables:
%%% mode: latex
%%% TeX-master: "main"
%%% End:


\section{Basic iterative methods}

\begin{Definition}{jacobi}
  The \define{Jacobi iteration} for a matrix $\mata\in\Rnn$ and a
  right hand side vector $\vb\in \R^n$ generates the iterate
  $\vx^{(k+1)}\in \R^n$ from $\vx^{(k)}\in \R^n$ as follows:
  \begin{gather}
     x^{(k+1)}_i = \frac1{a_{ii}}\left( b_i - \sum_{j\neq i} a_{ij}x^{(k)}_j\right).
  \end{gather}
\end{Definition}

\begin{Definition}{gauss-seidel}
  The \define{Gauss-Seidel iteration} for a matrix $\mata\in\Rnn$ and a
  right hand side vector $\vb\in \R^n$ generates the iterate
  $\vx^{(k+1)}\in \R^n$ from $\vx^{(k)}\in \R^n$ as follows:
  \begin{gather}
    x^{(k+1)}_i = \frac1{a_{ii}}
    \left( b_i
      - \sum_{j< i} a_{ij}x^{(k+1)}_j
      - \sum_{j> i} a_{ij}x^{(k)}_j
  \right).
  \end{gather}
\end{Definition}

\begin{Definition}{richardson-iteration}
  The \define{Richardson iteration} for a matrix $\mata\in\Rnn$ and a
  right hand side vector $\vb\in \R^n$ generates the iterate
  $\vx^{(k+1)}\in \R^n$ from $\vx^{(k)}\in \R^n$ as follows:
  \begin{gather}
    \vx^{(k+1)} = \vx^{(k)} - 
    \omega_k\left( \mata\vx^{(k)} - \vb
  \right).
\end{gather}
The relaxation parameter $\omega_k$ must be chosen carefully to obtain
convergence.
\end{Definition}

\begin{Definition}{matrix-iteration}
  We call a \define{matrix iteration} any iterative method of the structure
  \begin{gather}
    \vx^{(k+1)} = \matm \vx^{(k)} + \vg,
  \end{gather}
  with an \define{iteration matrix} $\matm\in\Rnn$ and an inhomogeneity $\vg$.
\end{Definition}

\begin{Theorem*}{bfpt}{Banach fixed-point theorem}
  Let $V$ be a vector space with norm $\norm{\cdot}$ and $M$ be a
  closed subset of $V$. Let $F\colon M\to M$ be a \define{contraction},
  that is, there is a \define{contraction number} $q\in[0,1)$ such that
  \begin{gather}
    \norm{F(\vx)-F(\vy)} \le q \norm{\vx-\vy}\qquad\forall \vx,\vy\in M.
  \end{gather}
  Then, there is a unique \define{fixed-point} $\vx^*\in M$ with the property
  \begin{gather}
    F(\vx^*) = \vx^*.
  \end{gather}
  The \define{fixed-point iteration}
  \begin{gather}
    \vx^{(k+1)} = F\bigl(\vx^{(k)}\bigr)
  \end{gather}
  converges to $\vx^*$ for any $\vx^{(0)}\in M$.
%and there holds
%  \begin{gather}
%    \norm{\vx^{(k)}-x^*} \le \frac{q^n}{1-q}\norm{\vx^{(1)}-\vx^{(0)}}.
%  \end{gather}
\end{Theorem*}

\begin{Corollary}{bfp-estimates}
  Let $F$ define a fixed-point iteration with contraction number $q<1$
  and let $x^*$ be the unique fixed-point. Then, the following
  estimates hold:
  \begin{align}
    \norm{x^{(k)} - x^*} &\le \frac{q}{1-q} \norm{x^{(k)}-x^{(k-1)}}\\
    \norm{x^{(k)} - x^*} &\le \frac{q^k}{1-q} \norm{x^{(1)}-x^{(0)}}
  \end{align}
\end{Corollary}  

\begin{Corollary}{matrix-norm-convergence}
  Let $\norm{\matm} < 1$ for some operator norm of a vector norm $\norm{\cdot}$ on $\R^n$. Then, the matrix iteration
  \begin{gather}
    \vx^{(k+1)} = \matm \vx^{(k)} + \vg
  \end{gather}
  converges for any initial value $\vx^{(0)}\in\R^n$.
\end{Corollary}

\begin{Example}{matrix-norm-convergence}
  Let $\mata\in\R^{2\times 2}$ be a rotation by 45\textdegree combined
  with a scaling,
  \begin{gather}
    \mata = 0.9
    \begin{pmatrix}
      \cos \tfrac\pi4 &\sin\tfrac\pi4\\
      -\sin\tfrac\pi4&\cos\tfrac\pi4
    \end{pmatrix}.
  \end{gather}
  Its spectral norm is
  \begin{gather}
    \norm{\mata}_2 = 0.9,
  \end{gather}
  while
  \begin{gather}
    \norm{\mata}_1 \ge 0.9\sqrt2 > 1.2
  \end{gather}
  which can be seen by mapping the vector $(1,0)^T$. We conclude, that
  a matrix iteration might be a contraction with respect to one norm,
  but not with respect to another.
\end{Example}

\begin{remark}
  The statement of \slideref{Corollary}{matrix-norm-convergence}
  implies, that it is sufficient to find one vector norm such that the
  iteration is a contraction to prove convergence. The example shows
  that this can only be a sufficient condition, but not a necessary
  one.

  After the following Lemma, we will prove a result which provides us
  with a sufficiant \emph{and} necessary condition.
\end{remark}

\begin{Lemma}{norm-spectral-radius}
  For any matrix $\mata\in\Rnn$ and for any $\epsilon>0$ there exist a
  vector norm $\norm{\cdot}_{\mata,\epsilon}$ and associated operator norm
  denoted by the same symbol, such that
  \begin{gather}
    \rho(\mata) \le \norm{\mata}_{\mata,\epsilon} \le \rho(\mata)+\epsilon.
  \end{gather}
\end{Lemma}

\begin{proof}
  See~\cite[Lemma 3.1]{Rannacher18nla}.
\end{proof}

\begin{Theorem}{matrix-radius-convergence}
  The matrix iteration
  \begin{gather}
    \vx^{(k+1)} = \matm \vx^{(k)} + \vg
  \end{gather}
  converges to a fixed-point $x^*$ for any start vector $x^{(0)}$,
  if and only if for the spectral radius there holds
  \begin{gather}
    \rho(\matm) < 1.
  \end{gather}
  Then, there holds asymptotically only
  \begin{gather}
    \operatorname*{lim\,sup}_{k\to\infty} \frac{\norm{x^{(k+1)}-x^*}}{\norm{x^{(k)}-x^*}}
    < \rho(\matm).
  \end{gather}
\end{Theorem}

\begin{proof}
  See~\cite[Theorem 3.1]{Rannacher18nla}.
\end{proof}

\begin{remark}
  While \slideref{Theorem}{matrix-radius-convergence} has the
  mathematically more pleasing statement ``if and only if'', its
  assumptions do not establish a contraction. In particular, the norms
  of iterates may grow in early steps of the iteration and only then
  start decreasing, something which is not desired from a practical
  point of view.
\end{remark}

%%% Local Variables:
%%% mode: latex
%%% TeX-master: "main"
%%% End:


\section{Krylov-space methods}
\input{Krylov}

\chapter{Large Sparse Eigenvalue Problems}

\begin{intro}
  In this chapter we study algorithms for finding a small number
  $n_{\text{ev}}$ of eigenvalues and corresponding eigenvectors of a
  large matrix $\mata\in\Rnn$, where $n$ is large, for instance
  $n>10^6$. Like in the previous chapter, this is achieved by
  projecting thw peoblem into smaller subspaces, where we can solve
  the eigenvalue problem using the methods of the first chapter.
\end{intro}

\section{Projected subspace iteration}
\begin{Definition}{galerkin-ev}
  Let $\mata\in\Rnn$. Then, the solutions to the eigenvalue problem
  \begin{align}
    \tilde\vx &\in K,\\
    \mata\tilde\vx - \mu\tilde\vx&\perp L.
  \end{align}
  are called the \define{Galerkin
    approximation} of the eigenvalue value problem in a subspace $K$ orthogonal to a subspace
  $L$. In the case $K=L$, it is also called the \define{Rayleigh-Ritz approximation}.
\end{Definition}

\begin{Algorithm*}{rayleigh-ritz}{Rayleigh-Ritz method}
  \begin{enumerate}
  \item Compute an orthonormal basis $\matq_m$ for the subspace $K$ and let
    \begin{gather}
      \matb_m = \matq_m^T\mata\matq_m \in \R^{m\times m}.
    \end{gather}
  \item Compute the eigenvalues of $\matb_m$ and select $k\le m$ ``desired'' eigenvalues
    \begin{gather}
      \mu_1, \dots,\mu_k.
    \end{gather}
  \item Compute the eigenvectors $\vy_1,\dots,\vy_k\in\R^m$ of $\matb$
    and the corresponding approximate eigenvectors of $\mata$:
    \begin{gather}
      \tilde\vu_i = \matq_m \vy_i,\qquad i=1,\dots,k.
    \end{gather}
  \end{enumerate}
\end{Algorithm*}

\begin{Definition}{ritz-values}
  Let $\mata$ be a matrix and $\matb_m = \matq_m^*\mata\matq_m$ be its
  orthogonal projection to the subspace $V$ spanned by the orthonormal
  basis $\matq_m$. Then, we refer to the eigenvalues $\mu_i$ of
  $\matb_m$ as \define{Ritz values}.  For each eigenvector $\vy_i$ of
  $\matb_m$, we refer to the vector $\vx_i = \matq_m\vy_i$ as
  \define{Ritz vector}. We call the pairs $(\mu_i,\matq_m\vy_i)$ the
  \define{Rayleigh-Ritz approximation} of an eigenpair of $\mata$ in the
  subspace $V$.
\end{Definition}

\begin{Lemma}{ritz-invariant}
  Let $V\subset\R^n$ be an invariant subspace under the action of
  $\mata$. Then, the Ritz values and associated Ritz vectors of the
  Rayleigh-Ritz approximation in the space $V$ are exact eigenpairs of $\mata$.
\end{Lemma}

\begin{proof}
  Let $\vq_1,\dots,\vq_m$ be an orthonormal basis for the invariant
  subspace $V$, and let $\matb_m = \matq_m^T\mata\matq$ be the projected
  matrix. For an eigenvector $\vy\in\R^m$ to eigenvalue $\lambda$ of
  $\matb_m$, compute the Ritz vector $\vv = \matq_m\vy$. Then,
  \begin{gather}
    \lambda \vv = \lambda\matq_m\vy = \matq_m\matb\vy
    = \matq_m\matq_m^T\mata\matq_m\vy = \matq_m\matq_m^T\mata\vv.
  \end{gather}
  Since $\matq_m\matq_m^T$ is the orthogonal projector to $V$, the
  term on the right is equal to $\mata\vv$ if and only if $V$ is
  invariant.
\end{proof}

% \begin{remark}
%   A variation of \slideref{Algorithm}{rayleigh-ritz} computes the
%   Schur vectors of $\matb$ instead of the eigenvectors. For symmetric
%   matrices, they are actually the same vectors, but for
%   non-diagonalizable matrices, this variant is preferred as it does
%   not require cumbersome computation of invariant subspaces.
% \end{remark}

\begin{intro}
  We now focus on the Rayleigh-Ritz method for real, symmetric
  matrices. The results transfer immediately to the complex symmetric
  (Hermitian) case, but we avoid formulating with complex numbers.
\end{intro}

\begin{Lemma}{Courant-Fischer-Ritz}
  Let the eigenvalues of a symmetric matrix $\mata\in\Rnn$ be ordered such that
  \begin{gather}
    \lambda_1\ge \lambda_2\ge \dots \ge \lambda_n.
  \end{gather}
  Let $\mu_1,\dots,\mu_m$ be the Ritz values in the subspace
  $V$. Then, there holds
  \begin{xalignat}2
    \lambda_i &\ge \mu_i, &i&=1,\dots,m,\\
    \lambda_{n-i}& \le \mu_{m-i}, &i&=0,\dots,m-1.
  \end{xalignat}
\end{Lemma}

\begin{proof}
  Due to the Courant-Fischer min-max theorem,
  \slideref{Theorem}{minmax}, we can characterize the Ritz values as
  \begin{alignat}2
    \mu_i &= \max_{\substack{W \subset V\\\dim W = i}} \min_{\vx\in W} R_\mata(\vx)
    &\le \max_{\substack{W \subset \R^n\\\dim W = i}} \min_{\vx\in W} R_\mata(\vx)
    &= \lambda_i\\
    \mu_{m-i}&= \min_{\substack{W \subset V\\\dim W = i}} \max_{\vx\in W} R_\mata(\vx)
    &\le \min_{\substack{W \subset \R^n\\\dim V = i}} \max_{\vx\in W} R_\mata(\vx)
    &= \lambda_{n-i}.
  \end{alignat}
\end{proof}

\begin{Lemma}{ritz-rayleigh-estimate}
  Let $(\lambda,\vu)$ be an eigenpair of the symmetric matrix
  $\mata\in\Rnn$. Let $P_V$ be the orthogonal projection to the
  subspace $V$. Then, the \putindex{Rayleigh quotient}
  $R_\mata(P_V \vu)$ admits the estimate
  \begin{gather}
    \abs*{\lambda-R_\mata(P_V \vu)}
    \le \norm{\mata-\lambda\id}
    \frac{\norm{\vu-P_V\vu}^2}{\norm{P_V\vu}^2}
    = \norm{\mata-\lambda\id} \tan^2\theta
    ,
  \end{gather}
  where $\theta$ is the angle between $V$ and the subspace spanned by
  the eigenvector $\vu$.
\end{Lemma}

\begin{proof}
  See also~\cite[Lemma 4.1]{Saad11}.

  Let $\lambda$ be an eigenvalue of $\mata$. We have
  \begin{gather}
    \lambda-R_\mata(P_K \vu)
    = \frac{\scal({(\lambda\id-\mata)} P_V\vu,P_V\vu)}{\scal(P_V\vu,P_V\vu)}.
  \end{gather}
  Furthermore, if $\vu$ is an eigenvector for $\lambda$, then
  \begin{gather}
    (\lambda\id-\mata)P_V\vu = (\lambda\id-\mata) \bigl[u-(\id-P_V)\vu\bigr]
    = (\mata-\lambda\id) (\id-P_V)\vu,
  \end{gather}
  and
  \begin{gather}
    0 = \scal(P_V\vu,{(\mata-\lambda\id)}u)
    = \scal({(\mata-\lambda\id)}P_V\vu,\vu).
  \end{gather}
  Hence,
  \begin{gather}
    \lambda-R_\mata(P_K \vu)
    = \frac{\scal({(\mata-\lambda\id)} {(\id-P_V)}\vu,{(\id-P_V)}\vu)}{\scal(P_V\vu,P_V\vu)}.
  \end{gather}
  
\end{proof}

\begin{Lemma}{ritz-estimate-sine}
  Let $(\lambda,\vu)$ be an eigenpair of the symmetric matrix
  $\mata\in\Rnn$, and let $(\mu,\tilde\vu)$ be the eigenpair
  of the projected problem such that $\mu$ is the closest
  Ritz value to $\lambda$.  Then,
  \begin{gather}
    \abs{\lambda-\mu}
    \le \norm{\mata-\lambda\id} \sin^2\theta,
  \end{gather}
  where $\theta$ is the angle between $\vu$ and $\tilde\vu$.
\end{Lemma}

\begin{proof}
  Note that
  \begin{gather}
    \mu-\lambda
    = \scal(\mata\tilde \vu,\tilde\vu)
    - \lambda\scal(\tilde \vu,\tilde\vu)
    = \scal({(\mata-\lambda)} \tilde\vu,\tilde\vu).
  \end{gather}
  Let now $\alpha = \scal(\vu,\tilde\vu) = \cos\theta$. Exploiting
  $(\mata-\lambda\id)\vu=0$, we write
  \begin{gather}
    \mu-\lambda
    = \scal({(\mata-\lambda\id)}{(\tilde\vu - \alpha\vu)},\tilde\vu)
    = \scal({(\mata-\lambda\id)}{(\tilde\vu - \alpha\vu)},\tilde\vu - \alpha\vu).
  \end{gather}
  Hence, by Bunyakovsky-Cauchy-Schwarz,
  \begin{align}
    \abs{\mu-\lambda}
    &\le \norm{(\mata-\lambda\id)(\tilde\vu - \alpha\vu)}\norm{\tilde\vu - \alpha\vu}\\
    &\le \norm{\mata-\lambda\id}\norm{\tilde\vu - \alpha\vu}^2\\
    &= \norm{\mata-\lambda\id} \sin^2\theta.
  \end{align}
\end{proof}


\begin{Algorithm*}{ev-projection}{Projected subspace iteration}
  \begin{algorithmic}[1]
    \Require $\matx^{(0)} = (\vx_1,\dots,\vx_m)$
    \For {$k=1,\dots$ until convergence}
    \State $\matw \gets \mata\matx^{(k-1)}$
    \State $\matq \gets ONB(\matw)$ \Comment{QR factorization}
    \State $\matb \gets \matq^*\mata\matq$ \Comment{Projected matrix}
    \State $\maty \gets EV(\matb)$ \Comment{Eigenvectors}
    \State $\matx^{(k)} \gets \matq\maty$ \Comment{Ritz vectors}
    \EndFor
  \end{algorithmic}
\end{Algorithm*}

\begin{Theorem}{ritz-convergence}
  Let $D_m(\mata)$ be the space spanned by the dominant $m$
  eigenvectors $\vv_1,\dots,\vv_m$ of $\mata$ and let $\matp_D$ be the
  orthogonal projector onto it. Let $S_k$ be the subspace spanned by
  the basis $\matx^{(k)}$.
  
  If $\matp_D S_0 = D_m(\mata)$, then there holds: for each
  eigenvector $\vv_i, i=1,\dots,m$, there is a vector
  $\vy_i\in S_0$ such that $\matp\vy_i = \vv_i$. Furthermore,
  \begin{gather}
    \norm{\vv_i - \matp_{S_k} \vv_i}_2 \le \norm{\vv_i-\vy_i}
    \left(\abs*{\frac{\lambda_{m+1}}{\lambda_i}}+\epsilon_k\right)^k,
  \end{gather}
  where $\matp_{S_k}$ is the orthogonal projector onto $S_k$ and
  $\epsilon_k\to 0$ as $k \to \infty$.
\end{Theorem}

\begin{proof}
  See~\cite[Theorem 5.2]{Saad11}
\end{proof}

\begin{remark}
  The estimates are all with respect to the $m+1$-st eigenvalue. This
  differs from the standard subspace iteration in the first chapter,
  where we always deal with the separation to the following
  eigenvalue.
  
  Thus, it is a reasonable approach to choose $m > n_{\text{ev}}$,
  such that for all relevant eigenvalues the quotient
  $\lambda_{m+1}/\lambda_i$ is sufficiently small. This is
  particularly true if there is a known gap in the spectrum.

  With respect to this property, this method seems superior to the
  orthogonal subspace iteration. Be aware though that a Schur
  decomposition by QR factorization or the computation of eigenvalues
  and eigenvectors of $\matb$ relies on methods from the first
  chapter.
\end{remark}



%%% Local Variables:
%%% mode: latex
%%% TeX-master: "main"
%%% End:


\section{The Arnoldi/Lanczos method}
\begin{Lemma}{ev-arnoldi-a-posteriori}
  Let $(\lambda_i^{(m)},\vy_i^{(m)})$ be an eigenpair of the projected
  matrix $\matH_m$ of Arnoldi's method. Then, the Ritz vector
  $\vu_i^{(m)} = \matv_m \vy_i^{(m)}$ has a residual represented as
  \begin{gather}
    \left(A-\lambda_i^{(m)}\identity\right)\vu_i^{(m)}
    = h_{m+1,m}\ve_m^*\vy_i^{(m)}\vv_{m+1}.
  \end{gather}
  In particular, its norm is
  \begin{gather}
    \norm*{\left(A-\lambda_i^{(m)}\identity\right)\vu_i^{(m)}}_2
    = h_{m+1,m} \abs{\ve_m^*\vy_i^{(m)}}.
  \end{gather}
\end{Lemma}

\begin{proof}
  See~\cite[Proposition 6.8]{Saad11}.
\end{proof}

\begin{Corollary}{ev-arnoldi-breakdown}
  If the Arnoldi algorithm breaks down, the generated Ritz vectors are
  eigenvectors.
\end{Corollary}

%%% Local Variables:
%%% mode: latex
%%% TeX-master: "main"
%%% End:



\appendix
\chapter{Basics from Linear Algebra}
\section{Bases and matrices}
\subsection{Matrix notation for bases}
\begin{Notation}{column-vectors}
  For a matrix $\mata\in \C^{n\times k}$ with entries $a_{ij}$, we denote its column vectors by $\va_j$ such that
  \begin{gather}
      (\va_j)_i = a_{ij}.
  \end{gather}
  Vice versa, given vectors $\vv_1,\dots,\vv_k$ in $\C^n$, we denote the matrix generated by those column vectors as
  \begin{gather}
      \matv = \bigl( \vv_1,\dots,\vv_k\bigr).
  \end{gather}
\end{Notation}

\begin{Lemma}{matrix-linear-combination}
  Let $\vv_1,\dots,\vv_k$ be a set of vectors in $\C^n$ and let the matrix $\matv=(v_1,\dots,v_k) \in \C^{n\times k}$. Then, for vectors $\va\in\C^k$ there holds
  \begin{gather}
      \vu = \sum_{i=1}^k a_i \vv_i
      \qquad \Longleftrightarrow \qquad
      u = \matv \va.
  \end{gather}
  Thus, we can write linear combinations as matrix vector products.
\end{Lemma}

\begin{proof}
  Direct application of the definition of the matrix product.
\end{proof}


%%% Local Variables:
%%% mode: latex
%%% TeX-master: "main"
%%% End:

\subsection{Similarity transformations}
\begin{Definition}{similar-matrix}
  Two matrices $\mata,\matb\in\C^{n\times n}$ are called \define{similar}, if there is a nonsingular matrix $\matv\in\C^{n\times n}$, such that
  \begin{gather}
      \mata = \matv \matb \matv^{-1}.
  \end{gather}
  We call the mapping
  \begin{gather}
      \matb \mapsto \matv \matb \matv^{-1}
  \end{gather}
  a \define{similarity transformation}.
\end{Definition}

\begin{Lemma}{similarity-equivalence}
  Similarity is an equivalence relation, that is, it is reflexive,
  symmetric, and transitive.
\end{Lemma}

\begin{Lemma}{matrix-basis-change}
  Let $\phi\colon \C^n\to \C^n$ be a linear mapping represented with
  respect to the canonical basis by the matrix $\mata$, such that
  \begin{gather}
    \phi(\vx) = \mata \vx.
  \end{gather}
  Then, the matrix $\matb = \matv^{-1}\mata\matv$ represents $\phi$
  with respect to the basis $\matv$, namely, if $\vu = \matv \vy$ and
  $\phi(\vu) = \matv \vz$, then
  \begin{gather}
    \vz = \matb \vy.
  \end{gather}
\end{Lemma}

\begin{Lemma}{similarity-eigenvalues}
  The eigenvalues of a matrix are invariant under similarity transformations.
\end{Lemma}

\begin{proof}
  We use the definition of eigenvalues as roots of the characteristic
  polynomial and show that the characteristic polynomial does not
  change under similarity transformation. Let
  $\matb = \matv^{-1}\mata\matv$. Note that
  \begin{gather}
    \matb-\lambda\id = \matv^{-1}\mata\matv - \lambda \matv^{-1}\matv
    = \matv^{-1}(\mata-\lambda\id)\matv.
  \end{gather}
  Hence,
  \begin{gather}
    \chi_{\matb}(\lambda) = \det(\matv^{-1}{(\mata-\lambda\id)}\matv)
    = \det(\matv^{-1})\det(\mata-\lambda\id)\det(\matv)
    = \chi_{\mata}(\lambda).
  \end{gather}
\end{proof}

\begin{Lemma}{invariant-blockmatrix}
  Let $\mata\in\Cnn$ have invariant subspaces $V_i$ of dimension
  $n_i$, $i=1,\dots,k$, such that $\C^n= \bigoplus V_i$. Then, there
  is a basis $\matv$ such that
  \begin{gather}
    \matb = \matv^{-1}\mata\matv =
    \begin{pmatrix}
      \matb_1\\&\ddots\\&&\matb_k,
    \end{pmatrix}
  \end{gather}
  where each block $\matb_i\in\C^{n_i\times n_i}$.
\end{Lemma}

\begin{proof}
  For each $i=1,\dots,k$ choose a basis $\vv_{i,1},\dots,\vv_{i,n_i}$
  of $V_i$. Concatenate these bases to obtain
  \begin{gather}
    \matv = \bigl(\vv_{1,1},\dots,\vv_{1,n_1},\vv_{2,1},\dots,\vv_{2,n_2},\dots,\vv_{k,1},\dots,\vv_{k,n_k}\bigr).
  \end{gather}
  Let now $\vu\in V_i$, then $\vu = \matv\vx$, where the coefficient
  vector has nonzero entries only betwen the positions of $\vv_{i,1}$
  and $\vv_{i,n_i}$. Since $\matb V_i\subset V_i$, the coefficient
  vector of $\matb\vu = \matv\vy$ is also nonzero only in this
  range. In particular, all entries of rows in this index range are
  zero for columns not in this index range.
\begin{todo}
    Improve this text!
\end{todo}
\end{proof}

\begin{Theorem*}{Jordan-canonical-form}{Jordan canonical form}
  Every matrix $A\in\C^{n\times n}$ is similar to a matrix with block structure
  \begin{gather}
    \begin{pmatrix}
      \matj_1\\&\matj_2\\&&\ddots\\&&&\matj_k
    \end{pmatrix},
  \end{gather}
  where each block has the form
  \begin{gather}
    \matj_i = \begin{pmatrix}
      \lambda_i&1\\&\ddots&\ddots\\
      &&\lambda_i&1\\
      &&&\lambda_i
    \end{pmatrix}.
  \end{gather}
\end{Theorem*}


\begin{Definition}{diagonalizable}
  A matrix is called \define{diagonalizable}, if it is similar to a diagonal matrix, that is,
  \begin{gather}
    \matlambda = \matv^{-1} \mata \matv
  \end{gather}
  with a diagonal matrix $\matlambda$ of eigenvalues.
\end{Definition}

\begin{remark}
  Let us apply \slideref{Lemma}{matrix-basis-change} to this
  definition such that $\matlambda$ is the matrix representing $\mata$
  in the basis $\matv$.

  By our method of writing bases as matrices, the transformation
  matrix $\matv$ corresponds to a basis $\vv_i = \matv\ve_i$.

  Since $\matlambda$ is diagonal, we have
  \begin{gather}
    \matlambda\ve_i = \lambda_e\ve_i.
  \end{gather}
  \slideref{Lemma}{matrix-basis-change} now tells us that
  \begin{gather}
    \mata\vv_i = \lambda_i\matv\ve_i=\lambda_i\vv_i.    
  \end{gather}
  Hence, the columns of $\matv$ are indeed eigenvectors of $\mata$.
\end{remark}

\begin{Theorem}{matrix-functions}
  Let $\mata\in \C^{n\times n}$ be diagonalizable such that
  \begin{gather}
    \mata = \matv \matlambda \matv^{-1},
    \qquad \matlambda = \diag(\lambda_1,\dots,\lambda_n).
  \end{gather}
  Then, for any analytic function $f\colon \C \to \C$, the matrix
  $f(\mata)$ is well defined by
  \begin{gather}
    f(\mata) = \matv f(\matlambda) \matv^{-1},
    \qquad f(\matlambda) = \diag\bigl(f(\lambda_1),\dots,f(\lambda_n)\bigr),
  \end{gather}
  if all eigenvalues are in the domain of convergence of the Tayor
  series of $f$.
\end{Theorem}

\begin{proof}
  First, we observe that
  \begin{gather}
    \mata^2 = \matv \matlambda \matv^{-1} \matv \matlambda \matv^{-1}
    = \matv \matlambda^2 \matv^{-1}.
  \end{gather}
  The square of the diagonal matrix $\matlambda$ is easily computed.
  By induction $\mata^k =  \matv \matlambda^k \matv^{-1}$.

  Let now $f(x) = \sum a_k x^k$ be the Taylor series of $f$. Then,
  \begin{gather}
    \begin{split}
      f(\mata)
      &= \sum_{k=0}^\infty a_k  \matv \matlambda^k \matv^{-1}\\
      &= \matv \left(\sum_{k=0}^\infty a_k \matlambda^k\right) \matv^{-1}\\
      &= \matv \diag\left(
        \sum_{k=0}^\infty a_k \lambda_1^k,\dots,
        \sum_{k=0}^\infty a_k\lambda_n^k
      \right)  \matv^{-1}.
    \end{split}
  \end{gather}
  All limits are well defined since we assume that all eigenvalues are
  in the domain of convergence.
\end{proof}

\begin{Theorem}{simultaneous-diagonalization}
  Two diagonalizable matrices $\mata, \matb\in \C^{n\times n}$ can be
  diagonalized by the same set of eigenvectors if and only if they
  commute. namely $\mata\matb = \matb\mata$.
\end{Theorem}

\begin{proof}
  First, we show that commutativity implies joint diagonalizability. To this end, 
  let $\vv$ be an eigenvector of $\mata$ such that $\mata\vv = \lambda\vv$. Then,
  \begin{gather}
    \label{eq:similarity:1}
    \mata\matb\vv = \matb\mata\matv = \lambda\matb\matv.
  \end{gather}
  Hence, $\matb\vv$ is also an eigenvector of $\mata$ for eigenvalue
  $\lambda$. Now, we distinguish two cases: if $\lambda$ is a simple
  eigenvalue of $\mata$, we immediately see that $\matb\vv$ must be a
  multiple of $\vv$ and hence it is an eigenvector of $\matb$ to some
  eigenvalue.

  If $\lambda$ is a multiple eigenvalue of $\mata$, we investigate its
  eigenspace $V_\lambda$. From~\eqref{eq:similarity:1}, we obtain that
  this space is an invariant subspace of $\matb$. Hence, there is a
  basis $\matv$ of eigenvectors of $\mata$ with those in $V_\lambda$
  first, such that
  \begin{gather}
    \matv^{-1}\matb\matv =
    \begin{pmatrix}
      \matb_1&\\&\matb_2
    \end{pmatrix}
  \end{gather}
  Since $\matb$ is diagonalizable, the basis for $V_\lambda$ can be
  chosen such that $\matb_1$ is diagonal. Doing this for any
  eigenspace of $\mata$ yields the joint diagonalizability.

  Assume now that there is a basis $\matv$ such that
  $\matlambda_{\mata} = \matv^{-1}\mata\matv$ and
  $\matlambda_{\matb} = \matv^{-1}\matb\matv$. Then,
  \begin{gather}
    \mata\matb = \matv\matlambda_{\mata}\matv^{-1}\matv\matlambda_{\matb}\matv^{-1}
    = \matv\matlambda_{\mata}\matlambda_{\matb}\matv^{-1}
    = \matv\matlambda_{\matb}\matlambda_{\mata}\matv^{-1}
    = \matb\mata.
  \end{gather}
\end{proof}

%%% Local Variables:
%%% mode: latex
%%% TeX-master: "main"
%%% End:

\section{Inner products and orthogonality}
\begin{Definition}{sesqui}
  A \define{sesquilinear form} on a complex vector space $V$ is a mapping
  \begin{gather}
      a\colon V\times V \to \C,
  \end{gather}
  which is \define{linear} and \define{semi-linear} in the first and second argument, respectively. That is, for all $\vu,\vv,\vw\in V$ and $\alpha\in\C$ holds
  \begin{xalignat}2
  a(\vu+\vv,\vw) &= a(\vu,\vw)+a(\vv,\vw),
  & a(\alpha \vu,\vw) &= \alpha a(\vu,\vw),\\
  a(\vu,\vv+\vw) &= a(\vu,\vv)+a(\vu,\vw),
  & a(\vu,\alpha \vw) &= \overline\alpha a(\vu,\vw).
  \end{xalignat}
  It is \define{Hermitian} or \define{complex symmetric}, if in addition there holds
  \begin{gather}
      \label{eq:inner:symmetry}
      a(\vu,\vv) = \overline{a(\vv,\vu)}.
  \end{gather}
\end{Definition}

\begin{remark}
  A sesquilinear form can be defined equivalently as being semi-linear in the second argument and linear in the first. Both versions can be found in the literature and we follow the convention in the book of Saad.
  
  The author prefers the term ``complex symmetric'' over the common and pompous ``Hermitian'', since it is the natural extension of symmetry to complex vector spaces, as the following asserts. The author would even prefer the term ``symmetric'', which is unfortunately understood in the real way in other publications. In particular,~\eqref{eq:inner:symmetry} implies that $a(\vu,\vu)$ is always real, such that the following makes sense:
\end{remark}

\begin{Definition}{inner-product}
  An \define{inner product} on the complex vector space $V$ is a sesqui-linear form $\scal(\cdot,\cdot)$ on $V$ which in addition is definite, namely, for all $\vv\in V$ there holds
  \begin{gather}
      \scal(\vv,\vv) \ge 0,
  \end{gather}
  and $\scal(\vv,\vv)=0$ implies $\vv=0$.
\end{Definition}

\begin{todo}
  Euclidean inner product
\end{todo}

\begin{todo}
  Define conjugate $\mata^*$
\end{todo}

\begin{todo}
  Define orthonormal set, unitary matrix
\end{todo}


%%% Local Variables:
%%% mode: latex
%%% TeX-master: "main"
%%% End:

\section{Projections}
\begin{Definition}{projection}
  A matrix $\matp\in\Cnn$ is called a \define{projection matrix} if
  \begin{gather}
    \matp^2 = \matp.
  \end{gather}

  It is called \define{orthogonal projection matrix}\index{orthogonal}
  if in addition
  \begin{gather}
    \matp^* = \matp.
  \end{gather}

  The associated operators are called (orthogonal) projection or
  (orthogonal) \define{projector}. If a projector is not orthogonal,
  it is also called \define{oblique}.
  \index{projector!orthogonal}
  \index{projector!oblique}
\end{Definition}

\begin{Lemma}{projection-range}
  Let $\matp\in\Cnn$ be an orthogonal projection matrix and
  $V = \range \matp$. Then, for any $\vx\in\C^n$ there holds
  \begin{gather}
    (\id-\matp)\vx \in V^\perp.
  \end{gather}
  
  Furthermore, the orthogonal projection $\matp$ is uniquely
  determined by its range.
\end{Lemma}

\begin{proof}
  Let $\vx,\vy\in \C^n$ arbitrary. Then,
  \begin{gather}
    \scal(\matp\vy,{(\id-\matp)}\vx) = \scal(\vy,\matp{(\id-\matp)}\vx)
    = \scal(\vv,{(\matp-\matp^2)}\vx) = \scal(\vv,{(\matp-\matp)}\vx) = 0.
  \end{gather}
  Since $V=\range \matp$, this implies that $(\id-\matp)\vx \in V^\perp$.

  Let now $\matp_1,\matp_2$ be orthogonal projectors. Then, for any
  $\vx\in\C^n$ there holds
  \begin{align}
    \norm{(\matp_1-\matp_2)\vx}_2^2
    &= \scal({(\matp_1-\matp_2)}\vx,{(\matp_1-\matp_2)}\vx)\\
    &= \scal(\matp_1\vx,{(\matp_1-\matp_2)}\vx) - \scal(\matp_2\vx,{(\matp_1-\matp_2)}\vx)\\
    &= \scal(\matp_1\vx,{(\id-\matp_2)}\vx)
      + \scal(\matp_2\vx,{(\id-\matp_1)}\vx).
  \end{align}
  if now $\range{\matp_1}=\range{\matp_2}$, then both inner products
  vanish. Thus, the operator norm of $\matp_1-\matp_2$ is zero and the
  projectors must be equal.
\end{proof}

\begin{Definition}{projection-distance}
  Let $U,V\subset \C^n$ be two subspaces with orthogonal projectors
  $\matp_U$ and $\matp_V$, respectively. Then, we define their
  \define{distance} as\index{dist}
  \begin{gather}
    \dist(U,V) = \norm{\matp_U-\matp_V}_2.
  \end{gather}
\end{Definition}

\begin{todo}
  \begin{enumerate}
  \item Lemma: the distance is indeed a metric
  \item Remove example below
  \end{enumerate}
\end{todo}

\begin{Lemma}{projection-distance}
  For two subspaces $U,V\subset \C^n$ of unequal dimension there holds
  $\dist(U,V)\ge 1$. The distance of subspaces of equal dimension is
  \begin{align}
    \dist(U,V) &= \max_{\vu\in U}\min_{\vv\in V} \sin\bigl(\angle(u,v)\bigr)\\
               &= \max_{\vv\in V}\min_{\vu\in U} \sin\bigl(\angle(u,v)\bigr).
  \end{align}
\end{Lemma}

\begin{proof}
  Let without loss of generality $\dim U >\dim V$. Then, $U$ contains
  a vector $\vu$ which is orthogonal to $V$. Hence,
  $(\matp_U-\matp_V)\vu=\vu$ and the norm must be greater or equal to
  one.

  Let now $\vu\in U$ with $\norm{\vu}_2=1$. Then
  \begin{gather}
    \norm{(\matp_U-\matp_V)\vu}_2
    = \norm{\vu-\matp_V\vu}_2
    = \norm{\vu-\scal(\vu,\vw)\vw}_2,
  \end{gather}
  for some normalized vector $\vw\in v$. Hence,
  \begin{gather}
    \norm{(\matp_U-\matp_V)\vu}_2^2
    = \norm{\vu}_2^2 - 2 \scal(\vu,\vw)^2 + \scal(\vu,\vw)^2\norm{w}_2^2
    = 1-\cos^2\alpha = \sin^2\alpha,
  \end{gather}
  where $\alpha$ is the angle between $\vu$ and $\vw$. By the
  minimization property of the orthogonal projection, $\alpha$
  minimizes this expression over $V$. For the norm
  $\norm{\matp_U-\matp_V}$, we must take the maximum over $\vu$.
\begin{todo}
  Do this in detail!
\end{todo}
  For the symmetry of the expression, we note that the pair
  $\vu\in U$, $\vw\in V$ already shows that the maximum over $V$
  cannot be smaller than the maximum over $U$. But if it were greater,
  say for a pair $\vv\in V$ and $\vy\in U$, this would imply that the
  maximum over $U$ was not attained at $\vu$.
\end{proof}

\begin{Example}{projection-distance}
  Since the orthogonal projection is uniquely defined by its range, there holds
  \begin{gather}
    \dist(U,U) = \norm{\matp_U-\matp_U}_2 = 0.
  \end{gather}
\end{Example}

% \begin{Example}{projection-distance}
%   Let $U = \spann \vu$ and $V=\spann\vv$ with
%   $\norm{\vu} = \norm{\vv} = 1$. Then,
%   \begin{align}
%     \dist(U,V)^2
%     &= \sup_{\vx\in\C^n}\frac{\norm{\matp_U\vx-\matp_V\vx}^2}{\norm{\vx}^2}\\
%     &= \sup_{\vx\in\C^n}\frac{\norm{\scal(\vx,\vu)\vu - \scal(\vx,\vv)\vv}^2}{\norm{\vx}^2}\\
%     &= \sup_{\vx\in\C^n}\frac{\scal(\vx,\vu)^2+\scal(\vx,\vv)^2 - 2\scal(\vx,\vu)\scal(\vx,\vv)\scal(\vu,\vv)}{\norm{\vx}^2}\\
%   \end{align}
% \end{Example}


\begin{Lemma}{projection-complement}
  If $\matp\in\Cnn$ is a projection operator, so is
  $\id-\matp$, and there holds
  \begin{gather}
    \ker{\matp} = \range{\id-\matp}.
  \end{gather}
  Furthermore, there holds
  \begin{gather}
    \C^n = \ker{\matp} \oplus \range{\matp}.
  \end{gather}
\end{Lemma}

\begin{proof}
  \cite[Section 1.12.1]{Saad00}.
\end{proof}

\begin{Lemma}{projection-spaces}
  Any pair of subspaces $N,R\subset \C^n$ such that $\C^n = N\oplus R$
  uniquely defines a projector $\matp$, such that $N = \ker\matp$ and
  $R=\range\matp$. There holds for $\vx\in\C^n$
  \begin{align}
    \matp \vx &\in R\\
    \vx - \matp \vx &\in N.
  \end{align}
  We say $\matp$ projects onto $R$ along $N$.
\end{Lemma}

\begin{proof}
  \cite[Section 1.12.1]{Saad00}.
\end{proof}

\begin{Definition}{projection-spaces-orthogonal}
  Given two subspaces $R,S\subset\C^n$ of equal dimension, a
  projector $\matp$ is said to project onto $R$ orthogonal to $S$, if
  for any $\vx\in\C^n$ there holds
  \begin{align}
    \matp\vx &\in R\\
    \vx-\matp\vx &\perp S.
  \end{align}
\end{Definition}

\begin{Lemma}{projection-spaces-orthogonal}
  Let $R,S\subset\C^n$ be two subspaces of equal dimension. Then, the
  projection of any vector $\vx\in\C^n$ onto $R$ orthogonal to $S$ is
  uniquely defined if and only if no vector in $R$ is orthogonal to
  $S$.
\end{Lemma}

\begin{proof}
  \cite[Lemma 1.36]{Saad00}.
\end{proof}

\begin{Definition}{biorthogonal}
  Two sets of vectors $\{\vv_1,\dots,\vv_m\}$ and  $\{\vw_1,\dots,\vw_m\}$ are called \define{biorthogonal}, if
  \begin{gather}
    \scal(\vv_i,\vw_j) = \delta_{ij}, \qquad i,j=1,\dots,m.
  \end{gather}
  In matrix notation
  \begin{gather}
    \matw^*\matv = \matv^*\matw = \id
  \end{gather}
\end{Definition}

\begin{Lemma}{projection-basis}
  Let $R,S\subset\C^n$ be two subspaces of equal dimension and let
  $\matp$ be the projector onto $R$ orthogonal to $S$. Let
  $\{\vv_1,\dots,\vv_m\}$ and $\{\vw_1,\dots,\vw_m\}$ be biorthogonal
  and bases of $R$ and $S$, respectively. Then,
  \begin{gather}
    \matp = \matv\matw^*.
  \end{gather}
\end{Lemma}

\begin{proof}
  \cite[Section 1.12.2]{Saad00}.
\end{proof}


%%% Local Variables:
%%% mode: latex
%%% TeX-master: "main"
%%% End:


\chapter{Basics from Numerical Analysis}

\section{QR decomposition}
\begin{Definition}{qr-decomposition}
  The \define{QR decomposition} of a matrix $\mata\in\C^{m\times n}$
  with $m\ge n$ is the product
  \begin{gather}
    \mata = \matq\matr,
  \end{gather}
  such that $\matq \in\C^{m\times n}$ is unitary and
  $\matr\in \C^{n\times n}$ is upper triangular.
\end{Definition}

\begin{Lemma}{qr-columns}
  Let $\mata = \matq\matr$. Then, the column vectors of $\mata$ and
  $\matq$ admit the relation
  \begin{gather}
    \va_k = \sum_{i=1}^k r_{ik} \vq_i.
  \end{gather}
  If $r_{ii}\neq 0$ for $i=1,\dots,k$, this relation is uniquely
  invertible. In particular,
  \begin{gather}
    \operatorname{span}\{\vq_1,\dots,\vq_k\}
    = 
    \operatorname{span}\{\va_1,\dots,\va_k\}.
  \end{gather}
\end{Lemma}

\begin{Theorem}{qr-existence}
  Every matrix $\mata\in\C^{m\times n}$ with $m\ge n$ of full rank
  admits a QR decomposition. It is unique under the condition that for
  all $i$ there holds $r_{ii} > 0$.
\end{Theorem}

%%% Local Variables:
%%% mode: latex
%%% TeX-master: "main"
%%% End:


\section{Matrix norms and spectral radius}

\begin{Definition}{spectral-radius}
  The \define{spectral radius} of a matrix $\mata\in\Cnn$ is the
  maximal absolute value of its eigenvalues, that is
  \begin{gather}
    \rho(\mata) = \max_{\lambda\in\sigma(\mata)} \abs{\lambda}.
  \end{gather}
\end{Definition}

\begin{Lemma*}{spectral-radius}{Properties of the spectral radius}
  For any matrix norm $\norm\cdot$ and for any matrix $\mata\in\Cnn$ there holds
  \begin{gather}
    \rho(\mata) = \lim_{k\to\infty} \norm{\mata^k}^{1/k}.
  \end{gather}

  The sequence $\vx^{(k)} = A^k\vx$ converges to zero for all
  $\vx\in\C^n$ if and only if $\rho(\mata)<1$.

  For any $\epsilon>0$
  there is a matrix norm $\norm\cdot$ such that
  \begin{gather}
    \norm{\mata} \le (1+\epsilon) \rho(\mata) \qquad \forall \mata\in\Cnn.
  \end{gather}
\end{Lemma*}

\section{Chebyshev polynomials}

\begin{Definition}{chebyshev-polynomials}
  The \define{Chebyshev polynomials} $\pchebyshev_k$ are defined by the
  three-term recurrence relation
  \begin{gather}
    \pchebyshev_{k}(x) = 2x \pchebyshev_{k-1}(x) - \pchebyshev_{k-2}(x),
  \end{gather}
  with $T_0 \equiv 1$ and $T_1(x) = x$.
  They are orthogonal with respect to the inner product
  \begin{gather}
    \scal(p,q) = \int_{-1}^1 \tfrac1{\sqrt{1-x^2}} \,p(x)q(x)\dx.
  \end{gather}
\end{Definition}

\begin{Lemma}{chebyshev-representation}
  Chebyshev polynomials admit the trigonometric representation
  \begin{gather}
    \pchebyshev_k(x) = \cos(k \operatorname{arccos} x).
  \end{gather}
  
  Furthermore, there holds
    \begin{gather}
    \pchebyshev_k = \frac12
    \left(
      \left(x-\sqrt{x^2-1}\right)^k
      +
      \left(x+\sqrt{x^2-1}\right)^k
    \right),\qquad\abs{x}\ge 1.
  \end{gather}
\end{Lemma}

\begin{Lemma}{chebyshev-abscissae}
  The Chebyshev polynomial $\pchebyshev_n$ has $n$ roots in the
  interval $(-1,1)$ at
  \begin{gather}
    x_k = \cos\left(\pi\frac{(k-\nicefrac12)}{n}\right),
    \qquad k=1,\dots,n.
  \end{gather}
  It alternatingly assumes the values $\pm1$ at the \define{Chebyshev abscissae}
  \begin{gather}
    x_k = \cos\left(\pi\frac kn\right),
  \end{gather}
  wiyh $\pchebyshev_n(1) = 1$ and $\pchebyshev_n(-1) = (-1)^n$.
\end{Lemma}

\begin{Theorem}{chebyshev-minimal-1}
  For every polynomial $p\in \P_n$ with leading coefficient 1 there is $x\in[-1,1]$ such that $\abs{p(x)} \ge \nicefrac1{2^{n-1}}$. There holds
  \begin{gather}
   \frac1{2^{n-1}} \pchebyshev_n(x)
   = \operatorname*{arg min}_{\substack{p\in\P_n\\p = x^n+\cdots}}
   \max_{x\in[-1,1]}\abs{p(x)}.
  \end{gather}
\end{Theorem}

\begin{proof}
  See \cite[Satz 7.19]{DeuflhardHohmann08}.  The three-term recursion
  for Chebyshev polynomials implies that the leading coefficient of
  $\pchebyshev_n$ is $2^{n-1}$.

  Assume that there is another polynomial $p\in \P_n$ with leading
  coefficient $2^{n-1}$ such that
  \begin{gather}
    \max_{x\in[-1,1]} \abs{p(x)} < 1.
  \end{gather}
  Then, $q_n = \pchebyshev_n-p \in \P_{n-1}$. Furthermore, for the
  Chebyshev abscissae $\tilde x_j = \cos(\nicefrac{j\pi}{n})$ with
  $j=0,\dots,n$ there holds
  \begin{xalignat}4
    \pchebyshev_n(\tilde x_j) &= 1,
    & p(\tilde x_j) &< 1
    & q_n(\tilde x_j) &> 0,
    & j&\text{ gerade}\\
    \pchebyshev_n(\tilde x_j) &= -1,
    & p(\tilde x_j) &> -1
    & q_n(\tilde x_j) &< 0,
    & j&\text{ ungerade}.
  \end{xalignat}
  Therefore, 
  $q_n$ changes sign at at least $n$ points and thus has at least $n$ roots.
  From
  $q_n\in\P_{n-1}$ we deduce $q_n=0$ and thus $p=\pchebyshev_n$ as a contradiction.
  After scaling by $2^{n-1}$ we obtain
  \begin{gather}
    \operatorname*{min}_{\substack{p\in\P_n\\p = x^n+\cdots}}
   \max_{x\in[-1,1]}\abs{p(x)} \ge 1,
 \end{gather}
 with equality for the scaled Chebyshev polynomial.
\end{proof}

\begin{Lemma}{chebyshev-growth}
  Let
  \begin{gather}
    K = \bigl\{ p\in \P_n \;\big|\; \max_{x\in[-1,1]} \abs{p(x)} =1 \bigr\}.
  \end{gather}
  Then,
  \begin{gather}
    \abs{\pchebyshev_n(x)} \ge p(x) \qquad \forall p\in K, \quad \forall \abs{x} > 1.
  \end{gather}
\end{Lemma}

\begin{proof}
  We conduct the proof for $x>1$ where $\pchebyshev_n(x) > 0$.
  Let $\tilde p\in K$ such that
  $\abs{\tilde p(y)} \ge \pchebyshev_n(y)$ for some $y>1$. Let
  $\gamma = \pchebyshev_n(y)/\tilde p(y)$ and
  $p(x) = \tilde p(x)\gamma$, such that
  $q(x) = p(x) - \pchebyshev_n(x) \in \P_n$ has a root in $y$.
  Furthermore,
  \begin{gather}
    \max_{x\in[-1,1]} \abs{p(x)} =\gamma < 1
  \end{gather}

  Thus, $q(x)$ has alternating sign in the $n+1$ Chebyshev abscissae and due to continuity $n$ roots in
  $(-1,1)$. Hence, it has $n+1$ roots and therefore
  $q \equiv 0$. We conclude $p \equiv \pchebyshev_n$ and since
  $\norm{p}_{\infty;[0,1]} = 1$ there holds $\tilde p = p$.
\end{proof}

\begin{Corollary}{chebyshev-minimal-2}
  Let $[a,b]$ be an interval with $0 < a$. Then, the polynomial
  \begin{gather}
    \widehat \pchebyshev_n(x)
    = \frac{\pchebyshev_n\left(\frac{a+b-2x}{b-a}\right)}%
    {\pchebyshev_n\left(\frac{a+b}{b-a}\right)}
  \end{gather}
  solves the minimization problem
  \begin{gather}
    \widehat \pchebyshev_n(x)
    = \operatorname*{argmin}_{\substack{p\in\P_n\\p(0) = 1}}
    \max_{x\in[a,b]}{\abs{p(x)}}.
  \end{gather}
  There holds
  \begin{align}
    \label{eq:chebyshev-cg1}
    \max_{x\in[a,b]}{\abs{\widehat \pchebyshev_n(x)}}
    &= \left(\pchebyshev_n\left(\frac{a+b}{b-a}\right)\right)^{-1}\\
    & \le 2 \left(\frac{\sqrt b-\sqrt a}{\sqrt b + \sqrt a}\right)^n.
  \end{align}
\end{Corollary}

\begin{proof}
  The Chebysshev polynomial $\pchebyshev_n$ is the one with minimal maximum on $[-1,1]$ and maximal growth outside this interval. Thus, if we transform it to the interval $[a,b]$ by mapping
  \begin{gather}
    x \mapsto \frac{a+b-2x}{b-a},
  \end{gather}
  it is the polynomial with maximal absolute value 1 inside $[a,b]$
  with maximal value at 0. Dividing by this value, it solves the
  stated minimization problem and \eqref{eq:chebyshev-cg1} holds.

  To further estimate this value note that
  \begin{align}
    \pchebyshev_n(x)
    &= \frac12
      \left(x-\sqrt{x^2-1}\right)^n
      +
      \left(x+\sqrt{x^2-1}\right)^n\\
    &\ge \left(x+\sqrt{x^2-1}\right)^n.
  \end{align}
  Entering $x = \frac{a+b}{b-a}$ yields
  \begin{align}
    \frac{a+b+\sqrt{(a+b)^2-(b-a)^2}}{b-a}
    &= \frac{(\sqrt a + \sqrt b)^2}{b-a}\\
    &= \frac{\sqrt b + \sqrt a}{\sqrt b-\sqrt a}.
  \end{align}
\end{proof}

\bibliographystyle{alpha}
\bibliography{all}
\printindex

%%% Local Variables:
%%% mode: latex
%%% TeX-master: "main"
%%% End:
