\begin{Definition}{erk}
  \defindex{Runge-Kutta method!explicit (ERK)}
  \index{ERK|see{Runge-Kutta method}} 
  An \textbf{explicit Runge-Kutta method (ERK)} is a
  one-step method with the representation
  \begin{subequations}
    \label{eq:explicit:1}
    \begin{xalignat}{2}
      \label{eq:explicit:1a}
      \rkg_i &= y_0 + h \sum_{j=1}^{i-1} \rka_{ij} k_j
      & i &= 1,\dots,\rks
      \\
      \label{eq:explicit:1b}
      k_i &= f\left(h \rkc_i, \rkg_i\right)
      & i &= 1,\dots,\rks
      \\
      \label{eq:explicit:1c}
      y_1 &= y_0 + h \sum_{i=1}^{\rks} \rkb_i k_i
    \end{xalignat}
  \end{subequations}
  In this method the values $h\rkc_i$ are the quadrature points
  on the interval $[0,h]$. The values $k_i$ are approximations to
  function values of the integrand in these points and the values $\rkg_i$ constitute
  approximations to the solution $u(h\rkc_i)$ in the quadrature
  points. This method uses $s$ intermediate values and is thus called
  an $\rks$-stage method.
\end{Definition}

%%% Local Variables:
%%% mode: latex
%%% TeX-master: "../notes"
%%% End:
