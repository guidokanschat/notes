\begin{Definition}{local-uniqueness}
  \defindex{solution!locally unique (BVP)} A solution $u(t)$ of the
  BVP~\eqref{eq:rwa:1} is called \textbf{locally unique}
  \defindex{locally unique solution (BVP)} or
  \textbf{isolated}\defindex{isolated solution}, if there is no second
  solution $v(t)$ of the BVP, which is arbitrary close to $u(t)$. In
  mathematical language: there exists an $\epsilon>0$, such that for
  any two solutions of the BVP there holds
  \begin{gather*}
    \max_{t\in [a,b]}|u(t)-v(t)| < \epsilon
    \qquad\Rightarrow\qquad
    u(t) = v(t) \quad \forall t\in [a,b].
  \end{gather*}
\end{Definition}
