\begin{Definition}{newton}
  The \define{Newton method} for finding the root of the nonlinear
  equation $f(x) = 0$ reads: given an initial value $x^{(0)}$, compute
  iterates $x^{(k)}$, $k=1,2,\ldots$ by the rule
  \begin{gather}
    \label{eq:newton-def:1}
    \begin{split}
      J &= \nabla f\left(x^{(k)}\right),
      \\
       J d^{(k)} &= f(x^{(k)}),
      \\
      x^{(k+1)} &= x^{(k)} - d^{(k)}.
    \end{split}
  \end{gather}
  We denote by the term \define{quasi-Newton method} any modification
  of this scheme employing an approximation $\widetilde J$ of the
  Jacobian $J$.
\end{Definition}

%%% Local Variables:
%%% mode: latex
%%% TeX-master: "../notes"
%%% End:
