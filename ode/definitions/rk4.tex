\begin{Example*}{rk4}{The classical Runge-Kutta method of 4th order}
  \defindex{Runge-Kutta method!four-stage}
  \begin{gather*}
    \begin{aligned}
      k_1 & = f(t_n, y_n) \\
      k_2 & = f(t_n + \frac12 h_n, y_n + \frac12 h_n k_1) \\
      k_3 & = f(t_n + \frac12 h_n, y_n + \frac12 h_n k_2) \\
      k_4 & = f(t_n + h_n, y_n + h_n k_3) \\
      y_{n+1} & = y_n + h_n ( \frac16 k_1 + \frac26 k_2 + \frac26 k_3 + \frac16 k_4 ) \\
    \end{aligned}
    \qquad
    \begin{array}{c|cccc}
      0 & \\
      \frac12 & \frac12 \\
      \frac12 & 0 & \frac12 \\
      1 & 0 & 0 & 1 \\
      \hline
        & \frac16 & \frac26 & \frac26 & \frac16 \\
    \end{array}
  \end{gather*}
  This formula is based on the Simpson rule as well, but it uses two
  approximations for the value in the center point. It is of fourth
  order.
\end{Example*}

%%% Local Variables:
%%% mode: latex
%%% TeX-master: "../notes"
%%% End:
