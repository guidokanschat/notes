\subsection{Polynomräume}

\begin{Satz}{nullstellen}
  Ein reelles Polynom vom Grad $n$ hat höchstens $n$ Nullstellen oder es ist das Nullpolynom.
\end{Satz}

\begin{proof}
  Für $n=1$ handelt es sich um ein lineares Polynom und die Aussage
  des Satzes ist unmittelbar klar. Sei nun $p$ ein Polynom strikt vom
  Grad $n>1$ mit Nullstelle $x_0$. Dann gibt es nach dem euklidischen
  Algorithmus zur Division mit Rest ein Polynom $q$ vom Grad $n-1$ und
  eine Konstante $c$, so dass
  \begin{gather}
    p(x) = (x-x_0)q(x)+c.
  \end{gather}
  Daraus folgt $p(x_0) = c$, so dass folgt $c=0$. Wir können dieses
  Verfahren für alle weiteren Nullstellen $x_1,\dots,x_m$ wiederholen
  und erhalten
  \begin{gather}
    p(x) =  r(x) \prod_{k=0}^m (x-x_i),
  \end{gather}
  wobei $r(x)$ ein Polynom vom Grad $n-m$ sein muss, da $p$ vom Grad
  $n$ ist. Insbesondere muss gelten $m\le n$.
\end{proof}

\begin{Korollar}{monome-linear-unabhaengig}
  Zwei reelle Polynome vom Grad $n$ sind identisch, wenn sie in
  mindestens $n+1$ Punkten übereinstimmen. Insbesondere ist jede Menge
  von Monomen $\{x^{k_0}, x^{k_1},\dots,x^{k_n}\}$ für paarweise
  verschiedene Exponenten $k_i$ linear unabhängig.
\end{Korollar}

\begin{Satz}{polynomraum}
  Die Polynome vom maximalen Grad $n$ bilden einen Vektorraum der
  Dimension $n+1$.  Wir bezeichnen ihn mit $\P_n$.
\end{Satz}

\subsection{Skalarprodukt und Orthogonalität}
\begin{Definition}{skalarprodukt}
  Sei $V$ ein reeller Vektorraum. Eine Abbildung
  $a\colon V \times V \to \R$ heißt \define{Bilinearform}, wenn für
  $u,v,w\in V$ und $\alpha,\beta\in \R$ gilt
  \begin{align}
    a(\alpha u + \beta v,w) &= \alpha a(u,w) + \beta a(v,w)\\
    a(w,\alpha u + \beta v) &= \alpha a(w,u) + \beta a(w,v).
  \end{align}
  Eine Bilinearform heißt \define{symmetrisch}, wenn für $u,v\in V$ gilt
  \begin{gather}
    a(u,v) = a(v,u).
  \end{gather}
  Sie heißt \define{positiv semi-definit}, wenn $a(u,u) \ge 0$ für alle
  $u\in V$ und \define{positiv definit}, wenn zusätzlich
  \begin{gather}
    a(u,u) = 0 \quad \Longrightarrow \quad u=0.
  \end{gather}
  Eine symmetrische, positiv definite Bilinearform heißt
  \define{Skalarprodukt}, in der Regel notiert als $\scal(\cdot,\cdot)$.
\end{Definition}

\begin{Lemma}{hilbertnorm}
  Sei $V$ ein reeller Vectorraum mit Skalarprodukt
  $\scal(\cdot,\cdot)$. Dann ist durch
  \begin{gather}
    \norm{u} = \sqrt{\scal(u,u)}
  \end{gather}
  auf $V$ eine Norm definiert. Ein reeller Vektorraum $V$ mit
  Skalarprodukt und zugehöriger Norm heißt \define{euklidischer
    Vektorraum}.
\end{Lemma}

\begin{Beispiel*}{l2-norm}{$L^2$-Skalarprodukt}
  Auf dem Raum $V=\P_n$ der reellen Polynome vom Grad bis zu $n$ ist durch
  \begin{gather}
    \scal(p,q) = \int_{-1}^1 p(x)q(x)\dx
  \end{gather}
  ein Skalarprodukt definiert.
\end{Beispiel*}

\begin{Definition}{orthogonal}
  Zwei Vektoren $u,v\in V$ heißen \define{orthogonal}, wenn
  \begin{gather}
    \scal(u,v) = 0.
  \end{gather}
  Ein Vektor $u\in V$ ist orthogonal zum Untervektorraum $W\subset V$, wenn
  \begin{gather}
    \scal(u,v) = 0\quad\forall v\in W.
  \end{gather}
\end{Definition}

\begin{Notation}{euklidischer-vr}
  Von nun an bezeichnet $V$ immer einen endlichdimensionalen, reellen,
  euklidischen Vektorraum.
\end{Notation}

\begin{Definition}{bestapproximation}
  Sei $A\subset V$ ein affiner Unterraum eines euklidischen
  Vektorraums. Dann ist die Bestapproximation $u_b\in A$ eines Vektors
  $u\in V$ in $A$ definiert durch die Beziehung
  \begin{gather}
    \norm{u-u_b} = \min_{v\in A} \norm{u-v}.
  \end{gather}
\end{Definition}

\begin{Satz}{bestapproximation}
  Sei $A=w+W$ ein nichtleerer, affiner Unterraum von $V$.  Dann
  existiert die Bestapproximation nach
  \slideref{Definition}{bestapproximation} und ist eindeutig
  bestimmt. Es gilt die notwendige und hinreichende Bedingung
  \begin{gather}
    \scal(u-u_b,v) = 0 \quad \forall v\in W.
  \end{gather}
\end{Satz}

\begin{proof}
  Siehe \cite[Satz 2.14]{Rannacher17} oder \cite[Satz 3.4]{DeuflhardHohmann08}.
\end{proof}

\begin{Definition}{komplement-projektion}
  Sei $W \subset V$ ein Untervektorraum. Dann gilt
  $V = W \oplus W^\perp$, wobei das \define{orthogonale Komplement}
  $W^\perp$ eindeutig definiert ist durch
  \begin{gather}
    W^\perp = \bigl\{ v\in V \big| \scal(v,w) = 0 \quad\forall w\in W\bigr\}.
  \end{gather}
  Die Lösung $\Pi_W u = u_b\in W$ der Bestaproximationsaufgabe nennen
  wir die \define{orthogonale Projektion} von $u\in V$ auf $W$.
\end{Definition}

\begin{lemma}
  Das orthogonale Komplement und die orthogonale Projektion sind wohldefiniert.
\end{lemma}

\begin{proof}
  \slideref{Satz}{bestapproximation}.
\end{proof}

\begin{Beispiel}{polynom-bestapproximation}
  Die Aufgabe der Gaußschen Ausgleichsrechnung lautet: finde zu einer
  gegebenen Funktion $f$ das Polynom vom Grad höchstens $n$, das auf
  dem Intervall $[-1,1]$ den mittleren quadratischen Abstand
  minimiert, also $p\in \P_n$ mit
  \begin{gather}
    \int_{-1}^1 \bigl(f(x)-p(x)\bigr)^2 \dx
    = \min_{q\in \P_n} \int_{-1}^1 \bigl(f(x)-q(x)\bigr)^2 \dx.
  \end{gather}
  Die Lösung erfüllt
  \begin{gather}
    \int_{-1}^1 p(x)q(x) \dx = \int_{-1}^1 f(x)q(x) \dx
    \qquad\forall q\in \P_n.
  \end{gather}
\end{Beispiel}


\subsection{Orthogonale Basen}



%%% Local Variables:
%%% mode: latex
%%% TeX-master: "main"
%%% End:
