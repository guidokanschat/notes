\documentclass[german,ignorenonframetext,notheorems,aspectratio=1610]{beamer}
\usetheme[compress]{Madrid}
\usecolortheme{iwr}
\usepackage{mathsim}
\lstset{language=Python}
\usetikzlibrary{snakes}
\tikzset{shape veloxy/.style={color=black,draw,fill=red,thick}}
\tikzset{shape pressure/.style={color=black,draw,fill=cyan,thick}}

\def\restrict{r}
\def\prolongate{p}

\usepackage{times}
\usepackage{xr}
\externaldocument{main}
\usepackage{mfirstuc}
\usepackage{mathtools}  
\mathtoolsset{showonlyrefs}

\def\footnote#1{}
\def\putindex#1{#1}
\title{Einführung in die Numerik}
\author{Guido Kanschat}
\date{\today}
\begin{document}
\frame{\maketitle}
\frame{\frametitle{Inhalt}\tableofcontents[hideallsubsections]}
\section{Orthogonale Projektionen}
\frame{\sectoc}

\subsection{Polynomräume}
\frame {\input {blocks/Satz-nullstellen.tex}
  \input {blocks/Korollar-monome-linear-unabhaengig.tex}
  \input {blocks/Satz-polynomraum.tex}}

\subsection{Skalarprodukt und Orthogonalität}
\frame{\subtoc}
\frame {\input {blocks/Definition-skalarprodukt.tex}}
\frame {\input {blocks/Lemma-hilbertnorm.tex}}
\frame {\input {blocks/Lemma-l2-norm.tex}}
\frame {\input {blocks/Definition-orthogonal.tex}
  \input {blocks/Notation-euklidischer-vr.tex}}
\frame {\input {blocks/Lemma-bcs.tex}}
\frame {\input {blocks/Lemma-pythagoras.tex}}

\subsection{Bestapproximation und orthogonale Projektion}
\frame{\subtoc}
\frame {\input {blocks/Definition-bestapproximation.tex}
  \input {blocks/Satz-bestapproximation.tex}}
\frame {\input {blocks/Definition-komplement-projektion.tex}}
\frame {\input {blocks/Beispiel-polynom-bestapproximation.tex}}

\subsection{Orthogonale Basen}
\frame{\subtoc}
\frame {\input {blocks/Lemma-gram-system.tex}}
\frame {\input {blocks/Definition-ortho-system.tex}
  \input {blocks/Lemma-ortho-lu.tex}}
\frame {\input {blocks/Lemma-parseval.tex}}
\frame {\input {blocks/Lemma-gram-system.tex}
  \input {blocks/Lemma-least-squares-orthogonal.tex}}
\frame {\input {blocks/Theorem-gram-schmidt.tex}}
\frame {\input {blocks/Algorithmus-gram-schmidt.tex}}
\frame {\input {blocks/Beispiel-gram-schmidt.tex}}
\frame {\input {blocks/Algorithmus-mgs.tex}}
\frame {\input {blocks/Beispiel-gs-mgs.tex}}

\subsection{Dreiterm-Rekursion}
\frame{\subtoc}
\frame {\input {blocks/Satz-dreiterm.tex}}
\frame {\input {blocks/Bemerkung-dreiterm-normierung.tex}}
\frame {\input {blocks/Definition-legendre-polynome.tex}}
\frame {\input {blocks/Beispiel-least-squares-legendre.tex}}
\frame {\input {blocks/Definition-chebyshev-polynome.tex}}

\section{Bibliography}
\frame{\bibliographystyle{alpha}
\bibliography{all}}
\end{document}

%%% Local Variables:
%%% mode: latex
%%% TeX-master: t
%%% End:
