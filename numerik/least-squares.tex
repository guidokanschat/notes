\begin{intro}
  Die Methode der kleinsten Fehlerquadrate führt auf die Minimierungsaufgabe
  \begin{gather}
    \norm{Ax-b}_2 = \min.
  \end{gather}
\end{intro}

\begin{Satz}{normalengleichungen}
  Sei $A\in \R^{m\times n}$ mit $m\ge n$ und $b\in \R^m$. Dann ist
  $x\in\R^n$ genau dann eine Lösung des linearen Ausgleichsproblems
  \begin{gather}
    \norm{Ax-b}_2 = \min,
  \end{gather}
  wenn $x$ Lösung der \define{Normalengleichungen}
  \begin{gather}
    A^TA x = A^Tb
  \end{gather}
  ist. Insbesondere ist die Minimierungsaufgabe eindeutig lösbar, wenn
  $A$ vollen Rang hat.
\end{Satz}

\begin{remark}
  Wir können die Normalengleichungen lösen, indem wir die symmetrische Matrix $C = A^TA\in \R^{n\times n}$ berechnen und dann eines der Verfahren der vorigen Abschnitte auf diese Matrix anwenden.
  
  Das Lemma nach der nächsten Definition legt nahe, dass das keine gute Idee ist, da sich die Konditionszahl durch das Matrixprodukt quadriert und damit die Lösungsgenauigkeit leidet.  
\end{remark}

\begin{Definition}
  Die Konditionszahl einer rechteckigen Matrix maximalen Rangs bezüglich der Operatornorm zur Vektornorm $\norm{\cdot}$ ist
  \begin{gather}
    \operatorname{cond}(A) = \frac{\sup_{\norm{x}=1}\norm{Ax}}{\inf_{\norm{x}=1}\norm{Ax}}.
    \end{gather}
  Die Definition ist konsistent zur Definition für invertierbare, quadratische Matrizen.
\end{Definition}

%%% Local Variables:
%%% mode: latex
%%% TeX-master: "main"
%%% End:
