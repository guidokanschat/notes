\section{Polynominterpolation}

\begin{Definition}{lagrange-interpolation}
  Die \define{Interpolation}saufgabe nach Lagrange lautet: seien $n+1$
  paarweise verschiedene \define{Stützstellen} $x_0,\dots,x_n$ mit
  zugehörigen Funktionswerten $f_i$ gegeben. Finde ein Polynom
  $p\in \P_n$ mit der Eigenschaft
  \begin{gather}
    p(x_i) = f_i.
  \end{gather}
  Alternativ ist die Interpolationsaufgabe aufzufassen als eine Abbildung
  \begin{gather}
    \begin{split}
      I_n\colon C[a,b] &\to \P_n\\
      p(x_i) &= f(x_i),
    \end{split}
  \end{gather}
  wobei das Interval $[a,b]$ alle Stützpunkte enthält. 
  Wir nennen diese Abbildung den
  \define{Lagrange-Interpolationsoperator} oder kurz
  \define{Lagrange-Interpolation}.
\end{Definition}

\begin{Satz}{lagrange-interpolation}
  Die Interpolationsaufgabe nach Lagrange hat eine eindeutige Lösung,
  bezeichnet als (Lagrange-)\define{Interpolierende} der Funktion $f$
  \begin{gather}
    p(x;f;x_0,\dots,x_n)
  \end{gather}
\end{Satz}
\begin{proof}
  Der Beweis ist eine direkte Konsequenz des folgenden Lemmas.
\end{proof}

\begin{Lemma}{lagrange-basis}
  Seien die Punkte $x_0,\dots,x_n$ paarweise verschieden. Dann gilt
  für die \define{Lagrange-Polynome}
  \begin{gather}
    L_i(x) = L_{i;n}(x) = L_{i}(x;x_0,\dots,x_n)
    = \prod_{\substack{j=0\\j\neq i}}^n \frac{x-x_j}{x_i-x_j}
  \end{gather}
  die Eigenschaft
  \begin{gather}
    L_i(x_j) = \delta_{ij},\qquad 0 \le i,j \le n.
  \end{gather}
  Die Lagrange-Polynome sind \putindex{orthonormal} bezüglich des
  Skalarprodukts
  \begin{gather}
    \scal(p,q) = \sum_{i=0}^n p(x_i)q(x_i).
  \end{gather}
  Daher sind sie linear unabhängig und formen eine Basis von $\P_n$.
\end{Lemma}

\begin{Korollar}{lagrange-interpolation}
  Die Lagrange-Interpolation eingeschränkt auf den Raum $\P_n$ ist die
  Identität
\end{Korollar}
\begin{remark}
  Die Lagrangesche Interpolationsaufgabe kann auch als Gaußsche
  Ausgleichsrechnung mit dem obigen Skalarprodukt aufgefasst werden.
\end{remark}

\begin{Satz*}{lagrange-kondition}{Konditionszahl der Lagrange-Interpolation}
  Die Konditionszahl des absoluten Fehlers in der Supremumsorm der
  Lagrange-Interpolation zu den Punkten $a = x_0 < \dots < x_n = b$
  ist die \define{Lebesgue-Konstante}
  \begin{gather}
    \Lambda_{x_0,\dots,x_n} = \max_{x\in [a,b]}
    \sum_{i=0}^n \abs{L_i(x;x_0,\dots,x_n)}.
  \end{gather}
  Es gilt also
  \begin{gather}
    \max _{x\in[a,b]} \abs{I_n f(x)}
    \le \Lambda_{x_0,\dots,x_n} \max _{x\in[a,b]} \abs{f(x)}.
  \end{gather}
  Diese Abschätzung ist scharf.
\end{Satz*}

\begin{proof}
  Siehe \cite[Satz 7.3]{DeuflhardHohmann08}.
\end{proof}

\begin{Beispiel}{lagrange-kondition-equi}
  Für äquidistante Stützstellen erhält man exemplarisch die Konditionszahlen
  \begin{center}
    \begin{tabular}{r|r}
      $n$ & $\Lambda_{0,\dots,n}$\\\hline
      5 & 3.1\\
      10 & 29.9\\
      15 & 512\\
      20 & 10986
    \end{tabular}
  \end{center}
  Quelle: \cite{DeuflhardHohmann08}
\end{Beispiel}

%%% Local Variables:
%%% mode: latex
%%% TeX-master: "main"
%%% End:
