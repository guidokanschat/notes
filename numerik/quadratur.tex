\subsection{Summierte Quadratur}

\begin{Definition}{quadratur}
  Eine \define{Quadraturformel} $Q_{[a,b]}(f)$ ist eine Approximation
  des Integrals
  \begin{gather}
    Q_{[a,b]}(f) \approx \int_a^b f(x)\dx
  \end{gather}
  in der Form
  \begin{gather}
    Q_{[a,b]}(f) = \sum_{i=0}^n \omega_i f(x_i).
  \end{gather}
  Die Stützstellen $x_i$ bezeichnen wir auch als
  \define{Quadraturpunkte}, die Zahlen $\omega_i$ als
  \define{Quadraturgewichte}.

  Lässt sich die Quadraturormel bezüglich einer Zerlegung
  $\mathcal I_h$ des Intervalls $[a,b]$ in der Form
  \begin{gather}
    Q_{[a,b]}(f) = \sum_{i=1}^n Q_{I_i} (f)
  \end{gather}
  schreiben, so sprechen wir von \textbf{summierter},
  \textbf{iterierter} oder \textbf{stückweiser Quadratur}.
\end{Definition}

\begin{Definition}{lokale-fehlerordnung}
  Gilt bei einer summierten Quadraturformel die Abschätzung
  \begin{gather}
    \left|\int_{I_i} f(x)\dx - Q_{I_i}(f)\right|
    =\bigo\left(h_i^{k+1}\right)
  \end{gather}
  für jedes Teilintervall $I_i$ und Funktionen $f\in C^{k+1}[a,b]$, so
  sprechen wir von der \textbf{lokalen
    Fehlerordnung}\defindex{Fehlerordnung}\defindex{lokale Fehlerordnung} $k+1$.
\end{Definition}

\begin{Satz}{summierte-quadratur}
  Sei $\mathcal I_h$ eine Zerlegung von $[a,b]$ der Feinheit $h$ und
  $c_q$ sei so gewählt, dass
  \begin{gather}
    c_q \min_{I_i\in \mathcal I_h} h_i \ge h.
  \end{gather}
  Sind dann die Formeln $Q_{I_i}$ von lokaler Fehlerordnung $k+1$ für
  $f\in C^{k+1}[a,b]$, so gilt für die summierte Quadratur $Q_{[a,b]}$
  die Abschätzung
  \begin{gather}
    \left|\int_a^b f(x)\dx - Q_{[a,b]}(f)\right|
    = \mathcal O\left(h^{k}\right).
  \end{gather}
\end{Satz}

\begin{proof}
  Das kleinste Intervall hat die Länge $h/c_q$. Damit ist die Anzahl
  der Intervalle beschränkt durch $n_{\max}=c_q (b-a)/h$. Aus der
  lokalen Fehlerordnung ergibt sich die Existenz einer Konstanten $c$,
  so dass
  \begin{gather}
    \left|\int_{I_i} f(x)\dx - Q_{I_i}(f)\right| \le c h_i^{k+1}.
  \end{gather}
  Damit schätzen wir ab
  \begin{align}
    \left|\int_a^b f(x)\dx - Q_{[a,b]}(f)\right|
    &= \sum_{I_i\in\mathcal I_h}  \left|\int_{I_i} f(x)\dx - Q_{I_i}(f)\right|\\
    &\le \sum_{I_i\in\mathcal I_h} c h^{k+1}\\
    & \le n_{\max} c h^{k+1} = \bigo\left(h^{k}\right).
  \end{align}
\end{proof}

\subsection{Quadratur auf Einzelintervallen}

\begin{Notation}{quadrature}
  In diesem Abschnitt integrieren wir wieder über das Intervall
  $I=[a,b]$, aber mit dem Gedanken, dass es sich eigentlich um die
  Teilintervalle $I_i$ einer summierten Quadratur handelt.

  Wir betrachten in der Regel Quadraturformeln mit $n$ Punkten
  $x_1,\dots,x_n$. Oft benutzen wir Ergebnisse aus den Abschnitten
  über Interpolation. Dabei ist jeweis darauf zu achten, dass die
  Indizes dort bei null loslaufen. Der Grund für diesen wechsel ist,
  dass wir bei der Interpolation den Grad der Polynome als führende
  Größe angesehen haben, während hier die Anzahl der Quadraturpunkte
  im Vordergrund steht.
\end{Notation}

\begin{Definition}{grad-exaktheit}
  Eine Quadraturformel $Q_I$ heißt \define{exakt vom Grad $k$} und $k$
  heißt der \define{Grad der Exaktheit} von $Q_I$, wenn sie exakt für
  alle Polynome vom Grad bis zu $k$ ist, also
  \begin{gather}
    \int_I p(x)\dx - Q_{I}(p) = 0 \qquad \forall p\in \P_k.
  \end{gather}
\end{Definition}

\begin{Lemma}{exakt-ordnung}
  Seid die Quadraturformel $Q_I$ exakt vom Grad $k$ und
  $\abs{I} \le h$. Dann gilt für $f\in C^{k+1}(I)$
  \begin{gather}
    \left|\int_{I} f(x)\dx - Q_{I}(f)\right| = \bigo\bigl(h^{k+2}\bigr)
  \end{gather}
\end{Lemma}

\begin{Definition}{interpolatorische-quadratur}
  Eine \define{interpolatorische Quadraturformel} mit $n$
  Quadraturpunkten $x_1,\dots,x_n$ approximiert das Integral einer
  Funktion $f$ durch das exakte Integral ihres Interpolationspolynoms
  $p\in \P_{n-1}$
\end{Definition}

\begin{Lemma}{interpolatorisch-omega}
  Seien $x_1,\dots,x_n$ die Quadraturpunkte einer interpolatorischen
  Quadraturformel $Q_I$, die exakt für Polynome vom Grad $n-1$
  ist. Dann sind die Gewichte gegeben durch
  \begin{gather}
    \omega_i = \int_I \plagrange_{i;x_1,\dots,x_n}(x)\dx,
  \end{gather}
  wobei $\plagrange_{i;x_1,\dots,x_n}$ das
  Lagrange-Interpolationspolynom zum Punkt $x_i$ ist.
\end{Lemma}

\begin{proof}
  Die Lagrange-Polynome $\plagrange_i$ sind Polynome vom Grad
  $n-1$. Es gilt daher
  \begin{gather}
    \int_I \plagrange_i = \sum_{k=1}^n \omega_k \plagrange_i(x_k) = \omega_i.
  \end{gather}
\end{proof}

\begin{Definition}{newton-cotes}
  Werden die Quadraturpunkte $x_1,\dots,x_n$ gleichmäßig im Intervall
  $[a,b]$ verteit, so spricht man von einer
  \define{Newton-Cotes-Formel}. Die ersten drei klassischen Formeln
  sind auf dem Einheitsintervall $[0,1]$ gegeben durch
  \begin{center}
    \begin{tabular}{l|c|cccc|cccc}
      & $n$ & \multicolumn{4}{|c}{$x_i$} & \multicolumn{4}{|c}{$\omega_i$}
      \\\hline
      Trapezregel & 2 & 0 & 1 &&& \nicefrac12 & \nicefrac12\\
      Simpson-Regel & 3 & 0 & \nicefrac12 & 1 &
                          & \nicefrac16& \nicefrac46& \nicefrac16\\
      \nicefrac38-Regel & 4 & 0 & \nicefrac13 & \nicefrac23 & 1
                          & \nicefrac18& \nicefrac38& \nicefrac38& \nicefrac18
    \end{tabular}
  \end{center}
\end{Definition}

\begin{Satz}{newton-cotes-error}
  Die Fehler der Newton-Cotes-Formeln auf dem Intervall $I$ der Länge
  $h$ lassen sich wie folgt abschätzen
  \begin{gather}
    \left|\int_I f\dx - Q_I(f)\right| \le
    \begin{cases}
      \dfrac{h^3}{12}\max\limits_{\xi\in I}\abs{f''(\xi)}
      &\text{Trapezregel}\\
      \dfrac{h^5}{2880}\max\limits_{\xi\in I}\abs{f^{(4)}(\xi)}
      &\text{Simpson-Regel}\\
      \dfrac{h^5}{6480}\max\limits_{\xi\in I}\abs{f^{(4)}(\xi)}
      &\text{\nicefrac38-Regel}
    \end{cases}
  \end{gather}
\end{Satz}

\begin{proof}
  Der Beweis für die Trapezregel und die \nicefrac38-Regel benutzt
  Interpolation in den Quadraturpunkten und die Fehlerdarstellung des
  Interpolationsfehlers. Für die Trapezregel ist er als Hausaufgabe
  gestellt.

  Hier führen wir nur den Beweis für die Simpson-Regel. Nachdem man
  experimentell beobachtet, dass die Formel exakt vom Grad 3 ist,
  nicht vom erwarteten Grad 2, konstruieren wir eine Interpolation auf
  $I=[x_1,x_3]$ mit Mittelpunkt $x_2$ wie folgt:
  \begin{xalignat}2
    p(x_1) &= f(x_1) & p(x_2) &= f(x_2) \\
    p(x_3) &= f(x_3) & p'(x_2) &= f'(x_2).
  \end{xalignat}
  Die letzte Bedingung ist aus der Quadraturformel nicht
  ersichtlich. Folgen wir jedoch der Basiskonstruktion im
  \slideref{Satz}{hermite-interpolation} über die Wohlgestelltheit der
  Hermite-Interpolationsaufgabe, so erhalten wir
  \begin{xalignat}2
    H_{10}(x) &= \tfrac{4(x-x_2)^2(x_3-x)}{h^3}&
    H_{20}(x) &= \tfrac{4(x-x_1)(x-x_3)}{h^2}\\
    H_{30}(x) &= \tfrac{4(x-x_2)^2(x-x_1)}{h^3}&
    H_{21}(x) &= \tfrac{4(x-x_2)(x-x_1)(x-x_3)}{h^2}
  \end{xalignat}

  \begin{figure}[tp]
    \centering
    \includegraphics[width=.8\textwidth]{graph/interpolation/simpson}
    \caption{Basisfunktionen für die Simpsonregel. Beachte, dass $H_21$ in allen Quadraturpunkten verschwindet und auch das Integral null ist.}
    \label{fig:simpson}
  \end{figure}
  Die Funktion $H_{21}(x)$ ist das Produkt der Parabel
  $(x-x_1)(x-x_3)$, die symmetrisch zur Intervallmitte ist mit einer
  linearen Funktion mit Nullstelle in der Intervallmitte. Daher
  verschwindet ihr Integral und das zugehörige Integrationsgewicht ist
  null. Die Simpson-Regel lässt sich also schreiben als
  \begin{gather}
    Q_I = \frac h6f(x_1)+\frac{4h}6f(x_2) + \frac h6f(x_3) + 0 f'(x_2).
  \end{gather}
  Für die obige Interpolation gilt nach
  \slideref{Satz}{Hermite-restglied} die Fehlerdarstellung
  \begin{gather}
    f(x) - p(x) = \frac{f^{4}(\xi(x)}{4!} \pnewton_{x_1,x_2,x_2,x_3}(x).
  \end{gather}
  Integration ergibt
  \begin{align}
    \left|\int_I f\dx - Q_I(f)\right|
    &= \left|\int_I \bigl(f(x) - p(x)\bigr)\dx\right|\\
    &\le \max_{\xi\in I} \frac{f^{4}(\xi}{4!}
      \int_I \pnewton_{x_1,x_2,x_2,x_3}(x) \dx.
  \end{align}
  Schließlich berechnen wir
  \begin{gather}
    \int_I \pnewton_{x_1,x_2,x_2,x_3}(x) \dx
    = \int_I (x-x_1)(x-x_2)^2(x-x_3)
    = \frac1{120}
  \end{gather}
\end{proof}

\begin{remark}
  Ab Grad ??? haben die Newton-Cotes-Formeln negative Gewichte...
\end{remark}

\subsection{Gauß-Quadratur}

\begin{Lemma}{quadratur-exakt-max}
  Sei $Q_n$ eine Quadraturformel auf einem Intervall $I$ mit $n$
  Quadraturpunkten. Dann ist $Q_n$ maximal exakt vom Grad $2n-1$.
\end{Lemma}

\begin{proof}
  Siehe \cite[Satz 3.1]{Rannacher17}
\end{proof}

\begin{Definition}{Gauss-Legendre}
  Die $n$-Punkt-\define{Gauß-Legendre-Formel} auf dem Intervall
  $I=[-1,1]$ benutzt als Stützstellen $x_1,\dots,x_n$ die Nullstellen
  des Legendre-Polynoms $\plegendre(x)$ vom Grad $n$. Ihre
  Quadraturgewichte sind die Integrale der Lagrange-Polynome
  \begin{gather}
    \omega_i = \int_I \plagrange_{i;x_1,\dots,x_n}(x)\dx.
  \end{gather}
\end{Definition}

\begin{Satz}{gauss-legendre}
  Die $n$-Punkt-Gauß-Legendre-Formel ist wohldefiniert und exakt für
  beliebige Polynome vom Grad $2n-1$.
\end{Satz}

\begin{Satz}{gauss-legendre-eindeutig}
  Seien die Quadraturformeln $Q_k$ mit Quadraturpunkten
  $x_1^{(k)},\dots,x_k^{(k)}$ für $k=1,\dots,n$ auf dem Intervall
  $I=[-1,1]$ exakt für beliebige $p\in \P_{2k-1}$. Dann sind die
  Polynome
  \begin{gather}
    p_k(x) = \prod_{i=1}^k (x-x_i^{(k)}) \in \P_k
  \end{gather}
  und $p_0(x) = 1 \in \P_0$ paarweise orthogonal bezüglich des
  $L^2$-Skalarprodukts. Insbesondere sind sie damit Vielfache der
  Legendre-Polynome $\plegendre_k$ und die
  $n$-Punkt-Gauss-Legendre-Formel ist die einzige Formel mit $n$
  Punkten, deren Grad der Exaktheit $2n-1$ ist.
\end{Satz}

\begin{Lemma}{gauss-legendre-gewichte}
  Die Gewichte der Gauss-Legendre-Formeln sind positiv und genügen der
  Darstellung
  \begin{gather}
    \omega_i = \int_{-1}^1 \prod_{j\neq i}
    \left(\frac{x-x_j}{x_i-x_j}\right)^2\dx.
  \end{gather}
\end{Lemma}

\begin{Lemma}{gauss-legendre-fehler}
  Für die Gauss-Legendre-Formel mit $n$ Quadraturpunkten auf
  $I=[-1,1]$ gilt die Fehlerabschätzung
  \begin{gather}
    \left|\int_I f\dx - Q_n(f)\right|
    \le \max_{\xi\in I}\frac{f^{(2n)}(\xi)}{(2n)!}
    \int_{-1}^1 \prod_{i=1}^n(x-x_i)^2.
  \end{gather}
\end{Lemma}

\begin{remark}
  Alle Resultate dieses Abschnitts gelten für Skalarprodukte der Form
  \begin{gather}
    \scal(p,q) = \int_I \omega(x)p(x)q(x)\dx
  \end{gather}
  mit einer positiven Gewichtsfunktion $\omega(x)$, wenn man die
  Legendre-Polynome durch die entsprechenden orthogonalen Polynome
  ersetzt.
\end{remark}

\subsection{Richardson-Extrapolation und Romberg-Quadratur}

\begin{Definition}{richardson-extrapolation}
  Sei $T(h)$ eine numerische Methode zur Approximation des
  tatsächlichen Wertes $T(0)$ mit Diskretisierungsparameter $h$ und
  Fehlerabschätzung $\abs{T(h)-T(0)} = \bigo(h^p)$. Zur
  \define{Richardson-Extrapolation} wertet man diese Methode mit einer
  Schrittfolge $h_1, h_2,\ldots,h_n$ aus, so dass die Schrittweite
  (theoretisch) gegen null geht. Wertet man dann das
  Interpolationspolynom $p(h^p)$ an der Stelle $h=0$ aus, so bekommt
  man unter stärkeren Voraussetzungen die verbesserte Approximation
  \begin{gather}
    \abs{T(0)-p(0)} = \bigo(h^{np}).
  \end{gather}
\end{Definition}

\begin{remark}
  Tatsächlich genügt die einfache Fehlerabschätzung
  $\abs{T(h)-T(0)} = \bigo(h^p)$ nicht, um die behauptete
  Konvergenzordnung zu beweisen. Man benötigt eine asymptotische
  Fehlerentwicklung der Form
  \begin{gather}
    T(h)-T(0) = \tau_1 h^p + \tau_2 h^{2p} + \dots \tau_n h^{np}
    + \bigo(h^{(n+1)p}).
  \end{gather}
\end{remark}

\begin{Definition}{Romberg-quadratur}
  Die \define{Romberg-Quadratur} beruht auf einer summierten
  Quadraturformel $Q_h$ der Ordnung $h^p$, die für eine Folge von
  Schrittweiten $h_1,\dots, h_n$ angewandt wird. Aus diesen berechnet
  man mit dem \putindex{Neville}-Algorithmus Approximationen für
  $Q_0$.
\end{Definition}

\begin{Algorithmus*}{romberg}{Romberg-Quadratur}
  \lstinputlisting{code/romberg.py}
\end{Algorithmus*}

\subsection{Praktische Aspekte}

\begin{remark}
  Die Konvergenzabschätzungen der Form
  \begin{gather}
    \left|\int_I f\dx - Q_h(f)\right| \le c h^p \norm{f^{p+1}}_{\infty;I}
  \end{gather}
  verlieren ihren Nutzen für große $h$, wenn die Ableitungen von $f$
  wachsen. Schlimmstenfalls bekommt man dann aus der
  Interpolationseigenschaft noch immer
  \begin{gather}
    \left|\int_I f\dx - Q_h(f)\right| \le c \norm{f}_{\infty;I}.
  \end{gather}
  Es gibt aber keine Garantie, dass der Fehler bei feinerer
  Unterteilung schrumpft.
  
  Ist aber $f\in C^{p+1}(I)$, so gilt die obige Abschätzung für
  hinreichend kleine $h_1>h_2$ in der stärkeren Form
  \begin{gather}
    \left|\int_I f\dx - Q_{h_2}(f)\right|
    \approx \left(\frac{h_2}{h_1}\right)^p
    \left|\int_I f\dx - Q_{h_1}(f)\right|.
  \end{gather}
  Man spricht hier vom asymptotischen Bereich, für größere $h$ vom
  präasymptotischen Bereich.
\end{remark}

%%% Local Variables:
%%% mode: latex
%%% TeX-master: "main"
%%% End:
