\subsection{Summierte Quadratur}

\begin{Definition}{quadratur}
  Eine \define{Quadraturformel} $Q_{[a,b]}(f)$ ist eine Approximation
  des Integrals
  \begin{gather}
    Q_{[a,b]}(f) \approx \int_a^b f(x)\dx
  \end{gather}
  in der Form
  \begin{gather}
    Q_{[a,b]}(f) = \sum_{i=0}^n \omega_i f(x_i).
  \end{gather}
  Die Stützstellen $x_i$ bezeichnen wir auch als
  \define{Quadraturpunkte}, die Zahlen $\omega_i$ als
  \define{Quadraturgewichte}.

  Lässt sich die Quadraturormel bezüglich einer Zerlegung
  $\mathcal I_h$ des Intervalls $[a,b]$ in der Form
  \begin{gather}
    Q_{[a,b]}(f) = \sum_{i=1}^n Q_{I_i} (f)
  \end{gather}
  schreiben, so sprechen wir von \textbf{summierter},
  \textbf{iterierter} oder \textbf{stückweiser Quadratur}.
\end{Definition}

\begin{Definition}{lokale-fehlerordnung}
  Gilt bei einer summierten Quadraturformel die Abschätzung
  \begin{gather}
    \left|\int_{I_i} f(x)\dx - Q_{I_i}(f)\right|
    =\bigo\left(h_i^{k+1}\right)
  \end{gather}
  für jedes Teilintervall $I_i$ und Funktionen $f\in C^{k+1}[a,b]$, so
  sprechen wir von der \textbf{lokalen
    Fehlerordnung}\defindex{Fehlerordnung}\defindex{lokale Fehlerordnung} $k+1$.
\end{Definition}

\begin{Satz}{summierte-quadratur}
  Sei $\mathcal I_h$ eine Zerlegung von $[a,b]$ der Feinheit $h$ und
  $c_q$ sei so gewählt, dass
  \begin{gather}
    c_q \min_{I_i\in \mathcal I_h} h_i \ge h.
  \end{gather}
  Sind dann die Formeln $Q_{I_i}$ von lokaler Fehlerordnung $k+1$ für
  $f\in C^{k+1}[a,b]$, so gilt für die summierte Quadratur $Q_{[a,b]}$
  die Abschätzung
  \begin{gather}
    \left|\int_a^b f(x)\dx - Q_{[a,b]}(f)\right|
    = \mathcal O\left(h^{k}\right).
  \end{gather}
\end{Satz}

\begin{proof}
  Das kleinste Intervall hat die Länge $h/c_q$. Damit ist die Anzahl
  der Intervalle beschränkt durch $n_{\max}=c_q (b-a)/h$. Aus der
  lokalen Fehlerordnung ergibt sich die Existenz einer Konstanten $c$,
  so dass
  \begin{gather}
    \left|\int_{I_i} f(x)\dx - Q_{I_i}(f)\right| \le c h_i^{k+1}.
  \end{gather}
  Damit schätzen wir ab
  \begin{align}
    \left|\int_a^b f(x)\dx - Q_{[a,b]}(f)\right|
    &= \sum_{I_i\in\mathcal I_h}  \left|\int_{I_i} f(x)\dx - Q_{I_i}(f)\right|\\
    &\le \sum_{I_i\in\mathcal I_h} c h^{k+1}\\
    & \le n_{\max} c h^{k+1} = \bigo\left(h^{k}\right).
  \end{align}
\end{proof}

\subsection{Quadratur auf Einzelintervallen}

\begin{Notation}{quadrature}
  In diesem Abschnitt integrieren wir wieder über das Intervall
  $I=[a,b]$, aber mit dem Gedanken, dass es sich eigentlich um die
  Teilintervalle $I_i$ einer summierten Quadratur handelt.

  Wir betrachten in der Regel Quadraturformeln mit $n$ Punkten
  $x_1,\dots,x_n$. Oft benutzen wir Ergebnisse aus den Abschnitten
  über Interpolation. Dabei ist jeweis darauf zu achten, dass die
  Indizes dort bei null loslaufen. Der Grund für diesen wechsel ist,
  dass wir bei der Interpolation den Grad der Polynome als führende
  Größe angesehen haben, während hier die Anzahl der Quadraturpunkte
  im Vordergrund steht.
\end{Notation}

\begin{Definition}{grad-exaktheit}
  Eine Quadraturformel $Q_I$ heißt \define{exakt vom Grad $k$} und $k$
  heißt der \define{Grad der Exaktheit} von $Q_I$, wenn sie exakt für
  alle Polynome vom Grad bis zu $k$ ist, also
  \begin{gather}
    \int_I p(x)\dx - Q_{I}(p) = 0 \qquad \forall p\in \P_k.
  \end{gather}
\end{Definition}

\begin{Lemma}{exakt-ordnung}
  Seid die Quadraturformel $Q_I$ exakt vom Grad $k$ und
  $\abs{I} \le h$. Dann gilt für $f\in C^{k+1}(I)$
  \begin{gather}
    \left|\int_{I} f(x)\dx - Q_{I}(f)\right| = \bigo\bigl(h^{k+2}\bigr)
  \end{gather}
\end{Lemma}

\begin{Definition}{interpolatorische-quadratur}
  Eine \define{interpolatorische Quadraturformel} mit $n$
  Quadraturpunkten $x_1,\dots,x_n$ approximiert das Integral einer
  Funktion $f$ durch das exakte Integral ihres Interpolationspolynoms
  $p\in \P_{n-1}$
\end{Definition}

\begin{Lemma}{interpolatorisch-omega}
  Seien $x_1,\dots,x_n$ die Quadraturpunkte einer interpolatorischen
  Quadraturformel $Q_I$. Dann sind die Gewichte gegeben durch
  \begin{gather}
    \omega_i = \int_I \plagrange_{i;x_1,\dots,x_n}(x)\dx,
  \end{gather}
  wobei $\plagrange_{i;x_1,\dots,x_n}$ das
  Lagrange-Inerpolationspolynom zum Punkt $x_i$ ist.
\end{Lemma}

\begin{Lemma}{exakt-omega}
  Eine $n$-Punkt-Quadraturformel $Q_n$ ist dann und nur dann exakt von
  Ordnung $k$, wenn
\end{Lemma}

\begin{Definition}{newton-cotes}
  Werden die Quadraturpunkte $x_1,\dots,x_n$ gleichmäßig im Intervall
  $[a,b]$ verteit, so spricht man von einer
  \define{Newton-Cotes-Formel}. Die ersten drei klassischen Formeln
  sind auf dem Einheitsintervall $[0,1]$ gegeben durch
  \begin{center}
    \begin{tabular}{l|c|cccc|cccc}
      & $n$ & \multicolumn{4}{|c}{$x_i$} & \multicolumn{4}{|c}{$\omega_i$}
      \\\hline
      Trapezregel & 2 & 0 & 1 &&& \nicefrac12 & \nicefrac12\\
      Simpson-Regel & 3 & 0 & \nicefrac12 & 1 &
                          & \nicefrac16& \nicefrac46& \nicefrac16\\
      \nicefrac38-Regel & 4 & 0 & \nicefrac13 & \nicefrac13 & 1
                          & \nicefrac18& \nicefrac38& \nicefrac38& \nicefrac18
    \end{tabular}
  \end{center}
\end{Definition}


%%% Local Variables:
%%% mode: latex
%%% TeX-master: "main"
%%% End:
