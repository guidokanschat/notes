\lstset{language=Python}
\usetikzlibrary{svg.path}
\excludecomment{solution}
\tikzset{shape veloxy/.style={color=black,draw,fill=red,thick}}
\tikzset{shape pressure/.style={color=black,draw,fill=cyan,thick}}


\def\constref#1{C_{\text{\ref{#1}}}}
\title{Einführung in die Numerik}
\author{Guido Kanschat}
\date{\today}

\newcommand{\rd}{\operatorname{rd}}
\newcommand{\eps}{\texttt{eps}}
\begin{document}
\maketitle
\tableofcontents
\chapter{Orthogonale Polynome}
\section{Polynomräume}

% \begin{Satz}{nullstellen}
%   Ein reelles Polynom vom Grad $n$ hat höchstens $n$ Nullstellen oder es ist das Nullpolynom.
% \end{Satz}

% \begin{proof}
%   Für $n=1$ handelt es sich um ein lineares Polynom und die Aussage
%   des Satzes ist unmittelbar klar. Sei nun $p$ ein Polynom strikt vom
%   Grad $n>1$ mit Nullstelle $x_0$. Dann gibt es nach dem euklidischen
%   Algorithmus zur Division mit Rest ein Polynom $q$ vom Grad $n-1$ und
%   eine Konstante $c$, so dass
%   \begin{gather}
%     p(x) = (x-x_0)q(x)+c.
%   \end{gather}
%   Daraus folgt $p(x_0) = c$, so dass folgt $c=0$. Wir können dieses
%   Verfahren für alle weiteren Nullstellen $x_1,\dots,x_m$ wiederholen
%   und erhalten
%   \begin{gather}
%     p(x) =  r(x) \prod_{k=0}^m (x-x_i),
%   \end{gather}
%   wobei $r(x)$ ein Polynom vom Grad $n-m$ sein muss, da $p$ vom Grad
%   $n$ ist. Insbesondere muss gelten $m\le n$.
% \end{proof}

% \begin{Korollar}{polynome-identisch}
%   Zwei reelle Polynome vom Grad $n$ sind identisch, wenn sie in
%   mindestens $n+1$ Punkten übereinstimmen. 
% \end{Korollar}


\begin{Lemma}{monome-linear-unabhaengig}
  Die Menge der Monome $\{x^0, x^1,\dots,x^n\}$ ist linear unabhängig.
\end{Lemma}

\begin{proof}
  Sei $p$ ein Polynom vom Grad $n$, also
  \begin{gather}
     p(x) = a_nx^n+a_{n-1}x^{n-1}+\dots+a_1x+a_0
   \end{gather}
   $p$ ist also gerade eine Linearkombination der Monome.  Zu zeigen
   ist, dass aus der Eigenschaft $p \equiv 0$ folgt, dass alle
   Koeffizienten verschwinden, also
  \begin{gather}
    p(x) \equiv 0
    \quad\Rightarrow\quad a_n = \dots = a_0 = 0.
  \end{gather}
  Zu diesem Zweck berechnen wir die $n$-te Ableitung von $p$ und
  erhalten, da mit $p$ auch alle seine Ableitungen identisch
  verschwinden,
  \begin{gather}
    n! a_n = 0.
  \end{gather}
  Daraus schließen wir $a_n = 0$. Nun gilt für die $(n-1)$-te Ableitung
  \begin{gather}
    n! a_n x + (n-1)! a_{n-1} = (n-1)! a_{n-1} = 0.
  \end{gather}
  Auf diese Weise schließen wir rekursiv bis $a_0$, dass alle Koeffizienten verschwinden. Damit ist das Lemma bewiesen.
\end{proof}

\begin{Satz}{polynomraum}
  Die Polynome vom maximalen Grad $n$ bilden einen Vektorraum der
  Dimension $n+1$.  Wir bezeichnen ihn mit $\P_n$.
\end{Satz}

\begin{proof}
  Es ist leicht nachzurechnen, dass sowohl die Summe, als auch reelle
  Vielfache von Polynomen wieder Polynome sind. Insbesondere erhöhen
  beide Operationen den Grad nicht. Damit ist $\P_n$ ein
  Vektorraum. Er wird per definitionem von den Monomen vom Grad bis zu
  $n$ erzeugt. Da diese nach
  \slideref{Lemma}{monome-linear-unabhaengig} linear unabhängig sind,
  bilden sie eine Basis und die Dimension von $\P_n$ ist $n+1$.
\end{proof}

\begin{Quiz}{Polynomräume}
  Gegeben beliebige Werte $x_j\in\R$ mit $j=1,\dots,n$. Die Menge der
  Polynome $p_i$ definiert durch
  \begin{align*}
    p_0(x) &= 1\\
    p_i(x) &= \prod_{j=1}^i (x-x_j),\qquad i=1,\dots,n
  \end{align*}
  \begin{enumerate}[A]
  \item ist linear unabhängig
  \item ist linear abhängig
  \item ist ein Erzeugendensystem für $\P_n$
  \item ist eine Basis von $\P_n$
  \end{enumerate}
\end{Quiz}
\section{Skalarprodukt und Orthogonalität}
\begin{Definition}{skalarprodukt}
  Sei $V$ ein reeller Vektorraum. Eine Abbildung
  $a\colon V \times V \to \R$ heißt \define{Bilinearform}, wenn für
  $u,v,w\in V$ und $\lambda,\mu\in \R$ gilt
  \begin{align}
    a(\lambda u + \mu v,w) &= \lambda a(u,w) + \mu a(v,w)\\
    a(w,\lambda u + \mu v) &= \lambda (w,u) + \mu a(w,v).
  \end{align}
  Eine Bilinearform heißt \define{symmetrisch}, wenn für $u,v\in V$ gilt
  \begin{gather}
    a(u,v) = a(v,u).
  \end{gather}
  Sie heißt \define{positiv semi-definit}, wenn $a(u,u) \ge 0$ für alle
  $u\in V$ und \define{positiv definit}, wenn zusätzlich
  \begin{gather}
    a(u,u) = 0 \quad \Longrightarrow \quad u=0.
  \end{gather}
  Eine symmetrische, positiv definite Bilinearform heißt
  \define{Skalarprodukt}, in der Regel notiert als $\scal(\cdot,\cdot)$.
\end{Definition}

%%%%%%%%%%%%%%%%%%%%%%%%%%%%%%%%%%%%%%%%%%%%%%%%%%%%%%%%%%%%%%%%%%%%%%
\begin{Lemma*}{bcs}{Bunjakowski-Cauchy-Schwarzsche Ungleichung}
  Sei $\scal(\cdot,\cdot)$ ein Skalarprodukt auf $V$.  Für zwei beliebige Elemente $u,v\in V$ gilt
  \begin{gather}
    \abs{\scal(u,v)} \le \sqrt{\scal(u,u)} \, \sqrt{\scal(v,v)}.
  \end{gather}
  Gleichheit gilt genau dann, wenn $u$ und $v$ kollinear sind, also
  $v=\alpha u$ mit einem skalaren Faktor $\alpha$.
\end{Lemma*}

\begin{proof}
  Zunächst zeigen wir nur die Ungleichung: Für $v=0\in V$ ist sie
  offensichtlich.
  
  Seien nun $v,u \in V$ keine Nullvektoren. Für beliebige $\mu, \lambda \in \R$
  gilt wegen der Bilinearität 
  \begin{gather}
   0 \le \scal(\lambda u + \mu v,\lambda u +  \mu v)
    = \lambda^{2} \scal(u,u)+2 \mu \lambda \scal(u,v) +\mu^{2} \scal(v,v)
  \end{gather}
  Setze $\lambda := \scal(v,v) \neq 0$
  \begin{gather}
   0 \le \scal(v,v)^{2} \scal(u,u) + 2\mu \scal(v,v)\scal(u,v) +\mu^{2}\scal(v,v)
  \end{gather}
  Dividiere durch$\scal(v,v)$
  \begin{gather}
   0 \le \scal(v,v) \scal(u,u) + 2\mu \scal(u,v) +\mu^{2}
  \end{gather}
  Setze nun $\mu := -\scal(u,v)$
  \begin{gather}
    0 \le \scal(v,v) \scal(u,u) -2\scal(u,v)^{2}+\scal(u,v)^{2}
  \end{gather}
  Daraus folgt
  \begin{gather}
    \scal(u,v)^{2} \le \scal(u,u) \scal(v,v)
  \end{gather}
  und mit der Monotonie der Quadratfunktion die Ungleichung.

  Nun bleibt die Äquivalenz für die Gleichheit zu zeigen.
  Für $v=0$ ist dies wieder trivial erfüllt. Seien zunächst $u,v$ linear abhängig, also zum Beispiel $u=av$.
  Dann gilt mit der Abkürzung $f(v) = \sqrt{\scal(v,v)}$
  \begin{gather}
    \abs{\scal(u,v)} = \abs{\scal(av,v)}
    = \abs{a} \cdot f(v) \cdot f(v)
    = f(av) \cdot f(v) =f(u) \cdot f(v).
  \end{gather}

  Gelte nun umgekehrt $\scal(u,v) = \sqrt{\scal(u,u)}\sqrt{\scal(v,v)}$.
  Es folgt
  \begin{gather}
     \scal(v,v) \scal(u,u) -2\scal(u,v)^{2}+\scal(u,v)^{2} = 0.
  \end{gather}
  Setze $\mu = \scal(u,v)\neq 0 $ und
  $\lambda = \scal(v,v)\neq 0$. Dann erhält man
  \begin{gather}
    \lambda \scal(u,u) - 2 \mu \scal(u,v) + \mu^2 = 0.
  \end{gather}
  Multipliplikation mit $\scal(v,v)$ ergibt
  \begin{gather}
   \lambda^2 \scal(u,u)+2\mu \scal(u,v)\scal(v,v) +\mu^{2}\scal(v,v) = 0 = \scal(\lambda u-\mu v,\lambda u-\mu v).
  \end{gather}
  
  Wegen der Definitheit folgt nun
  $\lambda u + \mu v = 0$ und da $\mu$ und $\lambda$ ungleich Null sind gilt,
  dass $ u,v$ linear abhängig sind
\end{proof}

%%%%%%%%%%%%%%%%%%%%%%%%%%%%%%%%%%%%%%%%%%%%%%%%%%%%%%%%%%%%%%%%%%%%%%
\begin{Lemma}{hilbertnorm}
  Sei $V$ ein reeller Vektorraum mit Skalarprodukt
  $\scal(\cdot,\cdot)$. Dann ist durch
  \begin{gather}
    \norm{u} = \sqrt{\scal(u,u)}
  \end{gather}
  auf $V$ eine Norm definiert. Ein reeller Vektorraum $V$ mit
  Skalarprodukt und zugehöriger Norm heißt \define{euklidischer
    Vektorraum}.
\end{Lemma}

\begin{proof}
  Das Skalarprodukt ist nicht negativ, daher ist die Abbildung $\norm{\cdot}\colon V \to \R$ wohldefiniert.
  Wir müssen nun die Normeigenschaften nachrechnen. Sei dazu $u \in V$. Es gilt
  \begin{enumerate}
  \item Nichtnegativität und Definitheit folgen sofort aus den entsprechenden Eigenschaften des Skalarprodukts.
  \item Homogenität
  \begin{gather}
    \norm{\lambda u} = \sqrt{\scal(\lambda u,\lambda u)}
    =\sqrt{\lambda^{2}\scal(u,u)}
    = \abs{\lambda}\sqrt{\scal(u,u)}
    =\abs{\lambda}\norm{u}
  \end{gather}
  \item Deiecksungleichung
  \begin{gather}
    \begin{aligned}
      \norm{u+v}^{2}
      &= \scal(u+v,u+v)\\
      &= \scal(u,u)+ 2 \scal(u,v) + \scal(v,v)\\
      \label{eq:orthopoly:1}
      &\le \scal(u,u)+ 2 \norm{u} \, \norm{v} + \scal(v,v)\\
      &=\norm{u}^{2}+ 2 \norm{u} \, \norm{v}+ \norm{v}^{2}\\
      & =(\norm{u}+\norm{v})^{2}\\
    \end{aligned}
  \end{gather}
  Daraus folgt durch Wurzelziehen auf beiden Seiten $\norm{u+v} \le \norm{u}+\norm{v}$.
  Für die Abschätzung in Zeile~\eqref{eq:orthopoly:1} haben wir die
  Bunyakovsky-Cauchy-Schwarz-Ungleichung aus \slideref{Lemma}{bcs} verwendet.
  \end{enumerate}
\end{proof}

\begin{remark}
  Ein reeller Vektorraum $V$ mit
  Skalarprodukt und zugehöriger Norm heißt \define{euklidischer
    Vektorraum}.
\end{remark}

\begin{Lemma*}{l2-norm}{$L^2$-Skalarprodukt}
  Auf dem Raum $V=\P_n$ der reellen Polynome vom Grad bis zu $n$ ist durch
  \begin{gather}
    \scal(p,q) = \int_{-1}^1 p(x)q(x)\dx
  \end{gather}
  ein Skalarprodukt definiert.
\end{Lemma*}

\begin{proof}
  Hier gilt es zu prüfen, ob die Abbildung auch die vier Eigenschaften eines
  Skalarprodukts erfüllt.\\
  Seien  $p,q,g \in \P_n$ in diesem Beweis.\\
  Da wir schon von einem Skalarprodukt ausgehen, empfiehlt es sich
  zuerst die Symmetrie zu zeigen.
  \begin{gather}
    \scal(p,q) =  \int_{-1}^1 p(x)q(x)\dx = \int_{-1}^1 q(x)p(x)\dx
    =\scal(q,p)
  \end{gather}
  Wenn wir nun zeigen, dass es eine Bilinearform ist müssen wir nur noch eine
  Identität zeigen, da wir schon wissen, dass die Symmetrieeigenschaft
  erfüllt ist.
  \begin{gather}
    \begin{aligned}
    \scal(\lambda p + \mu q, g)
    &= \int_{-1}^1 (\lambda p(x)+ \mu q(x))g(x)\dx\\
   & = \int_{-1}^1 \lambda p(x)g(x)+ \mu q(x)g(x)\dx \\
   &= \int_{-1}^1 \lambda  p(x)g(x)\dx + \int_{-1}^1 \mu q(x)g(x)\dx \\
   &= \lambda \int_{-1}^1 p(x)g(x)\dx + \mu  \int_{-1}^1 q(x)g(x)\dx \\
   &= \lambda \scal(p,g) + \mu \scal(q,g)
    \end{aligned}
  \end{gather}
  Da wir die Symmetrie vorher gezeigt haben, gilt Linearität auch
  im zweiten Argument.\\
  
  Als letztes zeigen wir, dass die Abbildung positiv definit ist.
  \begin{gather}
    0 = \scal(p,p) = \int_{-1}^1 p(x)p(x)\dx =\int_{-1}^1 p(x)^{2}\dx
  \end{gather}
  Aus den Integraleigenschaften folgt
  \begin {gather}
    0 = p(x)^{2} \quad \forall x
  \end{gather}
  Dies kann nur der Fall sein, wenn $p \equiv 0$ ist.\\
  Somit haben wir nachgerechnet, dass es sich um Skalarprodukt handelt.
  \end{proof}
    
  
\begin{Definition}{orthogonal}
  Zwei Vektoren $u,v\in V$ heißen \define{orthogonal}, wenn
  \begin{gather}
    \scal(u,v) = 0.
  \end{gather}
  Ein Vektor $u\in V$ ist orthogonal zum Untervektorraum $W\subset V$, wenn
  \begin{gather}
    \scal(u,v) = 0\quad\forall v\in W.
  \end{gather}
\end{Definition}

\begin{Notation}{euklidischer-vr}
  Von nun an bezeichnet $V$ immer einen endlichdimensionalen, reellen,
  euklidischen Vektorraum.
\end{Notation}

\begin{Lemma*}{pythagoras}{Pythagoras}
  Seien zwei Vektoren $u\in V$ und $v\in V$ orthogonal zueinander. Dann gilt
  \begin{gather}
    \norm{u+v}^{2} = \norm{u}^{2} + \norm{v}^{2}
  \end{gather}
\end{Lemma*}

\begin{proof}
  Seien $u,v \in V$. Es gilt $ 0 = \scal(u,v)$
   \begin{gather}
    \norm{u+v}^{2} = \scal(u+v,u+v)
    %=\scal(u+v,u)+\scal(u+v,v)
    %=\scal(u,u)+\scal(v,u)+\scal(u,v)+\scal(v,v)
    =\norm{u}^{2} + \norm{v}^{2} +2\scal(u,v) = \norm{u}^{2} + \norm{v}^{2}
  \end{gather}
\end{proof}

\section{Bestapproximation und orthogonale Projektion}
\begin{Definition}{bestapproximation}
  Sei $A\subset V$ ein affiner Unterraum eines euklidischen
  Vektorraums. Dann ist die Bestapproximation $u_b\in A$ eines Vektors
  $u\in V$ in $A$ definiert durch die Beziehung
  \begin{gather}
    \norm{u-u_b} = \min_{v\in A} \norm{u-v}.
  \end{gather}
\end{Definition}

\begin{Satz}{bestapproximation}
  Sei $w \in V$ und $W \subset V$.
  Sei $A=w+W$ ein nichtleerer, affiner Unterraum von $V$. Dann
  existiert die Bestapproximation nach
  \slideref{Definition}{bestapproximation} und ist eindeutig
  bestimmt. Es gilt die notwendige und hinreichende Bedingung
  \begin{gather}
    \scal(u-u_b,v) = 0 \quad \forall v\in W.
  \end{gather}
  Das heißt $ u_b$ ist Bestapproximation genau dann wenn $u-u_b$
  orthogonal zu $W$ bzgl. des Skalarprodukts $\scal(\cdot,\cdot)$ ist.
\end{Satz}

\begin{proof}
  Der Beweis gliedert sich in drei Teile. Zuert wird die Äquivalenz
  gezeigt danach zeigen wir die Eindeutigkeit und zum Schluss
  erst die Existenz.\\ \\
 $\glqq \Rightarrow \glqq$
  Sei $ u_b \in A$ die Bestapproximation des Vektors $ u \in V$\\
  Wir defnieren nun eine Funktion:
  \begin{gather}
    F_v(x):= \norm{u-u_b-xv}^{2}, x \in \R,  v\in A
  \end{gather}

  Diese Funktion besitzt ein Minimum bei x=0. Folglich gilt
  \begin{gather}
    \left. \frac{d}{dx} F(x) \right|_{x=0}
    =\left. \frac{d}{dx}\norm{u-u_b-xv}^{2} \right|_{x=0}=0
  \end{gather}
    
  Dies kann weiter umgeformt werden zu
  $\scal(u-u_b-xv,v)|_{x=0}=0 \ \forall v\in A$ und folglich zu
  \begin{gather}
   \scal(u-u_b,v)=0 \ \forall v\in A
  \end{gather}

  $\grqq \Leftarrow \glqq$
  Nun erfüllt $u_b\in A$ die Bedingung.\\
  Dann gilt mit einem beliebigen $v\in A$:
  \begin{gather}
   \norm{u-u_b}^{2}=\scal(u-u_b,u-u_b)\\
   = \scal(u-u_b,u-v)+\scal(u-u_b,v-u_b)\\
   \le \norm{u-u_b}\cdot\norm{u-v}\\
  \end{gather}
  Daraus folgt $\norm{u-u_b} \le \inf_{v\in A}{\norm{u-v}}$\\
  Damit erfüllt $u_b$ eben die Definiton der Bestapproximation\\ \\
  Nun zur Eindeutigkeit:\\
  Seien $u_b$ und $u_d \in$ A zwei Bestapproximationen.
  Dann gilt notwendigerweise
  \begin{gather}
   \scal(u-u_b,v) = 0 = \scal(u-u_d,v) \quad \forall v\in A
  \end{gather}
  Dies wird umgeformt zu
  \begin{gather}
  \scal(u-u_d,v)-\scal(u-u_b,v)=0 \quad  \forall v\in A \\
  \scal(u_b-u_d,v) = 0 \quad \forall v \in A
  \end{gather}
  Wähle nun $v:=u_b-u_d \in A$. Dies ergibt
  $\norm{u_b-u_d}^{2} =0$ und somit folgt $u_b = u_d$\\
  Die Existenz:\\
  Der endliche dimensionale Teilraum A$\subseteq$V besitzt eine Basis
  $(b_1,\dots, b_n)$ mit $n:=dim V$. Die gesuchte Approximation
  $u_b\in A$ lässt sich
  durch die Basis in folgender Form darstellen
  \begin{gather}
   u_b = \sum_{k=1}^n a_k b_k
  \end{gather}
  Dies wird in die notwendige Orthogonalitätsbedingung
  \slideref{Satz}{bestapproximation} eingesetzt.
  \begin{gather}
   \scal(u-\sum_{k=1}^n a_k b_k,v)=\scal(u,v)-\sum_{k=1}^n a_k\scal(b_k,v)=0
   \quad \forall v\in A
   \end{gather}
 Dies ist bei der Wahl von $v:=b_i \quad i=1,\dots,n$ äquivalent zu dem
 linearen $n$x$n$ Gleichungssystem.
 \begin{gather}
   \sum_{k=1}^n\scal(b_k,b_i) a_k= \scal(u,b_i) \quad i=1,\dots,n
 \end{gather}
 Definiere nun $A,x,b$ wie folgt
 \begin{gather}
  A:=(\scal(b_k,b_i))_{i,k=1}^n \quad x:=(a_k)_{k=1}^n\quad b:=(\scal(u,b_i))_{i=1}^n
 \end{gather}
 Dadurch lässt sich das LGS in der Form $Ax=b$ schreiben.
 Betrachte nun folgendes
 \begin{gather}
  x^{T}Ax =\sum_{i,k=1}^n a_i a_k\scal(b_k,b_i)=\scal(u_b,u_b)\ge 0
 \end{gather}
 $A$ ist folglich positiv definit. Das Gleichungssystem $Ax=b$ ist also für
 jede rechte Seite $b$, das heißt für jedes $u \in V$ eindeutig lösbar.
 Folglich bestimmt die Orthogonalitätsbedingung eindeutig ein Element
 $u_b \in A$, welches dann die Bestapproximation von $u$ ist.
 
\end{proof}

\begin{Definition}{komplement-projektion}
  Sei $W \subset V$ ein Untervektorraum. Dann gilt
  $V = W \oplus W^\perp$, wobei das \define{orthogonale Komplement}
  $W^\perp$ eindeutig definiert ist durch
  \begin{gather}
    W^\perp = \bigl\{ v\in V \big| \scal(v,w) = 0 \quad\forall w\in W\bigr\}.
  \end{gather}
  Die Lösung der Bestapproximationsaufgabe bezeichnen wir mit
  \begin{gather}
    \Pi_W u = u_b\in W
  \end{gather}
  und nennen es die \define{orthogonale Projektion} von $u\in V$ auf $W$.
\end{Definition}

\begin{Lemma}{komp-projekt-wohldefiniert}
  Das orthogonale Komplement und die orthogonale Projektion sind wohldefiniert.
\end{Lemma}

\begin{proof}
  \slideref{Satz}{bestapproximation}.
\end{proof}

\begin{Beispiel}{polynom-bestapproximation}
  Die Aufgabe der Gaußschen Ausgleichsrechnung lautet: finde zu einer
  gegebenen Funktion $f$ das Polynom vom Grad höchstens $n$, das auf
  dem Intervall $[-1,1]$ den mittleren quadratischen Abstand
  minimiert, also $p\in \P_n$ mit
  \begin{gather}
    \int_{-1}^1 \bigl(f(x)-p(x)\bigr)^2 \dx
    = \min_{q\in \P_n} \int_{-1}^1 \bigl(f(x)-q(x)\bigr)^2 \dx.
  \end{gather}
  Die Lösung erfüllt
  \begin{gather}
    \int_{-1}^1 p(x)q(x) \dx = \int_{-1}^1 f(x)q(x) \dx
    \qquad\forall q\in \P_n.
  \end{gather}
\end{Beispiel}

\section{Orthogonale Basen}

\begin{Lemma}{gram-system}
  Wählt man eine Basis $\{\phi_i\}$ für $W$, so transformiert wird die
  Orthogonalitätsbedingung in \slideref{Satz}{bestapproximation} zum
  linearen Gleichungssystem
  \begin{gather}
    \matg\vx = \vb.
  \end{gather}
  Hier sind $\vx$ der Koeffizientenvektor der Lösung $u_b$, $\matg$ die
  \define{Gramsche Matrix} und $\vb$ die rechte Seite gegeben durch
\begin{gather}
  g_{ij} = \scal(\phi_i,\phi_j), \qquad
  b_i = \scal(u,\phi_i).
\end{gather}
\end{Lemma}

\begin{remark}
  Das Gleichungssystem hängt nur von der Wahl einer Basis in $W$ ab,
  nicht in $V$.
\end{remark}

\begin{Definition}{ortho-system}
  Eine Menge von Vektoren $\{\phi_1,\dots,\phi_n\}\subset V$ bildet
  ein \define{Orthogonalsystem}, wenn
  \begin{gather*}
    \scal(\phi_i,\phi_j) = 0
    \qquad \forall 1\le i < j \le n.
  \end{gather*}
  Sie ist ein \define{Orthonormalsystem}, wenn zusätzlich
  $\norm{\phi_i} = 1$ für alle Elemente gilt. Ein Orthonormalsystem, das eine Basis bildet, heißt \define{Orthonormalbasis} (\define{ONB}).
\end{Definition}

\begin{Lemma}{ortho-lu}
  Jedes Orthogonalsystem ist linear unabhängig.
\end{Lemma}

\begin{Lemma*}{parseval}{Parsevalsche Gleichung}
  Sei $\{\phi_i\}$ für $i=1,\dots,n$ eine ONB von $V$. dann gilt für
  jedes $v\in V$ mit der Basisdarstellung
  \begin{gather}
    v = \sum_{i=1}^n x_i \phi_i
  \end{gather}
  die Identität
  \begin{gather}
    \norm{v}^2 = \sum_{i=1}^n x_i^2.
  \end{gather}
\end{Lemma*}
\begin{Lemma}{least-squares-orthogonal}
  Bezüglich einer ONB ist die Gramsche Matrix die
  Einheitsmatrix. Damit berechnen sich die Einträge des
  Koeffizientenvektors $\vx$ in \slideref{Lemma}{gram-system} durch
  die einfache Formel
  \begin{gather}
    x_i = b_i = \scal(u,\phi_i).
  \end{gather}
\end{Lemma}

\begin{Theorem*}{gram-schmidt}{Gram-Schmidt-Verfahren}
  Jede linear unabhängige Menge von Vektoren
  $\{v_1,\dots,v_n\}\subset V$ wird mit dem folgenden Verfahren in ein
  Orthonormalsystem $\{\phi_1,\dots,\phi_n\}\subset V$ umgeformt:
  \begin{gather}
    \begin{aligned}
      \phi_1 &= \tfrac1{\norm{v_1}} \,v_1\\
      w_j &= v_j - \sum_{i=1}^{j-1} \scal(v_j,\phi_i)\,\phi_i
      & \quad \phi_j &= \tfrac1{\norm{w_j}}\, w_j
      &\quad j=2,\dots,n
    \end{aligned}
  \end{gather}
  Für alle $1\le k \le n$ gilt
  \begin{gather}
    \operatorname{span}\{\phi_1,\dots,\phi_k\}
    =
    \operatorname{span}\{v_1,\dots,v_k\}
  \end{gather}
\end{Theorem*}

\begin{proof}
  Per Induktion über $n$ zeigen wir Orthogonalität und Normierung.\\

  $Indukionsanfang$ Sei $n=1$.\\
  Wird nur ein Vektor aus dem Raum gewählt, so erfüllt dieser
  die Orthogonalitätsbedingung, da er der einzige Vektor im System ist.
  Wird dieser Vektor zusätzlich normiert erhält man ein Orthonormalsystem.\\
  
  $Induktionsschritt$ Das Verfahren gelte
  für $\{v_1,\dots,v_{n-1}\}$ Vektoren aus V. \\
  $n-1 \rightarrow n$\\
  Sei $(\phi_1,\dots,\phi_{n-1})$ ein Orthonormalsystem\\
  Annahme $w_n$ nicht wohldefiniert. Dann gilt
  \begin{gather}
    w_n = v_n -\sum_{i=1}^{n-1}\scal(v_n,\phi_i)\,\phi_i = 0
  \end{gather}
  In diesem Fall sind $(v_1,\dots,v_n)$ linear abhängig.
  Das ist ein Widerspruch zur Voraussetzung,
  dass $(v_1,\dots,v_n)$ linear unabhängig sind.\\
  $w_n$ wird nun normiert über $\frac{1}{\norm{w_n}} \cdot w_n =\phi_n $.
  Nun zur Orthogonalität:
  \begin{gather}
    \scal(\phi_n,\phi_j)=\scal(v_n,v_j)-
    \sum_{i=1}^{n-1}\scal(v_n,\phi_i)
    \,\underbrace{\scal(\phi_i,\phi_j)}_{=\delta_{ij}}  = 0
    \quad j=1,\dots,n-1
  \end{gather}
\end{proof}

\begin{Algorithmus*}{gram-schmidt}{Gram-Schmidt}
  \lstinputlisting{code/gram-schmidt.py}
\end{Algorithmus*}

\begin{Beispiel}{gram-schmidt}
  Wir wählen für Polynome das $L^2$-Skalarprodukt aus
  \slideref{Lemma}{l2-norm} und die Basis $\{1,x,\dots,x^{n-1}\}$
  für $\P_{n-1}$. Wir verwenden die Iplementation in
  \slideref{Algorithmus}{gram-schmidt} und messen den Erfolg nach der
  Größe der Nebendiagonaleinträge der Gramschen Matrix.
  \begin{center}
    \begin{tabular}{c|c}
      $n$ & $\max_{i\neq j} \abs{g_{ij}}$ \\
      \hline
      5 & $8.9\cdot 10^{-16}$ \\
      10 & $9.1\cdot 10^{-12}$ \\
      15 & $1.2\cdot 10^{-7}$ \\
      20 & $0.23$
    \end{tabular}
  \end{center}
\end{Beispiel}

\begin{Algorithmus*}{mgs}{Modifizierter Gram-Schmidt}
  \lstinputlisting{code/modified-gram-schmidt.py}  
\end{Algorithmus*}

\begin{Beispiel}{gs-mgs}
  In dieser Tabelle wiederholen wir die Zahlen
  $\max_{i\neq j} \abs{g_{ij}}$ aus \slideref{Beispiel}{gram-schmidt}
  und stellen sie den entsprechenden Ergebnissen des modifizierten
  Verfahrens in \slideref{Algorithmus}{mgs} gegenüber.
  \begin{center}
    \begin{tabular}{c|cc}
      $n$ &  Gram-Schmidt & modifiziert\\
      \hline
      5 & $8.9\cdot 10^{-16}$ & $1.3\cdot 10^{-16}$ \\
      10 & $9.1\cdot 10^{-12}$ & $2.9\cdot 10^{-12}$ \\
      15 & $1.2\cdot 10^{-7}$ & $2.7\cdot 10^{-9}$ \\
      20 & $0.23$ & $3.9\cdot 10^{-5}$
    \end{tabular}
  \end{center}
\end{Beispiel}

\begin{remark}
  Wir sehen, dass die Wahl der Implementation eines Rechenverfahrens
  bei mathematischer Äquivalenz durchaus erheblichen Einfluss auf das
  Ergebnis haben kann. Dieses Phänomen werden wir in
  \Cref{sec:stability} näher untersuchen. Zunächst diskutieren wir
  aber eine weitere Variante der Erzeugung orthogonaler Basen in
  Polynomräumen.
\end{remark}

\section{Drei-Term-Rekursion}

\begin{Satz*}{dreiterm}{Dreiterm-Rekursion}
  Zu jedem Skalarprodukt $\scal(\cdot,\cdot)$ auf dem Raum der
  stetigen Funktionen gibt es genau eine Folge von orthogonalen
  Polynomen $p_k\in \P_k$ mit führendem Koeffizienten eins. Sie
  genügen der Dreiterm-Rekursionsformel
  \begin{gather}
    p_k(x) = (x-a_k)p_{k-1}(x) - b_k p_{k-2}(x),
    \qquad k=1,2,\ldots
  \end{gather}
  mit Startwerten $p_{-1} \equiv 0$ und $p_0 \equiv 1$. Die
  Koeffizienten sind
  \begin{gather}
    a_k = \frac{\scal(x p_{k-1},p_{k-1})}{\scal(p_{k-1},p_{k-1})}
    \qquad\text{und}\qquad
    b_k = \frac{\scal(p_{k-1},p_{k-1})}{\scal(p_{k-2},p_{k-2})}.
  \end{gather}
\end{Satz*}

\begin{proof}
  Siehe \cite[Satz 6.2]{DeuflhardHohmann08}
\end{proof}

\begin{Bemerkung}{dreiterm-normierung}
  Der Beweis ergibt, eigentlich die ``Eindeutigkeit einer Orthogonalfolge bis auf Normierung''. Tatsächlich werden in der Literatur immer wieder veschiedene Normierungen benutzt. Beispiele sind:
  \begin{enumerate}
  \item Führender Koeffizient eins, $p_k = x^k + \dots$
  \item $\norm{p_k} = 1$
  \item $p_k(1) = 1$
  \end{enumerate}
\end{Bemerkung}

\begin{Definition}{legendre-polynome}
  Die \define{Legendre-Polynome} $L_k$ sind definiert durch
  die Dreiterm-Rekursion
  \begin{gather}
    L_{k} = \tfrac{2k-1}{k}x L_{k-1}(x) - \tfrac{k-1}{k} L_{k-2}(x).
  \end{gather}
  Sie sind orthogonal bezüglich des $L^2$-Skalarprodukts in
  \slideref{Lemma}{l2-norm}.
\end{Definition}

\begin{Beispiel}{least-squares-legendre}
  Das Problem der Gaußschen Ausgleichsrechnung war: zu einer gegebenen
  Funktion $f$ finde $p\in \P_n$, so dass
  \begin{gather}
    \int_{-1}^1 (f-p)^2 \dx
    = \min_{q\in\P_n} \int_{-1}^1 (f-q)^2 \dx.
  \end{gather}
  Mit Hilfe der Legendre-Polynome können wir nun die Lösung explizit angeben als
  \begin{gather}
    p(x) = \sum_{i=0}^n \alpha_i L_i(x)
    \qquad\text{mit}\qquad
    \alpha_i = \frac1{\norm{L_i}^2}\int_{-1}^1 f L_i(x)\dx.
  \end{gather}
\end{Beispiel}

\begin{Definition}{chebyshev-polynome}
  Die \define{Tschebyscheff-Polynome} $T_k$ sind definiert durch
  die Dreiterm-Rekursion
  \begin{gather}
    T_{k} = 2x T_{k-1}(x) - T_{k-2}(x).
  \end{gather}
  Sie sind orthogonal bezüglich des Skalarprodukts
  \begin{gather}
    \scal(p,q) = \int_{-1}^1 \tfrac1{\sqrt{1-x^2}} \,p(x)q(x)\dx.
  \end{gather}
\end{Definition}

%%% Local Variables:
%%% mode: latex
%%% TeX-master: "main"
%%% End:


\chapter{Konditionierung und Stabilität}
\label{sec:stability}
\section{Fließkommazahlen}

\begin{Definition}{fliesskomma}
  Die Darstellung einer numerischen größe als \define{Fließkommazahl}
  (auch \define{Gleitkommazahl}) beruht auf einer Basis $2 \le b \in \mathbb N$. Sie
  besteht aus einer \define{Mantisse} $\nicefrac1b \le m < 1$ und einem
  Exponenten $e$, so dass eine Zahl $x\neq 0$ die Gestalt
  \begin{gather}
    x=\pm m \cdot b^{e}
  \end{gather}
  hat. Diese Darstellung ist durch die Normierung von $m$
  eindeutig. Sowohl die Mantisse, als auch der Exponent haben einen
  endlichen Wertebereich, typischerweise eine feste Anzahl von
  Stellen. Die endliche Menge der damit darstellbaren Zahlen
  bezeichnen wir mit $\mathbb M$.
\end{Definition}

\begin{intro}
  Die folgenden drei Beispiele beschreiben Teile der Implementation
  von Fließkommazahlen nach dem IEE-Standard 754. Die Quelle ist
  jeweils Wikipedia.
\end{intro}
\begin{Beispiel}{ieee754-double}
  Im Fließkommaformat mit 64 Bit (NumPy: \texttt{float64}) nach dem Standard
  \putindex{IEEE 754}, das auf Rechnern sehr weit verbreitet ist, wird
  die Basis 2 verwendet. Es hat
  \begin{itemize}
  \item 1 Bit Vorzeichen,
  \item 11 Bit Exponent und
  \item 53 Bit Mantisse (das erste ist immer 1 und wird nicht gespeichert)
  \end{itemize}
  Der Wertebereich ist zunächst
  \begin{gather}
    \left.
      \begin{matrix}
        2^{-1022} \\ \approx 2.25 \cdot 10^{-308}
      \end{matrix}
    \right\}
    \le x \le
    \left\{
      \begin{matrix}
        2^{1023}(2-2^{-52}) \\
        \approx 1.8 \cdot 10^{308}
      \end{matrix}
    \right.
  \end{gather}
  Tatsächlich liegt das Minimum durch Verkürzung der Mantisse bei
  $2^{-1074}$. Zusätzlich gibt es Darstellungen für $\pm 0$,
  unendlich, und illegale Zahlen.
\end{Beispiel}

\begin{Beispiel}{ieee754-single}
  Das Format mit 32 Bit (NumPy: \texttt{float32}) nach \putindex{IEEE 754} hat
  \begin{itemize}
  \item 1 Bit Vorzeichen,
  \item 8 Bit Exponent und
  \item 24 Bit Mantisse.
  \end{itemize}
\end{Beispiel}

\begin{Beispiel}{ieee754-half}
  Das Format mit 16 Bit (NumPy: \texttt{float16}) nach \putindex{IEEE 754} hat
  \begin{itemize}
  \item 1 Bit Vorzeichen,
  \item 5 Bit Exponent und
  \item 11 Bit Mantisse.
  \end{itemize}
\end{Beispiel}

\begin{Definition}{runden}
  Zahlen, die durch die endliche Mantisse nicht dargestellt werden
  können, unterliegen der \define{Rundung} auf eine benachbarte
  Fließkommazahl, notiert als $\rd(x)$. Besitzt die Mantisse $r$
  Stellen zum Exponenten $b$, so ist der relative Fehler, der dabei
  entsteht beschränkt durch $b^{1-r}$ bei Rundung zur nächsten
  Fließkommazahl sogar durch $\tfrac12 b^{1-r}$. Wir bezeichnen das
  Maximum des möglichen relativen Rundungsfehlers für ein
  Fließkommaformat als \define{Maschinengenauigkeit}, abgekürzt mit
  $\eps$. Es gild also per definitionem
  \begin{gather}
    \left\lvert\frac{x-\rd(x)}{x}\right\rvert
    \le \eps.
  \end{gather}
\end{Definition}

\begin{Beispiel}{eps-ieee}
  Bei den Fließkommaformaten nach IEEE 754 gilt die Rundung zur
  nächsten darstellbaren Zahl. Sollte eine Zahl exakt zwischen zwei
  darstellbaren Zahlen liegen, so wird zur nächsten darstellbaren Zahl
  mit gerader Mantisse gerundet.

  Die Maschinengenauigkeit liegt bei
  \begin{center}
    \begin{tabular}[l]{l|ll}
      Format & \multicolumn{2}{c}{\eps}\\\hline
      \texttt{float64} & $2^{-53}$ & $\approx 1.11\cdot 10^{-16}$ \\
      \texttt{float32} & $2^{-24}$ & $\approx 5.96\cdot 10^{-8}$ \\
      \texttt{float16} & $2^{-11}$ & $\approx 4.88\cdot 10^{-4}$ \\
    \end{tabular}
  \end{center}
\end{Beispiel}

\begin{Definition}{maschinenoperationen}
  Die Implementation der Grundrechenarten für Fließkommazahlen
  beinhaltet immer eine Rundung, damit das Ergebnis darstellbar
  ist. Wir kennzeichnen diese \define{Maschinenoperationen} für
  $x,y\in \mathbb M$ mit den Symbolen
  \begin{xalignat}2
    x \oplus y &= \rd(x+y) & x \odot y &= \rd(xy)\\
    x \ominus y &= \rd(x-y) & x \oslash y &= \rd(x/y).
  \end{xalignat}
\end{Definition}

\begin{Lemma}{nichtassoziativ}
  Die Maschinenoperation $\oplus$ und $\odot$ sind weder assoziativ
  noch distributiv, wenn auch die Unterschiede nur in der Größenordnung der Maschinengenauigkeit $\eps$ liegen.
\end{Lemma}

\begin{proof}
  Die Rundung am Ende einer Operation kann so verstanden werden, dass
  jeweils ein unbekannter, relativer Fehler $\abs{\epsilon} \le \eps$
  zum Ergebnis addiert wird. Damit gilt
  \begin{gather}
    \begin{split}
      (x\oplus y) \oplus z
      &= \bigl((x+y)(1+\epsilon_1)+z\bigr)(1+\epsilon_2)\\
      &= (x+y+z + \epsilon_1x+\epsilon_1y)(1+\epsilon_2)\\
      x\oplus (y \oplus z)
      &= (x+y+z + \epsilon_3y+\epsilon_3z)(1+\epsilon_4).
    \end{split}
  \end{gather}
  Selbst wenn die Werte der $\epsilon_i$ alle etwa gleich groß sind,
  so differieren doch die Fehler, wenn $x$ und $z$ sehr verschieden
  sind. Die Rechnungen für die Multiplikation und das
  Distributivgesetz sind ähnlich.
\end{proof}

\begin{remark}
  Das vorherige Lemma gibt einen ersten Hinweis, warum die beiden
  Varianten des Gram-Schmidt-Verfahrens sich so verschieden verhalten.
\end{remark}

\begin{Beispiel*}{harmonisch}{Harmonische Reihe in Fließkommaarithmetik}
  \lstinputlisting[frame=single]{code/harmonic.py}

  Bricht das Programm ab? Warum?
\end{Beispiel*}

\begin{Aufgabe}{rundung}
  Schreiben Sie ein Programm, das bis auf 10\% Genauigkeit die
  kleinsten Zahlen $a$ und $b$ ermittelt, so dass $1.0+a=1.0$ und
  $1.9+b=1.9$. Bestimmen Sie damit $\eps$ für mindestens eines der
  IEEE 754 Fließkommaformate.
\end{Aufgabe}

\begin{Fazit}{rundung}
  \begin{enumerate}
  \item Fließkommazahlen haben endlichen Wertebereich
  \item Die Eingabe reeller Zahlen sowie die Ergebnisse von
    Rechenoperationen werden durch Rundung verfälscht.
  \item Rundungsfehler sind relative Fehler beschränkt durch die
    Maschinengenauigkeit $\eps$
  \item Grundrechenarten mit Fließkommazahlen sind nicht assoziativ
  \end{enumerate}
\end{Fazit}

\section{Konditionierung einer Rechenaufgabe}

\begin{intro}
  In diesem Abschnitt nehmen wir zunächst an, die Berechnungen seien
  exakt und nur die Eingabedaten durch Rundung verfälscht. Daraufhin
  untersuchen wir, wie stark sich die Lösung einer Rechenaufgabe
  abhängig von Variationen der Eingabedaten verändert.
\end{intro}

\subsection{Einführung der Konditionierung}

\begin{Definition}{aufgabe}
  Eine \define{numerische Aufgabe} ist die Berechnung endlich vieler
  \define{Ausgabedaten} $y_i$, $i=1,\dots,n$ aus ebenfalls endlich
  vielen Eingabedaten $x_j$, $j=1,\dots,m$. Wir schreiben
  \begin{gather}
    y_i = f_i(x_1,\dots,x_m).
  \end{gather}
  Zur Lösung der Aufgabe verwenden wir als Rechenvorschrift einen
  \define{Algorithmus}, bzw.\ seine Implementation $f$ auf einem Computer.
\end{Definition}

\begin{Definition}{datenfehler}
  Aus der Verwendung fehlerhafter Eingabedaten $x+\delta x$ ergeben
  sich fehlerhafte Resultate $y+\delta y$. Mit $\norm{\delta x}$ und
  $\norm{\delta y}$ bezeichnen wir den \textbf{absoluten
    Fehler}\index{Fehler!absolut} in der Norm, mit $\abs{\delta x_j}$
  und $\abs{\delta y_i}$ die komponentenweisen absoluten Fehler. Der
  \textbf{relative Fehler}\index{Fehler!relativ} ist
  $\norm{\delta x}/\norm{x}$, und $\norm{\delta y}/\norm{y}$ bzw.\
  $\abs{\delta x_j}/\abs{x_j}$ und $\abs{\delta y_i}/\abs{y_i}$.

  Eine numerische Aufgabe heißt \define{gut konditioniert}, wenn es
  eine moderate Konstante $\kappa$ bzw. Konstanten $\kappa_{ij}$ gibt, so dass die Abschätzung
  \begin{gather}
    \frac{\norm{\delta y}}{\norm{y}}
    \le \kappa \frac{\norm{\delta x}}{\norm{x}}
    \qquad\text{bzw.}
    \frac{\abs{\delta y_i}}{\abs{y_i}}
    \le \kappa_{ij} \frac{\abs{\delta x_j}}{\abs{x_j}}
  \end{gather}
  für den bestmöglichen Algorithmus zur Lösung der Aufgabe
  gilt. Andernfalls heißt sie \define{schlecht konditioniert}.
\end{Definition}

\begin{remark}
  Die Begriffe \glqq gut\grqq, bzw. \glqq schlecht
  konditioniert\grqq{} sind nicht scharf definiert. In der Tat hängt
  die Grenze, ab der die Konstante $\kappa$ nicht mehr als \glqq
  moderat\grqq{} angesehen wird, von außermathematischen Faktoren wie
  den Ansprüchen der Anwendung oder dem persönlichen Geschmack des
  Anwenders ab. Dennoch werden wir uns nun um eine Quantifizierung der
  Konditionierung bemühen, die bei der Entscheidung, ob eine Aufgabe
  berechenbar ist, helfen kann.
\end{remark}

\begin{remark}
  Von entscheidender Bedeutung ist, dass die Konditionierung einer
  numerischen Aufgabe das Optimum über alle Algorithmen ist und damit
  vom konkreten Algorithmus unabhängig. Die ungeschickte Wahl eines
  Verfahrens führt natürlich zu einer schlechteren Konstanten in der
  Konditionsabschätzung.
\end{remark}

\subsection{Differenzielle Fehleranalyse}

\begin{intro}
  Besonders einfach lassen sich die Relationen zwischen den Fehlern
  der Eingabe- und Ausgabedaten über Ableitungen der Funktion $f$ in
  \slideref{Definition}{aufgabe} beschreiben. Für diesen Fall stehen
  uns alle Rechenregeln wie Ketten- und Produktregel oder der Satz von
  Taylor zur Verfügung. Natürlich gelten die Aussagen dann nur
  asymptotisch für $\eps \to 0$.

  Andererseits ist $\eps$ in der Regel sehr klein, weshalb die
  asymptotische Analyse oft hinreichend genau ist. Und schließlich
  bemühen wir uns, wo immer möglich, gesicherte Scharanken einzubauen.
\end{intro}

\begin{Definition}{landau}
  Zur quantitativen Beschreibung von Grenzprozessen dienen die
  \define{Landauschen Symbole} $\bigo(\cdot)$ und
  $\smallo(\cdot)$. Für Folgen/Funktionen $f(x)$ und $g(x)$ bedeuten
  \begin{xalignat}3
    f &= \smallo(g)
    &:\Leftrightarrow&
    & \lim\limits_{x\to a} \frac{\abs{f(x)}}{\abs{g(x)}} &= 0
    \\
    f &= \bigo(g)
    &:\Leftrightarrow&
    & \operatorname*{lim sup}_{x\to a} \frac{\abs{f(x)}}{\abs{g(x)}} & < \infty.
  \end{xalignat}  
  Dabei darf $a$ eine feste Zahl oder den Limes gegen $\pm\infty$
  bezeichnen. Zusätzlich definieren wir \define{gleich in erster
    Näherung}
  \begin{xalignat}3
    f &\doteq g
    &:\Leftrightarrow&
    & f(t) = g(t) + \bigo(t).
  \end{xalignat}
\end{Definition}

\begin{Beispiel}{smallo-differential}
  Als Definition der Ableitung der Funktion $f$ im Punkt $x$ kennen wir
  \begin{gather}
    \lim\limits_{h\to 0}\frac{f(x+h)-f(x)}{h} = f'(x).
  \end{gather}
  In unserer Schreibweise
  \begin{gather}
    \begin{split}
      f(x+h)-f(x) - h f'(x) &= \smallo(h)\\
      \text{oder: }\qquad
      \frac{f(x+h)-f(x)}{h} - f'(x) &= \smallo(1)
      \qquad\text{für } h\to 0.
    \end{split}
  \end{gather}
\end{Beispiel}

\begin{Beispiel}{bigo-taylor}
  Nach dem Satz von Taylor gilt für eine zweimal stetig
  differenzierbare Funktion $f$ mit $\xi\in(x,x+h)$
  \begin{gather}
 f(x+h) = f(x) + h f'(x) + \tfrac{h^2}{2} f''(\xi)   
\end{gather}
Damit können wir schreiben
\begin{gather}
  \begin{split}
    f(x+h)-f(x) &= \bigo(h)\\
    f(x+h)-f(x) &=  h f'(x) + \bigo(h^2).
  \end{split}
\end{gather}
Oder
\begin{gather}
  \begin{split}
    \frac{f(x+h)-f(x)}h &\doteq f'(x)
  \end{split}
\end{gather}
\end{Beispiel}

\begin{Lemma}{diff-fehler}
  Sei die Funktion $f$ in \slideref{Definition}{aufgabe} stetig
  differenzierbar um das Datum $x$. Dann gilt für die relativen Fehler
  \begin{gather*}
    \frac{\delta y_i}{y_i}
    \doteq \sum_{j=1}^m \kappa_{ij}\frac{\delta x_j}{x_j}
  \end{gather*}
  mit den \define{Konditionszahlen}
  \begin{gather}
    \kappa_{ij} = \frac{\d f_i}{\d x_j}(x)
    \frac{x_j}{y_i}
  \end{gather}
\end{Lemma}

\begin{Beispiel*}{kond-mult}{Konditionierung der Multiplikation}
  Es gilt
  \begin{gather}
    y_1 = f(x_1,x_2) = x_1 x_2,
    \quad \frac{\d f}{\d x_1} = x_2,
    \quad \frac{\d f}{\d x_2} = x_1.
  \end{gather}
  Damit folgt
  \begin{gather}
    \kappa_{11} = \kappa_{12} = 1,
  \end{gather}
  die Multiplikation ist also gut konditioniert, da die relativen
  Fehler der Ausgabedaten gleich denen der Eingabedaten sind.
\end{Beispiel*}

\begin{Beispiel*}{kond-add}{Konditionierung der Addition}
  Es gilt
  \begin{gather}
    y_1 = f(x_1,x_2) = x_1 + x_2,
    \quad \frac{\d f}{\d x_1} = 1,
    \quad \frac{\d f}{\d x_2} = 1.
  \end{gather}
  Damit folgt
  \begin{gather}
    \kappa_{11} = \frac{1}{1+\frac{x_2}{x_1}},
    \qquad\kappa_{12} = \frac{1}{1+\frac{x_1}{x_2}}.
  \end{gather}
  Für den Fall $x_1 \approx -x_2$ ist die Addition also schlecht konditioniert.
\end{Beispiel*}

\begin{Bemerkung}{ausloeschung}
  Man nennt die schlechte Konditionierung der Subtraktion fast
  gleicher Zahlen auch anschaulich \define{Auslöschung}, was wir an
  folgendem Beispiel erklären:
  \begin{center}
    \begin{tabular}{r@{}l}
      0.1234569&\\
      -0.1234567&\\\hline
      0.0000002&=0.2 $\cdot 10^{-6}$.
    \end{tabular}
  \end{center}
  Bei der Subtraktion zweier Zahlen mit 7-stelliger Mantisse haben
  sich 6 Stellen ausgelöscht und es bleibt nur eine einzige
  signifikante Stelle.
\end{Bemerkung}

\section{Stabilität eines Algorithmus}

%%% Local Variables:
%%% mode: latex
%%% TeX-master: "main"
%%% End:


\chapter{Interpolation und Quadratur}

\begin{intro}
  Ziel dieses Kapitels ist die Herleitung von Methoden zur
  Approximation des Integrals einer Funktion über ein Intervall
  $[a,b]$. Diese Aufgabe wird in zwei Teile geteilt:
  \begin{enumerate}
  \item Wir unterteilen das Intervall in Subintervalle und summieren
    die Teilintegrale
    \begin{gather}
      \int_a^b f \dx = \sum_{i=1}^n \int_{x_{i-1}}^{x_i} f \dx,
      \qquad
      a = x_0 < x_1 < \dots < x_n = b.
    \end{gather}
  \item Auf jedem Teilintervall finden wir Approximationen für das
    lokale Integral.
  \end{enumerate}
  Da wir Polynome exakt integrieren können, nutzen wir wieder die
  Approximation von Funktionen durch Polynome, um uns diesem Problem
  zu nähern.
\end{intro}

\section{Polynominterpolation}
\section{Polynominterpolation}

\subsection{Definition und Konditionsabschätzung}

\begin{Definition}{lagrange-interpolation}
  Die \define{Interpolation}saufgabe nach Lagrange lautet: seien $n+1$
  paarweise verschiedene \define{Stützstellen} $x_0,\dots,x_n$ mit
  zugehörigen Funktionswerten $f_i$ gegeben. Finde ein Polynom
  $p\in \P_n$ mit der Eigenschaft
  \begin{gather}
    p(x_i) = f_i.
  \end{gather}
  Alternativ ist die Interpolationsaufgabe aufzufassen als eine Abbildung
  \begin{gather}
    \begin{split}
      I_n\colon C[a,b] &\to \P_n\\
      p(x_i) &= f(x_i),
    \end{split}
  \end{gather}
  wobei das Interval $[a,b]$ alle Stützpunkte enthält. 
  Wir nennen diese Abbildung den
  \define{Lagrange-Interpolationsoperator} oder kurz
  \define{Lagrange-Interpolation}.
\end{Definition}

\begin{Satz}{lagrange-interpolation}
  Die Interpolationsaufgabe nach Lagrange hat eine eindeutige Lösung,
  bezeichnet als (Lagrange-)\define{Interpolierende} der Funktion $f$
  \begin{gather}
    p(x;f;x_0,\dots,x_n)
  \end{gather}
\end{Satz}
\begin{proof}
  Der Beweis ist eine direkte Konsequenz des folgenden Lemmas.
\end{proof}

\begin{Lemma}{lagrange-basis}
  Seien die Punkte $x_0,\dots,x_n$ paarweise verschieden. Dann gilt
  für die \define{Lagrange-Polynome}
  \begin{gather}
    L_i(x) = L_{i;n}(x) = L_{i}(x;x_0,\dots,x_n)
    = \prod_{\substack{j=0\\j\neq i}}^n \frac{x-x_j}{x_i-x_j}
  \end{gather}
  die Eigenschaft
  \begin{gather}
    L_i(x_j) = \delta_{ij},\qquad 0 \le i,j \le n.
  \end{gather}
  Die Lagrange-Polynome sind \putindex{orthonormal} bezüglich des
  Skalarprodukts
  \begin{gather}
    \scal(p,q) = \sum_{i=0}^n p(x_i)q(x_i).
  \end{gather}
  Daher sind sie linear unabhängig und formen eine Basis von $\P_n$.
\end{Lemma}

\begin{Korollar}{lagrange-interpolation}
  Die Lagrange-Interpolation eingeschränkt auf den Raum $\P_n$ ist die
  Identität
\end{Korollar}
\begin{remark}
  Die Lagrangesche Interpolationsaufgabe kann auch als Gaußsche
  Ausgleichsrechnung mit dem obigen Skalarprodukt aufgefasst werden.
\end{remark}

\begin{Lemma}{linear-bounded}
  Sei $f\colon X \to Y$ eine lineare Abbildung zwischen Vektorräumen
  $X$ und $Y$. Dann sind folgende Aussagen äquivalent:
  \begin{enumerate}
  \item In einem beliebigen Punkt $x\in X$ gilt für das gestörte Problem
    $y+\delta y = f(x+\delta x)$ die Abschätzung
    \begin{gather}
      \norm{\delta y} \le \kappa^{\text{abs}} \norm{\delta x}
      \qquad\forall \delta x \in X.
    \end{gather}
  \item Für $y = f(x)$ gilt die Abschätzung
    \begin{gather}
      \norm{y} \le \kappa^{\text{abs}} \norm{x}
      \qquad\forall x \in X.
    \end{gather}
  \end{enumerate}
\end{Lemma}

\begin{remark}
  Es genügt also, die Konditionierung um die null zu untersuchen, was
  die Analyse vereinfacht.

  Nun gilt für eine lineare Abbildung $f(0) = 0$. In diesem Falle ist
  also die Konditionszahl für den relativen Fehler aus
  \slideref{Definition}{datenfehler}
  bzw. \slideref{Lemma}{diff-fehler} nicht sinnvoll definiert. Wir
  benutzen daher die Konditionszahlen für den absoluten Fehler. 
\end{remark}

\begin{Satz*}{lagrange-kondition}{Konditionszahl der Lagrange-Interpolation}
  Die Konditionszahl des absoluten Fehlers in der Supremumsorm der
  Lagrange-Interpolation zu den Punkten $a = x_0 < \dots < x_n = b$
  ist die \define{Lebesgue-Konstante}
  \begin{gather}
    \Lambda_{x_0,\dots,x_n} = \max_{x\in [a,b]}
    \sum_{i=0}^n \abs{L_i(x;x_0,\dots,x_n)}.
  \end{gather}
  Es gilt also
  \begin{gather}
    \max _{x\in[a,b]} \abs{I_n f(x)}
    \le \Lambda_{x_0,\dots,x_n} \max _{x\in[a,b]} \abs{f(x)}.
  \end{gather}
  Diese Abschätzung ist scharf.
\end{Satz*}

\begin{proof}
  Siehe \cite[Satz 7.3]{DeuflhardHohmann08}.
\end{proof}

\begin{Beispiel}{lagrange-kondition-equi}
  Für äquidistante Stützstellen erhält man exemplarisch die Konditionszahlen in der zweiten Spalte. Später entwickeln wir einen optimalen Satz von Stützstellen. Die Konditionszahlen dazu sind in der rechten Spalte.
  \begin{center}
    \begin{tabular}{r|rr}
      & \multicolumn{2}{c}{ $\Lambda_{0,\dots,n}$}\\
      $n$ & äquidistant & optimal\\\hline
      5 & 3.1 & 2.1\\
      10 & 30 & 2.5 \\
      15 & 512 & 2.7 \\
      20 & 10986 & 2.9
    \end{tabular}
  \end{center}
  Quelle: \cite{DeuflhardHohmann08}
\end{Beispiel}

\subsection{Rekursive Interpolation}

\begin{Lemma*}{Aitken}{Aitken}
  Für das Interpolationspolynom
  \begin{gather}
    p_{0,\dots,n}(x) = p(x;f;x_0,\dots,x_n)
  \end{gather}
  zu paarweise verschiedenen Stützstellen $x_0,\dots,x_n$ gilt die
  Rekursionsformel
  \begin{gather}
    p_{0,\dots,n}(x)
    = \frac{(x-x_0) p_{1,\dots,n}(x) - (x-x_n) p_{0,\dots,n-1}(x)}{x_n-x_0}.
  \end{gather}
\end{Lemma*}

\begin{proof}
  Der Beweis benutzt wieder Induktion. Für eine einzige Stützstelle ist das Interpolationspolynom konstant, $p_i(x) = f_i$ und daher $p_i\in P_0$.
  Sei nun $\phi(x)$ der Bruch auf der rechten Seite. Durch Induktion sehen wir sofort, dass $\phi\in \P_n$. Ferner gilt für $i=1,\dots,n-1$
  \begin{gather}
    \begin{split}
      \phi(x_i)
      &= \frac{(x_i-x_0) p_{1,\dots,n}(x_i) - (x_i-x_n)p_{0,\dots,n-1}(x_i)}{x_n-x_0}\\
      &= \frac{(x_i-x_0) f_i - (x_i-x_n) f_i}{x_n-x_0}\\
      &= f_i.
    \end{split}
  \end{gather}
  Ebenso verschwindet für $x_0$ und $x_n$ je ein Term und es gilt
  dieselbe Aussage.
\end{proof}

\begin{Algorithmus*}{Neville}{Neville}
  Sei für eine Stelle $x$ an der das Interpolationspolynom berechnet
  werden soll $p_{ik} = p_{i-k,\dots,i}(x)$ für $i\ge k$. Dann lässt
  sich $p_{0,\dots,n}(x) = p_{nn}$ rekursiv berechnen durch
  \begin{enumerate}
  \item Für $k=0$ setze
    \begin{gather}
      p_{i0} = f_i \qquad i=0,\dots,n.
    \end{gather}
  \item Für $k=1,\dots,n$ berechne
    \begin{gather}
      p_{ik} = p_{i,k-1} + \frac{x-x_i}{x_i-x_{i-k}}
      \bigl( p_{i,k-1} - p_{i-1,k-1} \bigr)
      \qquad i=k,\dots,n.
    \end{gather}
  \end{enumerate}
\end{Algorithmus*}

\begin{Definition}{newton-basis}
  Als \define{Newton-Basis} der Lagrange-Interpolation bezeichnen wir
  die Polynome
  \begin{gather}
    \omega_i(x)
    = \omega_{0,\dots,i}(x)
    = \prod_{j=0}^{i-1} (x-x_j),
    \qquad i=0,\dots,n
  \end{gather}
  wobei die leere Summe für $i=0$ den Wert 1 annehme.
\end{Definition}

\begin{Lemma}{newton-basis}
  Sei $Q_k\in \P_n$ ein Polynom dargestellt bezüglich der Newton-Basis
  durch
  \begin{gather}
    Q_k(x) = \sum_{i=0}^k a_i \omega_i(x),\qquad k=0,\dots,n.
  \end{gather}
  Dann gilt
  \begin{align}
    Q_k(x) &= Q_{k-1}(x) + a_k \omega_k(x),
  \end{align}
  und $a_k$ ist der Koeffizient vor $x^k$ in der Monomdarstellung von
  $Q_k(x)$.
\end{Lemma}

\begin{Definition}{div-diff-1}
  Als \define{dividierte Differenzen} zur
  Lagrange-Interpolationsaufgabe bezeichnen wir die rekursiv
  definierten Werte
  \begin{align}
    [x_i]f
    &= f_i \\
    [x_i,\dots,x_{i+k}]f
    &= \frac{[x_{i+1},\dots,x_{i+k}]f - [x_i,\dots,x_{i+k-1}]f}{x_{i+k}-x_i}
  \end{align}
\end{Definition}

\begin{Satz}{newton-lagrange}
  Für das Lagrange-Interpolationspolynom $p_{i,\dots,i+k}(x)$ zu den
  paarweise verschiedenen Stützpunkten $x_i,\dots,x_{i+k}$ gilt
  \begin{gather}
    p_{i,\dots,i+k}(x)
    = \sum_{j=i}^{i+k} [x_i,\dots,x_{i+k}]f \frac{\omega_j(x)}{\omega_i(x)}.
  \end{gather}
\end{Satz}

\begin{remark}
  Der Bruch im vorherigen Satz ist nicht problematisch, da
  \begin{gather}
    \frac{\omega_j(x)}{\omega_i(x)} = \prod_{\ell=i}^{j-1} (x-x_\ell).
  \end{gather}
\end{remark}

\begin{Satz}{Lagrange-restglied}
  Sei $f \in C^{n+1}[a,b]$ und $p\in \P_n$ die die
  Lagrange-Interpolierende zu den Stützstellen
  $a=x_0,\dots,x_n=b$. Dann gibt es zu jedem $x\in \R$ einen Punkt
  $\xi$ im kleinsten Intervall $I$, das die Punkte $x$, $a$ und $b$
  enthält, so dass
  \begin{gather}
    f(x)- p(x) = \frac{f^{(n+1)}(\xi)}{(n+1)!} \omega_{0,\dots,n}(x).
  \end{gather}
\end{Satz}

\begin{proof}
  Der Beweis folgt \cite{Satz 2.1.4.1}[Stoer83].  Zunächst bemerken
  wir, dass für alle Stützstellen $x_i$ gilt, dass
  $f(x_i) - p(x_i) = 0$. Dort ist also nichts zu beweisen.
  Sei nun
  \begin{gather}
    \label{eq:interpolation:1}
    F(y) = f(y)-p(y) - \alpha \omega_n(y)
  \end{gather}
  und $\alpha$ soll so gewählt werden, dass $F(x) = 0$. Damit hat $F(y)$ im
  Intervall $I$ insgesamt die $n+2$ Nullstellen $x,x_0,\dots,x_n$.
  Wiederholte Anwendung des Satzes von Rolle ergibt, dass $F'(y)$
  insgesamt $n+1$ Nullstellen hat und das $F^{(n+1)}(y)$ eine
  Nullstelle $\xi$ besitzt. Da $p\in \P_n$ gilt
  \begin{gather}
    0 = F^{(n+1)}(\xi) = f^{(n+1)}(\xi) - \alpha (n+1)!
  \end{gather}
  und damit
  \begin{gather}
    \alpha = \frac{f^{(n+1)}(\xi)}{(n+1)!}.
  \end{gather}
\end{proof}

\begin{Korollar}{Lagrange-restglied}
  Sei $f \in C^{n+1}[a,b]$ und alle Stützstellen $x_i$ im Intervall
  $[a,b]$. Dann gibt es $\xi\in[a,b]$, so dass
  \begin{gather}
    [x_0,\dots,x_n]f = \frac{f^{(n)}}{n!}(\xi).
  \end{gather}
\end{Korollar}

\begin{proof}
  Formel~\eqref{eq:interpolation:1} gibt gerade an, dass $\alpha$ der
  Koeffizient vor dem nächsten Newton-Basispolynom ist, wenn man den
  Punkt $x$ der Menge der Stützstellen hinzufügt.
\end{proof}

\subsection{Hermite-Interpolation und dividierte Differenzen}

\begin{Definition}{hermite-interpolation}
  Die \define{Hermite-Interpolation} benutzt neben Funktionswerten
  auch Ableitungswerte zur Interpolation. Das Interpolationspolynom
  $p\in \P_n$ genügt in $m$ paarweise verschiedenen Punkten den
  Bedingungen
  \begin{gather}
    \frac{d^j p}{dx^j}(x_i) = f_i^{j},
    \qquad i = 0,\dots, m, \quad j=0,\dots,n_i,
  \end{gather}
  und es gilt
  \begin{gather}
    \sum_{i} n_i = n+1.
  \end{gather}
  Die definierenden Funktionale\footnote{Als Funktional bezeichnet man
    eine Abbildung aus einem Vektorraum in den zugehörigen Körper}
  $\nicefrac{d^j}{dx^j} p(x_i)$ werden auch als \define{Knotenwerte}
  oder \define{Knotenfunktionale} bezeichnet.
\end{Definition}

\begin{Satz}{hermite-interpolation}
  \slideref{Definition}{hermite-interpolation} bestimmt das
  Interpolationspolynom eindeutig.
\end{Satz}

\begin{proof}
  Analog zur Lagrange-Interpolation identifizieren wir wieder eine
  Basis $\{H_{ij}(x)\}$, diesmal doppelt indiziert, die bezüglich der
  Interpolationsbedingungen orthogonal ist. Damit stellen wir das
  Interpolationspolynom dar als
  \begin{gather}
    p(x) = \sum_{i=1}^m \sum_{j=0}^{n_i-1} f_i^j H_{ij}(x).
  \end{gather}
  Zunächst führen wir die Hilfspolynome
  \begin{gather}
    q_{ij}(x) = \frac{(x-x_i)^j}{j!}\prod_{k\neq i}
    \left(\frac{x-x_k}{x_i-x_k}\right)^{n_i}
  \end{gather}
  ein. Es gilt
  \begin{gather}
    \begin{aligned}
      \frac{d^j q_i}{d x^j} (x_k) &=0,
      \quad &k\neq i,&\quad& j&=0,\dots,n_{k}-1,\\
      \frac{d^j q_i}{d x^j} (x_i) &=0,
      \qquad &&& j&=0,\dots,n_{i}-1,\\   
      \frac{d^{n_{i}-1} q_i}{d x^{n_{i}-1}} (x_i) &=1.
      \qquad &&&& 
    \end{aligned}
  \end{gather}
  Damit können wir rekursiv definieren
  \begin{gather}
    \begin{aligned}
      H_{i,n_i-1}(x) &= q_{i,n_i-1}(x)
      & i&= 1,\dots,m\\
      H_{ij}(x) &= q_{ij}(x) - \sum_{k=j+1}^{n_i-1} q_{ij}^{(k)}(x_i) q_{ik}(x),
    \end{aligned}
  \end{gather}
  wobei die letzte Zeile die Anwendung des Gram-Schmidt-Verfahrens
  ist. Per constructionem gilt für diese Basis
  \begin{gather}
    \frac{d^\ell}{dx^\ell} H_{ij}(x_k) = \delta_{ik}\delta_{j\ell}.
  \end{gather}
\end{proof}

\begin{Notation}{interpolation-ascending}
  Bei der Polynominterpolation ist die Anordnung der
  Interpolationspunkte beliebig. Das ist auch weiterhin der Fall. Für
  die Darstellung der Resultate und Beweise ist es aber oft hilfreich
  anzunehmen, dass sie in aufsteigender Folge angeordnet sind. Wir
  nehmen daher ab jetzt an, dass
  \begin{gather}
    a = x_0 \le x_1 \le \dots \le x_n = b.
  \end{gather}
  Dabei sollen $k$-fach wiederholte Stützstellen bedeuten, dass dort
  nicht nur der der Funktionswert, sondern auch die ersten $k-1$
  Ableitungen interpoliert werden. Damit haben wir für die
  Interpolation in $\P_n$ immer eine Folge von $n+1$ Stützstellen.
\end{Notation}

\begin{Beispiel}{taylor-polynom}
  Sind alle Stützstellen $x_0 = \dots = x_n$ identisch, so erhalten
  wir duch Interpolation einer Funktion $f\in C^n[a,b]$ das
  Taylor-Polynom vom Grad $n$
  \begin{gather}
    p(x;f;x_0,\dots,x_n) = \sum_{k=0}^n \frac{(x-x_0)^k}{k!} f^{(k)}(x_0).
  \end{gather}
\end{Beispiel}

\begin{Beispiel}{hermite-kubisch}
  Die kubische Hermite-Interpolation auf dem Intervall $[a,b]$ ist
  definiert durch die Knotenwerte
  \begin{gather}
    p(a), p'(a), p(b), p'(b).
  \end{gather}
\end{Beispiel}



%%% Local Variables:
%%% mode: latex
%%% TeX-master: "main"
%%% End:


\section{Interpolation mit Splines}
\section{Interpolation mit Splines}

\subsection{Interpolation auf Teilintervallen}

\begin{Notation}{indices}
  $\P_k$, $n$ Intervalle indiziert mit $i$
\end{Notation}

\begin{Lemma*}{scaling-interpolation}{Skalierungsargument}
  Seit $\hat I$ ein fest gewähltes Intervall mit
  Interpolations-Stützpunkten $\hat x_0,\dots,\hat x_n$. Sei das
  Intervall $[a,b]$ unterteilt in Teilintervalle
  $I_k = \Phi_k(\hat I)$ der Länge $h_k \le h$. Eine Funktion
  $f\in C^{n+1[a,b]}$ werde in den Stützpunkten
  $x_{ki} = \Phi_k(\hat x_i)$ durch eine Funktion $v\in S^{n,0}$
  interpoliert.  Dann gibt es eine Konstante $C$, die nur von den
  Punkten $\hat x_0,\dots,\hat x_n$ abhängt, so dass der
  Interpolationsfehler beschränkt ist durch
  \begin{gather}
    \norm{f-v}_{\infty;[a,b]}
    \le C \frac{h^{n+1}}{(n+1)!} \norm{f^{(n+1)}}_{\infty;[a,b]}.
  \end{gather}
\end{Lemma*}


\subsection{Splines}
%%% Local Variables:
%%% mode: latex
%%% TeX-master: "main"
%%% End:


%\section{Trigonometrische Interpolation}
%\begin{Satz}{trigonometrische-interpolation}
  Zu gegebenen Zahlen $f_0,\dots f_n\in \C$ gibt es genau eine
  Funktion der Gestalt
  \begin{gather}
    t_n(x) = \sum_{k=0}^n c_k e^{ikx}.
  \end{gather}
  die den Interpolationsbedingungen
  \begin{gather}
    t_n(x_j) = = f_j, \quad x_j = \frac{2\pi k}{n+1}, j=0,\dots,n
  \end{gather}
  genügt. Ihre Koeffizienten sind
  \begin{gather}
    c_k = \frac{1}{n+1} \sum_{j=0}^n f_j e^{-ijx_k}.
  \end{gather}
\end{Satz}



%%% Local Variables:
%%% mode: latex
%%% TeX-master: "main"
%%% End:


\section{Interpolatorische Quadratur}
\subsection{Summierte Quadratur}

\begin{Definition}{quadratur}
  Eine \define{Quadraturformel} $Q_{[a,b]}(f)$ ist eine Approximation
  des Integrals
  \begin{gather}
    Q_{[a,b]}(f) \approx \int_a^b f(x)\dx
  \end{gather}
  in der Form
  \begin{gather}
    Q_{[a,b]}(f) = \sum_{i=0}^n \omega_i f(x_i).
  \end{gather}
  Die Stützstellen $x_i$ bezeichnen wir auch als
  \define{Quadraturpunkte}, die Zahlen $\omega_i$ als
  \define{Quadraturgewichte}.
\end{Definition}

\begin{Definition}{quadratur-summiert}
  Ist eine Quadraturormel bezüglich einer Zerlegung
  $\mathcal I_h$ des Intervalls $[a,b]$ in der Form
  \begin{gather}
    Q_{[a,b]}(f) = \sum_{i=1}^n Q_{I_i} (f)
  \end{gather}
  definiert, so sprechen wir von \textbf{summierter},
  \textbf{iterierter} oder \textbf{stückweiser Quadratur}.
\end{Definition}

\begin{Satz}{quadratur-kondition}
  Die Integration einer Funktion $f\in C[a,b]$ über das Intervall $[a,b]$ genügt der Konditionsabschätzung
  \begin{gather}
    \left|\int_a^b f(x)\dx \right| \le \kappa_{\text{abs}} \max_{x\in[a,b]}\abs{f(x)},
    \qquad \kappa_{\text{abs}} = b-a.
  \end{gather}
  Für die Quadraturaufgabe gilt
  \begin{gather}
    \left| Q_{[a,b]}(f)\right| \le \kappa_{\text{abs}} \max_{x\in[a,b]}\abs{f(x)},
    \qquad \kappa_{\text{abs}} = \sum_i \abs{\omega_i}.
  \end{gather}
\end{Satz}

\begin{proof}
  Für die Integration ist aus der Analysis die Abschätzung
  \begin{gather}
    \left|\int_a^b f(x)\dx\right| \le (b-a) \max_{x\in[a,b]}\abs{f(x)}
  \end{gather}
  bekannt.
  Für die Quadratur gilt genauso
  \begin{gather}
    \left|\sum_{i=1}^n \omega_i f(x_i)\right|
    \le \left(\sum_{i=1}^n \abs{\omega_i}\right) \max_{x\in[a,b]}\abs{f(x)}.
  \end{gather}
  Wählen wir eine Funktion mit
  $f(x_i) = \operatorname{sign} \omega_i$, so sehen wir, dass die
  zweite Abschätzung scharf ist. Für die erste ist das offensichtlich,
  wenn $f$ konstant ist.
\end{proof}

\begin{Definition}{quadratur-lokale-fehlerordnung}
  Gilt bei einer summierten Quadraturformel die Abschätzung
  \begin{gather}
    \left|\int_{I_i} f(x)\dx - Q_{I_i}(f)\right|
    =\bigo\left(h_i^{k+1}\right)
  \end{gather}
  für jedes Teilintervall $I_i$ und Funktionen $f\in C^{k+1}[a,b]$, so
  sprechen wir von der \textbf{lokalen
    Fehlerordnung}\defindex{Fehlerordnung}\defindex{lokale Fehlerordnung} $k+1$.
\end{Definition}

\begin{Satz}{summierte-quadratur}
  Sei $\mathcal I_h$ eine Zerlegung von $[a,b]$ der Feinheit $h$ und
  $c_q$ sei so gewählt, dass
  \begin{gather}
    c_q \min_{I_i\in \mathcal I_h} h_i \ge h.
  \end{gather}
  Sind dann die Formeln $Q_{I_i}$ von lokaler Fehlerordnung $k+1$ für
  $f\in C^{k+1}[a,b]$, so gilt für die summierte Quadratur $Q_{[a,b]}$
  die Abschätzung
  \begin{gather}
    \left|\int_a^b f(x)\dx - Q_{[a,b]}(f)\right|
    = \mathcal O\left(h^{k}\right).
  \end{gather}
\end{Satz}

\begin{proof}
  Das kleinste Intervall hat die Länge $h/c_q$. Damit ist die Anzahl
  der Intervalle beschränkt durch $n_{\max}=c_q (b-a)/h$. Aus der
  lokalen Fehlerordnung ergibt sich die Existenz einer Konstanten $c$,
  so dass
  \begin{gather}
    \left|\int_{I_i} f(x)\dx - Q_{I_i}(f)\right| \le c h_i^{k+1}.
  \end{gather}
  Damit schätzen wir ab
  \begin{align}
    \left|\int_a^b f(x)\dx - Q_{[a,b]}(f)\right|
    &= \sum_{I_i\in\mathcal I_h}  \left|\int_{I_i} f(x)\dx - Q_{I_i}(f)\right|\\
    &\le \sum_{I_i\in\mathcal I_h} c h^{k+1}\\
    & \le n_{\max} c h^{k+1} = \bigo\left(h^{k}\right).
  \end{align}
\end{proof}

\subsection{Quadratur auf Einzelintervallen}

\begin{Notation}{quadrature}
  In diesem Abschnitt integrieren wir wieder über das Intervall
  $I=[a,b]$, aber mit dem Gedanken, dass es sich eigentlich um die
  Teilintervalle $I_i$ einer summierten Quadratur handelt.

  Wir betrachten in der Regel Quadraturformeln mit $n$ Punkten
  $x_1,\dots,x_n$. Oft benutzen wir Ergebnisse aus den Abschnitten
  über Interpolation. Dabei ist jeweis darauf zu achten, dass die
  Indizes dort bei null loslaufen. Der Grund für diesen Wechsel ist,
  dass wir bei der Interpolation den Grad der Polynome als führende
  Größe angesehen haben, während hier die Anzahl der Quadraturpunkte
  im Vordergrund steht.

  Bei der summierten Quadratur steht die Anzahl der Intervalle im
  Vordergrund. Deswegen werden dort die Punkte weiterhin mit null
  beginnend numeriert.
\end{Notation}

\begin{Definition}{grad-exaktheit}
  Eine Quadraturformel $Q_I$ heißt \define{exakt vom Grad $k$} und $k$
  heißt der \define{Grad der Exaktheit} von $Q_I$, wenn sie exakt für
  alle Polynome vom Grad bis zu $k$ ist, also
  \begin{gather}
    \int_I p(x)\dx - Q_{I}(p) = 0 \qquad \forall p\in \P_k.
  \end{gather}
\end{Definition}

\begin{remark}
  Als unmittelbare Folgerung erhalten wir für jede Quadraturformel,
  die mindestens exakt vom Grad null ist, dass
  \begin{gather}
    \sum \omega_i = b-a.
  \end{gather}
  Insbesondere gilt dann, dass die Konditionierung der Quadratur
  gleich der der Integration ist, wenn alle Gewichte positiv
  sind. Umgekehrt führen negative Gewichte automatisch zur
  Verschlechterung der Konditionierung, weswegen solche Formeln
  vermieden werden.
\end{remark}

\begin{Lemma}{exakt-ordnung}
  Sei die Quadraturformel $Q_I$ exakt vom Grad $k$ und
  $\abs{I} \le h$. Dann gilt für $f\in C^{k+1}(I)$
  \begin{gather}
    \left|\int_{I} f(x)\dx - Q_{I}(f)\right| = \bigo\bigl(h^{k+2}\bigr)
  \end{gather}
\end{Lemma}

\begin{proof}
  Ersetzen wir $f$ durch sein Tailorpolynom $p\in \P_k$ um einen Punkt $x_0\in I$, so erhalten wir
  \begin{gather}
    f(x) = p(x) + r(x),
    \qquad r(x) = (x-x_0)^{k+1}\frac{f^{(k+1)}(\xi)}{(k+1)!}
  \end{gather}
  für einen Punkt $\xi$ zwischen $x_0$ und $x$. Aufgrund der Exaktheit gilt
  \begin{gather}
    \int_{I} f(x)\dx - Q_{I}(f)
    = \underbrace{\int_{I} p(x)\dx - Q_{I}(p)}_{=0}
    + \int_{I} r(x) \dx - Q_{I}(r).
  \end{gather}
  Für den Rest schätzen wir ab:
  \begin{gather}
    (x-x_0) \le h,\quad \int_{I}(x-x_0)^{k+1} = \frac1{k+2} (x-x_0)^{k+2},
    \qquad \sum_{\omega_i} = h,
  \end{gather}
  und die Ableitung von $f$ durch ihr Maximum auf $I$.
\end{proof}

\begin{remark}
  Die Überlegungen des vorhergehenden Lemmas sind insbesondere dann
  nützlich, wenn man die Konvergenzordnung einer Methode schnell
  überschlagen möchte. Man spart sich auf diese Art genauere
  Untersuchungen der Fehlerdarstellung. Umgekehrt sind die Konstanten
  in den Abschätzungen auch scharf und die Analysis erhebt auch nicht
  den Anspruch. Im Detail werden wir daher auch noch schärfere
  Abschätzungen machen.
\end{remark}

\begin{Definition}{interpolatorische-quadratur}
  Eine \define{interpolatorische Quadraturformel} mit $n$
  Quadraturpunkten $x_1,\dots,x_n$ approximiert das Integral einer
  Funktion $f$ durch das exakte Integral ihres Interpolationspolynoms
  $p\in \P_{n-1}$
\end{Definition}

\begin{Lemma}{interpolatorisch-omega}
  Seien $x_1,\dots,x_n$ die Quadraturpunkte einer interpolatorischen
  Quadraturformel $Q_I$, die exakt für Polynome vom Grad $n-1$
  ist. Dann sind die Gewichte gegeben durch
  \begin{gather}
    \omega_i = \int_I \plagrange_{i;x_1,\dots,x_n}(x)\dx,
  \end{gather}
  wobei $\plagrange_{i;x_1,\dots,x_n}$ das
  Lagrange-Interpolationspolynom zum Punkt $x_i$ ist.
\end{Lemma}

\begin{proof}
  Die Lagrange-Polynome $\plagrange_i$ sind Polynome vom Grad
  $n-1$. Es gilt daher
  \begin{gather}
    \int_I \plagrange_i = \sum_{k=1}^n \omega_k \plagrange_i(x_k) = \omega_i.
  \end{gather}
\end{proof}

\begin{Definition}{newton-cotes}
  Werden die Quadraturpunkte $a=x_1,\dots,x_n=b$ gleichmäßig im
  Intervall $[a,b]$ verteit, so spricht man von einer
  \define{Newton-Cotes-Formel}. Die
  ersten drei klassischen Formeln sind auf dem Einheitsintervall
  $[0,1]$ gegeben durch
  \begin{center}
    \begin{tabular}{l|c|cccc|cccc}
      & $n$ & \multicolumn{4}{|c}{$x_i$} & \multicolumn{4}{|c}{$\omega_i$}
      \\\hline
      Trapezregel & 2 & 0 & 1 &&& \nicefrac12 & \nicefrac12\\
      Simpson-Regel\footnote{Auch Keplersche Fassregel} & 3 & 0 & \nicefrac12 & 1 &
                          & \nicefrac16& \nicefrac46& \nicefrac16\\
      \nicefrac38-Regel\footnote{Von Newton auch mit dem Adjektiv \glqq pulcherrima\grqq{} belegt.}
      & 4 & 0 & \nicefrac13 & \nicefrac23 & 1
                          & \nicefrac18& \nicefrac38& \nicefrac38& \nicefrac18
    \end{tabular}
  \end{center}
\end{Definition}

\begin{remark}
  Genauer gesagt handelt es sich bei den Formeln in
  \slideref{Definition}{newton-cotes} um \textbf{geschlossene}
  Newton-Cotes-Formeln, die die Intervallenden als Quadraturpunkte
  enthalten.  Bei offenen Formeln sind die Intervallenden keine
  Quadraturpunkte. Während offene Newton-Cotes-Formeln von geringem
  Interesse sind, werden wir später bei der Gauß-Quadratur offene
  Formeln kennenlernen.

  Summierte geschlossene Formeln werden of direkt als Summe über die
  Zerlegung geschrieben. Dazu numerieren wir nun alle Stützpunkte der
  Reihe nach, egal ob es sich um Intervallenden oder innere
  Stützpunkte handelt von 0 bis $n$.
  Sei $h$ nun der Abstand zweier Quadraturpunkte
  der summierten Formel. dann gilt für die summierte Trapezregel
  \begin{gather}
    Q(f) = h\left(\tfrac12 f_0 + f_1 +\dots+f_{n-1}+\tfrac12 f_n\right).
  \end{gather}
  Bei der summierten Simpson-Regel (man beachte die Umskalierung der Intervall-Länge) ergibt sich
  \begin{gather}
    Q(f) = \tfrac h3\left(f_0 + 4 f_1 + 2 f_2 + 4 f_3 + 2 f_4
      + \dots + 2 f_{n-2} + 4 f_{n-1} + f_{n}\right).
  \end{gather}
  Die Anzahl der Teilintervalle ist bei dieser Darstellung $n$ für die
  Trapezregel und $2n$ für die Simpson-regel.
  
  Der Aufwand für Funktionsauswertungen in den Intervallenden halbiert
  sich bei dieser Darstellung der summierten Regeln. Natürlich lassen
  sich die summierten Formeln für Intervalle wechselnder Länge
  umschreiben. Bei der Simpson-Regel muss man aber dabei Bedenken,
  dass immer 2 Subintervalle ein Intervall der Zerlegung ergeben.
\end{remark}


\begin{Satz}{newton-cotes-error}
  Die Fehler der Newton-Cotes-Formeln auf dem Intervall $I$ der Länge
  $h$ lassen sich wie folgt abschätzen
  \begin{gather}
    \left|\int_I f\dx - Q_I(f)\right| \le
    \begin{cases}
      \dfrac{h^3}{12}\max\limits_{\xi\in I}\abs{f''(\xi)}
      &\text{Trapezregel}\\
      \dfrac{h^5}{2880}\max\limits_{\xi\in I}\abs{f^{(4)}(\xi)}
      &\text{Simpson-Regel}\\
      \dfrac{h^5}{6480}\max\limits_{\xi\in I}\abs{f^{(4)}(\xi)}
      &\text{\nicefrac38-Regel}
    \end{cases}
  \end{gather}
\end{Satz}

\begin{proof}
  Der Beweis für die Trapezregel und die \nicefrac38-Regel benutzt
  Interpolation in den Quadraturpunkten und die Fehlerdarstellung des
  Interpolationsfehlers. Für die Trapezregel ist er als Hausaufgabe
  gestellt.

  Hier führen wir nur den Beweis für die Simpson-Regel. Nachdem man
  experimentell beobachtet, dass die Formel exakt vom Grad 3 ist,
  nicht vom erwarteten Grad 2, konstruieren wir eine Interpolation auf
  $I=[x_1,x_3]$ mit Mittelpunkt $x_2$ wie folgt:
  \begin{xalignat}2
    p(x_1) &= f(x_1) & p(x_2) &= f(x_2) \\
    p(x_3) &= f(x_3) & p'(x_2) &= f'(x_2).
  \end{xalignat}
  Die letzte Bedingung ist aus der Quadraturformel nicht
  ersichtlich. Folgen wir jedoch der Basiskonstruktion im
  \slideref{Satz}{hermite-interpolation} über die Wohlgestelltheit der
  Hermite-Interpolationsaufgabe, so erhalten wir
  \begin{xalignat}2
    H_{10}(x) &= \tfrac{4(x-x_2)^2(x_3-x)}{h^3}&
    H_{20}(x) &= \tfrac{4(x-x_1)(x-x_3)}{h^2}\\
    H_{30}(x) &= \tfrac{4(x-x_2)^2(x-x_1)}{h^3}&
    H_{21}(x) &= \tfrac{4(x-x_2)(x-x_1)(x-x_3)}{h^2}
  \end{xalignat}

  \begin{figure}[tp]
    \centering
    \includegraphics[width=.8\textwidth]{graph/interpolation/simpson}
    \caption{Basisfunktionen für die Simpsonregel. Beachte, dass $H_21$ in allen Quadraturpunkten verschwindet und auch das Integral null ist.}
    \label{fig:simpson}
  \end{figure}
  Die Funktion $H_{21}(x)$ ist das Produkt der Parabel
  $(x-x_1)(x-x_3)$, die symmetrisch zur Intervallmitte ist mit einer
  linearen Funktion mit Nullstelle in der Intervallmitte. Daher
  verschwindet ihr Integral und das zugehörige Integrationsgewicht ist
  null. Die Simpson-Regel lässt sich also schreiben als
  \begin{gather}
    Q_I = \frac h6f(x_1)+\frac{4h}6f(x_2) + \frac h6f(x_3) + 0 f'(x_2).
  \end{gather}
  Für die obige Interpolation gilt nach
  \slideref{Satz}{Hermite-restglied} die Fehlerdarstellung
  \begin{gather}
    f(x) - p(x) = \frac{f^{4}(\xi(x)}{4!} \pnewton_{x_1,x_2,x_2,x_3}(x).
  \end{gather}
  Integration ergibt
  \begin{align}
    \left|\int_I f\dx - Q_I(f)\right|
    &= \left|\int_I \bigl(f(x) - p(x)\bigr)\dx\right|\\
    &\le \max_{\xi\in I} \frac{f^{4}(\xi}{4!}
      \int_I \pnewton_{x_1,x_2,x_2,x_3}(x) \dx.
  \end{align}
  Schließlich berechnen wir
  \begin{gather}
    \int_I \pnewton_{x_1,x_2,x_2,x_3}(x) \dx
    = \int_I (x-x_1)(x-x_2)^2(x-x_3)
    = \frac1{120}
  \end{gather}
\end{proof}

\begin{remark}
  Die Newton-Cotes-Formel mit 9 Punkten hat negative Gewichte, was
  auch bei höheren Formeln wieder auftritt. Bei solchen Formeln wird
  die Konditionierung der Quadraturaufgabe schlechter als die der
  Integration. Auch kann es sein, dass eine nirgendwo negative
  Funktion durch eine solche Quadratur ein negatives Integral erhält.
  Deswegen werden solche Formeln nicht verwendet.
\end{remark}

\subsection{Gauß-Quadratur}

\begin{Lemma}{quadratur-exakt-max}
  Sei $Q_n$ eine Quadraturformel auf einem Intervall $I$ mit $n$
  Quadraturpunkten. Dann ist $Q_n$ maximal exakt vom Grad $2n-1$.
\end{Lemma}

\begin{proof}
  Siehe auch \cite[Satz 3.1]{Rannacher17}.  Für die QuadraturPunkte $x_i$
  definieren wir das quadrierte Newton-Polynom
  \begin{gather}
    p(x) = \omega_{1,\dots,n}^2 = \prod_{i=1}^n (x-x_i)^2 \in \P_{2n}.
  \end{gather}
  Da es in allen Stützstellen verschwindet, gilt $Q{p} = 0$. Da es
  aber nichtnegativ und nicht das Nullpolynom ist, so ist sein
  Integral größer als null. Damit gibt es für jede $n$-Punkt-Formel
  ein Polynom, für das sie nicht exakt ist.
\end{proof}

\begin{Satz}{gauss-legendre-eindeutig}
  Sei in der Folge von Quadraturformeln $\{Q_n\}$ mit Quadraturpunkten
  $x_1^{(n)},\dots,x_n^{(n)}$ für $k=1,\dots$ auf dem Intervall
  $I=[-1,1]$ jede Formel $Q_n$ exakt für beliebige $p\in
  \P_{2n-1}$. Dann sind die Polynome
  \begin{gather}
    p_n(x) = \prod_{i=1}^n \left(x-x_i^{(n)}\right) \in \P_n
  \end{gather}
  und $p_0(x) = 1 \in \P_0$ paarweise orthogonal bezüglich des
  $L^2$-Skalarprodukts. Insbesondere sind sie damit Vielfache der
  Legendre-Polynome $\plegendre_n$ und die Formeln sind eindeutig
  bestimmt.
\end{Satz}

\begin{proof}
  Siehe~\cite[Lemma 9.9 und 9.10]{DeuflhardHohmann08}.
  Sei $j<n$. Dann ist $p_jp_n\in \P_{2n-1}$. Es gilt also
  \begin{gather}
    \int_{-1}^1 p_jp_n \dx = Q_n(p_jp_n)
    = \sum_{i=0}^n \omega_i p_j(x_i^{(n)}) p_n(x_i^{(n)}).
  \end{gather}
  Da die Punkte $x_i^{(n)}$ aber gerade die Nullstellen von $p_n$
  sind, muss dieser Term null sein. Wir haben damit eine Folge
  orthogonaler Polynome steigenden Grades. Dazu hatten wir in
  \slideref{Satz}{dreiterm} nachgewiesen, dass eine solche Folge bis
  auf Skalierung eindeutig bestimmt ist. Die Polynome $p_n$ sind also
  Vielfache der Legendre-Polynome $\plegendre_n$ und insbesondere die
  Nullstellen durch die Bedingung eindeutig festgelegt.
\end{proof}

\begin{Definition}{Gauss-Legendre}
  Die $n$-Punkt-\define{Gauß-Legendre-Formel} auf dem Intervall
  $I=[-1,1]$ benutzt als Stützstellen $x_1,\dots,x_n$ die Nullstellen
  des Legendre-Polynoms $\plegendre(x)$ vom Grad $n$. Ihre
  Quadraturgewichte sind die Integrale der Lagrange-Polynome
  \begin{gather}
    \omega_i = \int_I \plagrange_{i;x_1,\dots,x_n}(x)\dx.
  \end{gather}
\end{Definition}

\begin{Satz}{gauss-legendre}
  Die $n$-Punkt-Gauß-Legendre-Formel wohldefiniert. Sie ist exakt für
  beliebige Polynome vom Grad $2n-1$ genau dann, wenn sie exakt vom
  Grad $n-1$ ist.
\end{Satz}

\begin{proof}
  Zunächst müssen wir zeigen, dass das Legendre-Polynom $\plegendre_n$
  genau $n$ paarweise verschiedene, reelle Nullstellen hat. Seien dazu
  $\lambda_1,\dots,\lambda_m$ die Nullstellen ungerade Vielfachheit
  mit $m\le n$. Wir führen das Hilfspolynom $q$ ein, für das gelte
  $q\equiv 1$, falls es keine solche Nullstelle gibt und sonst
  \begin{gather}
    q(x) = \prod_{i=1}^m (x-\lambda_i).
  \end{gather}
  Dieses Polynom hat $m$ reelle Nullstellen.
  Das Polynom $\plegendre_nq \in \P_{n+m}$ hat dann keinen Vorzeichenwechsel
  und es gilt daher
  \begin{gather}
    \int_{-1}^1 \plegendre_n q \dx \neq 0.
  \end{gather}
  Aufgrund der Orthogonalität folgt damit $m=n$ und $q=\plegendre_n$.

  Nun wenden wir uns dem Grad der Exhaktheit zu.  Ein Polynom
  $p\in \P_{2n-1}$ können wir durch Division mit Rest als Summe
  \begin{gather}
    p(x) = q(x)\plegendre_n(x) + r(x)
  \end{gather}
  darstellen, wobei $q,r\in \P_{n-1}$. Es gilt dann wegen der Orthogonalität
  \begin{gather}
    \int_{-1}^1 p\dx
    = \int_{-1}^1 q\plegendre_n \dx + \int_{-1}^1 r\dx
    = \int_{-1}^1 r\dx.
  \end{gather}
  Für die Quadratur gilt, da die Quadraturpunkte die Nullstellen von
  $\plegendre_n$ sind,
  \begin{gather}
    Q_n(p) = Q_n(q\plegendre_n) + Q(r) = Q(r).
  \end{gather}
  Die Quadratur ist also genau dann exakt, wenn für beliebiges $r\in \P_{n-1}$
  \begin{gather}
    Q_n(r) = \int_{-1}^1 r\dx.
  \end{gather}
\end{proof}

\begin{Lemma}{gauss-legendre-gewichte}
  Die Gewichte der Gauss-Legendre-Formeln sind positiv und genügen der
  Darstellung
  \begin{gather}
    \omega_i = \int_{-1}^1 \prod_{j\neq i}
    \left(\frac{x-x_j}{x_i-x_j}\right)^2\dx.
  \end{gather}
\end{Lemma}

\begin{proof}
  Die Darstellung ist das Integral der Lagrange-Polynome zu den
  Punkten $x_1,\dots,x_n$, was nach
  \slideref{Lemma}{interpolatorisch-omega} und
  \slideref{Satz}{gauss-legendre} notwendig und hinreichend für die
  Exaktheit ist.

  Sei nun
  \begin{gather}
    q(x) = \left(\frac{\plegendre_n(x)}{x-x_i}\right)^2.
  \end{gather}
  Dann ist $q\in \P_{2n-2}$ und $q(x_j) = 0$ für $j\neq i$. Daher gilt
  auf Grund der Exaktheit
  \begin{gather}
    \omega_i = \frac1{q(x_i)} \int q\dx.
  \end{gather}
  Da $q(x)\ge 0$ ist die rechte Seite positiv. 
\end{proof}

\begin{Lemma}{gauss-legendre-fehler}
  Für die Gauss-Legendre-Formel mit $n$ Quadraturpunkten auf
  $I=[-1,1]$ gilt die Fehlerabschätzung
  \begin{gather}
    \left|\int_I f\dx - Q_n(f)\right|
    \le \max_{\xi\in I}\frac{f^{(2n)}(\xi)}{(2n)!}
    \int_{-1}^1 \prod_{i=1}^n(x-x_i)^2.
  \end{gather}
\end{Lemma}

\begin{proof}
  Dies ist eine direkte Anwendung der Fehlerdarstellung für die Lagrange-Interpol
\end{proof}

\begin{remark}
  Alle Resultate dieses Abschnitts gelten für Skalarprodukte der Form
  \begin{gather}
    \scal(p,q) = \int_I \omega(x)p(x)q(x)\dx
  \end{gather}
  mit einer positiven Gewichtsfunktion $\omega(x)$, wenn man die
  Legendre-Polynome durch die entsprechenden orthogonalen Polynome
  ersetzt.
\end{remark}

\subsection{Richardson-Extrapolation und Romberg-Quadratur}

\begin{Definition}{richardson-extrapolation}
  Sei $T(h)$ eine numerische Methode zur Approximation des
  tatsächlichen Wertes $T(0)$ mit Diskretisierungsparameter $h$ und
  Fehlerabschätzung $\abs{T(h)-T(0)} = \bigo(h^p)$. Zur
  \define{Richardson-Extrapolation} wertet man diese Methode mit einer
  Schrittfolge $h_1, h_2,\ldots,h_n$ aus, so dass die Schrittweite
  (theoretisch) gegen null geht. Wertet man dann das
  Interpolationspolynom $p(h^p)$ an der Stelle $h=0$ aus, so bekommt
  man unter stärkeren Voraussetzungen die verbesserte Approximation
  \begin{gather}
    \abs{T(0)-p(0)} = \bigo(h^{np}).
  \end{gather}
\end{Definition}

\begin{remark}
  Tatsächlich genügt die einfache Fehlerabschätzung
  $\abs{T(h)-T(0)} = \bigo(h^p)$ nicht, um die behauptete
  Konvergenzordnung zu beweisen. Man benötigt eine asymptotische
  Fehlerentwicklung der Form
  \begin{gather}
    T(h)-T(0) = \tau_1 h^p + \tau_2 h^{2p} + \dots \tau_n h^{np}
    + \bigo(h^{(n+1)p}).
  \end{gather}
\end{remark}

\begin{Definition}{Romberg-quadratur}
  Die \define{Romberg-Quadratur} beruht auf einer summierten
  Quadraturformel $Q_h$ der Ordnung $h^p$, die für eine Folge von
  Schrittweiten $h_1,\dots, h_n$ angewandt wird. Aus diesen berechnet
  man mit dem \putindex{Neville}-Algorithmus Approximationen für
  $Q_0$.
\end{Definition}

\begin{Algorithmus*}{romberg}{Romberg-Quadratur}
  \lstinputlisting{code/romberg.py}
\end{Algorithmus*}

\begin{Aufgabe*}{romberg}{Romberg-Quadratur}
  Schreiben Sie eine Funktion, die die Funktion $f(x) = \sin(\pi x)$
  über das Intervall $[0,1]$ mit der iterierten Trapezregel
  integriert. Wenden Sie die Romberg-Quadratur mit der Schrittfolge
  \begin{gather}
    h = 1,\frac12,\frac14,\frac18,\dots
  \end{gather}
  an und beobachten Sie die Konvergenz gegen den exakten Integralwert
  $2/\pi$ für die verschiedenen Spalten im Tableau des
  Neville-Algorithmus.
\end{Aufgabe*}

\subsection{Praktische Aspekte}

\begin{remark}
  Die Konvergenzabschätzungen der Form
  \begin{gather}
    \left|\int_I f\dx - Q_h(f)\right| \le c h^p \norm{f^{p+1}}_{\infty;I}
  \end{gather}
  verlieren ihren Nutzen für große $h$, wenn die Ableitungen von $f$
  wachsen. Schlimmstenfalls bekommt man dann aus der
  Interpolationseigenschaft noch immer
  \begin{gather}
    \left|\int_I f\dx - Q_h(f)\right| \le c \norm{f}_{\infty;I}.
  \end{gather}
  Es gibt aber keine Garantie, dass der Fehler bei feinerer
  Unterteilung schrumpft.

  Dies gilt auch für große $h$, falls $f\in C^{p+1}(I)$ aber die
  Ableitungen mit steigender Ordnung schnell wachsen. Dann kann
  zunächst der Gewinn durch Wahl einer feineren Unterteilung durch das
  Wachsen der Ableitung annulliert werden. Man nennt das Verhalten
  dann \define{präasymptotisch}.

Für hinreichend feine Unterteilungen $h_1$ und $h_2$ gilt die obige
Abschätzung aber in der stärkeren Form
  \begin{gather}
    \left|\int_I f\dx - Q_{h_2}(f)\right|
    \approx \left(\frac{h_2}{h_1}\right)^p
    \left|\int_I f\dx - Q_{h_1}(f)\right|.
  \end{gather}
  Man beobachtet also die Konvergenzordnung als direkte Verbesserung
  mit jeder Wahl eines feineren Parameters. Dieses Verhalten nennt man
  \define{asymptotisch}.
\end{remark}

\begin{remark}
  Die Konvergenzordnung\index{Konvergenzordnung!experimentell} eines
  Verfahrens lässt sich auch experimentell bestimmen. Sei dazu $T(h)$
  eine numerische Methode mit Diskretisierungsparameter $h$ und der
  Fehler verhalte sich wie
  \begin{gather}
    \abs{T(h) - T(0)} = c h^p + \smallo(h^p).
  \end{gather}
  Wenn die exakte Lösung $T(0)$ bekannt ist, so lässt sich die linke
  Seite für verschiedene Parameter $h$ berechnen. Benutzt man zwei
  verschiedene Schrittweiten, so lassen sich mit der Abkürzung
  $e(h) = \abs{T(h) - T(0)}$ die Werte $p$ und $c$ aus dem
  linearen Gleichungssystem
  \begin{gather}
    \begin{aligned}
      \log c + p \log h_1 &= \log e(h_1),\\
      \log c + p \log h_2 &= \log e(h_2).
    \end{aligned}
  \end{gather}
  bestimmen. Da hier die unbekannten Terme $\smallo(h)$ weggelassen
  wurden, ist diese Bestimmung nicht exakt, konvergiert aber für
  $h\to 0$. Führt man die Bestimmung für jeweils aufeinanderfolgende
  Paare von Parametern $h$ und einer Folge durch, so konvergieren die
  Werte von $p$ und $c$, so dass man dadurch die Zuverlässigkeit der
  Schätzung einschätzen kann.

  Findet man keine relevante Aufgabe, deren exakte Lösung bekannt ist,
  so kann man den Wert $T(0)$ durch Richardson-Extrapolation nähern.

  Der Nutzen dieser Technik liegt nicht nur in der experimentellen
  Bestätigung der theoretischen Beweise. Sie erlaubt es, die Konstante
  $c$ zu Schätzen, für die der Beweis oft nur Existenz liefert. Auch
  kann dadurch die Optimalität des theoretischen Ergebnisses überprüft
  werden. Schließlich erlaubt diese Technik auch, die
  Konvergenzordnung zu schätzen um dann erst zu sehen, wie man diese
  Ordnung auch beweist.
\end{remark}

\begin{remark}
  Wenn eine ekakte Lösung nicht verfügbar ist und Extrapolation zur
  null nicht sinnvoll, so gibt es noch die Option, die
  \textbf{intrinsische
    Konvergenzordnung}\index{Konvergenzordnung!intrinsisch} zu
  betrachten. Dazu betrachten wir zum Beispiel die Folge
  \begin{gather}
    d_k = \abs{T(2^kh) - T(2^{k-1}h)}, \qquad k=1,\dots
  \end{gather}
  Gilt $d_k \approx 2^p$ mit $p>0$, so gilt aufgrund der Konvergenz
  der geometrischen Reihe
  \begin{gather}
    \abs{T(h)-T(0)} \le \sum_{k=1}^\infty \abs{T(2^kh) - T(2^{k-1}h}
    \le 2 \abs{T(h)-T(h/2)}.
  \end{gather}
  Daraus schließt man, dass auch die Konvergenzordnung etwa gleich $p$
  ist. Auch dies funktioniert nur, wenn $h$ bereits im asymptotischen
  Bereich liegt.
\end{remark}

\begin{Aufgabe}{konvergenzraten}
  Bestimmen sie aus dem Neville-Tableau aus
  \slideref{Aufgabe}{romberg} die experimentellen und intrinsischen
  Konvergenzraten der Spalten.
\end{Aufgabe}

%%% Local Variables:
%%% mode: latex
%%% TeX-master: "main"
%%% End:


\chapter{Lösung linearer Gleichungssysteme}

\section{Vektor- und Matrixnormen}
\subsection{Grundlagen}

\begin{Definition}{norm}
  Eine \define{Norm} $\norm{\cdot}$ auf dem Vektorraum $V$ ist eine Abbildung
  \begin{gather}
    \begin{split}
      \norm{\cdot}\colon V &\to \R\\
      x&\mapsto \norm{x}
    \end{split}
  \end{gather}
  mit den Eigenschaften
    \begin{xalignat}3
    &\text{Homogenität:}
    &\norm{\alpha x} &= \abs{\alpha}\norm{x}
    &\forall \alpha&\in\R,x\in V
    \\
    &\text{Dreiecksungleichung:}
    &\norm{x+y} &\le \norm{x}+\norm{y}
    &\forall x,y&\in V
    \\
    &\text{Definitheit:}
    &\norm{x} & \ge 0
    &\forall x&\in V
    \\
    &&\norm{x} &\neq 0
    &\forall x&\neq0
    \end{xalignat}
  Verzichtet man auf die zweite Definitheitsbedingung, so erhält man eine \define{Seminorm}.
\end{Definition}

\begin{Definition}{norm-aequivalenz}
  Sei $V$ ein reeller oder komplexer Vektorraum. Zwei Normen
  $\norm{\cdot}_X$ und $\norm{\cdot}_Y$ auf $V$ heißen äquivalent,
  wenn es Konstanten $c>0$ und $C>0$ gibt, so dass
  \begin{gather}
    c \norm{v}_X \le \norm{v}_Y \le C \norm{v}_X
    \qquad\forall v\in V.
  \end{gather}
\end{Definition}

\begin{Definition}{rn-konvergenz}
  Eine Folge $\{x^{(k)}\}\subset \R^n$ für $k=1,2,\dots$ heißt
  \textbf{komponentenweise konvergent} gegen $x\in \R^n$, wenn gilt
  \begin{gather}
    \forall \epsilon>0\;
    \exists k_0\in \mathbb N\;
    \forall k\ge k_0, i=1,\dots,n
    : \abs{x^{(k)}_i - x_i} < \epsilon.
  \end{gather}
  Die Folge heißt konvergent unter der Norm $\norm{\cdot}$ wenn gilt
  \begin{gather}
    \forall \epsilon>0\;
    \exists k_0\in \mathbb N\;
    \forall k\ge k_0
    : \norm{x^{(k)} - x} < \epsilon.
  \end{gather}
\end{Definition}

\begin{Lemma}{norm-aequivalenz}
  Sei $\norm{\cdot}$ eine beliebige Norm auf $\R_n$. Dann ist die Abbildung
  \begin{gather}
    f\colon x \mapsto \norm{x}
  \end{gather}
  stetig bezüglich der komponentenweisen Konvergenz. Ferner ist die
  Norm $\norm{\cdot}$ äquivalent zur Maximumsnorm.
\end{Lemma}

\begin{proof}
  Für den ersten Teil ist zu zeigen, dass zu einer komponentenweise
  konvergenten Folge von Vektoren auch deren Norm konvergiert. Sei
  $\{x^{(k)}\}$ eine solche Folge und dazu $k_0$ so gewählt, dass
  \begin{gather}
    \max_{i=1,\dots,n}\left\lvert\left(x_i^{(k)}-x_i\right) \norm{e_i}\right\rvert
      < \frac\epsilon n
    \qquad \forall k\ge k_0.
  \end{gather}
  Hier ist $e_i$ der $i$-te Einheitsvektor. Dann folgt
  \begin{align}
    \norm{x^{(k)}-x}
    &= \left\lVert\sum_{i=1}^n \left(x_i^{(k)}-x_i\right) e_i\right\rVert
    \\
    &\le \sum_{i=1}^n \left\lVert\left(x_i^{(k)}-x_i\right) e_i\right\rVert
    \\
    & < n \frac\epsilon n = \epsilon.
  \end{align}
  Hiermit haben wir bereits bewiesen, dass komponentenweise Konvergenz
  auch Normkonvergenz impliziert.

  Die \glqq{}Einheitssphäre\grqq{}
   \begin{gather}
     S = \bigl\{ x\in \R^n \big| \norm{x}_\infty = 1 \bigr\}
   \end{gather}
   ist beschränkt und bezüglich der komponentenweisen Konvergenz
   abgeschlossen. Die Norm $\norm.$ nimmt dort als stetige Funktion
   ihr Minimum $c$ und ihr Maximum $C$ an. Insbesondere gilt aber
   wegen der Definitheit $c > 0$. Für einen beliebigen Vektor
   $x\in \R^n$ ist $x/\norm{x}_\infty \in S$, so dass gilt
   \begin{gather}
     c \norm{x}_\infty \le \norm{x} \le C \norm{x}_\infty.
   \end{gather}
\end{proof}

\begin{Satz}{norm-aequivalenz}
  Auf $\R^n$ sind zwei beliebige Normen $\norm._X$ und $\norm._Y$ äquivalent.
\end{Satz}

\begin{proof}
  Nach dem vorherigen Lemma sind beide Normen äquivalent zur Maximumsnorm. Es gibt also Konstanten $c_X, c_Y, C_X, C_y>0$ mit
  \begin{gather}
    \begin{aligned}
     c_X \norm{x}_\infty &\le &\norm{x}_X \;&\le C_X \norm{x}_\infty\\
     c_Y \norm{x}_\infty &\le &\norm{x}_Y \;&\le C_Y \norm{x}_\infty.
    \end{aligned}
  \end{gather}
  Daher gilt
  \begin{gather}
    \begin{split}
      \norm{x}_Y \le C_Y\norm{x}_\infty &\le \frac{C_Y}{c_X} \norm{x}_X\\
      \norm{x}_X \le C_X\norm{x}_\infty &\le \frac{C_X}{c_y} \norm{x}_Y
    \end{split}
  \end{gather}
\end{proof}


\begin{Definition}{matrix-norm}
  Auf dem Vektorraum der Matrizen $\R^{m\times n}$ ist durch
  \slideref{Definition}{norm} eine Norm definiert. Gilt zusätzlich
  \begin{gather}
    \norm{Ax} \le \norm{A}\norm{x}
    \qquad\forall A\in \R^{m\times n}, x\in \R^n,
  \end{gather}
  so heißt die Norm $\norm.$ der Matrix \define{verträglich} mit der
  Vektornorm $\norm.$. Wir sprechen von einer \define{Matrixnorm},
  wenn sie zusätzlich \define{submultiplikativ} ist, dass heißt,
  für alle Matrizen $A,B$ passender Dimensionen gilt
  \begin{gather}
    \norm{AB} \le \norm{A}\norm{B}.
  \end{gather}
  Ferner definieren wir die \define{Operatornorm} oder \define{natürliche Norm}
  \begin{gather}
    \norm{A} = \sup_{\substack{x\in\R^n\\x\neq0}} \frac{\norm{Ax}}{\norm{x}}
    = \sup_{\norm{x}=1}\norm{Ax}.
  \end{gather}
\end{Definition}

\begin{Lemma}{operator-norm}
  Die Operatornorm ist verträglich und submultiplikativ.
\end{Lemma}

\begin{Beispiel}{Zeilensummen}
  Die Operatornormen zu den Vektornormen $\norm._1$ und
  $\norm._\infty$ sind die \define{Spaltensummennorm} und die
  \define{Zeilensummennorm}
  \begin{align}
    \norm{A}_1 &= \max_{i=1,\dots,n} \sum_{j=1}^n \abs{a_{ji}}\\
    \norm{A}_\infty &= \max_{i=1,\dots,n} \sum_{j=1}^n \abs{a_{ij}}
  \end{align}
\end{Beispiel}

\subsection{Eigenwerte und die Spektralnorm}

\begin{Definition}{eigenwert}
  Sei $A\in \R^{n\times n}$. Gilt für einen Vektor $0\neq x\in \R^n$
  \begin{gather}
    Ax = \lambda x,
  \end{gather}
  so nennen wir $\lambda$ \define{Eigenwert} von $A$ und $x$ einen
  zugehörigen \define{Eigenvektor}. Wir notieren die Zugehörigkeit zur
  Matrix $A$ auch explizit durch $\lambda(A)$.
\end{Definition}

\begin{Lemma}{ew-norm}
  Für alle Eigenwerte $\lambda\in\C$ einer Matrix $A\in\R^{n\times n}$ gilt
  \begin{gather}
    \abs{\lambda} \le \norm{A}
  \end{gather}
  für jede zu einer beliebigen Vektornorm verträglichen Norm.
\end{Lemma}

\begin{Satz}{onb-ev}
  Sei $A\in \R^n\times n$ eine symmetrische Matrix. Dann gibt es eine
  orthonormalbasis des $\R^n$ von Eigenvektoren $v^{(i)}$ mit zugehörigen
  reellen Eigenwerten $\lambda_i$.
\end{Satz}

\begin{proof}
  Resultat der linearen Algebra.
\end{proof}

\begin{Satz}{spektralnorm}
  Die Operatornorm zur euklidischen Norm\index{Norm|euklidisch} ist
  die \define{Spektralnorm}
  \begin{gather}
    \norm{A}_2 = \max_{\substack{x\in\R^n\\x\neq0}}
    \sqrt{\frac{x^TA^TAx}{x^Tx}} = \sqrt{\lambda_{\max}(A^TA)}.
  \end{gather}
  Insbesondere gilt für symmetrische Matrizen
  $\norm{A}_2 = \max_i\abs{\lambda_i(A)}$.
\end{Satz}

\begin{proof}
  Nach \slideref{Satz}{onb-ev} gibt es eine Basis des $\R^n$ von
  Eigenvektoren $v^{(i)}$ von $A^TA$. Jeder beliebige Vektor $x$
  besitzt damit die Darstellung
  \begin{gather}
    x = \sum_{i=1}^n \alpha_i v^{(i)}.
  \end{gather}
  Es gilt nach der Parsevalschen Gleichung
  $\norm{x}_2 = \norm{\alpha}_2$. Ferner gilt mit den Eigenwerten
  $\lambda_i = \lambda_i(A^TA)$
  \begin{gather}
    \norm{Ax}_2^2 = x^TA^TAx = \sum_{i=1}^n \lambda_i \alpha_i^2.
  \end{gather}
  Daher gilt
  \begin{gather}
    \norm{A}_2^2 = \max_{x\in \R^n} \frac{\norm{Ax}^2}{\norm{x}^2}
    = \max_{\alpha} \frac{\sum \lambda_i\alpha_i^2}{\sum\alpha_i^2}
    \le \lambda_{\max}(A^TA).
  \end{gather}
  Da für symmetrische Matrizen $A=A^T$, so ist
  \begin{gather}
    \lambda_{\max}(A^TA) = \lambda_{\max}(A^2) = \lambda_{\max}^2(A)
  \end{gather}
\end{proof}

\begin{Definition}{pos-def}
  Eine Matrix $A\in\R^{n\times n}$ heißt \define{positiv definit}, wenn
  \begin{gather}
    x^TAx > 0 \qquad\forall 0\neq x\in \R^n.
  \end{gather}
\end{Definition}

\begin{Satz}{spd}
  Eine symmetrische Matrix $A\in\R^{n\times n}$ ist positiv definit genau dann, wenn ihre Eigenwerte alle positiv sind.
\end{Satz}

\section{Fehleranalyse}

\subsection{Konditionierung der Lösung}

\begin{Definition}{aufgabe-loesung}
  Die Aufgabe, das lineare Gleichungssystem
  \begin{gather}
    Ax=b
  \end{gather}
  zu lösen wandelt die Eingabedaten $(A,b)$ in das Ausgabedatum $x$
  um. Die zugehörige gestörte Aufgabe ist
  \begin{gather}
    (A+\delta A) (x+\delta x) = b+ \delta b,
  \end{gather}
  wobei $\delta A$ und $\delta b$ eine Matrix und ein Vektor sind, um
  die die Eingabedaten gestört sind. $\delta x$ ist die resultierende
  Störung der Lösung.
  
  Die Untersuchung der Konditionierung dieser Aufgabe besteht in
  der Bestimmung einer relativen \putindex{Konditionszahl} $\kappa$, so dass
  \begin{gather}
    \frac{\norm{\delta x}}{\norm{x}}
    \le \kappa \left[\frac{\norm{\delta A}}{\norm{A}}
      +\frac{\norm{\delta b}}{\norm{b}}\right].
  \end{gather}
\end{Definition}

\begin{Lemma}{gestoert-invertierbar}
  Sei $B\in \R^{n\times n}$ mit $\norm{B} < 1$. Dann ist $I-B$
  invertierbar und es gilt
  \begin{gather}
    \norm{(I-B)^{-1}} \le (1-\norm{B})^{-1}
  \end{gather}
\end{Lemma}

\begin{proof}
  Siehe \cite[Hilfssatz 4.4]{Rannacher17}.
\end{proof}

\begin{Satz}{kondition-lgs}
  Sei die Matrix $A\in\R^{n\times n}$ invertierbar und
  \begin{gather}
    \norm{\delta A} < \norm{A}^{-1}.
  \end{gather}
  Dann ist die gestörte Matrix $A+\delta A$ ebenfalls invertierbar und
  es gilt die Fehlerabschätzung
  \begin{gather}
    \frac{\norm{\delta x}}{\norm{x}}
    \le \frac{\cond(A)}{1-\cond(A)\nicefrac{\norm{\delta A}}{\norm{A}}}
    \left[\frac{\norm{\delta A}}{\norm{A}}
      +\frac{\norm{\delta b}}{\norm{b}}\right].
  \end{gather}
  Hierzu definieren wir die \define{Konditionszahl} der Matrix $A$ zur
  Norm $\norm{\cdot}$
  \begin{gather}
    \cond(A) = \norm{A} \,\norm{A^{-1}}
  \end{gather}
\end{Satz}

\begin{proof}
  Siehe \cite[Satz 4.1]{Rannacher17}.
\end{proof}

%%% Local Variables:
%%% mode: latex
%%% TeX-master: "main"
%%% End:


\section{Die LR-Zerlegung}
\begin{notation}%{quadratische-matrizen}
  Da wir uns in diesem Abschnitt mit der Lösung quadratischer
  Gleichungssysteme beschäftigen, gelte für alle Matrizen, soweit
  nicht anders vermerkt, dass ihre Dimension $n\times n$ sei.
\end{notation}

\subsection{Dreiecksmatrizen und Frobeniusmatrizen}

\begin{Definition}{dreiecksmatrix}
  Für eine \define{untere Dreiecksmatrix} $L \in \R^{n\times n}$ gilt
  \begin{gather}
    \ell_{ij} = 0,\qquad j>i.
  \end{gather}
  Für eine \define{obere Dreiecksmatrix} $R \in \R^{n\times n}$ gilt
  \begin{gather}
    r_{ij} = 0,\qquad j<i.
  \end{gather}
\end{Definition}

\begin{Satz}{dreieck-gruppe}
  Die Mengen der invertierbaren oberen und unteren Dreiecksmatrizen
  bilden jeweils eine multiplikative Gruppe. Die Determinante einer
  Dreiecksmatrix ist das Produkt ihrer Diagonalelemente.
\end{Satz}

\begin{proof}
  Hausaufgabe
\end{proof}

\begin{Korollar}{dreieck-inverse}
  Eine Dreiecksmatrix ist invertierbar genau dann, wenn alle ihre
  Diagonalelemente von null verschieden sind.
\end{Korollar}

\begin{Algorithmus}{vorwaerts-rueckwaerts}
  Die Lösung der linearen Gleichungssysteme
  \begin{gather}
    Lx = b \qquad Rx = b
  \end{gather}
  mit einer unteren Dreiecksmatrix $L$ und einer oberen Dreiecksmatrix
  $R$ lässt sich sukzessive durch Vorwärts- bzw.\ Rückwärtseinsetzen
  berechnen.
  \begin{minipage}[t]{.45\linewidth}
    \lstinputlisting[basicstyle=\footnotesize]{code/forsub.py}    
  \end{minipage}
  \begin{minipage}[t]{.45\linewidth}
    \lstinputlisting[basicstyle=\footnotesize]{code/backsub.py}    
  \end{minipage}
\end{Algorithmus}

\begin{Definition}{frobenius-matrix}
  Eine Matrix der Gestalt
  \begin{gather}
    G_k=\begin{bmatrix}
      1 & & & & & \\
      &\ddots & & & & \\
      &   & 1& & &\\
      &   & g_{k+1,k}&1 & &\\
      &   & \vdots& &\ddots &\\
      &   & g_{nk}& & &1
    \end{bmatrix}
  \end{gather}
  mit von null verschiedenen Subdiagonaleinträgen nur in Spalte $k$
  heißt \define{Frobenius-Matrix}.
\end{Definition}

\begin{Lemma}{frobenius-matrix}
  Das Ergebnis des Produktes $G_kA$ einer Frobeniusmatrix mit einer
  beliebigen Matrix ergibt sich aus $A$ dadurch, dass auf die $j$-te
  Zeile das $g_{jk}$-fache der $k$-ten Zeile addiert wird.

  Für Frobenius-Matrizen gilt
  \begin{gather}
    G_k^{-1} = 2\identity-G_k.
  \end{gather}
  
  Sei $k_1<\dots<k_m$ eine aufsteigende Folge von Indizes. Dann gilt für Produkte die Darstellung
  \begin{gather}
    G_{k_1}\cdots G_{k_m} = \sum_{i=1}^m G_i - (m-1) \identity.
  \end{gather}
  \begin{gather}
    \text{Insbesondere gilt }\qquad
    G_1\cdots G_{n-1} =
    \begin{bmatrix}
      1\\
      g_{21} & 1 \\
      \vdots & \ddots & \ddots \\
      g_{n1}  & \dots & g_{n,n-1} & 1
    \end{bmatrix}
  \end{gather}
\end{Lemma}

\subsection{Konstruktion der LR-Zerlegung}

\begin{Lemma}{elimination-1}
  Bei der Gauß-Elimination lässt sich die Elimination der
  Subdiagonalelemente der $k$-ten Spalte als Matrix-Produkt
  \begin{gather}
    A^{(k+1)} = L^{-1}_k A^{(k)},
    \qquad b^{(k+1)} = L^{-1}_k b^{(k)},
    \qquad k=1,\dots,n-1
  \end{gather}
  mit $A^{(1)}=A$, $b^{(1)}=b$ und den Frobenius-Matrizen
  \begin{gather}
    L_k =\begin{bmatrix}
      1 & & & & & \\
      &\ddots & & & & \\
      &   & 1& & &\\
      &   & \ell_{k+1,k}&1 & &\\
      &   & \vdots& &\ddots &\\
      &   & \ell_{nk}& & &1
    \end{bmatrix},
    \qquad
    \ell_{ik} = \frac{a_{ik}^{(k)}}{a_{kk}^{(k)}}
  \end{gather}
  schreiben.
\end{Lemma}

\begin{Satz}{elimination-2}
  Nach $n-1$ Schritten der Gauß-Elminiation erhält man das transformierte lineare Gleichungssystem
  \begin{gather}
    R x = y,\qquad R = L^{-1}A, \qquad y=L^{-1}b,
    \qquad L = L_1\cdots L_{n-1},
  \end{gather}
  und die LR-Zerlegung
  \begin{gather}
    A = LR
  \end{gather}
  mit einer oberen Dreiecksmatrix $R$ und einer unteren Dreiecksmatrix
  $L$, deren Diagonale aus Einsen besteht.
\end{Satz}

\begin{Algorithmus*}{lr}{LR-Zerlegung}
  \lstinputlisting{code/lu.py}
\end{Algorithmus*}

\begin{remark}
  Im vorigen Algorithmus wird die LR-Zerlegung (engl. LU
  decomposition) so durchgeführt, dass sie die Matrix $A$
  ersetzt. Dadurch wird kein zusätzlicher Speicher benötigt. Nach
  ausführen der Funktion hat dann $A$ nicht mehr die Bedeutung einer
  Matrix, sondern ist ein quadratisches Zahlenfeld, für dessen
  Einträge $a_{ij}$ gilt
  \begin{gather}
    a_{ij} =
    \begin{cases}
      r_{ij} & i \le j\\
      \ell_{ij} & i>j.
    \end{cases}
  \end{gather}
  Von den Diagonaleinträgen von $L$ wissen wir, dass sie den Wert 1
  haben, deswegen werden sie nicht gespeichert.

  Für dieses spezielle Datenformat gibt es dann auch eine
  spezialisierte Version der Auflösung der gestaffelten
  Gleichungssysteme:
\end{remark}

\begin{Algorithmus*}{for-back}{Vorwärts-Rückwärts-Einsetzen}
  \lstinputlisting{code/forward_backward.py}
\end{Algorithmus*}

\begin{Lemma}{aufwand-LR}
  Der Aufwand der LR-Zerlegung einer $n\times n$-Matrix ist
  \begin{gather}
    \tfrac13 n^3 + \bigo(n^2).
  \end{gather}
\end{Lemma}

\begin{Satz}{existenz-LR}
  Ist die Matrix $A$ invertierbar, dann ist im $k$-ten Schritt der
  Gauß-Elimination wenigstens eins der Elemente $a^{(k)}_{jk}$ mit
  $j\ge k$ von null verschieden. für den Fall, dass
  $a^{(k)}_{kk} = 0$, kann damit die Elimination nach Vertauschen der
  Zeilen $j$ und $k$ fortgesetzt werden.
\end{Satz}

\begin{Definition}{spalten-pivot}
  Führt man im $k$-ten Schritt der Gauß-Elimination eine
  Zeilenvertauschung durch, so dass
  \begin{gather}
    \abs{a^{(k)}_{kk}} = \max_{j\ge k} \abs{a^{(k)}_{jk}},
  \end{gather}
  so spricht man von Gauß-Elimination mit
  \define{Spalten-Pivotierung}. Vertauscht man sogar die verbleibenden
  Zeilen und spalten, so dass
  \begin{gather}
    \abs{a^{(k)}_{kk}} = \max_{i,j\ge k} \abs{a^{(k)}_{ij}},
  \end{gather}
  handelt es sich um \define{vollständige Pivotierung}.
\end{Definition}

\begin{Lemma}{spalten-pivot}
  Führt man die Gauß-Elimination mit Spalten-Pivotierung durch, so gilt
  für die Matrix $L$:
  \begin{gather}
    \abs{\ell_{ij}} \le 1,\qquad1\le i,j\le n.
  \end{gather}
\end{Lemma}

\begin{Lemma}{l-p-vertauschung}
  Sei $\pi$ eine Permutation der Zahlen $1,\dots,n$, so dass die
  Zahlen $1,\dots,k$ unverändert bleiben, und $P_{\pi}$ die Matrix der
  entsprechenden Zeilenvertauschungen. Dann ist
  \begin{gather}
    P_\pi L_k P_\pi^{-1} =
    \begin{bmatrix}
      1 & & & & & \\
      &\ddots & & & & \\
      &   & 1& & &\\
      &   & \ell_{\pi(k+1),k}&1 & &\\
      &   & \vdots& &\ddots &\\
      &   & \ell_{\pi(n)k}& & &1
    \end{bmatrix}
  \end{gather}
  wieder eine Frobenius-Matrix gleicher Struktur wie $L_k$.
\end{Lemma}

\begin{proof}
  Jede Permutation ist das Produkt von Transpositionen. Für solche
  wird die Aussage in der Hausaufgabe gezeigt.
\end{proof}

\begin{Satz}{lr-pivot}
  Nach $n-1$ Schritten der Gauß-Elimination mit Spalten-Pivotierung
  erhält man die Zerlegung
  \begin{gather}
    PA = LR
  \end{gather}
  mit einer Permutationsmatrix $P$ und den Dreiecksmatrizen $L$ und $R$.
\end{Satz}

\begin{proof}
  Siehe \cite[Abschnitt 1.3]{DeuflhardHohmann08}.
\end{proof}

\subsection{Fehleranalyse}

Ohne Beweis geben wir die folgenden Resultate zur
Rundungsfehleranalyse der Lösung linearer Gleichungssysteme mit der
LR-Zerlegung an. Teile der Beweise finden sich in~\cite{Stoer83}.

\begin{Notation}{abs-matrix}
  Zu einer Matrix $A$ sei $\abs{A}$ die Matrix der Absolutbeträge, also
  \begin{gather}
    \abs{A} =
    \begin{pmatrix}
      \abs{a_{11}} & \cdots & \abs{a_{1n}} \\
      \vdots && \vdots\\
      \abs{a_{nn}} & \cdots & \abs{a_{nn}}
    \end{pmatrix}.
  \end{gather}
\end{Notation}

% Stoer:
\begin{Lemma}{rundung-lr-1}
  Die Berechnung der LR-Zerlegung einer Matrix $A$ in
  Fließkommaarithmetik resultiert in einer Zerlegung
  $\widehat L \widehat R = A+\delta A$ und es gilt die Abschätzung
  \begin{gather}
    \abs{\delta A} \le 2\alpha_{\max} \frac{\eps}{1-\eps}
    \begin{pmatrix}
      0&0&0&\cdots&0&0\\
      1&1&1&\cdots&1&1\\
      1&2&2&\cdots&2&2\\
      1&2&3&\cdots&3&3\\
      \vdots&\vdots&\vdots&\cdots&n-1&n-1
    \end{pmatrix}
  \end{gather}
  Hierbei ist $\alpha_{\max}$ der betragsmäßig größte Eintrag
  aller Matrizen $A^{(k)}$, die im Verfahren auftreten,
  \begin{gather}
    \alpha_{\max} = \max_{1\le k,i,j\le n} \abs{a^{(k)}_{ij}}.
  \end{gather}
\end{Lemma}

% Deuflhard:
% \begin{Lemma}{rundung-lr-1}
%   Die Matrix $A$ besitze eine LR-Zerlegung. Dann berechnet das
%   Gaußsche Eliminationsverfahren für das Gleichungssystem $AX=b$ die
%   Lösung $\widehat x$ des Systems $(A+\delta A) \widehat x = b$ für eine Matrix
%   $A+\delta A$ mit
%   \begin{gather}
%     \abs{\delta a_{ij}} \lessdot 2n \abs{\widehat \ell_{ij}}
%     \abs{\widehat r_{ij}}\eps.
%   \end{gather}
% \end{Lemma}

\begin{Lemma}{rundung-vor-rueck}
  Die Realisierung des Vorwärtseinsetzens für das Gleichungssystem
  $Lx = b$ in Fließkommaarithmetik berechnet die Lösung $\widehat x$ des
  gestörten Systems $\widehat L \widehat x = b$ mit
  \begin{gather}
    \abs{L - \widehat L}
    \le \frac{\eps}{1-n\eps} \left(\abs{L}
      \begin{pmatrix}
        1\\&2\\&&\ddots\\&&&n
      \end{pmatrix}
      .-\id \right)
  \end{gather}
\end{Lemma}


\begin{Satz*}{rundung-lr-2}{Wilkinson}
  Das Gaußsche Eliminationsverfahren mit Spaltenpivotierung für das
  Gleichungssystem $Ax=b$ berechnet die Lösung $\widehat x$ des Systems
  $\widehat A \widehat x = b$ für eine Matrix $\widehat A$ mit
  \begin{gather}
    \frac{\norm{\delta A}_\infty}{\norm{A}_\infty}
    \lessdot 2 n^3 \frac{\alpha_{\max}}{\max \abs{a_{ij}}} \eps,
  \end{gather}
  wobei
  \begin{gather}
    \alpha_{\max} = \max_{1\le k,i,j\le n} \abs{a^{(k)}_{ij}}.
  \end{gather}
\end{Satz*}

\begin{remark}
  Im Allgemeinen kann die Konstante $\alpha_{\max}$ durch den Wert
  $2^{n-1} \max a_{ij}$ abgeschätzt werden. Dies führt bei
  größeren Matrizen sehr schnell zu inakzeptablen Fehlern.
  Es gibt aber einige Aussagen über die LR-Zerlegung von Matrizen mit
  spezielleren Strukturen. So gilt
  \begin{enumerate}
  \item Ist die Matrix $A$ invertierbar und schwach diagonaldominant,
    das heißt,
    \begin{gather}
      a_{ii} \ge \sum_{j\neq i} \abs{a_{ij}},
      \qquad i=1,\dots,n,
    \end{gather}
    so kann die LR-Zerlegung ohne Pivotierung durchgeführt werden.
  \item Ist die Matrix $A$ positiv definit, so kann die LR-Zerlegung
    ohne Pivotierung durchgeführrt werden und alle Diagonalelemente
    $a_{kk}^{(k)}$ sind positiv (siehe \cite[Satz 4.7]{Rannacher17}).
  \item Für symmetrisch positiv definite Matrizen führt man die
    Gauß-Elimination in der Variante des
    \putindex{Choleski-Verfahren}s durch, das eine $LL^T$-Zerlegung
    mit dem halben Aufwand der LR-Zerlegung produziert.
  \item Hat die Matrix eine Struktur, bei der sich alle von null
    verschiedenen Einträge um die Diagonale konzentrieren, man spricht
    von Band- und Skyline-Matrizen, so kann man bei der LR-Zerlegung
    diese Struktur ausnutzen und erheblich an Operationen sparen.
  \end{enumerate}
\end{remark}

\begin{remark}
  Hat man über die LR-Zerlegung eine Näherungslösung
  $\widehat x = x+\delta x$ von $Ax=b$ berechnet, so kann das Residuum
  \begin{gather}
    r(\widehat x) = b-A \widehat x
  \end{gather}
  berechnet werden und gibt Aufschluss über den Fehler. Insbesondere gilt
  \begin{gather}
    \delta x = A^{-1} r(\widehat x),
  \end{gather}
  was aber nicht berechenbar ist, da wir die Inverse von $A$ nicht
  exakt kennen. Wir können aber $\delta x$ mit derselben relativen
  Genauigkeit wie $x$ durch den Vektor $\widehat{\delta x}$ approximieren, indem
  wir mit der bereits berechneten LR-Zerlegung
  \begin{gather}
    \widehat L \widehat R \widehat{\delta x} = r(\widehat x)
  \end{gather}
  lösen. Dann ist aber $\widehat x + \widehat{\delta x}$ eine
  Approximation an $x$, deren absoluter Fehler dem von
  $\widehat{\delta x}$ entspricht, ist also genauer als $\widehat x$.
  Offenbar lässt sich dieser Prozess wiederholen
\end{remark}

\begin{Definition}{nachiteration}
  Sei $\widehat L\widehat R$ die fehlerbehaftete LR-Zerlegung der Matrix $A$
  und $x^{(0)} = \widehat R^{-1}\widehat L^{-1} b$. Dann besteht die
  \define{Nachiteration} (engl. \define{iterative refinement}) aus der
  Iterationsvorschrift
  \begin{gather}
    x^{(k+1)} = x^{(k)} + \widehat R^{-1}\widehat L^{-1}
    \bigl(b-Ax^{(k)}\bigr).
  \end{gather}
\end{Definition}

\begin{Aufgabe}{nachiteration-kontraktion}
  Nutzen Sie \slideref{Satz}{rundung-lr-2} und die Konditionierung der
  Lösung eines linearen Gleichungssystems um eine Bedingung an die
  Genauigkeit der Zerlegung zu stellen, unter der Sie die Konvergenz
  der Nachiteration durch den Banachschen Fixpunktsatz beweisen
  können.
\end{Aufgabe}

\begin{remark}
  Die Funktionsbibliothek LAPACK zur linearen Algebra wird heute als
  Standardimplementation für viele der hier diskutierten Algorithmen
  benutzt, zum Beispiel die LR-Zerlegung. Sie enthält viele
  Optimierungen und benutzt auch automatisch Spaltenpivotierung.
\end{remark}



%%% Local Variables:
%%% mode: latex
%%% TeX-master: "main"
%%% End:


\section{Die QR-Zerlegung}
\subsection{Derivation from orthogonal subspace iteration}

\begin{todo}
  rename vectors in OSI
\end{todo}

\begin{intro}
  \label{par:qr:intro}
  In this section, we call the orthogonal vectors of the orthogonal subspace iteration $\matu^{(k)}$ to distinguish them from the new orthogonal basis constructed here.
  
  If the \putindex{orthogonal subspace iteration} converges, then
  there is an unitary matrix $\matu$ such that
  $\lim_{k \to \infty} \matu_k = \matu$.  From the two assignments of
  the algorithm, we get
  \begin{gather}
    \lim_{k \to \infty} \matu_k^* \mata \matu_k = \lim_{k \to \infty} \matu_k^* \maty_k = \lim_{k \to \infty}
    \matu_k^*\matu_{k+1}\matr_{k+1} = \matr,
  \end{gather}
  where $\matr\leftarrow\matr_k$. Hence, the matrices
  \begin{gather}
    \label{eq:qr:intro:1}
    \mata_k = \matu_k^*\mata\matu_k
  \end{gather}
  converge to an upper triangular matrix with the converged
  eigenvalues on the diagonal. Our next goal ist the modification of
  the orthogonal subspace iteration, such that we compute the sequence
  $\mata_k$ directly, without the intermediate $\maty_k$. To this end,
  we require $m=n$, that is, we compute the whole spectrum of $\mata$.
\end{intro}

\begin{Lemma}{qr-iteration-derivation}
  The matrices $\mata_k = \matu_k^*\mata\matu_k$ obtained from the
  orthogonal subspace iteration follow the recursion formula
  \begin{gather}
    \label{eq:qr:iteration-derivation:2}
    \mata_k = \matq_k^*\mata_{k-1}\matq_k,
  \end{gather}
  where $\matq_k = \matu_{k-1}^*\matu_k$ for $k=1,\dots,n-1$. There holds
  \begin{gather}
    \label{eq:qr:iteration-derivation:1}
    \matu_k=\matu_0\matq_1\dots\matq_k.
  \end{gather}
  Furthermore, $\mata_k$ can be
  computed from the QR factorization $\mata_{k-1}= \matq_k\matr_k$ as
  \begin{gather}
    \mata_k = \matr_k\matq_k.
  \end{gather}
\end{Lemma}

\begin{proof}
  By the definition of $\mata_{k-1}$ and
  \slideref{Algorithm}{subspace-iteration}, we have
  \begin{gather*}
    \mata_{k-1} = \matu_{k-1}^*\mata\matu_{k-1}
    = \matu_{k-1}^* \maty_{k-1}
    = \left(\matu_{k-1}^*\matu_{k}\right)\matr_k = \matq_k\matr_k,
  \end{gather*}
  where we see that on the right we have obtained the QR factorization of $\mata_{k-1}$.
  On the other hand,
  \begin{align*}
    \mata_k
    &= \matu_k^*\mata\matu_k\\
    &= \matu_k^*\mata\matu_{k-1}\matu_{k-1}^*\matu_k\\
    &= \matu_k^*\maty_{k-1} \matu_{k-1}^*\matu_k\\
    &= \matr_k \left(\matu_{k-1}^*\matu_k\right) = \matr_k\matq_k.
  \end{align*}
  Thus, we see that $\mata_k$ is obtained by multiplying the QR
  factorization of $\mata_{k-1}$ in reverse order.  We have already
  replaced $\matq_k = \matu_{k-1}^*\matu_k$ above. Mutliplying from
  the left by $\matu_{k-1}$ and using $\matu_0=\matq_0=\id$ yields the
  formula for $\matu_k$ by induction.

  Now, let $\matq_0=\id$ such that
  $\mata_0 = \matq_0^*\mata\matq_0=\mata$. Then, for $k\ge 1$ we
  obtain the recursion
  \begin{gather}
    \mata_{k} = \matr_k\matq_k = \matq_k^* \matq_k \matr_k \matq_k = \matq_k^* \mata_{k-1} \matq_k.
  \end{gather}
\end{proof}

\begin{Algorithm*}{qr-iteration}{QR iteration}
  \begin{algorithmic}[1]
    \Require $\mata_0\in\Cnn$.
    \For{$k=1,\dots$ until convergence}
    \State $\matq_k\matr_k \gets \mata_{k-1}$ \Comment{QR factorization}
    \State $\mata_{k} = \matr_k\matq_k$
    \EndFor
  \end{algorithmic}
\end{Algorithm*}

\begin{Lemma}{qr-Schur}
  If convergent, the QR iteration converges to the Schur canonical
  form of the matrix $\mata$ with eigenvalues sorted according to
  their modulus.
\end{Lemma}

\begin{proof}
  See~\ref{par:qr:intro}.
\end{proof}

\begin{Lemma}{qr-1}
  The matrices $\mata_k$ of the QR-iteration with $\mata_0 = \mata$
  have the following properties:
  \begin{enumerate}
  \item If $\matq_0=\id$, $\mata_{k} = \matq_k^*\mata_{k-1}\matq_k = \matq_k^*\dots\matq_0^*\mata\matq_0\dots\matq_k$.
  \item $\mata^k=\matq_1\dots\matq_k\matr_k\dots\matr_1$.
  \item If $\mata$ is normal, so is $\mata_k$ for any $k$.
  \item If $\mata$ is complex symmetric, so is $\mata_k$ for any $k$.
  \end{enumerate}
\end{Lemma}

\begin{proof}
  The first relation follows from~\eqref{eq:qr:intro:1}
  and~\eqref{eq:qr:iteration-derivation:1} letting $\matu_0=\id$,
  which corresponds to $\mata_0 = \matu_0^*\mata\matu_0 = \matq_0^*\mata\matq_0 = \mata$.
  
  The second, we prove by induction, where
  $\mata = \mata_0 = \matq_1\matr_1$ follows directly from the first
  step of the algorithm. For the induction step, we abbreviate
  \begin{gather}
    \matu_k = \matq_1\dots\matq_k,\qquad \mats_k = \matr_k\dots\matr_1.
  \end{gather}
  Assuming $\mata^k = \matu_k\mats_k$, we obtain
  \begin{gather}
    \mata^{k+1} = \mata\matu_k\mats_k = \matu_k\mata_{k+1}\mats_k
    = \matu_k\matq_{k+1}\matr_{k+1}\mats_k = \matu_{k+1}\mats_{k+1}.
  \end{gather}
\end{proof}

%%%%%%%%%%%%%%%%%%%%%%%%%%%%%%%%%%%%%%%%%%%%%%%%%%%%%%%%%%%%%%%%%%%%%%
\subsection{Hessenberg matrices}
%%%%%%%%%%%%%%%%%%%%%%%%%%%%%%%%%%%%%%%%%%%%%%%%%%%%%%%%%%%%%%%%%%%%%%
\begin{intro}
  In each step of the QR-iteration, a QR factorization of the matrix
  is needed, which requires $\bigo(n^3)$ operations. Thus, the
  complexity of the iteration is highly unfavorable. The following
  discussion will provide us with means to reduce the complexity of
  the QR factorization to $\bigo(n^2)$, in the symmetric case even to
  $\bigo(n)$.
\end{intro}

\begin{Definition}{hessenberg}
  A matrix is in \define{Hessenberg form} or is a \define{Hessenberg
    matrix}, if all its entries below the first subdiagonal are zero. Visually,
  \begin{gather}
    H =
    \begin{pNiceMatrix}
      *&\Cdots&\Cdots&\Cdots&*\\
      *&\Ddots&&&\Vdots\\
      &\Ddots&\Ddots&&\Vdots\\
      &&\Ddots&\Ddots&\Vdots\\
      &&&*&*
    \end{pNiceMatrix}
  \end{gather}
  A symmetric or Hermitian Hessenberg matrix is \define{tridiagonal}.
\end{Definition}

\begin{Algorithm*}{Hessenberg-qr-1}{Explicit Hessenberg QR step}
  \index{Hessenberg QR!explicit}
  \begin{algorithmic}[1]
    \Require $\matH\in\Cnn$ in Hessenberg form
    \For{$k=1,\dots,n-1$}
    \Comment{factorization $\matq\matr = \matH$, $\matr$ stored in $\matH$}
    \State $\givens_{k,k+1} \gets$ Givens rotation for $h_{kk},h_{k+1,k}$
    \State $\matH\gets \givens^*_{k,k+1}\matH$
    \EndFor
    \For{$k=1,\dots,n-1$}
    \Comment{$\matH = \matr\matq$}
    \State $\matH\gets \matH\givens_{k,k+1}$
    \EndFor
  \end{algorithmic}
\end{Algorithm*}

\begin{example}
  At the example of a 5-by-5-matrix, we show how this algorithm works.
  \begin{enumerate}
  \item Apply a Givens rotation from the left which eliminates the value $Y$ from the matrix. It affects the two top rows.
    \begin{gather*}
      \left(\begin{NiceArray}[margin,hvlines,corners=SW]{ccccc}
          \CodeBefore
          \rectanglecolor{yellow}{1-2}{2-5}
          \Body
          \cellcolor{green}x&\Block{1-4}{}*&*&*&*\\
          \cellcolor{green}y&\Block{1-4}{}*&*&*&*\\
          &\Block{1-4}{}*&*&*&*\\
          &&\Block{1-3}{}*&*&*\\
          &&&\Block{1-2}{}*&*
        \end{NiceArray}\right)
      \underrightarrow{\quad\matg_{12}^*\matH\quad}
      \left(\begin{NiceArray}[margin,hvlines,corners=SW]{ccccc}
          \Block{1-5}{}*&*&*&*&*\\
          &\Block{1-4}{}*&*&*&*\\
          &\Block{1-4}{}*&*&*&*\\
          &&\Block{1-3}{}*&*&*\\
          &&&\Block{1-2}{}*&*
        \end{NiceArray}\right)    
    \end{gather*}

  \item Do the same with the following row
    \begin{gather*}
      \left(\begin{NiceArray}[margin,hvlines,corners=SW]{ccccc}
          \CodeBefore
          \rectanglecolor{yellow}{2-3}{3-5}
          \Body
          \Block{1-5}{}*&*&*&*&*\\
          &\cellcolor{green}x&\Block{1-3}{}*&*&*\\
          &\cellcolor{green}y&\Block{1-3}{}*&*&*\\
          &&\Block{1-3}{}*&*&*\\
          &&&\Block{1-2}{}*&*
        \end{NiceArray}\right)
      \underrightarrow{\quad\matg_{23}^*\matH\quad}
      \left(\begin{NiceArray}[margin,hvlines,corners=SW]{ccccc}
          \Block{1-5}{}*&*&*&*&*\\
          &\Block{1-4}{}*&*&*&*\\
          &&\Block{1-3}{}*&*&*\\
          &&\Block{1-3}{}*&*&*\\
          &&&\Block{1-2}{}*&*
        \end{NiceArray}\right)    
    \end{gather*}
    to the last pair
    \begin{gather*}
      \left(\begin{NiceArray}[margin,hvlines,corners=SW]{ccccc}
          \CodeBefore
          \rectanglecolor{yellow}{4-5}{5-5}
          \Body
          \Block{1-5}{}*&*&*&*&*\\
          &\Block{1-4}{}*&*&*&*\\
          &&\Block{1-3}{}*&*&*\\
          &&&\cellcolor{green}x&*\\
          &&&\cellcolor{green}y&*\\
        \end{NiceArray}\right)
      \underrightarrow{\quad\matg_{45}^*\matH\quad}
      \left(\begin{NiceArray}[margin,hvlines,corners=SW]{ccccc}
          \Block{1-5}{}*&*&*&*&*\\
          &\Block{1-4}{}*&*&*&*\\
          &&\Block{1-3}{}*&*&*\\
          &&&\Block{1-2}{}*&*\\
          &&&&*
        \end{NiceArray}\right)    
    \end{gather*}
    Note that columns with two zero entries remain unchanged and will not have to be processed.
  \item Now the matrix is upper triangular and the transformation was
    \begin{gather*}
      \matq^* = \givens_{45}^*\givens_{34}^*\givens_{23}^*\givens_{12}^*.
    \end{gather*}

  \item When we go back, we apply Givens rotations from the right, thus affecting columns of the matrices.
    \begin{multline*}\small
      \left(\begin{NiceArray}[margin,hvlines,corners=SW]{ccccc}
          \CodeBefore
          \rectanglecolor{yellow}{1-1}{1-1}
          \rectanglecolor{yellow}{1-2}{2-2}
          \Body
          *&\Block{2-1}{}*&\Block{3-1}{}*&\Block{4-1}{}*&\Block{5-1}{}*\\
          &*&*&*&*\\
          &&*&*&*\\
          &&&*&*\\
          &&&&*
        \end{NiceArray}\right)
      \underrightarrow{\quad\matH\matg_{12}\quad}
      \left(\begin{NiceArray}[margin,hvlines,corners=SW]{ccccc}
          \CodeBefore
          \rectanglecolor{yellow}{1-2}{2-2}
          \rectanglecolor{yellow}{1-3}{3-3}
          \Body
          \Block{2-1}{}*&\Block{2-1}{}*&\Block{3-1}{}*&\Block{4-1}{}*&\Block{5-1}{}*\\
          *&*&*&*&*\\
          &&*&*&*\\
          &&&*&*\\
          &&&&*
        \end{NiceArray}\right)
      \underrightarrow{\quad\matH\matg_{23}\quad}
      \cdots\\\small
      \underrightarrow{\quad\matH\matg_{34}\quad}
      \left(\begin{NiceArray}[margin,hvlines,corners=SW]{ccccc}
          \CodeBefore
          \rectanglecolor{yellow}{1-4}{4-4}
          \rectanglecolor{yellow}{1-5}{5-5}
          \Body
          \Block{2-1}{}*&\Block{3-1}{}*&\Block{4-1}{}*&\Block{4-1}{}*&\Block{5-1}{}*\\
          *&*&*&*&*\\
          &*&*&*&*\\
          &&*&*&*\\
          &&&&*
        \end{NiceArray}\right)
      \underrightarrow{\quad\matH\matg_{45}\quad}
      \left(\begin{NiceArray}[margin,hvlines,corners=SW]{ccccc}
          \Block{2-1}{}*&\Block{3-1}{}*&\Block{4-1}{}*&\Block{5-1}{}*&\Block{5-1}{}*\\
          *&*&*&*&*\\
          &*&*&*&*\\
          &&*&*&*\\
          &&&*&*
        \end{NiceArray}\right)
    \end{multline*}    
  \end{enumerate}
\end{example}

\begin{remark}
  While the previous algorithmdoes the job, it has some
  disadvantages. For once, the Givens rotations accumulated in the
  first part must be stored to be applied in the second. Then, if
  applied to a tridiagonal matrix, the result of the first step has no
  subdiagonal, but two superdiagonals. And finally, if applied to a
  symmetric matrix, all intermediate results are nonsymmetric.

  These drawbacks have led to a second algorithm, where each Givens
  rotation is applied from the left and the right at the same time and
  $\matr$ is never explicitly computed.

  After introducing this algorithm, we will obviously have to make
  sure that its result is useful.
\end{remark}

\begin{Algorithm*}{Hessenberg-qr-2}{Implicit Hessenberg QR step}
  \begin{algorithmic}[1]
    \Require $\matH\in\Cnn$ in Hessenberg form
    \State $\givens_{1,2} \gets$ Givens rotation for $h_{11},h_{21}$
    \State $\matH \gets \givens^*_{1,2}\matH \givens_{1,2}$
    \For{$k=2,\dots,n-1$}
    \State $\givens_{k,k+1}\gets \givens_{k,k+1}[h_{k,k-1},h_{k+1,k-1}]$ Givens rotation
    \State $\matH \gets \givens^*_{k,k+1}\matH \givens_{k,k+1}$
    \EndFor
  \end{algorithmic}
\end{Algorithm*}

\begin{example}
  At the example of a 5-by-5-matrix, we show how this algorithm works.
  \begin{enumerate}
  \item Apply a Givens rotation from the left which eliminates the value $Y$ from the matrix. It affects the two top rows. By
    applying the Givens rotation from the right an additional non zero entry below the subdiagonal is created.
    \begin{gather*}\small
      \left(\begin{NiceArray}[margin,hvlines,corners=SW]{ccccc}
          \CodeBefore
          \rectanglecolor{yellow}{1-2}{2-5}
          \Body
          \cellcolor{green}x&\Block{1-4}{}*&*&*&*\\
          \cellcolor{green}y&\Block{1-4}{}*&*&*&*\\
          &\Block{1-4}{}*&*&*&*\\
          &&\Block{1-3}{}*&*&*\\
          &&&\Block{1-2}{}*&*
        \end{NiceArray}\right)
      \to%\underrightarrow{\quad\matg_{12}^*\matH\quad}
      \left(\begin{NiceArray}[margin,hvlines,corners=SW]{ccccc}
          \CodeBefore
          \rectanglecolor{yellow}{1-1}{3-2}
          \Body
          *&\Block{3-1}{}*&\Block{4-1}{}*&\Block{5-1}{}*&\Block{5-1}{}*\\
          &*&*&*&*\\
          &*&*&*&*\\
          &&*&*&*\\
          &&&*&*
        \end{NiceArray}\right)
      \to%\underrightarrow{\quad\matg_{12}^*\matH\matg_{12}\quad}
      \left(\begin{NiceArray}[margin,hvlines,corners=SW]{ccccc}
          \Block{3-1}{}*&\Block{3-1}{}*&\Block{4-1}{}*&\Block{5-1}{}*&\Block{5-1}{}*\\
          *&*&*&*&*\\
          \cellcolor{red}*&*&*&*&*\\
          &&*&*&*\\
          &&&*&*
        \end{NiceArray}\right)    
    \end{gather*}
\item Apply a Givens rotation from the left to restore Hessenberg form of the first column.
    \begin{gather*}\small
      \left(\begin{NiceArray}[margin,hvlines,corners=SW]{ccccc}
          \CodeBefore
          \rectanglecolor{yellow}{2-2}{3-5}
          \Body
          \Block{1-5}{}*&*&*&*&*\\
          \cellcolor{green}x&\Block{1-4}{}*&*&*&*\\
          \cellcolor{green}y&\Block{1-4}{}*&*&*&*\\
          &&\Block{1-3}{}*&*&*\\
          &&&\Block{1-2}{}*&*
        \end{NiceArray}\right)
      \to%\underrightarrow{\quad\matg_{12}^*\matH\quad}
      \left(\begin{NiceArray}[margin,hvlines,corners=SW]{ccccc}
          \CodeBefore
          \rectanglecolor{yellow}{1-2}{4-3}
          \Body
          \Block{2-1}{}*&\Block{3-1}{}*&\Block{4-1}{}*&\Block{5-1}{}*&\Block{5-1}{}*\\
          *&*&*&*&*\\
          &*&*&*&*\\
          &&*&*&*\\
          &&&*&*
        \end{NiceArray}\right)    
      \to%\underrightarrow{\quad\matH\matg_{12}\quad}
      \left(\begin{NiceArray}[margin,hvlines,corners=SW]{ccccc}
          \Block{2-1}{}*&\Block{4-1}{}*&\Block{4-1}{}*&\Block{5-1}{}*&\Block{5-1}{}*\\
          *&*&*&*&*\\
          &*&*&*&*\\
          &\cellcolor{red}*&*&*&*\\
          &&&*&*
        \end{NiceArray}\right)    
    \end{gather*}
    
  \item Do the same with the second column.
    
    \begin{gather*}\small
      \left(\begin{NiceArray}[margin,hvlines,corners=SW]{ccccc}
          \CodeBefore
          \rectanglecolor{yellow}{3-3}{4-5}
          \Body
          \Block{1-5}{}*&*&*&*&*\\
          \Block{1-5}{}*&*&*&*&*\\
          &\cellcolor{green}x&\Block{1-3}{}*&*&*\\
          &\cellcolor{green}y&\Block{1-3}{}*&*&*\\
          &&&\Block{1-2}{}*&*
        \end{NiceArray}\right)
      \to%\underrightarrow{\quad\matg_{23}^*\matH\quad}
      \left(\begin{NiceArray}[margin,hvlines,corners=SW]{ccccc}
          \CodeBefore
          \rectanglecolor{yellow}{1-3}{5-4}
          \Body
          \Block{2-1}{}*&\Block{3-1}{}*&\Block{4-1}{}*&\Block{5-1}{}*&\Block{5-1}{}*\\
          *&*&*&*&*\\
          &*&*&*&*\\
          &&*&*&*\\
          &&&*&*
        \end{NiceArray}\right)
      \to%\underrightarrow{\quad\matH\matg_{23}\quad}
      \left(\begin{NiceArray}[margin,hvlines,corners=SW]{ccccc}
          \Block{2-1}{}*&\Block{3-1}{}*&\Block{5-1}{}*&\Block{5-1}{}*&\Block{5-1}{}*\\
          *&*&*&*&*\\
          &*&*&*&*\\
          &&*&*&*\\
          &&\cellcolor{red}*&*&*
        \end{NiceArray}\right)    
    \end{gather*}
  
  \item Finally with the last row
    
    \begin{gather*}\small
      \left(\begin{NiceArray}[margin,hvlines,corners=SW]{ccccc}
          \CodeBefore
          \rectanglecolor{yellow}{4-4}{5-5}
          \Body
          \Block{1-5}{}*&*&*&*&*\\
          \Block{1-5}{}*&*&*&*&*\\
          &\Block{1-4}{}*&*&*&*\\
          &&\cellcolor{green}x&\Block{1-2}{}*&*\\
          &&\cellcolor{green}y&\Block{1-2}{}*&*
        \end{NiceArray}\right)
      \to%\underrightarrow{\quad\matg_{23}^*\matH\quad}
      \left(\begin{NiceArray}[margin,hvlines,corners=SW]{ccccc}
          \CodeBefore
          \rectanglecolor{yellow}{1-4}{5-5}
          \Body
          \Block{2-1}{}*&\Block{3-1}{}*&\Block{4-1}{}*&\Block{5-1}{}*&\Block{5-1}{}*\\
          *&*&*&*&*\\
          &*&*&*&*\\
          &&*&*&*\\
          &&&*&*
        \end{NiceArray}\right)
      \to%\underrightarrow{\quad\matH\matg_{23}\quad}
      \left(\begin{NiceArray}[margin,hvlines,corners=SW]{ccccc}
          \Block{2-1}{}*&\Block{3-1}{}*&\Block{4-1}{}*&\Block{5-1}{}*&\Block{5-1}{}*\\
          *&*&*&*&*\\
          &*&*&*&*\\
          &&*&*&*\\
          &&&*&*
        \end{NiceArray}\right)    
    \end{gather*}

    Note that columns with two zero entries remain unchanged and will not have to be processed.
  \end{enumerate}
\end{example}

\begin{Remark}{bulge-chasing}
  This algorithm is called \define{bulge chasing} with the following
  image in mind. After the application of the first rotation, there is
  a bulge protruding down from the Hessenberg form in the furst
  column. This bulge is then ``chased'' down row by row until it
  leaves the matrix at the bottom.
    \begin{gather}\arraycolsep0pt
    \begin{pNiceMatrix}
      *&\Cdots&\Cdots&\Cdots&\Cdots&*\\[-5pt]
      *&\Ddots&&&&\Vdots\\[-5pt]
      &\Ddots&\Ddots&&&\Vdots\\[-5pt]
      &*&\Ddots&\Ddots&&\Vdots\\[-5pt]
      &&&\Ddots&\Ddots&\Vdots\\[-5pt]
      &&&&*&*
    \end{pNiceMatrix}
    \to
    \begin{pNiceMatrix}
      *&\Cdots&\Cdots&\Cdots&\Cdots&*\\[-5pt]
      *&\Ddots&&&&\Vdots\\[-5pt]
      &\Ddots&\Ddots&&&\Vdots\\[-5pt]
      &&\Ddots&\Ddots&&\Vdots\\[-5pt]
      &&*&\Ddots&\Ddots&\Vdots\\[-5pt]
      &&&&*&*
    \end{pNiceMatrix}
  \end{gather}
\end{Remark}

\begin{Definition}{hessenberg-unreduced}
  A Hessenberg matrix is called \define{unreduced} if all entries on
  the first subdiagonal are nonzero. It is called \define{reduced}
  otherwise.
\end{Definition}

\begin{Theorem*}{implicit-Q}{Implicit Q Theorem}
  Let $\mata\in\Cnn$ arbitrary, let $\matq,\matv\in\Cnn$ unitary such that
  \begin{gather}
    \matq^*\mata\matq = \matH,\qquad \matv^*\mata\matv = \matg,
  \end{gather}
  where $\matH$ and $\matg$ are Hessenberg matrices.
  Let $k$ denote
  the smallest integer such that $h_{k+1,k} = 0$, or $k=n$ if $\matH$
  is unreduced. Assume $\vv_1 \parallel \vq_1$. Then,
  $\vv_j\parallel \vq_j$ and $\abs{h_{j+1,j}} = \abs{g_{j+1,j}}$
  for $j=1,\dots,k-1$. If $k<n$, then $g_{k+1,k} = 0$.
\end{Theorem*}

\begin{proof}
  We define the matrix $\matw = \matv^*\matq$, which is unitary as the product of unitary matrices. There holds
  \begin{gather}
    \matg\matw = \matv^*\mata\matv\matv^*\matq = \matv^*\mata\matq
    = \matv^*\matq\matq^*\mata\matq = \matw\matH.
  \end{gather}
  Spelling out column $j$ of this product, we obtain
  \begin{gather}
    \label{eq:qr:implicitq-1}
    \matg\vw_j = \sum_{k=1}^{j+1} h_{kj} \vw_{k}.
  \end{gather}
  We use this equality to show by induction over the columns that the
  entries $w_{ij}$ of $\matw$ are zero for $i>j$. In other words,
  $\matw$ is upper triangular.  For $j=1$, we obtain from the
  unitarity of $\matq$ and $\matv$ and the parallelity of $\vq_1$
  and $\vv_1$ that $\vw_1 = e^{i\phi} \ve_1$ for some argument $\phi$.

  Now let the statement be proven for all columns of $\matw$ up to
  column $j$. Then, from~\eqref{eq:qr:implicitq-1} we obtain
  \begin{gather}
    h_{j+1,j}\vw_{j+1} = \matg\vw_j - \sum_{\nu=1}^j h_{\nu j} \vw_{\nu}.
  \end{gather}
  For each vector in the sum, there holds $(\vw_{\nu})_i = 0$ for
  $i>j$. Since $\matg$ is Hessenberg and applied to $\matw_j$, the
  last possibly nonzero entry of the product is in position $j+1$,
  what we wanted to show.

  Since every unitary matrix is normal and due to
  \slideref{Problem}{normal-triangular-diagonal} every triangular
  normal matrix is diagonal, the matrix $(\vw_1,\dots,\vw_k)$ is
  diagonal with diaognal entries of the form $w_{jj} = e^{i\phi_j}$
  with some arguments $\phi_j$. Hence, $\vv_j = e^{-i\phi_j} \vq_j$
  for $j=1,\dots,k$ and thus $\vv_j$ is parallelto $\vq_j$.

  There holds for $j<k$
  \begin{gather}
    \abs{h_{j+1,j}} = \abs{\ve_{j+1}^*\matH\ve_j} = \abs{\vq_{j+1}^*\mata\vq_j}
    = \abs{\vv_{j+1}^*\mata\vv_j} = \abs{g_{j+1,j}}.
  \end{gather}

  If $k<n$, that is, $h_{k+1,k}=0$, it remains to show that
  \begin{multline}
    g_{k+1,k} = \ve_{k+1}^*\matg\ve_k = e^{i\phi_k}\ve_{k+1}^*\matg\matw\ve_k
    =  e^{i\phi_k}\ve_{k+1}^*\matw\matH\ve_k\\
    =  e^{i\phi_k}\ve_{k+1}^* \sum_{j=1}^k h_{jk} \matw\ve_j
    =  \sum_{j=1}^k e^{i\phi_k} h_{jk} \ve_{k+1}^* \ve_j = 0.
  \end{multline}
\end{proof}

\begin{Definition}{essentially-equal}
  The \putindex{Implicit Q Theorem} says that two Hessenberg forms of $\mata$ with the same initial reduction vector are \define{essentially equal} in the sense that they only differ by the diagonal scaling $\matg = \matd^{-1}\matH\matd$ where $\matd=\diag(d_1,\dots,d_n)$ matrix with $\abs{d_{i}} = 1$.
\end{Definition}

\begin{Corollary}{Hessenberg-qr-equivalence}
  The two versions of the Hessenberg QR step are essentially equal.
\end{Corollary}

\begin{Problem}{Hessenberg-qr-effort}
  \begin{enumerate}
  \item How many operations do the two versions of the Hessenberg QR step require?
  \item Show that if $\matH$ is Hermitian, the result of the
    Hessenberg QR step is Hermitian as well.
  \end{enumerate}
\end{Problem}

\begin{Corollary}{Hessenberg-qr}
  The complexity of each step of the implicit QR-iteration for Hessenberg matrices is $\bigo(n^2)$. For tridiagonal (complex) symmetric matrices, it is $\bigo(n)$.
\end{Corollary}

\begin{Theorem}{Hessenberg-householder}
  Every matrix $\mata\in\Cnn$ is unitarily similar to a Hessenberg matrix $\matH$, that is,
  \begin{gather}
    \matH = \matq^* \mata \matq.
  \end{gather}
  The matrix $\matq$ can be obtained by $n-2$ \putindex{Householder
    reflection}s.
\end{Theorem}

\begin{proof}
  The proof is constructive and relies on Householder
  transformations. We begin by partitioning the matrix $\mata$ as
  \begin{gather}
    \mata =
    \left(\begin{NiceArray}[margin,hvlines]{ccw{c}{4em}c}
        \Block{1-4}{}*&\Cdots&\Cdots&*\\
        \Block{4-1}{\vv_1}&\Block{4-3}<\huge>{*}&&\\
        &&&\\
        &&&\\
        &&&
      \end{NiceArray}\right)
  \end{gather}
  Now we find the Householder vector $\tilde\vw_1\in\C^{n-1}$ which transforms $\vv_1$ to a multiple of $\ve_1\in \C^{n-1}$ and let
  \begin{gather}
    \vw_1 =
    \begin{pmatrix}
      0\\ \tilde\vw_1
    \end{pmatrix},
    \qquad
    \matq_1 = \id - 2 \frac{\vw_1\vw_1^*}{\vw_1^*\vw_1}.
  \end{gather}
  Note that the multiplication $\matq_1\mata$ leaves the first row of
  $\mata$ unchanged, while $\mata\matq_1$ leaves the first column
  unchanged. Hence,
  \begin{gather}
    \matq_1 \mata =
    \left(\begin{NiceArray}[margin,hvlines]{ccw{c}{4em}c}
        \Block{1-4}{}*&\Cdots&\Cdots&*\\
        *&\Block{4-3}<\huge>{*}&&\\
        \Block{3-1}{0}&&&\\
        &&&\\
        &&&
      \end{NiceArray}\right), 
  \end{gather}
  and this structure does not change by multiplication with $\matq_1$
  from the right. Now partition
  \begin{gather}
    \matq_1\mata\matq_1 =
    \left(\begin{NiceArray}[margin,hvlines]{cccw{c}{4em}c}
        \Block{1-5}{}*&\Cdots&\Cdots&\Cdots&*\\
        *&\Block{1-4}{}*&\Cdots&\Cdots&*\\
        \Block{4-1}{0}&\Block{4-1}{\vv_2}&\Block{4-3}<\huge>{*}&&\\
        &&&&\\
        &&&&\\
        &&&&
      \end{NiceArray}\right),     
  \end{gather}
  and choose the Householder vector $\tilde\vw_2\in\C^{n-2}$ which
  maps $\vv_2$ to a multiple of $\ve_1\in\C^{n-2}$. Let
  \begin{gather}
    \vw_2 =
    \begin{pmatrix}
      0\\0\\ \tilde\vw_2
    \end{pmatrix},
    \qquad
    \matq_2 = \id - 2 \frac{\vw_2\vw_2^*}{\vw_2^*\vw_2},
  \end{gather}
  and observe that multiplication with $\matq_2$ from left and right leaves the first two rows and columns untouched, respectively. Hence,
  \begin{gather}
    \matq_2\matq_1\mata\matq_1\matq_2 =
    \left(\begin{NiceArray}[margin,hvlines]{cccw{c}{4em}c}
        \Block{1-5}{}*&\Cdots&\Cdots&\Cdots&*\\
        *&\Block{1-4}{}*&\Cdots&\Cdots&*\\
        \Block{4-1}{0}&*&\Block{4-3}<\huge>{*}&&\\
        &\Block{3-1}{0}&&&\\
        &&&&\\
        &&&&
      \end{NiceArray}\right).
  \end{gather}
  This algorithm can be continued until the third last entry in the
  last row is set to zero. Note that the operations from the left
  mimic the QR factorization, but start to operate one row below the
  diagonal.
\end{proof}

\begin{Problem}{Hermitian-tridiagonal}
  Show that every (complex) Hermitian matrix is unitarily similar
  to a symmetric tridiagonal matrix with real entries.
\end{Problem}

\begin{Algorithm*}{qr-method}{The Hessenberg QR-Method}
  Compute the spectrum of a matrix $\mata\in\Cnn$ by
  \begin{enumerate}
  \item Use $n-2$ Householder transformations to transform $\mata$ to
    Hessenberg form
    \begin{gather}
     \matH_0 = \matq^*\mata\matq.
   \end{gather}
 \item QR-iteration: perform the implicit Hessenberg QR step until convergence
 \item Store Householder vectors as well as $r$ and $c$ for each
   Givens rotation \textbf{only} if the eigenvectors are desired in the end.
  \end{enumerate}
\end{Algorithm*}

\begin{remark}
  In this algorithm, we focus on computing eigenvalues. The
  eigenvectors are neglected. The could be computed by storing the
  $n-2$ Householder vectors of the transformation to Hessenberg form
  and all parameters of the Givens rotations of the iteration. In the
  end, the vectors can be computed by applying all these unitary
  matrices in the right order to an identity matrix.

  Usually, this is not done, since it results in an inefficient
  algorithm. Furthermore, we will modify the algorithm to deal with
  reduced Hessenberg matrices, which may appear anytime in the
  iteration. Actually, we should point out here that the implicit QR
  step should not be continued in this case since the Implicit Q
  Theorem dos not apply anymore.
\end{remark}

\begin{Lemma}{Hessenberg-rank}
  The rank of an unreduced Hessenberg matrix of dimension $n$ is at
  least $n-1$. In particular, the geometric multiplicity of any
  eigenvalue of such a matrix is one.
\end{Lemma}

\begin{proof}
  If $\mata$ is in Hessenberg form and unreduced, the $k+1$-st entry
  of the $k$-th column vector $\va_k$ for $k\le n-1$ is nonzero, while
  the same entry of all previous column vectors is zero. Hence, the
  first $n-1$ column vectors are linearly independent and the rank of
  $\mata$ is at least $n-1$. Note that this argument does not apply to
  the last column.

  If $\mata$ is unreduced Hessenberg, so is $\mata-\lambda\id$, which
  proves the second statement.
\end{proof}

\subsection{Deflation and shifts}

\begin{intro}
  The goal of this section is the development and justification of a
  method which accelerates convergence of the QR-iteration and
  reducing the effort at the same time. It is based on shifts, like
  for the simple or inverse power method. But, shifts are much more
  powerful here, since we compute not only ``converging subspace'',
  but also its complement. The presentation follows
  mostly~\cite{GolubVanLoan83}.
\end{intro}

\begin{Theorem}{qr-reduction}
  Let the matrix $\matH^{(k)}\in\Cnn$ in the QR iteration be of the
  form
  \begin{gather}
    \matH^{(k)} =
    \begin{pmatrix}
      \matH_{11} & \mata_{12}\\0 & \matH_{22}
    \end{pmatrix}
  \end{gather}
  with Hessenberg matrices $\matH_{11}\in\C^{p\times p}$,
  $\matH_{22}\in \C^{n-p\times n-p}$ and an arbitrary matrix
  $\mata_{12}\in \C^{p\times n-p}$. Then, the matrix $\matq^{(k)}$
  decouples into two diagonal blocks and $\matH^{(k+1)}$ has the same
  form. Thus, the iteration decouples into two separate iterations. If
  $p=n-1$, then $h_{nn}$ approximates an eigenvalue.
\end{Theorem}

\begin{proof}
  We consider the algorithm in its explicit form.  The Givens
  transformation $\givens_{p,p+1}$ can be chosen as identity, since
  the subdiagonal entry in column $p$ is already zero. Hence, the
  product of all givens transformations is
  \begin{gather}
    \matq = \givens_{12}\dots\givens_{p-1,p}\givens_{p+1,p+2}\dots\givens_{n-1,n},
  \end{gather}
  where the first $p-1$ only act on the first $p$ rows/columns and the
  remaining ones on th last $n-p$. Therefore, $\matr\matq$ has the
  same block structure as $\matH^{(k)}$.
\end{proof}


\begin{Algorithm*}{qr-deflation}{Deflation}
  After each step of the shifted QR-iteration monitor the subdiagonal
  elements of $\matH^{(k)}$. Whenever
  \begin{gather}
    \abs{h_{j,j-1}} \le \eps \bigl(\abs{h_{j-1,j-1}}+\abs{h_{jj}}\bigr)
  \end{gather}
  set $h_{j,j-1}=0$.

  If this happens in the last row, consider $h_{nn}=\lambda_n$
  converged and proceed with a matrix of dimension $n-1\times n-1$.

  If this happens in the center of the matrix, proceed with both
  remaining diagonal blocks separately.

\end{Algorithm*}

\begin{Remark}{qr-deflation}
  Deflation changes the matrix in a ``nonorthogonal'' way and thus
  changes the eigenvalues. Their accuracy will be determined by the
  parameter $\eps$ in the end.
\end{Remark}

\begin{remark}
  The purpose of deflation is removing Schur vectors from the
  iteration. Thus, if one of the Schur vectors for a multiple
  eigenvalue has converged, the remaining iterations will deal with
  reduced multiplicity.

  Deflation by itself will not help us to deal with the
  requirement that all eigenvalues must have different modulus, but
  this is solved below in combination with shifts.
\end{remark}

\begin{Algorithm*}{shifted-qr-iteration}{QR iteration with shift}
  \begin{algorithmic}[1]
    \Require $\matH_0 \in\Cnn$, Hessenberg, unreduced
    \For {$k=1,\ldots$ until convergence}
    \State $\matq_k\matr_k = \matH_{k-1} - \sigma_k\id$\Comment{QR factorization}
    \State $\matH_{k} = \matr_k\matq_k + \sigma_k\id$
    \EndFor
  \end{algorithmic}
  There is an implicit form of the shifted QR step which follows
  exactly the version outlined for the unshifted case.
\end{Algorithm*}

\begin{Lemma}{shifted-qr-similarity}
  The matrices $\matH_k$ generated by the QR iteration with shifts
  admit the recurrence relation
  \begin{gather}
    \matH_k = \matq_k^*\matH_{k-1}\matq_k.
  \end{gather}
\end{Lemma}  

\begin{proof}
  The proof is almost identical to \slideref{Lemma}{qr-1}. There holds
  \begin{multline}
    \matH_k = \matr_k\matq_k + \sigma_k \id
    = \matq_k^*\matq_k\matr_k\matq_k + \sigma_k \id\\
    =\matq_k^*\left(\matH_{k-1}-\sigma_k\id\right)\matq_k + \sigma_k \id
    =\matq_k^*\matH_{k-1}\matq_k.
  \end{multline}
\end{proof}

\begin{Lemma*}{perfect-shift}{Perfect shift}
  Let $\matH\in\Cnn$ be an unreduced Hessenberg matrix with eigenvalue
  $\sigma$. Let $\matq\matr = \matH - \sigma\id$ be a QR factorization
  and $\widetilde\matH = \matr\matq+\sigma\id$. Then,
  $\tilde h_{n,n-1}=0$ and $\tilde h_{nn} =\sigma$.
\end{Lemma*}

\begin{proof}
  See also~\cite[Theorem 7.5.1]{GolubVanLoan83}.  Since $\matH$ is
  unreduced, its first $n-1$ columns are linearly independent. Hence,
  if $\matq\matr=\matH-\sigma\id$ is a QR factorization, then
  $r_{ii} \neq 0$ for $i=1,\dots,n-1$.

  Since $\matH-\sigma\id$ is singular, we conclude $r_{nn}=0$. Thus,
  the last row of $\matr\matq$ is zero and the statement holds.
\end{proof}

\begin{intro}
  Obviously, if we knew an eigenvalue, we could deflate right
  away. Thus, the next step in the development of the algorithm is the
  determination of a \define{shift strategy} which drives $h_{n,n-1}$
  to zero by approximating the last eigenvalue.

  Such a shift strategy selects a new shift parameter $\sigma_k$ in
  every step of the algorithm. The shift strategies differ in the
  approximation of the eigenvalue which is closest to $h_{nn}$.
\end{intro}

\begin{Example*}{rayleigh-shift}{Rayleigh shift}
  The Rayleigh quotient for the smallest eigenvalue by magnitude
  converges to $h_{nn}$, as
  \begin{gather}
    \ve_n^* H^{(k)} \ve_n = h_{nn}^{(k)}
  \end{gather}
  and $\vq_n$ is orthogonal to all eigenvectors for eigenvalues of
  greater magnitude. Therefore, using $\sigma_k = h_{nn}^{(k)}$ seems
  a good idea, and often is. But it is not reliable, as in the example
  \begin{gather}
    H =
    \begin{pmatrix}
      0 & 1 \\ 1 & 0
    \end{pmatrix}.
  \end{gather}
\end{Example*}

\begin{Definition*}{wilkinson-shift}{Wilkinson shift}
  Let
  \begin{gather}
    \matm =
    \begin{pmatrix}
      h_{n-1,n-1}^{(k)}&h_{n-1,n}^{(k)}\\h_{n,n-1}^{(k)}&h_{nn}^{(k)}
    \end{pmatrix}.
  \end{gather}
  Then, for $\sigma_k$ use the eigenvalue of $\matm$ which is closer
  to $h_{nn}^{(k)}$.
\end{Definition*}

\begin{Example}{wilkinson-failure}
  Consider the orthogonal matrix
  \begin{gather}
    \mata =
    \begin{pmatrix}
      0&0&1\\1&0&0\\0&1&0
    \end{pmatrix}.
  \end{gather}
  The lower right block has a single eigenvalue zero, such that the
  Wilkinson shift and the Rayleigh shift for this matrix are zero. The eigenvalues of $\mata$ are
  \begin{gather}
    \sigma(\mata) = \left\{1, -\tfrac12 \pm \sqrt{\tfrac34}i\right\},
  \end{gather}
  which all have the same modulus. Thus, the algorithm will not converge with either shift.
\end{Example}

\begin{remark}
  The structure of this example for the Wilkinson shift suggests, that
  for every simple shift strategy relying on submatrices, we will find
  a matrix which defeats it. This could be either because the
  iteration stagnates or since it runs in a loop. Hence, we have to
  break this situatuin, which can be achieved by applying a random
  shift.

  A notable exception from this rule are symmetric tridiagonal
  matrices, where we actually have a proof of convergence, see
  \slideref{Theorem}{wilkinson-convergence}.
\end{remark}

\begin{Algorithm}{exceptional-shift}
  If no deflation has ocurred for a given number of iteration steps of
  the shifted QR iteration, the chosen shift strategy has failed.

  In this case, perform a single step with a random shift parameter, a
  so-called \define{exceptional shift}.
\end{Algorithm}

\begin{remark}
  From the necessity to introduce exceptional shifts, we realize that
  a convergence result for the shifted QR iteration is hard to
  obtain. Nevertheless, from the result for the orthogonal subspace
  iteration, we have
  \begin{gather}
    h_{j+1,j}^{(k)} = \bigo \left(\abs*{\frac{\lambda_{j+1}-\sigma}{\lambda_j-\sigma}}^k\right).
  \end{gather}
  Hence, the situation is not hopeless and typically, the algorithm
  converges again after an exceptional shift.
\end{remark}

\begin{Algorithm*}{qr-step-deflation}{QR step with deflation}
  In each step of the QR iteration, first set
  \begin{gather}
    h_{i,i-1} = 0, \qquad \text{where}\quad
    \abs{h_{i,i-1}} \le \eps \left(\abs{h_{i,i}}+\abs{h_{i-1,i-1}}_{\vphantom{g}}\right).
  \end{gather}

  Then, partition the matrix
  $\matH$ as
  \begin{gather}
    \matH =
    \begin{pmatrix}
      \matH_{11} & \matH_{12} & \matH_{13} \\
      & \matH_{22} & \matH_{23}\\
      && \matH_{33}
    \end{pmatrix},
  \end{gather}
  where $\matH_{22}\in\R^{q\times q}$ and
  $\matH_{33}\in\R^{p\times p}$ are chosen maximal such that
  $\matH_{33}$ is upper \putindex{triangular} and $\matH_{22}$
  is unreduced.

  The shifted QR step is then applied to $\matH_{22}$ only.
\end{Algorithm*}

\begin{remark}
  The three diagonal blocks in the preceding algorithm represent different stages in the algorithm:
  \begin{itemize}
  \item $\matH_{33}$ is converged to an upper triangular matrix and
    hence corresponds to a converged Schur form with eigenvalues on
    the diagonal.
  \item $\matH_{22}$ is an unreduced Hessenberg matrix. The shifted QR
    step operates on this block with the goal of driving at least one
    subdiagonal element to zero.
  \item $\matH_{11}$ is the part of the matrix waiting to be handled
    by the algorithm later.
  \end{itemize}
  If a subdiagonal entry of $\matH_{22}$ is in the last row, this last
  row is transferred to the block $\matH_{33}$ in the next step, hence
  reducing the dimension of $\matH_{22}$ by one.

  If such a subdiagonal entry is in an earlier row, we split the
  matrix $\matH_{22}$ at this point. The lower diagonal block becomes
  the new $\matH_{22}$, while the upper one is becoming a part of
  $\matH_{11}$.

  We might save additional iterations by allowing two-by-two blocks on
  the diagonal of $\matH_{33}$. Their eigenvalues can still be
  computed easily by computing the roots of a quadratic polynomial.
  
  We do not compute eigenvectors with this algorithm, such that the
  transformations of $\matH_{13}$ and $\matH_{23}$ can be avoided.
\end{remark}

\begin{remark}
  When implementing \slideref{Algorithm}{qr-step-deflation}, we have
  to be able to run a QR step on a submatrix. Allocating new memory
  and copying the submatrix should be avoided since it comes at
  considerable cost.

  This means on the other hand, that the matrix $\matH_{22}$ will not
  be stored as a consecutive array of $q\times q$ numbers. Depending
  on whether the entries are sorted in row-major or column-major
  order, there will either be a gap between each consecutive element
  of a column or between the last element of one column and the first
  element of the next.

  Thus, the QR step operations must allow for a \define{stride}
  between rows or columns. This can be achieved either by storing the
  matrix as a sequence of column vectors, or by using strided versions
  of the algorithms.

  In FORTRAN, the function DAXPY of the basic linear algebra
  subprograms (BLAS) library has explicit parameters for this
  stride. The submatrix objects of the Armadillo library implement
  this in a transparent way.
\end{remark}

%%% Local Variables:
%%% mode: latex
%%% TeX-master: "main"
%%% End:


\chapter{Iterationsverfahren}

\section{Dünnbesetzte lineare Gleichungssysteme}

\section{Nichtlineare Gleichungssysteme}

\bibliographystyle{apalike}
\bibliography{all}
\printindex

%%% Local Variables:
%%% mode: latex
%%% TeX-master: "main"
%%% End:
