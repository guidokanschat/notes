
\subsection{Vektor- und Matrixnormen}

\begin{Definition}{norm}
  Eine \define{Norm} $\norm{\cdot}$ auf dem Vektorraum $V$ ist eine Abbildung
  \begin{gather}
    \begin{split}
      \norm{\cdot}\colon V &\to \R\\
      x&\mapsto \norm{x}
    \end{split}
  \end{gather}
  mit den Eigenschaften
  \begin{enumerate}
  \item Homogenität
  \item Dreiecksungleichung
  \item Definitheit
  \end{enumerate}
\end{Definition}

\begin{Definition}{norm-aequivalenz}
  Sei $V$ ein reeller Vektorraum. Zwei Normen $\norm{\cdot}_X$ und
  $\norm{\cdot}_Y$ auf $V$ heißen äquivalent, wenn es Konstanten $c>0$
  und $C>0$ gibt, so dass
  \begin{gather}
    c \norm{v}_X \le \norm{v}_Y \le C \norm{v}_X
    \qquad\forall v\in V.
  \end{gather}
\end{Definition}

\begin{Definition}{rn-konvergenz}
  Eine Folge $\{x^{(k)}\}\subset \R^n$ für $k=1,2,\dots$ heißt
  \textbf{komponentenweise konvergent} gegen $x\in \R^n$, wenn gilt
  \begin{gather}
    \forall \epsilon>0\;
    \exists k_0\in \mathbb N\;
    \forall k\ge k_0, i=1,\dots,n
    : \abs{x^{(k)}_i - x_i} < \epsilon.
  \end{gather}
  Die Folge heißt konvergent unter der Norm $\norm{\cdot}$ wenn gilt
  \begin{gather}
    \forall \epsilon>0\;
    \exists k_0\in \mathbb N\;
    \forall k\ge k_0
    : \norm{x^{(k)} - x} < \epsilon.
  \end{gather}
\end{Definition}

\begin{Lemma}{norm-stetig}
  Sei $\norm{\cdot}$ eine Norm auf $\R_n$. Dann ist die Abbildung
  \begin{gather}
    f\colon x \mapsto \norm{x}
  \end{gather}
  stetig bezüglich der komponentenweisen
  Konvergenz.

  Komponentenweise Konvergenz und Normkonvergenz sind äquivalent auf
  $\R^n$ .
\end{Lemma}

\begin{proof}
  
\end{proof}


\begin{Satz}{norm-aequivalenz}
  
\end{Satz}

\subsection{Konditionierung der Lösung}


%%% Local Variables:
%%% mode: latex
%%% TeX-master: "main"
%%% End:
