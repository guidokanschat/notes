Dieser Abschnitt folgt recht eng der Darstellung in \cite[Abschnitt 2.3]{Rannacher17}.

\subsection{Interpolation auf Teilintervallen}

\begin{Notation}{indices}
  In diesem Abschnitt bezeichne für die monotone Folge
  \begin{gather}
    a = x_0 < x_1 \dots < x_n = b
  \end{gather}
  stets
  \begin{gather}
    \mathcal I_h = \bigl\{ I_i = [x_{i-1},x_i] \big|
    \; i=1,\dots,n\bigr\}
  \end{gather}
  eine \define{Zerlegung} des Intervalls $I=[a,b]$, also
  \begin{gather}
    [a,b] = \bigcup_{i=1}^n I_h.
  \end{gather}
  Die Länge der Teilintervalle bezeichnen wir mit
  $h_i = \abs{I_i} = x_{i} - x_{i-1}$, mit $h=\max h_i$ die
  \define{Feinheit} der Unterteilung.
\end{Notation}

\begin{Notation}{reference-interval}
  Wir bezeichnen $\hat I = [-1,1]$ als Referenzintervall. Jedes
  Intervall $I_i$ einer Zerlegung $\mathcal I_h$ ergibt sich als Bild
  von $\hat I$ unter der affinen Abbildung
  \begin{gather}
    \begin{split}
      \Phi_i\colon \hat I &\to I_i\\
      \hat x &\mapsto \tfrac{x_{i}+x_{i-1}}{2} + \tfrac{h_i}{2} \hat x.
    \end{split}
  \end{gather}
\end{Notation}

\begin{Definition*}{piecewise-interpolation}{Stückweise Interpolation}
  Sei $\mathcal I_h$ eine Zerlegung von $[a,b]$. Auf dem
  Referenzintervall $\hat I$ sei eine Interpolationsaufgabe durch die
  Stützstellen $\hat x_0,\dots, \hat x_k$ definiert. Dann lautet die
  Aufgabe der stückweisen Interpolation auf $\mathcal I_h$: finde eine
  Funktion $s$ auf $[a,b]$, so dass für jedes $i=1,\dots,n$ die
  Einschränkung $s_{|I_i} \in \P_k$ der Interpolationsaufgabe mit den
  Stützstellen
  \begin{gather}
    x_{ij} = \Phi_i(\hat x_j),\qquad j=1,\dots,k
  \end{gather}
  genügt.
\end{Definition*}

\begin{Lemma}{piecewise-solvable}
  Die stückweise Interpolationsaufgabe hat eine eindeutige Lösung,
  wenn die Interpolationsaufgabe auf dem Referenzintervall eine solche
  besitzt.
\end{Lemma}

\begin{Lemma*}{scaling-interpolation}{Skalierungsargument}
  Für die Lösung $\hat p\in \P_k$ der Interpolationsaufgabe auf dem
  Referenzintervall gelte mit einer Konstanten $C$ unabhängig von
  $\hat f\in C^{k+1}(\hat I)$ die Fehlerabschätzung
  \begin{gather}
    \norm{\hat f- \hat p}_{\infty;\hat I} \le C \norm{\hat
      f^{(k+1)}}_{\infty;\hat I}.
  \end{gather}
  Dann ist der Fehler der stückweisen Interpolation beschränkt ist durch
  \begin{gather}
    \norm{f-s}_{\infty;[a,b]}
    \le \frac{C}{2^{k+1}} h^{k+1} \norm{f^{(k+1)}}_{\infty;[a,b]}.
  \end{gather}
\end{Lemma*}

\begin{Bemerkung}{scaling-interpolation-local}
  Genauere Betrachtung der Analyse ergibt die schärfere Abschätzung
  \begin{gather}
    \norm{f-s}_{\infty;[a,b]}
    \le \frac{C}{2^{k+1}} \max_{i=1,\dots,n}
    \Bigl(h_i^{k+1}\norm{f^{(k+1)}}_{\infty;I_i}\Bigr).
  \end{gather}  
\end{Bemerkung}

\subsection{Splines}

\begin{Definition}{spline-raum}
  Für stückweise Polynome auf dem Intervall $[a,b]$ mit einer
  Zerlegung $\mathcal I_h$ definieren wir die
  \textbf{Spline-Räume}\index{Spline-Raum}
  \begin{gather}
    S^{(k,m)}_h \ \bigl\{ s\in C^m[a,b]
    \big| s_{|I_i} \in \P_k, i=1,\dots,n\bigr\}
  \end{gather}
  mit $m<k$.
\end{Definition}

\begin{Lemma}{spline-raum}
  Die Dimension von $S^{(k,m)}_h$ ist
  \begin{gather}
    \operatorname{dim}S^{(k,m)}_h = (k-m)n + m+1
  \end{gather}
\end{Lemma}

\begin{proof}
  Betrachten wir die $n$ Wiederholungen des Raums $\P_k$, eine für
  jedes Intervall $I_i$, so ergibt sich $(k+1)n$.  Die Bedingung
  $s\in C^m[a,b]$ bedeutet, dass die Werte und die ersten $m$
  Ableitungen der Funktionen in $S^{(k,m)}$ in jedem inneren Punkt
  $x_i$ für die beiden Intervalle $I_i$ und $I_{i+1}$
  übereinstimmen. Daraus ergeben sich $(n-1)(m+1)$ lineare
  Beschränkungen, so dass die Dimension $(k+1)n - (n-1)(m+1)$ ist.
\end{proof}

\begin{Definition}{cubic-spline}
  Die Interpolationsaufgabe mit kubischen \define{Splines} lautet:
  finde eine Funktion $s\in S_h^{(3,2)}$, so dass
  \begin{gather}
    s(x_i) = f_i,\qquad i=0,\dots,n.
  \end{gather}
\end{Definition}

\begin{Definition}{cubic-spline-bc}
  Da die Anzahl der Interpolationsbedingungen um 2 geringer ist als
  die Dimension des Raumes $S_h^{(3,2)}$ definieren wir folgende,
  alternative Randbedingungen:
  \begin{description}
  \item[Natürlich]
    \begin{gather}
      s''(a) = s''(b) = 0
    \end{gather}
  \item[Periodisch]
    \begin{gather}
      s'(a) = s'(b) \quad \wedge \quad s''(a) = s''(b)
    \end{gather}
  \item[Eingespannt]
    \begin{gather}
      s'(a) = f'(a) \quad \wedge \quad s'(b) = f'(b)
    \end{gather}
  \end{description}
\end{Definition}

\begin{Satz}{cubic-spline}
  Die stückweise kubische Spline-Interpolierende $s\in S_h^{(3,2)}$
  mit natürlicher Randbedingung existiert und ist eindeutig bestimmt.
\end{Satz}

\begin{proof}
  Wie meistens beginnen wir mit der Eindeutigkeit. Seinen $s_1$ und
  $s_2$ zwei Interpolierende der Werte $f_i$ in den Punkten $x_i$,
  $i=0,\dots,n$ und $s=s_2-s_1$. Dann gilt
  \begin{gather}
    \label{eq:splines:n}
    s \in N_h = \bigl\{ w\in C^2[a,b]
    \;\big|\; w(x_i) = 0, \quad i=0,\dots,n \bigr\}.
  \end{gather}
  Zusätzlich gilt $s_{| I_i}\in \P_3$ für alle Intervalle. Wir beobachten, dass für beliebiges $w\in N_h$ gilt
  \begin{align}
    \int_{I_i} s''(x) w''(x)\dx
    &= s''w'\Bigr|^{x_i}_{x_{i-1}} - \int_{I_i} s^{(3)}(x) w'(x)\dx\\
    &= s''w'\Bigr|^{x_i}_{x_{i-1}} - s^{(3)}w \Bigr|^{x_i}_{x_{i-1}}
      + \int_{I_i} s^{(4)}(x) w(x)\dx\\
    &= s''w'\Bigr|^{x_i}_{x_{i-1}}.
  \end{align}
  Summieren wir über alle Intervalle, so ergibt sich
  \begin{gather}
    \int_a^b s''(x) w''(x)\dx = \sum_{i=1}^n s''w'\Bigr|^{x_i}_{x_{i-1}}
    = s''(b) w'(b) - s''(a)w'(a).
  \end{gather}
  Wegen der natürlichen Randbedingung ist dies aber null. Insbesondere
  können wir $w=s$ einsetzen und erhalten
  \begin{gather}
    \int_a^b \abs{s''(x)}^2 \dx = 0
  \end{gather}
  und $s$ muss ein lineares Polynom sein. Aus $s(a) = s(b) = 0$ folgt
  damit $s\equiv 0$ im Widerspruch zur Annahme, dass es zwei Lösungen
  gebe.

  Nach \slideref{Lemma}{spline-raum} hat $S_h^{(3,2)}$ die Dimension
  $n+3$. Andererseits haben wir $n+1$ Interpolationsbedingungen und 2
  Randbedingungen, so dass aus der Eindeutigkeit die Existenz folgt.
\end{proof}

\begin{Lemma}{spline-optimality}
  Unter allen Funktionen $f\in C^2[a,b]$ mit vorgegebenen
  Funktionswerten $f(x_i) = y_i$, $i=0,\dots,n$ ist der natürliche
  Spline $s\in S_h^{(3,2)}$, der diese Punkte interpoliert, diejenige
  mit der kleinsten mittleren zweiten Ableitung, es gilt also
  \begin{gather}
    \int_a^b \abs{s''(x)}^d\dx \le \int_a^b \abs{f(x)}^2 \dx
    \qquad \forall f\in C^2[a,b].
  \end{gather}
\end{Lemma}

\begin{proof}
  Siehe \cite[Satz 2.9]{Rannacher17}
\end{proof}

\begin{Lemma}{splines-konkret}
  Sind die Polynome auf den Teilintervallen $I_i$ dargestellt durch
  \begin{gather}
    s_{|I_i}(x) = a_{i0} + a_{i1} (x-x_i) + a_{i2}(x-x_i)^2 + a_{i3}(x-x_i)^3,
  \end{gather}
  so ergeben sich für $i=1,\dots,n$ die Koeffizienten $a_{i0}$,
  $a_{i1}$ und $a_{i3}$ der natürlichen Splines aus den Gleichungen
  \begin{gather}
    a_{i0} = f_i,
    \quad a_{i1} = \tfrac{f_i-f_{i-1}}{h_i}
    + \tfrac{h_i(2 a_{i2} + a_{i-1,2})}{3} ,
    \quad a_{i3} = \tfrac{a_{i2} - a_{i-1,2}}{3h_i}.
  \end{gather}
  Die Koeffizienten $a_{i2}$ für $i=1,\dots,n-1$ lösen das Gleichungssystem
  
  \begin{gather}
    \begin{pmatrix}
      \overline{h}_2 & h_2\\
      h_2 & \overline{h}_3 & h_3\\
      & \ddots& \ddots& \ddots\\
      &&h_{n-2} & \overline{h}_{n-1} & h_{n-1}\\
      &&&h_{n-1} & \overline{h}_n
    \end{pmatrix}
    \begin{pmatrix}
      a_{12}\\\\\vdots\\\\a_{n-1,2}
    \end{pmatrix}
    =
    \begin{pmatrix}
      3\tfrac{f_2-f_1}{h_2} - 3\tfrac{f_1-f_0}{h_1}
      \\\\\vdots\\\\
      3\tfrac{f_{n}-f_{n-1}}{h_n} - 3\tfrac{f_{n-1}-f_{n-2}}{h_{n-1}}
    \end{pmatrix}
  \end{gather}
  mit $\overline{h}_i = 2 (h_{i-1}+h_i)$ und es gilt
  $a_{12} = 3a_{13}h_1$ sowie $a_{n2}=0$.
\end{Lemma}

%%% Local Variables:
%%% mode: latex
%%% TeX-master: "main"
%%% End:
