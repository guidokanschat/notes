\subsection{Orthogonale Matrizen}

\begin{Definition}{ortho-matrix}
  Eine \define{orthogonale Matrix} ist eine quadratische Matrix, deren
  Spaltenvektoren bzw. deren Zeilenvektoren eine Orthonormalbasis des
  $\R^n$ bilden.
\end{Definition}

\begin{Satz}{ortho-matrix}
  Für eine orthogonale Matrix $Q$ gilt
  \begin{gather}
    \label{eq:qr:1}
    Q^{-1} = Q^T.
  \end{gather}
  Umgekehrt folgt aus dieser Beziehung die Orthogonalität der
  Zeilenvektoren und Spaltenvektoren.
\end{Satz}

\begin{proof}
  Nehmen wir an, die Spaltenvektoren $q^{(1)},\dots,q^{(n)}$ von $Q$
  seinen eine ONB. Dann gilt für die Matrix $A = Q^TQ$:
  \begin{gather}
    a_{ij} = \sum_{k=1}^n q_{ki}q_{kj}
    = \sum_{k=1}^n q^{(i)}_k q^{(j)}_k
    = \bigl(q^{(i)}\bigr)^T q^{(j)} = \delta_{ij}.
  \end{gather}
  Daher gilt $Q^TQ=I$.  Multiplizieren wir diese Gleichung von rechts
  mit $Q^{-1}$, so erhalten wir~\eqref{eq:qr:1}. Setzen wir umgekehrt
  $Q^TQ=I$, so ergibt obige Rechnung die Orthogonalität der
  Spaltenvektoren.

  Aus $Q^T = Q^{-1}$ folgt aber durch Transponieren
  \begin{gather}
    Q = Q^{-T},
  \end{gather}
  wobei $Q^{-T}$ die Inverse von $Q^T$ ist. Multiplizieren wir die
  letzte Gleichung von rechts mit $Q^T$, so erhalten wir $QQ^T = I$,
  was äquivalent zur Orthonormalität der Zeilenvektoren ist.

  Wir hätten diesen Beweis auch mit den Zeilenvektoren beginnen können
  und $QQ^T=I$ folgern. Der Rest verläuft dann analog.
\end{proof}

\begin{example}
  Die Rotationsmatrix
  \begin{gather}
    Q = \begin{bmatrix}
      \cos \phi & \sin \phi\\-\sin\phi\cos\phi
    \end{bmatrix}
  \end{gather}
  ist orthogonal. Dasselbe gilt für die Reflexionsmatrix an einem
  normierten Vektor $w\in \R^n$,
  \begin{gather}
    Q = I-2ww^T
  \end{gather}
\end{example}

\subsection{Konstruktion der QR-Zerlegung}

\subsection{Fehleranalyse}


%%% Local Variables:
%%% mode: latex
%%% TeX-master: "main"
%%% End:
